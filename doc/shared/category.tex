\begin{plSection}{Key ideas in category/topos theory}
\label{sec:Key-ideas}
  
Things various authors claim are important or valuable;
things that ought to be understood,
and ought to be presentable in a natural, intuitive way.
Ordered by mean rank of selected references:
\begin{enumerate}
  \item Topos, elementary topos: 
  category ``essentially the same as \textbf{Set}''
  \item subobject (classifier)
  \item Cartesian closed category
  \item initial, terminal object
  \item product, sum/coproduct, exponential, meet, join
  \item monic/epic morphism
  \item split/retraction,/isomorphism
  \item duality/opposite
  \item equalizers
  \item limits
  \item pullback/pushout
  \item completeness
  \item universal (mapping) property
  \item adjointness
\end{enumerate}

\end{plSection}%{Key ideas in category/topos theory}
%-----------------------------------------------------------------
%-----------------------------------------------------------------
\begin{plSection}{Example categories}
\label{sec:Example-categories}

One question to be examined here is: are there any meaningful
categories that aren't some version of a collection of sets
and a collection of functions between? 
By ``meaningful'', I mean that the key ideas 
in section~\ref{sec:Key-ideas} apply
and are useful in some way.
And, in any of the cases, 
does categorical language help or hurt?

\begin{itemize}
  \item Sets of sets and sets of functions between them.
Sometimes relations rather than functions.
More specifically, mathematical structures and
structure-preserving functions between them.

  \item Monoids, groups, etc. \par
  Each monoid category has a single object which can be taken as
  corresponding to the set of elements. The arrows correspond to 
  individual elements, or to the functions obtained from partial
  evaluation of the monoid operation: 
  $f_A(\_)= \comop (A,\_) = A \comop \_$.\par
  One obvious problem with this is that it excludes about half of
  the group-like, one set, one operation structures~\cite{wiki:Magma}: 
  the ones without identity. 
  An example where identity arrows are a problem. \par
  Also, seems clear that many of the key categorical constructs
  don't apply, raising even more questions about whether there's
  any value.
  
  \item Pre-ordered sets, partially ordered sets, including
  inclusion ordering in topology.
  
  
  \item ``Discrete'' categories. Object are elements of a set; 
  only arrows are identities, one per element. 
  (Another example where identity arrows are problematic.)
  Almost surely of no value.
  
  \item Completion of arbitrary directed graph; 
  almost surely of no value, at least on its own.
  This includes things commonly represented as directed graphs,
  without any additional categorical structure, 
  like ontologies~\cite{Spivak:2013:CatTheoryForScientists},
  state transition diagrams, dependency graphs, etc.
  Do categorical completion and constructs help with anything?
  
  \item Types signatures in ``functional'' programming.
  
  \item Categories of arrows, categories, functors, etc.
  Is there anything here beyond certain sets and functions between
  them?
  \item Slice categories~\cite[sec~2.6.10]{BarrWells:2020}
\end{itemize}
\end{plSection}%{Example categories}
%-----------------------------------------------------------------
%-----------------------------------------------------------------
\begin{plSection}{Bottom up}
\label{sec:Bottom-up}

Start from directed (multi)graph~\cite{wiki:DirectedGraph},
define key categorical ideas (section~\ref{sec:Key-ideas}) 
in those terms.

Claim: categorical constructs are 
pretty much impossible to motivate in this context.
(Like starting from matrices rather than linear functions.)

Look for more natural, simpler alternative definitions?

In this section, I'm going to use graph terminology,
eg, ``vertex'' and ''edge'', rather than ``object'' and ``arrow'',
but I will follow the category theory convention of
upper case letters for vertices/objects and lower case for
edges/arrows.

%-----------------------------------------------------------------
\begin{plSection}{Directed graph}
\label{sec:Directed_graph}

(\textbf{Question:}
 any value is thinking about undirected graphs?
 Simplicial complexes?)

\begin{plDefinition}{Directed graph}{}
A \ding{directed (multi)graph} (aka \ding{digraph}) is
an ordered pair $G=\left( \Set{V}, \Set{E} \right)$,
where $\Set{V}$ is a set of \ding[vertex]{vertices}, 
$\Set{E}$ a set of \ding[edge]{edges}, ordered pairs of vertices.
$V_0 = \tail(e)$ and $V_1 = \head(e)$
for edge $e = \left[ V_0, V_1 \right]$.
\end{plDefinition}

In (computational) applications, $\Set{V}$ and $\Set{E}$ 
are finite, perhaps unbounded finite,
but, in general, they need not be.

It's common to assume 
\begin{enumerate}
\item there is at most one edge
in a graph connecting any $2$ vertices.\par
\item the vertices in an edge must be 
distinct (no loop edges).
\end{enumerate}
(\textbf{Question:} does ``ordered pair'' imply uniqueness?
In other words, can we have multiple distinct copies of
 $\left(A,B\right)$, pairing the same elements $A$ and $B$?
 If we use one of the primitive set theory ``implementations'',
 then there is only one $\left(A,B\right)$ 
 for given $A$ and $B$.)

A graph may be called a \ding{multigraph} when $1$ is
not assumed.

In a context where loops are allowed,
a graph with no loops may be called 
\ding[simple graph]{simple}.

(\textbf{Question:} 
Digraph special case of (oriented) simplicial complex.
What's the difference between a simplicial complex 
and a hypergraph?)

A \ding{path} is a sequence of edges where the head of each
edge matches the tail of the following edge, if there is one.

An \ding{directed acyclic graph} (aka DAG) has no paths
connecting a vertex to itself.

A directed graph is 
\ding[connected graph]{(strongly) connected} 
if there is a path
from any vertex to any other (weakly if there is path
ignoring direction.

(\textbf:{TODO:} straighten out \ding{path} vs \ding{walk} vs
\ding{trail}; directed vs undirected; sequence of alternating
vertices and edges, starting and beginning with a vertex.)

A \ding{root} vertex (aka source) has no entering edges.

A \ding{leaf} vertex (aka sink) has no departing edges.

A \ding{tree} is a directed graph with one root, 
where all vertices have at most entering edge.

(Mesh) \ding[dual graph]{dual} of a graph interchanges the roles of edges 
and vertices, resulting in a hypergraph where every dual vertex
has exactly $2$ edges. 
(\textbf{Question:} orientation/direction 
of dual edges?~\cite{Rusnak:2012:OrientedHyperGraphs})
(Other notion of dual of planar graph interchanges 
vertices and ``faces'',
orientation from $90^{\comop}$ 
turn of primal edge.~\cite{wiki:DualGraph})

\begin{plDiagram}[nofloat]
{A directed (acyclic) graph}
{digraph}
\centering
\begin{tikzcd}[
row sep=1.2cm, 
column sep=2.5cm,
ampersand replacement = \&,
%execute at begin picture={
%     \useasboundingbox (-5.15,-3.35) rectangle (4.25,3.4); }
    ]
V_0
\arrow[drr, "e_{02}", bend left]
\arrow[ddr, "e_{03}"', bend right]
\arrow[dr, "{e_{01}}"] 
\& 
\& 
\\
\& 
V_1 
\arrow[r, "e_{12}"] 
\arrow[d, "e_{13}"']
\& 
V_2 
\arrow[d, "e_{24}"] 
\\
\& 
V_3 
\arrow[r, "e_{34}"']
\& V_4
\end{tikzcd}
\end{plDiagram}

An example in \cref{diagram:digraph}:
$V_0$ is the only root; $V_4$ is the only leaf;
the graph is simple, connected, and acyclic.

Applications of directed graphs:
(see, eg, 
\citeAuthorYearTitle{CormenLeisersonRivestStein:2009:Algorithms})
\begin{itemize}
  \item Transportation networks
  \item Shortest path
  \item Predict travel time
  \item Maximum flow
  \item network design/optimization
  \item connected components
  \item spanning trees
  \item dependencies among modules, tasks, definitions/theorems
  \item finite state machines, state-transition diagrams
  \item continuation-passing code (no automatic return to caller)
  \item graph layout (mapping vertices to $2$d points).
\end{itemize}
\end{plSection}
%-----------------------------------------------------------------
\begin{plSection}{Category}
\label{sec:Category_from_digraph}

%-----------------------------------------------------------------
\begin{plSection}{Category (standard)}

\begin{plDefinition}{Category (standard)}{}
A \ding{category} is a directed (multi) graph 
together with two functions that impose constraints
on the graph:
\begin{description}
\item[\compose \textrm{\textup{(infix: $\comop$)}}]\mbox{}\\
At least $2$ ways to think about this. 
More common way first, better (?) way second.\\
The usual way this is posed is for pairs of connected edges:
Whenever $\tail(e0) = \head(e_1)$,
then
\[ \compose(e_0,e_1) = e_0 \comop e_1 = e_{10} \]
is defined
for some $e_{10}$ among the edges of the graph,
with $\tail(e_{10}) = \tail(e_1)$
and $\head(e_{10}) = \head(e_0)$.
(Note the potential for confusion 
in the ordering of the arguments to \compose;
general idea is that $e_1$ is traversed/applied first,
then $e_0$.) \\
In addition, \compose is required to be associative:
\begin{align*}
\compose(\compose(e_0,e_1),e_2) 
&
= \compose(e_0,\compose(e_1,e_2))
\\
&
= \compose(e_0,e_1,e_2)
\end{align*}
\\
An alternative is define the domain of \compose
to be the graph's paths. 
Then, for all paths $p$ in the graph,
there exists some edge $e$ in the graph
with 
\begin{equation*}
\compose(p) = e
\end{equation*} 
such that 
$p$ and $e$ have the same head and tail.
Associativity here means that if we partially reduce a path
by applying \compose to a sub-path,
and then \compose the partially reduced path,
we get the same edge as composing the whole path.
\\
Either way we define \compose (no reason not to do both
simultaneously), it is clearly (to me) not a natural construct
for directed graphs: for every path, there is also 
a single edge directly connecting its tail to its head.
And associativity imposes strong constraints on which of the possibly
many such single edges can be chosen as the value of \compose.
\\
I don't know of any non-categorical applications of directed graphs
for which this makes sense.
Moreover, it's not clear that any of this is really necessary.
\\
\item[\identity]\mbox{}\\
For every vertex $V$, there is a self-loop edge 
$\identity(V)$ whose head and tail
are both $V$.
\compose and \identity
must be defined so that  $\identity(v)$ 
``disappears'' in any composition with edges 
entering or leaving $V$:
\[ 
\compose(\identity(V),e_0) = e_0 \]
and
\[ \compose(e_1,\identity(V)) = e_1 \]
\\
Again, a lot of room for doubt that this extra structure 
is really necessary.
\\
For example, we could extend \compose 
to take vertices as well as edges,
and return both edges and vertices.
Edge-vertex composition just returns the edge (assuming head/tail match).
Composition of ``inverse'' edges is allowed to return the common vertex,
rather than an artificially added self-loop edge.
\end{description}
\end{plDefinition}

\Cref{diagram:aCompletedDigraph} shows the additional
edges needed to make the graph in \cref{diagram:digraph}
into a category: the identity loops are in blue, first order
compositions red, and second order green.

% Note tkzcd bug: bounding box for diagram includes the control 
% pts of the curved edges! 
\begin{plDiagram}
{Completion of \cref{diagram:digraph} to a category}
{aCompletedDigraph}
\centering
%\fbox{
\begin{tikzcd}[
row sep=1.2cm, 
column sep=2.5cm,
ampersand replacement = \&,
execute at begin picture={
     \useasboundingbox (-5.15,-3.35) rectangle (4.25,3.4); }
    ]
V_0
\arrow[loop left, blue, crossing over]
%\arrow[drr, bend left=40, "e_{02}" near start]
%\arrow[ddr, bend right=40, "e_{03}"' near start]
\arrow[drr, bend left, "e_{02}" near start]
\arrow[ddr, bend right, "e_{03}"' near start]
\arrow[dr, "{e_{01}}" ] 
\arrow[drr, "e_{12} \comop e_{01}", bend left=10, dashed, red]
\arrow[ddr, "e_{13} \comop e_{01}", bend right=10, dashed, red]
\& 
\& 
\\
\& 
V_1 
\arrow[loop above, blue,crossing over]
\arrow[r, "e_{12}"] 
\arrow[d, "e_{13}"']
\arrow[dr, bend left, "e_{24} \comop e_{12}", dashed, red]
\arrow[dr, bend right, "e_{34} \comop e_{13}"' , dashed, red]
\& 
V_2 
\arrow[loop right, crossing over, blue]
\arrow[d, "e_{24}"] 
\\
\& 
V_3 
\arrow[loop below, blue,crossing over]
\arrow[r, "e_{34}"']
\&
V_4
\arrow[loop right, blue,crossing over]
\arrow["e_{24} \comop e_{02}" {near end},
from=1-1,to=3-3, red, dashed, bend left=100,]
\arrow["e_{34} \comop e_{03}"' {near end},
from=1-1,to=3-3, red, dashed, bend right=100,]
\arrow["e_{24} \comop e_{12} \comop e_{01}" {near start},
from=1-1,to=3-3, draw=DarkGreen, dashed, bend left=120]
\arrow["e_{34} \comop e_{13} \comop e_{01}"' {near start},
from=1-1,to=3-3, draw=DarkGreen, dashed, bend right=120]
\end{tikzcd}
%}
\end{plDiagram}

Complete categories like this are rarely drawn,
which seems to follow a propaganda-ish pattern 
I've observed in other graphical languages:
It is common to only draw partial graphs,
where what's omitted is arguably as or more important 
than what's drawn.
I think this is usually an unconscious choice,
but, even so, it has the effect of hiding and distracting 
from the parts of the graphical representation that would
be most open fo questions and doubt.
An example is probabilistic graphical models/bayesian networks
in machine learning, where highly questionable
independence assumptions correspond to edges that 
are \textsl{not} drawn.
\end{plSection}
%-----------------------------------------------------------------
\begin{plSection}{Kategory}
\label{sec:Kategory_from_digraph}

(Until I think of a better name\ldots)

\compose is used primarily, if not exclusively,
to define whether $2$ paths are equivalent, by asserting they 
compose to the same edge.
(\textbf{Question:} Is this absolutely true?
Are \compose and the resulting edges
used for anything other than defining path equivalence?)
This suggests replacing \compose with 
either an equivalence relation on path pairs,
or a path ``evaluation'' function $\pvalue$
(\textbf{TODO:} need a better name for this),
with path equivalence classes
being the level sets of $\pvalue$.
That's essentially the same as \compose,
except we no longer assume the values are also edges in the graph.
Offhand, this seems a lot more intuitive than \compose,
adding all these additional, apparently unnecessary, edges to
the graph.

For sets and functions categories, the natural equivalence relation
is that the composed functions corresponding to the paths 
have the same graph relation (map each given domain element to the 
same codomain element. 
In this case, the $\pvalue$ function maps each path
to its graph relation.

For transportation graphs, and similar, the $\pvalue$ 
function could return a tuple of all path properties relevant to
the problem being solved, eg,  expected travel time, likely fuel 
cost, etc.

\begin{plDefinition}{Paths}{}
In this context, a \ding{path} is taken to be a sequence of
alternating vertices and edges, beginning and ending 
with vertices. 
Note that this allows us to have a path consisting 
of a single vertex.
I will often omit the vertices when writing the sequence,
whenever convenient.
The main operation on paths is \concatenate
(infix operator: \conop).
With paths $p_0,p_1,p_2$,
assuming $\head(p_0) = \tail(p_1)$: 
\begin{align*}
p_2 & = \concatenate(p_0,p_1) = p_0 \diamond p_1
\end{align*}
\textbf{Note:} \concatenate reverses the arguments
of \compose. 
We start at $\tail(p_0)$ and end at 
$\head(p_1)$.
(I personally find this less confusing.)
In more detail, with edges $e_{ij}$,
omitting vertices, 
and assuming all heads and tails match as needed:
\begin{align*}
\left[ e_{00} e_{01} \ldots e_{0m}
e_{10} e_{11} \ldots e_{1n}
 \right]
& = 
\left[ e_{00} e_{01} \ldots e_{0m} \right]
\diamond
\left[ e_{10} e_{11} \ldots e_{1n} \right]
\end{align*}
\end{plDefinition}

\begin{plDefinition}{Kategory (explicit path-equivalence)}{}
(Until I think of a better name\ldots)\\
A \ding{kategory} is a directed (multi) graph together with
a path evaluation function: $\left[G, \pvalue\right]$.
Two paths are \ding[path equivalence]{equivalent} ($\simeq$) 
if they have the same 
$\head$, $\tail$, and $\pvalue$:
\[
\left[ p_0 \simeq p_1 \right]
\;  \Leftrightarrow \;
\left[
\left( \head(p_0) = \head(p_1) \right)
\wedge
\left( \tail(p_0) = \tail(p_1) \right)
\wedge
\left( \pvalue(p_0) = \pvalue(p_1) \right)
\right]
\]
\end{plDefinition}

\Cref{diagram:digraph} is the graph part of a kategory.
No additional edges, either compositions or identities, are
needed. 
The $\pvalue$  function is not shown,
but that matches typical
use of diagrams, which omit the implied
composition and identity edges.
Note that \identity is not needed at all;
the constraints between \identity and \compose
are dropped; and \concatenate is associative by definition.
\end{plSection}

%-----------------------------------------------------------------
\begin{plSection}{Kategory from category}
\label{sec:Kategory_from_category}

To go from category to kategory, we need to define the 
path $\pvalue$ function.
It is \compose for general paths, and, 
for single edge identity loop paths,
it returns the 
path consisting of the single 
$\head$/$\tail$ vertex of the loop.

For kategories derived from categories
\begin{equation*}
\left[ \left( p_0 \simeq q_0 \right) \wedge 
 \left( p_1 \simeq q_1  \right) \right]
\; \implies \; 
\left[
\left( p_0 \diamond p_1 \right)
\simeq
\left( q_0 \diamond q_1 \right)
\right]
\end{equation*}
where $p_0$, $p_1$, $q_0$, $q_1$ are paths 
whose ends match appropriately.
This follows from the associativity of \compose:
\begin{align*}
\compose(p_0 \diamond p_1) 
& = 
\compose \left( 
\compose(p_1),
\compose(p_0) \right)
\\
& = 
\compose \left( 
\compose(q_1),
\compose(q_0) \right)
\\
& = 
\compose(q_0 \diamond q_1) 
\end{align*}
 
(\textbf{Question:} should this be required of all kategories?
If so, kategories may be no more intuitive than categories.)
\end{plSection}
%-----------------------------------------------------------------
\end{plSection}
%-----------------------------------------------------------------
\begin{plSection}{Functor}
\label{sec:functor}

For all the fuss, there doesn't seem to be much to functors:
A \ding{functor} is simply a structure-preserving function
between categories (or kategories), in the usual sense:
it can be applied any entity 
making up the structure,
and it ``commutes'' or ``distributes'' appropriately
over the structure functions.

\begin{plDefinition}{Structure-preserving function between directed
graphs\\ (aka directed graph homomorphism)}{}
$f:G_0 \rightarrow G_1$ is a structure-preserving function
between directed graphs, if and only if:
\begin{align*}
  & f(\vertices(G_0)) \, \subseteq \, \vertices(G_1) \\
  & f(\edges(G_0)) \, \subseteq \, \edges(G_1) \\
  & \forall e \in \edges(G_0) 
  \begin{cases}
  & f(\head(e)) = \head(f(e)) \\
  & f(\tail(e)) = \tail(f(e))
  \end{cases}
\end{align*}
We apply $f$ to paths in the obvious way.
\end{plDefinition}

\begin{plDefinition}{Functor (category)}{}
A \ding{functor} $F:C_0 \rightarrow C_1$ is 
a structure-preserving function
between categories, which requires that is a structure-preserving 
function between directed graphs, and:
\begin{align*}
  F(\compose_{C_0}(e_0,e_1)) & = \compose_{C_1}(F(e_0),F(e_1))\\
  F(\identity_{C_0}(V_0)) & = \identity_{C_1}(F(V_0)
\end{align*}
  (Subscripts on \compose and \identity 
  are usually dropped, but we might 
  encounter two categories that have the same vertices and edges,
  but differ in the definition of the two functions.)
\end{plDefinition}

\begin{plDefinition}{Phunctor (kategory)}{}
(Very unlikely to stick with this name.)

A \ding{phunctor} $P:K_0 \rightarrow K_1$ is 
a structure-preserving function
between kategories, which requires that is a structure-preserving 
function between graphs, and:
\begin{align*}
  \left( \pvalue_{K_0}(p_0) = \pvalue_{K_0}(p_1) \right)
  \; \Leftrightarrow \; 
  \left( \pvalue_{K_1}(P(p_0)) = \pvalue_{K_1}(P(p_1)) \right)
\end{align*}
\end{plDefinition}
\end{plSection}
%-----------------------------------------------------------------
\begin{plSection}{Diagram}
\label{sec:Diagram}

Diagrams are in some sense the core of category theory;
they are, however, defined ambiguously, incorrectly, 
or not at all, in the references I've seen:

\begin{plDefinition}
{Diagram (pictorial)\\
\textmd{
\citeAuthorYear{Geroch:1985:MathPhysics,Goldblatt:1984:Topoi,
Hillman:2001:CatPrimer}
}}
{diagram_pictorial}
``By \ding{diagram} we mean any collection of objects
along with a collection of morphisms between various of those
objects. (E.g., a diagram is what is pictured.)''~\cite{Geroch:1985:MathPhysics} 
\par
This is wrong, because it doesn't deal with the common
case of multiple vertices and edges with the same labels.
\end{plDefinition}

\begin{plDefinition}{Diagram (graph to graph)
\textmd{
\citeAuthorYear{BarrWells:2020}
}}
{}
Let $I$ and $G$ be {[directed) graphs]}.
A \ding{diagram} in $G$ of shape $I$
is a graph homomorphism $D : I \rightarrow G$ of graphs.
\end{plDefinition}

\begin{plDefinition}{Diagram (graph to category)
\textmd{
\citeAuthorYear{AspertiLongo:1991}
}}
{}
Let $I$ be a directed graph
 and $C$ be a category.
A \ding{diagram} in $G$ of shape $I$
is a graph homomorphism $D : I \rightarrow C$ of graphs.
\end{plDefinition}

\begin{plDefinition}{Diagram (category to category)
 \textmd{(broken)}}
{}
\label{def:diagram_cat_to_cat}
Let $I$ and $A$ be categories.
A \ding{diagram} is a functor $D : I \rightarrow A$.
$I$ is the \ding{shape} (aka \ding{scheme},
\ding{index category}) of the diagram $D$.\\
Sometimes $I$ is required to a small category
(sets vs classes crap).\\
This is often introduced in the context of limits,
after $100$s of pages
with multiple diagrams on each:
\begin{itemize}
\item \citeAuthorYear[p$193/524$]{AdamekHerrlichStrecker:1990}
\item \citeAuthorYear[p$101/311$]{Awodey:2010}
\item \citeAuthorYear[p$118/183$]{Leinster:2016:BasicCategoryTheory}
(commutative diagrams introduced on p$11$!)
\item \citeAuthorYear[p$38/240$]{Riehl:2017:CatTheory}
 (earlier than most),
\item \citeAuthorYear[p$180/267$]{Spivak:2013:CatTheoryForScientists}
\item \citeAuthorYear[p$16/86$]{VanOosten:2002:CatTheory} (also relatively early)
\end{itemize}
These definitions are clearly \textbf{wrong}.
What's drawn is not a category.
Drawing a category would make impossible the primary use of diagrams,
that is to aid in reasoning
about the equivalences of path in the diagram (``commuting'').
\end{plDefinition}

\citeAuthorYear[p$18/181$]{Perrone:2019:CatTheory}
gives ``an informal (but consistent) definition'' early,
followed by the incorrect category to
category definition \ref{def:diagram_cat_to_cat}, still relatively 
early~\cite[p$53/181$]{Perrone:2019:CatTheory}:
\begin{plDefinition}{Diagram (informal but consistent)\\
 \textmd{{\citeAuthorYear[Definition 1.1.33]{Perrone:2019:CatTheory}}})}
{diagram_perrone}
``A diagram in a category $C$ is a directed (multi-)graph formed 
out of objects and arrows of $C$ such that:
\begin{itemize}
\item Each object and morphism 
may appear more than once in the diagram;
\item Between any two objects 
there may be also more than one morphism;
\item For each object X in the diagram, 
the identity is implicitly present in the diagram (but generally
not drawn);
\item For each {[pair of]} composable edges 
(arrows) $f : X \rightarrow Y$ and $g : Y \rightarrow Z$
which are appearing head-to-tail
in the diagram, the composite $g\comop f : X \rightarrow Z$
 is implicitly present in the diagram (but generally
not drawn).
\end{itemize}
Since objects and morphisms may appear 
more than once in the diagram, as different vertices
and edges, we will refer to vertices and edges to avoid ambiguity.''
\end{plDefinition}

The informal definition is the only one I've seen which
explicitly acknowledges some of
what is \emph{not} drawn. 
The later formal definition does not,
and is therefore obviously wrong (as a definition of how diagrams
are drawn and used).

\begin{plDefinition}{Diagram (this time for 
real) \textmd{{\citeAuthorYear[Definition 1.4.20]{Perrone:2019:CatTheory}}})}
{}
``Let $C$ be a category, and $I$ be a small category. A diagram in
$C$ of shape $I$ is a functor $I \rightarrow C$.''
\end{plDefinition}

(\citeAuthorYear[p$200/390$]{LawvereSchanuel:2009:ConceptualMath} 
doesn't quite fit, due to vagueness; roughly
map from directed graph to category, 
not specifying structure-preserving.)

Undefined in:
\citeAuthorYear{MacLane:1998:CategoriesWorking2}.
\end{plSection}
%-----------------------------------------------------------------
\begin{plSection}{Commuting Diagram}
\label{sec:Commuting_Diagram}

\begin{plDiagram}
{Commuting version of \cref{diagram:aCompletedDigraph}}
{aCommutingDiagram}
\centering
\fbox{
\begin{tikzcd}
[
row sep=1.2cm, 
column sep=2.5cm,
ampersand replacement = \&,
execute at begin picture={
     \useasboundingbox (-3.3,-2.5) rectangle (4.25,3.4); }
    ]
V_0
\arrow[loop above, blue, crossing over]
\arrow[dr, "{e_{01}}" ] 
\arrow[drr, "e_{12} \comop e_{01}"', "e_{02}"  {near start},
bend left]
\arrow[ddr, "e_{13} \comop e_{01}", "e_{03}"' {near start},
bend right]
\& 
\& 
\\
\&
V_1 
\arrow[loop above, blue,crossing over]
\arrow[r, "e_{12}"] 
\arrow[d, "e_{13}"']
\arrow[dr, "e_{24} \comop e_{12}", "e_{34} \comop e_{13}"',
dashed, red]
\& 
V_2 
\arrow[loop right, crossing over, blue]
\arrow[d, "e_{24}"] 
\\
\& 
V_3 
\arrow[loop below, blue,crossing over]
\arrow[r, "e_{34}"']
\& 
V_4
\arrow[loop right, blue, crossing over]
\arrow[
"e_{24} \comop e_{02}",
"e_{34} \comop e_{03}" {near end},
"e_{24} \comop e_{12} \comop e_{01}" {near start},
"e_{34} \comop e_{13} \comop e_{01}"' {near start},
from=1-1,to=3-3, draw=DarkGreen, dashed, bend left=100,]
\end{tikzcd}
}
\end{plDiagram}

In \cref{diagram:aCompletedDigraph}, 
any of the pair of edges with the same head and tail might be
taken to be the same edge, and still satisfy the definition of a
category. 
If all such ``redundant'' pairs are reduced to single edges,
then the diagram is said to \ding{commute}.
This is shown in \cref{diagram:aCommutingDiagram}

Commutative diagrams are in some sense the heart of category
theory,
but both ``commutative'' and ``diagram'' are problematic.

I have been unable to find an explanation for the use of
the word ``commute'' to mean ``all displayed paths
with the same head and tail are equivalent''.
This is one of the early symptoms that there's something wrong:
The fact that a key term is used for, as far as I can tell, 
an unrelated concept
from closely related areas of mathematics (eg commutative groups)
is at best, evidence of flawed exposition, if not confused thinking.
To use ``commute'' in the context of the binary \compose
operation to mean anything other than $e_0 \comop e_1 = e_1 \comop e_0$
adds pointlessly to the cognitive load of someone new to the subject.

The convention in category theory is to leave out identity loops
and composition edges when drawing diagrams,
which is a symptom that perhaps the identities and compositions
ought not be there.

It worth noting that ``diagram'' is rarely defined in introductory
texts, at least, 
not until long after diagrams have been in use.
And the standard definition---functor from indexing category
to ambient category
(eg \cite[][Definition 1.6.4]{Riehl:2017:CatTheory})---doesn't reflect how
diagrams are actually drawn and used.

What's drawn is a subgraph, not a category (unless we use
a path-equivalence/identity-free definition of categories).
The functor is essentially irrelevant.
The displayed subgraph is a device for 
reducing the cognitive burden of certain (mental) computations
(like a chess board and pieces);
what's drawn is (should be) the minimum necessary 
to get to the desired result.~\cite{DutilhNovaes:2012:FormalLanguages}

(\textbf{Question:} is diagram chasing the same thing as proving
a diagram commutes?)
\end{plSection}

%-----------------------------------------------------------------
\begin{plSection}{Monic, epic, isic (Monepisic?) edges}
\label{sec:monic_epic_isic}

Two edges are \ding[parallel edges]{parallel} 
if they have the same head and tail.
Same for paths.

\begin{plSection}{Category}

See, for example, \cite{Perrone:2019:CatTheory}.

\begin{plDiagram}
{Monic edge $e$: 
$\left( f \neq g \right) 
\implies 
\left( e \comop f \, \neq \, e \comop g \right)$.}
{monic}
\centering
\begin{tikzcd}
V_0 
\arrow[r, bend left, "f"]
\arrow[r, bend right, swap, "g"]
& V_1 
\arrow[r, "e"]
& V_2 
\end{tikzcd}
\end{plDiagram}

An edge $e$ is \ding{monic} (aka a monomorphism, left-cancelable)
if, for any matching parallel edges $f,g$:
\begin{align*}
\left( e \comop f = e \comop g \right) 
& 
\implies 
\left( f = g \right) 
\\
\left( f \neq g \right) 
&
\implies 
\left( e \comop f \neq e \comop g \right) 
\end{align*}

\begin{plDiagram}
{Epic edge $e$: 
$\left( f \neq g \right)
 \implies 
 \left( f \comop e \, \neq \, g \comop e \right)$.}
{epic}
\centering
\begin{tikzcd}
V_0 \arrow[r, "e"]
& V_1 
\arrow[r, bend left, "f"]
\arrow[r, bend right, "g"']
& V_2 
\end{tikzcd}
\end{plDiagram}

An edge $e$ is \ding{epic} (aka epimorphism, right-cancelable)
if, for any matching parallel edges $f,g$:
\begin{align*}
\left( f \comop e = g \comop e \right) 
& 
\implies 
\left( f = g \right)
\\
\left( f \neq g \right)
& 
\implies 
\left( f \comop e \neq g \comop e \right) 
\end{align*}
(\textbf{Question:} 
what is an edge/path that is both epic and monic?
Clumsy ``epic-monic'' may be only option.)

First duality: reverse all edges gives monic $\,\leftrightarrow\,$ epic.

\begin{plDiagram}
{Split monic edge $e$, if the two edge path
$V_0 \longrightarrow V_0$
composes to the identity loop.}
{splitMonic}
\centering
\begin{tikzcd}
V_0 
\arrow[loop left, blue, "I_{V_0}"]
\arrow[r, "e", bend left]
\arrow[r, "e^{-left}"', leftarrow, bend right]
& 
V_1 
\end{tikzcd}
\end{plDiagram}

An edge $e$ is \ding{split monic} (aka has a left inverse)
if there exists an $e^{-\text{left}}$, the \ding{retraction},
such that
\[
e^{-\text{left}} \comop e = \identity(\tail(e))
\]
Every split monic edge is monic.

\begin{plDiagram}
{Split epic edge $e$, if the two edge path
$V_1 \longrightarrow V_1$
composes to the identity loop.}
{splitEpic}
\centering
\begin{tikzcd}
V_0 
\arrow[r, "e", bend left]
\arrow[r, "e^{-right}"', leftarrow, bend right]
& 
V_1 
\arrow[loop right, blue, "I_{V_1}"]
\end{tikzcd}
\end{plDiagram}

An edge $e$ is \ding{split epic} (aka has a right inverse)
if there exists an $e^{-\text{right}}$, the \ding{splitting}, 
such that
\[
e \comop e^{-\text{right}} = \identity(\head(e))
\]
Every split epic edge is epic.

\begin{plDiagram}
{Isic edge $e$, if both paths $V_0 \longrightarrow V_0$
are equivalent (compose to the identity loop) and the 
same for both paths $V_1 \longrightarrow V_1$.}
{isic}
\centering
\begin{tikzcd}
V_0 
\arrow[loop left, blue, "I_{V_0}"]
\arrow[r, "e", bend left]
\arrow[r, "e^{-1}"', leftarrow, bend right]
& 
V_1 
\arrow[loop right, blue, "I_{V_1}"]
\end{tikzcd}
\end{plDiagram}

An edge $e$ is \ding{isic} (aka an isomorphism, invertible)
if there exists an edge $e^{-1}$ such that
\[
e^{-1} \comop e = \identity(\tail(e))
\]
and
\[
e \comop e^{-1} = \identity(\head(e))
\]
$e$ is isic \liff it is both split monic and split epic.\\
Proof (belabored, probably unnecessary).
We need to show that $e^{-\text{left}} = e^{-\text{right}}$:
\begin{align*}
e \comop e^{-\text{right}} 
& = 
\identity(\head(e)) 
\\
e^{-\text{left}} \comop \left( e \comop e^{-\text{right}} \right)
& = 
e^{-\text{left}} \comop \identity(\head(e)) 
\\
\left( e^{-\text{left}} \comop e \right) \comop e^{-\text{right}} 
& = 
e^{-\text{left}}
\\
\identity(\head(e)) \comop e^{-\text{right}} 
& = 
e^{-\text{left}} 
\\
e^{-\text{right}} 
& = 
e^{-\text{left}} 
\end{align*}
(Note that this seems much simpler in left/right invertible language.)

(\textbf{Note:} in a topos monic$\,\wedge\,$epic $\implies$ isic.
Is that because all monic/epic are split?)

(\textbf{TODO:} describe these ideas in the context of
structures and structure-preserving functions vs
set and all functions, to distinguish epic/monic from split versions.
eg non-surjective epimorphisms, non-injective monomorphism.)

Monic, epic, and isic are ``abstracted'' 
from one-to-one, onto, and invertible functions.
At this point, it is, at best, not clear that we have gained
anything and we definitely have lost something.
I put quotes around ``abstracted'' because the new concepts are
not simplifications of the originals.
First of all, 
we've broken the modularity of the original concepts:
one-to-one, onto, and invertible can be determined from
the function alone, and the fact that the domain and codomain
somewhat slippery can be handled directly.
(\textbf{TODO:} find the right way to say this.)
Monic, epic, and, particularly, isic depend other edges in the category, 
and details of how \compose is defined.
Specifically, note that,
a function that is one-to-one and onto is
therefore invertible, and vice versa.
On the other hand, all isic edges are both monic and epic,
but an edge that is both monic and epic need not be isic.
For an edge to be isic, it requires the existence of an ``inverse''
edge, that cancels when composed with the original edge 
in either order.

Difference between split monic/epic and simple monic/epic
is that inverse edges need not exist in simple case.
In a general sets-and-functions context, split and simple versions
are the same.
In a structures-and-structure-preserving-functions context,
this is no longer true, because, although the inverse functions
will exist, they may not be structure-preserving.
Is this any clearer in category language vs sets-and-functions?
I don't think so.
\end{plSection}
%-----------------------------------------------------------------
\begin{plSection}{Misleading (?) diagrams}

Standard definitions refer to pairs of edges $f, g$,
which may lead the less careful to unconsciously assume
the condition is defined as 
``there exists some pair $f,g$ such that \ldots'',
rather than
``for all such pairs $f,g$ \ldots''.

A realistic diagram of the monic case 
is in \cref{diagram:monicDiagram}.
Even this is idealized, since common categories (eg Sets)
would have an uncountable number of such edges.

\begin{plDiagram}
{More realistic monic diagram.}
{monicDiagram}
\centering
\begin{tikzcd}
V_0 
\arrow[r, bend left=90, "f_0"]
\arrow[r, bend left=30, "f_1","{\vdots}"' inner sep=0.0ex]
%\arrow[r, bend right=30, swap, "f_{n-1}"]
%\arrow[r, bend right=100, swap, "f_{n}"]
& V_1 
\arrow[r, "e"]
& V_2 
\arrow["e \comop f_0",
from=1-1,to=1-3, red, dashed, bend right=30,]
\arrow["e \comop f_1",
from=1-1,to=1-3, red, dashed, bend right=60,
"{\vdots}"' inner sep=0.0ex]
\end{tikzcd}
\end{plDiagram}

An edge $e$ is \ding{monic} (aka a monomorphism, left-cancelable)
if, for any matching parallel edges $f_i,f_j$:
\begin{align*}
\left( e \comop f_i = e \comop f_j \right) 
& 
\implies 
\left( f_i = f_j \right) 
\\
\left( f_i \neq f_j \right) 
&
\implies 
\left( e \comop f_i \neq e \comop f_j \right) 
\end{align*}
\end{plSection}

%-----------------------------------------------------------------
\begin{plSection}{Do monepisic paths make sense?}

Questions:
\begin{enumerate}
  \item What happens if we replace the edges $f_0,f_1$
  by paths $q_0,q_1$ in the monic/epic definitions?
  
  Then $e$ is monic if for all parallel paths $q_0,q_1$:
  \[
  \left( \compose(q_0) \neq \compose(q_1) \right) 
  \implies 
  \left( e \comop \compose(q_0) \neq e \comop \compose(q_1) \right)
  \]
  (for the monic case).
  
  In kategory language:
  \begin{align*}
  \left( q_0 \not\peq q_1 \right) 
  & \implies 
  \left( q_0 \conop e \not\peq q_1 \conop e \right)
  \\
  \left( q_0 \conop e \peq q_1 \conop e \right)
  & \implies 
  \left( q_0 \peq q_1 \right) 
  \end{align*}

  \item What happens if we replace the edge $e$ by a path $p$?
  
  A path $p$ would be epic if the edge $\compose(p)$ was.
  
  In kategory language:
  \begin{align*}
  \left( f_0 \not\peq f_1 \right) 
  & \implies 
  \left( f_0 \conop p \not\peq f_1 \conop p \right)
  \\
  \left( f_0 \conop p \peq f_1 \conop p \right)
  & \implies 
  \left( f_0 \peq f_1 \right) 
  \end{align*}
  
  \item What happens if we replace $e,f,g$ by paths?
  
  \begin{align*}
  \compose(q_0) & \neq \compose(q_1) \\
  & \Downarrow \\
  \compose(p) \comop \compose(q_0) 
  & \neq \compose(p) \comop \compose(q_1) 
  \end{align*}
  
  In kategory language:
  \begin{align*}
  \left( q_0 \not\peq q_1 \right) 
  & \implies 
  \left( q_0 \conop p \not\peq q_1 \conop p \right)
  \\
  \left( q_0 \conop p \peq q_1 \conop p \right)
  & \implies 
  \left( q_0 \peq q_1 \right) 
  \end{align*}
  
  \item Do split monic/epic and isic definitions work if we replace
edge $e$ by a path $p$?
\end{enumerate}
\end{plSection}

%-----------------------------------------------------------------
\begin{plSection}{Kategory version}

Definitions equivalent to ones with the parallel
edges replaced by parallel paths and equality/identity 
by equivalence? 

An path $p$ is \ding{left-cancelable} (aka a monomorphism, monic)
if, for any matching parallel paths $q_0,q_1$:
\begin{align*}
\left( q_0 \conop p \peq q_1 \comop p \right) 
& 
\implies 
\left( q_0 \peq q_1 \right) 
\\
\left( q_0 \not\peq q_1 \right) 
& \implies 
\left( q_0 \conop p \not\peq q_1 \conop p \right) 
\end{align*}
Should it be ``right-cancelable'' in path language?

\end{plSection}
%-----------------------------------------------------------------
\begin{plSection}{What is the meaning of this?}

Are there any intuitive interpretations in the direct graph,
category, or kategory contexts? As opposed to 
structures and structure-preserving functions.

Does it help to think about factoring edges/paths
as a step in solving for something?
For example, if we have $2$ equivalent paths 
and we can factor both into a prefix and a common monic
suffix, then we can infer that the prefixes are equivalent:
\begin{align*}
& r_0 \peq r_1 \\
& r_0 = q_0 \conop p \\
& r_1 = q_1 \conop p \\
& p \text{ monic } \\
& \Downarrow \\
& q_0 \peq q_1
\end{align*}
\end{plSection}%{What is the meaning of this?}
\end{plSection}
%-----------------------------------------------------------------
\end{plSection}%{Category and Topos}
%-----------------------------------------------------------------
%-----------------------------------------------------------------
