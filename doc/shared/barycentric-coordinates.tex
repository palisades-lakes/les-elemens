\begin{plSection}{Barycentric coordinates}
\label{sec:barycentric-coordinates}

In this section, I describe how to compute
the barycentric coordinates of a point $\Point{q}$
with respect to a $n$-simplex in
a finite dimensional inner product space $\Space{V}$
(with $\dimension (\Space{V} ) = m \ge n$) ---
and the derivatives with respect to the vertex positions.

Let $\Simplex{S}$ be the geometric $n$-simplex
with vertex positions $\left\{\Point{p}_0  \ldots  \Point{p}_n\right\}$;
in $\Space{V}$.

The {\it span} of $\Simplex{S}$, $\affinespan ( S )$,
is the affine span of its vertex positions
$\affinespan \left\{\Point{p}_0  \ldots  \Point{p}_n\right\}$,
that is, the set of points $\Point{q}$ such that
$\Point{q} = \sum_{j=n}^{m} b_j \Point{p}_j $ 
where $1 = \sum_{j=0}^{n} b_j $.
$b_j$ are {\it barycentric coordinates} of $\Point{q}$ 
with respect to $S$.
Barycentric coordinates of an arbitrary point 
$\Point{q} \in \Space{V}$,
are barycentric coordinates of the closest point in $\affinespan(S)$.
If $S$ is non-degenerate (the dimension of its span is $m$),
then the barycentric coordinates of any point are unique.

Assume in what follows that $S$ is non-degenerate.
When faced with for a degenerate simplex, there are several options.
One can use the minimum norm $\Vector{b}$.
It is often possible to use the barycentric coordinates
with respect to the 'largest' non-degenerate face instead.
We may also want to choose $\Vector{b}$ to be as close to convex,
$0 \le b_j \le 1$, as possible. {\it (How?)}

and let $Vector{V} =
 \sum_{j=0}^{n-1} ( \Vector{v}_j \otimes \Vector{e}_j^{\Reals^n} )$,
the linear map from $\Reals^n \mapsto \Space{V}$,
whose 'columns' are the $\Vector{v}_j$.
Let $\Vector{b} = \left( b_0 \ldots b_{n-1} \right) \in \Reals^n$
be the unconstrained vector of the first $n$ barycentric coordinates.
For $\Point{q} \in \affinespan ( S )$, 
$\Point{q} = Vector{V} \Vector{b} + \Point{p}_m$
and $\Vector{b} = 
Vector{V}^{-1} \left ( \Point{q} - \Point{p}_m \right)$.
For arbitrary $\Point{q} \in \Space{V}$,
$\Vector{b} = Vector{V}^{-} (\Point{q} - \Point{p}_m)$
(see \cref{sec:Inverses-and-pseudo-inverses}).

\end{plSection}%{Barycentric coordinates}
