%-----------------------------------------------------------------
% The alttree type of glossary styles need to know the
 % widest entry name for each level
 \iftoggle{printglossary}
 {
\glssetwidest{Rational numbers} % level 0 widest name
\glssetwidest[1]{Homogeneous space}      % level 1 widest name
\makeglossaries

\newglossarystyle{cites}
{% based on list style
  \setglossarystyle{list}%
    \renewcommand*{\glossentry}[2]{%
    \item[\glsentryitem{##1}%
          \glstarget{##1}{\glossentryname{##1}}]
       \glossentrydesc{##1}\glspostdescription
    \ifglshasfield{useri}{##1}{\space
     % in the event of multiple cites (as in the vestibulum2
     % sample entry), \glsentryuseri{##1} needs to be expanded
     % before being passed to \cite.
     \glsletentryfield{\thiscite}{##1}{useri}%
     (See \expandafter\cite\expandafter{\thiscite})}{}%
    \space ##2}%
}

\newglossarystyle{citeshyper}
{% based on list style
  \setglossarystyle{alttreehypergroup}%
    \renewcommand*{\glossentry}[2]{%
    \item[\glsentryitem{##1}%
          \glstarget{##1}{\glossentryname{##1}}]
       \glossentrydesc{##1}\glspostdescription
    \ifglshasfield{useri}{##1}{\space
     % in the event of multiple cites (as in the vestibulum2
     % sample entry), \glsentryuseri{##1} needs to be expanded
     % before being passed to \cite.
     \glsletentryfield{\thiscite}{##1}{useri}%
     (See \expandafter\cite\expandafter{\thiscite})}{}%
    \space ##2}}
}
{}
%-----------------------------------------------------------------
\newglossaryentry{Sets}{
  name={Sets},
  text={Sets},
  description={\nopostdesc},
  %user1=Halmos:1960:NaiveSetTheory
  }

\newglossaryentry{Spaces}{
  name={Spaces},
  text={Spaces},
  description={\nopostdesc}}

  
\newglossaryentry{Set}{
  name={Set},
  text={set},
  first={a generic set},
  symbol={\Set{S}},
  description={a generic \emph{set}},
  sort=set,
  parent=Sets}
  
\newglossaryentry{HomogeneousSpace}{
  name={Homogeneous Space},
  text={Homogeneous space},
  %first={a generic set},
  %symbol={\ensuremath{\mathcal{S}}},
  description={a \emph{space} where every point looks the same},
  %sort=set,
  parent=Spaces}
  
\newglossaryentry{elementOf}{
  name={elementOf},
  text={elementOf},
  first={elementOf},
  description={$x \in \mathcal{S}$ means $x$ is an element of
  the set $\mathcal{S}$}, 
  sort=element,
  symbol={\ensuremath{\in}},
  parent=Sets}

\newglossaryentry{Integers}{
  name={integers},
  text={integers},
  symbol={\Space{Z}},
  description={the integers}}

\newglossaryentry{PositiveIntegers}{
  name={positive integers},
  text={positive integers},
  symbol={\ensuremath{\Space{Z}_{+}}},
  description={\SetSpec{i\glssymbol{elementOf}\glssymbol{Integers}}{i>0}},
  parent=Integers}

\newglossaryentry{NaturalNumbers}{
  name={natural numbers},
  text={natural numbers},
  symbol={\Space{N}},
  description={\SetSpec{i\glssymbol{elementOf}\glssymbol{Integers}}{i \geq 0}}}

\newglossaryentry{RationalNumbers}{
  name={rational numbers},
  text={rational numbers},
  symbol={\Space{Q}},
  description={
  \SetSpec{i/j}{
  i \glssymbol{elementOf} \glssymbol{Integers}, 
  j \glssymbol{elementOf} \glssymbol{PositiveIntegers}}}}

\newglossaryentry{RealNumbers}{
  name={real numbers},
  text={real numbers},
  symbol={\Space{R}},
  description={the real numbers}}

\newglossaryentry{DoublePrecisionFloat}{
  name={doubles},
  text={doubles},
  symbol={\Space{D}},
  description={the {IEEE} 754 64 bit floating point numbers}}

\newglossaryentry{SinglePrecisionFloat}{
  name={floats},
  text={floats},
  symbol={\Space{F}},
  description={the {IEEE} 754 32 bit floating point numbers}}

\newglossaryentry{GenericSpace}{
  name={a generic space},
  text={a generic space},
  symbol={\Space{S}},
  description={a generic space}}
