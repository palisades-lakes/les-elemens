%-----------------------------------------------------------------
\begin{plSection}{Functions and relations}
\label{sec:Functions-and-relations}

In mathematics literature, \textit{function} is usually defined as
a special kind of \textit{relation} --- essentially a set of
ordered pairs with certain properties.

This approach is useful for some purposes, but here we will be
more interested in ``computable'' functions, at least in the sense
that a function is a ``machine'' that takes an element of one set
as input and returns an element of another, with the constraint
that a given input always returns the same output.

Because we need the notion of ``relation'' anyway, I'm going to
provide both definitions.

%-----------------------------------------------------------------
\begin{plSection}{Relations}
\label{sec:Relations}

A \textit{relation}, 
$\Set{R}$, on $\Set{S}_0, \Set{S}_1, \ldots \Set{S}_{n-1}$,  
is any subset 
$\Set{R} \subseteq \Set{S}_0 \times \Set{S}_1 \times \ldots 
\times \Set{S}_{n-1}$,
that is, a set of tuples.

Ambiguity note: a given set of tuples, $\Set{R}$, can be
considered as a relation over many cartesian product sets.
The minimal such set is 
$\Set{R}_0 \times \Set{R}_1 \times \ldots \times
\Set{R}_{n-1}$, 
where 
$\Set{R}_i = \SetSpec{ x }{ \exists r 
\in \Set{R} \text{ s.t. } r_i = x }$.

A general \textit{binary relation} is a set of ordered
pairs $\Set{R} \subseteq \Set{S}_0 \times \Set{S}_1$.
It's common to write a binary relation as a predicate, 
$r(s_0,s_1) = ([s_0 \, s_1] \in \Set{R})$,
or $(r \, s_0 \, s_1)$ in pseudo-code,
or as a binary operation $s_0 \, R \, s_1$.

An important special case
are binary relations on $\Set{S}^2 = \Set{S} \times \Set{S}$
(often just written as ``binary relation on $\Set{S}$''). 
In this case we can define certain properties:

\begin{description}
\item[Transitive]
$r(s_0,s_1) \text{ and } r(s_1,s_2) \Rightarrow r(s_0,s_2)$.
\item[Reflexive] $r(s,s)$ is always true.
\item[Symmetric] For $s_0 \neq s_1$, $r(s_0,s_1)$ implies
$r(s_1,s_0)$.
\item[Antisymmetric] For $s_0 \neq s_1$, $r(s_0,s_1)$ implies not
$r(s_1,s_0)$.
\end{description}
These properties determine 2 important classes of relations:
\begin{description}
\item[Equivalence] Transitive, reflexive, and symmetric.
\item[Partial order] \label{def:partial-order}
Transitive, reflexive, and antisymmetric.
\end{description}

\end{plSection}%{Relations}
%-----------------------------------------------------------------
\begin{plSection}{Functions}
\label{sec:Functions}

A \textit{functional} binary relation, $\Set{F}$ on $\Set{X}
\times \Set{Y}$ has exactly one $y \in \Set{Y}$ for each
$x \in \Set{X}$ such that $[x \, y] \in \Set{F}$.
More conventional notation writes the \textit{function}, 
$f : \Set{X} \rightarrow \Set{Y}$ as $y = f(x)$.
In that case, $\Set{X}$ is the \textit{domain} of $f$
and $\Set{Y}$ is the $codomain$.

It is nearly universal to extend or identify a function $f$ that maps 
elements of the domain $\Set{X}$
to elements of 
the codomain $\Set{Y}$
the closely related function that maps
subsets of $\Set{X}$ to subsets of $\Set{Y}$,
via $f\left( \Set{A} \right) = 
\SetSpec{y \in \Set{Y}}{\exists \, x \in \Set{A} 
\text{ such that } f\left( x \right) = y}$

The \textit{range} of $f$ is then $f\left( \Set{X} \right)$.

Arbitrariness of codomain:
Note that $\text{range}\left( f \right)$ 
is unambiguous.
If  $f_0 : \Set{X} \rightarrow \Set{Y}_0$,
and $\Set{Y}_0 \subset \rightarrow \Set{Y}_1$,
then is $f_1 : \Set{X} \rightarrow \Set{Y}_1$
defined by $f_1(x) = f_0(x)$
the same function as $f_0$?

We will also frequently encounter the notion of the 
\textit{support} of a function, which is less consistently
defined. In general, the support is a subset of the domain
where the function takes on some non-trivial value, for example,
where a numerically valued function is non-zero, or sometimes,
finite.

Partial functions: support vs domain.

Any function, $f : \Set{X} \rightarrow \Set{Y}$ defines an
equivalence relation on $\Set{X}$ via 
$\Set{E}_f = \SetSpec{[x_0 \, x_1]}{f(x_0) = f(x_1)}$
(see~\ref{sec:Implicit-functions}).

$\Set{F}\left( \Set{D}, \Set{C} \right)$ is the set of all 
functions with domain $\Set{D}$ and codomain $\Set{C}.$

\end{plSection}%{Functions}
%-----------------------------------------------------------------
\begin{plSection}{Equivalence classes and quotient sets}

If $\Set{E}$ is an equivalence relation on $\Set{S}^2$, then we
can define the \textit{equivalence class} of $s_0$, 
$E(s_0) = \SetSpec{s_1 \in \Set{S}}{[s_0 \, s_1] \in \Set{E} }$.
The set of distinct equivalence classes partitions $\Set{S}$,
and is called the \textit{quotient set}: $\Set{S} / \Set{E}$.

In the case where the equivalence relation is derived from a
function, we write $\Set{S} / f$.

Equivalence classes and quotient sets will turn out to be
important. A common representational/implementation trick is to
use a larger but simpler (in some sense) space for calculations
which are meant to apply to a quotient space that actually has the
properties of interest. Homogeneous coordinates for affine and
projective spaces  are an important example.
The first one we will encounter here will be the rational numbers,
which are represented/implemented as pairs of integers
$[\text{numerator} \, \text{denominator}]$, but the actual
rational numbers are the equivalence classes defined by 
$\SetSpec{ [[p_0 \, q_0] \, [p_1 \, q_1]]}{p_0*q_1 = p_1*q_0 }$,
that is, $[1 \, 2]$ and $[2 \, 4]$ represent the same rational.
\end{plSection}%{Equivalence classes and quotient sets}
%-----------------------------------------------------------------
\begin{plSection}{Inverses and pseudo-inverses}
\label{sec:Inverses-and-pseudo-inverses}

Given a function $f : \Set{D} \mapsto \Set{C}$,
the \textit{inverse} of $f$ is 
$f^{-1}(c) = \SetSpec{d}{f(d) = c}$.
Note that $f^{-1} : \Set{C}  \mapsto \PowerSet{\Set{D}}$.

$\Set{L}\left( f , c \right) = f^{-1}(c)$ is a \textit{level set} of
$f$ at $c$.

For more symmetry, we can extend $f$ and $f^{-1}$ to 
$\PowerSet{\Set{D}}  , \PowerSet{\Set{C}}$
by defining
$f(\Set{S}) = 
\SetSpec{c}{\exists \, d \, \in \Set{S} \; \text{s.t.} \; c = f \left( d \right)}$

The usual definition of inverse treats $f^{-1}$
as a function from $\Set{C} \mapsto \Set{D}$,
which is undefined where the value of the true
inverse is not a set containing a single point.

\textbf{TODO:} When is a function 
$\PowerSet{\Set{D}}  \mapsto \PowerSet{\Set{C}}$
derivable from a function $\Set{D} \mapsto \Set{C}$?

Inverse function theorem~\cite[Theorem~2-1]{Spivak:1965:CalculusOnManifolds}
for $\Space{R}^n$. 

Generalizations?~\cite{wiki:InverseFunctionTheorem}

\end{plSection}%{Inverses and pseudo-inverses}
%-----------------------------------------------------------------
\begin{plSection}{Level sets and implicit functions}
\label{sec:Implicit-functions}

Suppose 
$f : \left( \Set{D}_0 \times \Set{D}_1 \right) \mapsto \Set{C}$.
Then each level set $f^{-1}\left( c \right)$ defines a relation on 
$\Set{R}_{f^{-1}(c)} \left( \Set{D}_0 , \Set{D}_1 \right)$.
If that relation is a function
(one $d_1$ paired to each $d_0$),
we call it an \textit{implicit function},
which we might write $g_{f^{-1}(c)} : \Set{D}_0 \mapsto \Set{D}_1$

Implicit functions are generally difficult to use,
because they don't tell us how to compute 
$g_{f^{-1}(c)} \left( d_0 \right)$.
Essentially, one has to use an iterativc zero-finding 
algorithm, which is difficult and non-robust except
for very low dimensional problems.
An alternative is to minimize something like
$\| c - f\left( d_0, d_1 \right) \|^2$ over $d_0$,
and check that the minimum value is close enough to $0$.

\textbf{TODO:} When does a level set correspond to a function?
Implicit function theorem~\cite[Theorem~2-2]{Spivak:1965:CalculusOnManifolds}
for $\Space{R}^n$. 

Generalizations?~\cite{wiki:ImplicitFunctionTheorem,
wiki:InverseFunctionTheorem}

\end{plSection}%{Level sets and implicit functions}
%-----------------------------------------------------------------
\begin{plSection}{Multiple arguments}
\label{sec:Multiple-arguments}

Often useful to consider 
functions of several arguments
$f \left( x_0, x_1, \ldots x_{n-1}\right)$.

Simpler in general to deal with functions of one argument.
Two ways to identify a function of several argmuments
with a single argument function:
\begin{description}
\item[Catesian product domains]
Define $f \left( x_0, x_1, \ldots x_{n-1} \right) 
= f \left( \left[x_0, x_1 \ldots  x_{n-1} \right]\right)$
where $\left[x_0, x_1, \ldots x_{n-1} \right] \in
\Set{X}_0 \times \Set{X}_1 \times \ldots \times \Set{X}_{n-1}$. 
(Note that domain/support may be a strict subset of full cartesian 
product.)
\item[Currying]
In pseudo-lisp: $\left(f \, x_0 \, x_1 \, \ldots x_{n-1} \right) 
= 
\left(
\left(
\ldots 
\left( 
\left( 
f 
\, x_0 
\right) 
\, x_1
\right) 
\ldots 
\right) 
\, x_{n-1}
\right) 
$.
This is easier to follow in the $2$ argument case.
Then we interpret $f : \Set{X}_0 \rightarrow 
\Set{F}\left(\Set{X}_1, \Set{Y} \right)$,
that is, the value $f \left( x_0 \right)$ is a function 
$\Set{X}_1 \mapsto \Set{Y}$,
so, in confusing infix notation,
$f \left( x_0 , x_1 \right) = 
f \left( x_0 \right) \left( x_1 \right) \Rightarrow y \in \Set{Y}$,
or, in pseudo-lisp, 
$\left( f \, x_0 \, x_1 \right) = \left( \left( f \, x_0 \right) x_1 \right) \Rightarrow y$
\end{description}

\end{plSection}%{Multiple arguments}
%-----------------------------------------------------------------
\begin{plSection}{Computable functions}
\end{plSection}%{Computable functions}
%-----------------------------------------------------------------
\begin{plSection}{implementation}
\end{plSection}%{implementation}
%-----------------------------------------------------------------
\begin{plSection}{examples}
\end{plSection}%{examples}
%-----------------------------------------------------------------
\end{plSection}%{Functions and relations}

