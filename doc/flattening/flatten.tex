\section{Classifying projected simplexes}
\label{sec:classifying}

%--------------------------------------------------------------------

The convex hull of a $(d-1)$ dimensional projection of a $d$-simplex
is either a $(d-1)$-simplex or a $(d-1)$ dimensional cross polytope.

\begin{Lemma}
\label{rambau-lemma}
Any set $Z$ of $(d+2)$ points whose convex hull is of dimension $d$
has exactly two triangulations denoted $T_{Z^+}$ and $T_{Z^-}$.
\end{Lemma}

See Rambau \cite{rambau-jorg-1996}, Lemma 1.1.2.

Let $S=(\p_0, \p_1, \ldots , \p_m)$ be an $m$-simplex in $\Re^{n}$.
Let $\pi$ be a projection from $\Re^{n}$ to $Q$, an $(m-1)$-dimensional
affine subspace of $\Re^{n}$.
Assume $\pi$ is chosen so that the points
$\{\pi \p_0, \pi \p_1, \ldots , \pi \p_m\}$ are in {\it general position},
that is, any $l+1$ of the projected points spans an $l$-dimensional
affine subspace of $Q$.

By Lemma \ref{rambau-lemma},
there are two exactly triangulations of the convex hull of the projected points.
The $(m-1)$-simplexes of the triangulations are images of the $(m-1)$-simplexes of $S$,
and two triangulations correspond to a partition of $(m-1)$-simplexes of $S$
into two subsets, the "top" and "bottom" of $S$ with respect to $\pi$.

To see why this is true, and to further classify the triangulations,
consider the fact that the boundary of the convex hull of $\pi S$
must contain either $m$ or $m+1$ of the $\pi \p_i$.
(Any fewer and the points cannot be in general position.)

\begin{Theorem}
\label{one-simplex-case}
If the boundary of the convex hull of $\pi S$
contains $m$ of the $\pi \p_i$,
then it is a $(m-1)$-simplex
and the image of one of the $(m-1)$-simplexes, $F$, in $S$.
The first triangulation is just $\pi F$.
The second triangulation consists of the images of
all the remaining $(m-1)$-simplexes of $S$.
The second triangulation is itself the mutual refinement of both.
\end{Theorem}

The first 2 statements are obvious.
If we label the vertices so that $\{\pi \p_1, \ldots , \pi \p_m\}$
are on the boundary, and $\pi \p_0$ is in the interior,
then the second triangulation results from refining the first
triangulation {\it pulling}
\cite{lee-hdcg-2004} the vertex $\pi \p_0$.
The faces of the triangulation formed by pulling
are the images of the $m$ $(m-1)$-simplexes
of $S$ that contain $\p_0$, that is, all the $(m-1)$-simplexes in $S$,
other than $(\p_1, \ldots , \pi \p_m)$.

\begin{Theorem}
\label{two-simplex-case}
If the boundary of the convex hull of $\pi S$
contains all $m+1$ of the $\pi \p_i$,
then it is the image of 2 of the $(m-1)$-simplexes in $S$,
which share a common $(m-2)$-simplex.
These 2 simplexes are the first triangulation.
The second triangulation consists of the images
of the remaining $m-1$ $(m-1)$-simplexes of $S$,
which share a common $1$-simplex.
The mutual refinement is formed by splitting either
the shared $(m-2)$-simplex in the first triangulation
or the shared $1$-simplex in the second.
\end{Theorem}



