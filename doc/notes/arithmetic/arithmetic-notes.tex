% !TEX TS-program = arara
% arara: xelatex: { synctex: on, options: [--halt-on-error] } 
% arara: biber
% arara: makeglossaries
% arara: makeindex
% arara: xelatex: { synctex: on, options: [-halt-on-error] } 
% arara: xelatex: { synctex: on, options: [-halt-on-error] } 
% arara: clean: { files: [arithmetic-notes.aux,arithmetic-notes.bbl] }
% arara: clean: { files: [arithmetic-notes.bcf,arithmetic-notes.blg] }
% arara: clean: { files: [arithmetic-notes.glg,arithmetic-notes.glo] }
% arara: clean: { files: [arithmetic-notes.gls,arithmetic-notes.gls] }
% arara: clean: { files: [arithmetic-notes.idx,arithmetic-notes.ilg] }
% arara: clean: { files: [arithmetic-notes.ind,arithmetic-notes.loe] }
% arara: clean: { files: [arithmetic-notes.lof,arithmetic-notes.log] }
% arara: clean: { files: [arithmetic-notes.log,arithmetic-notes.lol] }
% arara: clean: { files: [arithmetic-notes.out] }
% arara: clean: { files: [arithmetic-notes.run.xml] }
% arara: clean: { files: [arithmetic-notes.toc,arithmetic-notes.xdy] }
% arara: clean: { files: [arithmetic-notes.synctex.gz] }
%-------------------------------------------------------------------------------
\documentclass[11pt,openany]{book}
\usepackage{import}
\def \texFolder {../../tex/}
\def \bibFolder {../../bib/}
\def \sharedFolder {../../shared/}
\import{\texFolder}{head}
\import{\texFolder}{landscape-2col-8x17}
\import{\texFolder}{clj-listings}
\import{\sharedFolder}{glossary}
\addbibresource{{\bibFolder}algebra.bib}
\addbibresource{{\bibFolder}arithmetic.bib}
\addbibresource{{\bibFolder}cactus.bib}
\addbibresource{{\bibFolder}proglang.bib}
\addbibresource{{\bibFolder}halmos.bib}
\addbibresource{{\bibFolder}interpolation.bib}
\addbibresource{{\bibFolder}logic.bib}
\addbibresource{{\bibFolder}mcdonald.bib}
\addbibresource{{\bibFolder}mop.bib}
\addbibresource{{\bibFolder}numbers.bib}
\addbibresource{{\bibFolder}sets.bib}
\addbibresource{{\bibFolder}tex.bib}
%\addbibresource{{\bibFolder}ieeestd.bib}
%\addbibresource{{\bibFolder}fparith.bib}
%-----------------------------------------------------------------
\title{Notes on arithmetic}
\author{\textsc{John Alan McDonald}}
%\email{mcdonald.john.alan at gmail.com}
\date{draft of \today}
%-----------------------------------------------------------------
\begin{document}

\maketitle

%\frontmatter

% \begingroup
% \let\onecolumn\twocolumn
% \sffamily
% \tableofcontents
% \rmfamily
% \endgroup
% 
% \mainmatter

%-----------------------------------------------------------------
% 7 for part level
% 6 for chapter level
% 5 for section level
% 4 for subsection level
% 3 for subsubsection level
% 2 for paragraph level
% 1 for subparagraph level
\setcounter{baseSectionLevel}{6}
%-----------------------------------------------------------------
\epigraph{
\textsl{das Wesen der Mathematik liegt gerade in ihrer Freiheit.}
\par
(The essence of mathematics lies precisely in its freedom.)}%
{Cantor~\cite{Cantor1883},
as quoted in~\cite{ferreiros2007labyrinth}.}

motivation from empirical problems; 
see eg~\cite{maclane1981mathModels}.
goal ``breadth, clarity, and depth''.

\epigraph{
Mathematics is not the rigid and rigidity-producing schema 
that the layman thinks it is; 
rather, in it we find ourselves at that meeting point 
of constraint and freedom 
that is the very essence of human nature.}%
{Hermann Weyl ?}


adjoint functors?

axioms on composition of functions as alternative to set theory
as foundation of mathematics

Cantorian set theory: platonic sets with $2$nd order axioms.

my point of view: 
mathematics (notation) and programming languages
as \textit{written} languages,
used to formulate real world problems,
and construct their solutions.

express an idea in at most a couple pages of text.

say enough and not too much, especially not irrelevant things.
abstract just enough; 
concrete details of model/representation/implementation.

key issue: is the solution correct?
mathematics evaluated by consensus;
software by testing.

\epigraph{Mathematics as we practice it is much more formally 
complete and precise than
other sciences, but it is much less formally complete and precise 
for its content
than computer programs.}%
{Thurston, \textit{On proof and progress in mathematics}~\cite{thurston1994proof}}


 \setcounter{currentlevel}{\value{baseSectionLevel}}
\import{\sharedFolder}{mathematics}

\setcounter{currentlevel}{\value{baseSectionLevel}}
\import{\sharedFolder}{logic}
\setcounter{currentlevel}{\value{baseSectionLevel}}
\import{\sharedFolder}{sets}
\setcounter{currentlevel}{\value{baseSectionLevel}}
\import{\sharedFolder}{functions}
\setcounter{currentlevel}{\value{baseSectionLevel}}
\import{\sharedFolder}{algebra}
\setcounter{currentlevel}{\value{baseSectionLevel}}
\import{\sharedFolder}{numbers}
\setcounter{currentlevel}{\value{baseSectionLevel}}
\import{\sharedFolder}{arithmetic}

%-----------------------------------------------------------------
%\pagebreak
%\appendix
%\part{Appendices}
%\import{\texFolder}{typesetting}
%-----------------------------------------------------------------
\backmatter


%\part{Backmatter}
% un-comment \input and comment \printbibiliography
% to get index, glossary, list of theorems, etc.
%\import{\texFolder}{tail}
\printbibliography[heading=bibintoc, title={References}]
%-----------------------------------------------------------------
\end{document}
