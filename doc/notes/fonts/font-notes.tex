% !TEX TS-program = arara
% arara: xelatex: { synctex: on, options: [-halt-on-error] } 
% arara: biber
% % arara: texindy: { markup: xelatex }
% %% arara: makeglossaries
% % arara: xelatex: { synctex: on, options: [-interaction=batchmode, -halt-on-error] }
% % arara: xelatex: { synctex: on, options: [-interaction=batchmode, -halt-on-error]  }
% % arara: clean: { extensions: [ aux, log, out, run.xml, ptc, toc, mw, synctex.gz, ] }
% % arara: clean: { extensions: [ bbl, bcf, blg, ] }
% % arara: clean: { extensions: [ glg, glo, gls, ] }
% % arara: clean: { extensions: [ idx, ilg, ind, xdy, ] }
% % arara: clean: { extensions: [ plCode, plData, plMath, plExercise, plNote, plQuote, ] }
%-----------------------------------------------------------------
\documentclass[11pt]{PalisadesLakesBook}
\geomHDTV
\geomLandscape
%-----------------------------------------------------------------

%\AsanaFonts % misssing \mathhyphen; less on page than Cormorant/Garamond
%\CormorantFonts % light, missing unicode greek, needs scaling relative to monfont
\EBGaramondFonts % fewest pages
%\ErewhonFonts
%\FiraFonts % tall lines, all sans, much less per page, missing \in?
%\GFSNeohellenicFonts 
%\KpFonts
%\LatinModernFonts
%\LegibleFonts
%\LibertinusFonts
%\NewComputerModernFonts
%\STIXTwoFonts
%\BonumFonts % most pages
%\PagellaFonts
%\ScholaFonts
%\TermesFonts
%\XITSFonts

%-----------------------------------------------------------------
\togglefalse{plMath}
\togglefalse{plCode}
\togglefalse{plData}
\togglefalse{plNote}
\togglefalse{plExercise}
\togglefalse{plQuote}
\togglefalse{printglossary}
\togglefalse{printindex}
%-----------------------------------------------------------------
\title{Notes on fonts and typesetting, text and mathematics}
\author{John Alan McDonald 
(palisades dot lakes at gmail dot com)}
\date{draft of \today}
%-----------------------------------------------------------------
\begin{document}
\maketitle
\PalisadesLakesTableOfContents{7}
%-----------------------------------------------------------------
\def\sharedFolder{../../shared/}

\begin{plSection}{Introduction}

This is a collection of notes on font design,
typesetting (both for text and mathematics),
and possibly related layout problems. 
Partially notes on reading; 
partially my own analysis and implementation.

Although it's unlikely I will finish such a thing,
I will be approaching this topic with the mindset of:
``what would I do if I wer going to replace \TeX, \LaTeX, etc.''.


\end{plSection}%{Introduction}
%-----------------------------------------------------------------
\begin{plSection}{Reading}
%-----------------------------------------------------------------
\begin{plSection}{Noordzij}

\citeAuthorTitle{Devroye:2014:Noordzij}

\citeAuthorTitle{Middendorp:2019:Noordzij}

\end{plSection}%{Noordzij}

%-----------------------------------------------------------------
\begin{plSection}{Rhatigan}

\begin{plSection}{The Monotype 4-line system}

\citeAuthorYearTitle{Rhatigan:2007:Monotype4Line}

\end{plSection}%{The Monotype 4-line system}

\begin{plSection}{Three typefaces for mathematics}

\citeAuthorYearTitle{Rhatigan:2007:MathTypefaces}

\end{plSection}%{Three typefaces}

\begin{plSection}{Gina}

\citeAuthorYearTitle{Rhatigan:2007:GinaPractice}

\citeAuthorYearTitle{Rhatigan:2007:GinaSpecimen}

Tabular oldstyle figures (digits) are nice and, I think,
relatively unusual.

(\emph{Question:} any font with quasi-oldstyle digits that alternate 
ascending above x-height and dropping below baseline?
Eg, 0 rising, 1 with a descender of some kind, 2 rising, 3 with
a descender (usual for oldstyle), 4 ascending (usual is descending).
The remaining 6--9 alternate ascending/descending in most
oldstyle digit renderings.
Don't know if this would actually look good.
Might be a project for learning FontForge?)


Mathematics examples are too limited to judge, 
but operators ($\int$, $\prod$) look too small. 
(How was the math in the specimen doc typeset?)

\end{plSection}%{Gina}

\end{plSection}%{Rhatigan}
%-----------------------------------------------------------------
\end{plSection}%{Reading}
%-----------------------------------------------------------------
\BeginAppendices
\input{typesetting}
%-----------------------------------------------------------------
%-----------------------------------------------------------------
\end{document}
%-----------------------------------------------------------------
