\begin{plSection}{Polylines}
%-----------------------------------------------------------------
\begin{plSection}{Vertex Bends}
\label{sec:Vertex-Bends}

\begin{plDiagram}
{Polyline vertex neighborhood}
{PolylineVertexNeighborhood}
\centering
\begin{verbatim}
                   e1
             p1 o--------o p0
                        a \
                           \ e2
                            \
                             o p2
\end{verbatim}
\end{plDiagram}

In this section, I disuss measures of curvature
based on the bending at each vertex in a polyline.

\end{plSection}%{Vertex Bends}
%-----------------------------------------------------------------
\begin{plSection}{Cosine}
\label{sec:polyline-vertex-cosine}

Consider the non-boundary vertex
at point $\Vector{p}_0 \in \Reals^3$ in \cref{diagram:PolylineVertexNeighborhood}.
$\Simplex{v}$ has degree $2$;
the incident edges are labeled $\Simplex{e}_1$ and $Simplex{e}_2$;
and the neighboring vertices are at points $\Vector{p}_1 \in \Reals^3$
and $\Vector{p}_2 \in \Reals^3$.
The shape of the neighborhood is determined by
$\Vector{p} = (\Vector{p}_0, \Vector{p}_1, \Vector{p}_2) \in \Reals^9$
The unsigned angle between edges 
$\Simplex{e}_1$ and $\Simplex{e}_2$ is $\alpha(\Vector{p})$.

One measure of the amount of bending is the cosine of $\alpha$:
\begin{equation}
\cos(\alpha(\Vector{p})) =
{{(\Vector{p}_1 - \Vector{p}_0)} \over {\| \Vector{p}_1 - \Vector{p}_0 \|} }
\bullet
{{(\Vector{p}_2 - \Vector{p}_0)} \over {\| \Vector{p}_2 - \Vector{p}_0 \|} }
\end{equation}

A little calculus shows that the partial gradients are:
\begin{eqnarray}
\label{eq:polyline-vertex-cosine-gradient}
\Gf{\Vector{p}_0}{\cos(\alpha(\Vector{p}))}
& = &
-
\left[
{{(\Vector{p}_1 - \Vector{p}_0) \perp  (\Vector{p}_2 - \Vector{p}_0)}
\over
{\| \Vector{p}_1 - \Vector{p}_0 \| \| \Vector{p}_2 - \Vector{p}_0 \|} }
+
{{(\Vector{p}_2 - \Vector{p}_0) \perp  (\Vector{p}_1 - \Vector{p}_0)}
\over
{\| \Vector{p}_1 - \Vector{p}_0 \| \| \Vector{p}_2 - \Vector{p}_0 \|} }
\right]
\\
\Gf{\Vector{p}_1}{\cos(\alpha(\Vector{p}))}
& = &
{{(\Vector{p}_2 - \Vector{p}_0) \perp  (\Vector{p}_1 - \Vector{p}_0)}
\over
{\| \Vector{p}_1 - \Vector{p}_0 \| \| \Vector{p}_2 - \Vector{p}_0 \|} }
\nonumber
\\
\Gf{\Vector{p}_2}{\cos(\alpha(\Vector{p}))}
& = &
{{(\Vector{p}_1 - \Vector{p}_0) \perp  (\Vector{p}_2 - \Vector{p}_0)}
\over
{\| \Vector{p}_1 - \Vector{p}_0 \| \| \Vector{p}_2 - \Vector{p}_0 \|} }
\nonumber
\end{eqnarray}

\end{plSection}%{Cosine}
%-----------------------------------------------------------------
\begin{plSection}{Squared Cosine}
\label{sec:polyline-vertex-squared-cosine}

For any positive measure of bending,
minimizing the sum of squared bends,
rather than simply the sum of bends,
will tend to produce a more even distribution of curvature.
Since $\cos(\alpha)$ ranges over $[-1,1]$,
we use $\left( \frac{1 + \cos(\alpha)}{2} \right)^2$.

The gradient is:
\begin{equation}
\Gf{\Vector{p}}{\left( {{1 + \cos(\alpha(\Vector{p}))} \over 2} \right)^2}
=
\left( 1 + \cos(\alpha(\Vector{p})) \right)
\Gf{\Vector{p}}{\cos\left(\alpha(\Vector{p})\right)}
\end{equation}

$\Gf{\Vector{p}}{\cos\left(\alpha(\Vector{p})\right)}$ is given
in \cref{eq:polyline-vertex-cosine-gradient}.

\end{plSection}%{Squared Cosine}
%-----------------------------------------------------------------
\begin{plSection}{Angle}
\label{sec:polyline-vertex-angle}


The angle $\alpha$ is $\pi$ for a straight polyline,
and $0$ for a maximally bent vertex.
Therefore we may choose to minimize the sum of negative angles:
\begin{equation}
-\alpha(\Vector{p}) =
-\cos^{-1} \left(
{{(\Vector{p}_1 - \Vector{p}_0)} \over {\| \Vector{p}_1 - \Vector{p}_0 \|} }
\bullet
{{(\Vector{p}_2 - \Vector{p}_0)} \over {\| \Vector{p}_2 - \Vector{p}_0 \|} }
\right)
\end{equation}

The gradient is:
\begin{equation}
\Gf{\Vector{p}}{-\alpha(\Vector{p})}
=
{{1} \over {\sqrt{ 1 - \cos(\alpha(\Vector{p}))^2}}}
\Gf{\Vector{p}}{\cos\left(\alpha(\Vector{p})\right)}
\end{equation}

$\Gf{\Vector{p}}{\cos\left(\alpha(\Vector{p})\right)}$ is given
in \cref{eq:polyline-vertex-cosine-gradient}.

\end{plSection}%{Angle}
%-----------------------------------------------------------------
\begin{plSection}{Squared Angle}
\label{sec:polyline-vertex-squared-angle}

Since $-\alpha$ ranges over $[-\pi,0]$
we square $\pi - \alpha$.

The gradient is:
\begin{equation}
\Gf{\Vector{p}}{\left( \pi - \alpha(\Vector{p}) \right)^2}
=
{{2 \left( \pi - \alpha(\Vector{p}) \right)}
\over
{\sqrt{ 1 - \cos(\alpha(\Vector{p}))^2}}}
\Gf{\Vector{p}}{\cos\left(\alpha(\Vector{p})\right)}
\end{equation}

$\Gf{\Vector{p}}{\cos \left( \alpha(\Vector{p}) \right)}$ is given
in \cref{eq:polyline-vertex-cosine-gradient}.

\end{plSection}%{Squared Angle}
%-----------------------------------------------------------------
\end{plSection}%{Polylines}
