% !TEX TS-program = arara
% arara: xelatex: { synctex: on, options: [--halt-on-error] } 
% arara: biber
% arara: makeglossaries
% arara: makeindex
% arara: xelatex: { synctex: on, options: [-halt-on-error] } 
% arara: xelatex: { synctex: on, options: [-halt-on-error] } 
% arara: clean: { files: [topos-notes.aux,topos-notes.bbl] }
% arara: clean: { files: [topos-notes.bcf,topos-notes.blg] }
% arara: clean: { files: [topos-notes.glg,topos-notes.glo] }
% arara: clean: { files: [topos-notes.gls,topos-notes.gls] }
% arara: clean: { files: [topos-notes.idx,topos-notes.ilg] }
% arara: clean: { files: [topos-notes.ind,topos-notes.loe] }
% arara: clean: { files: [topos-notes.lof,topos-notes.log] }
% arara: clean: { files: [topos-notes.log,topos-notes.lol] }
% arara: clean: { files: [topos-notes.out,topos-notes.thm] }
% arara: clean: { files: [topos-notes.run.xml] }
% arara: clean: { files: [topos-notes.toc,topos-notes.xdy] }
% arara: clean: { files: [topos-notes.synctex.gz] }
%-------------------------------------------------------------------------------
\documentclass[11pt,openany]{article}
\usepackage{import}
\def \texFolder {../../tex/}
\def \bibFolder {../../bib/}
\def \sharedFolder {../../shared/}
\import{\texFolder}{head}
\import{\texFolder}{landscape-2col-8x17}
\import{\texFolder}{clj-listings}
\import{\sharedFolder}{glossary}
\addbibresource{{\bibFolder}algebra.bib}
\addbibresource{{\bibFolder}arithmetic.bib}
\addbibresource{{\bibFolder}cactus.bib}
\addbibresource{{\bibFolder}proglang.bib}
\addbibresource{{\bibFolder}halmos.bib}
\addbibresource{{\bibFolder}interpolation.bib}
\addbibresource{{\bibFolder}logic.bib}
\addbibresource{{\bibFolder}mcdonald.bib}
\addbibresource{{\bibFolder}mop.bib}
\addbibresource{{\bibFolder}numbers.bib}
\addbibresource{{\bibFolder}sets.bib}
\addbibresource{{\bibFolder}tex.bib}
%\addbibresource{{\bibFolder}ieeestd.bib}
\addbibresource{{\bibFolder}fparith.bib}
\addbibresource{{\bibFolder}topos.bib}
%-----------------------------------------------------------------
\title{Notes on category and topos theory}
\author{\textsc{John Alan McDonald}}
%\email{mcdonald.john.alan at gmail.com}
\date{draft of \today}
%-----------------------------------------------------------------
\begin{document}

\maketitle

%-----------------------------------------------------------------
% 7 for part level
% 6 for chapter level
% 5 for section level
% 4 for subsection level
% 3 for subsubsection level
% 2 for paragraph level
% 1 for subparagraph level
\setcounter{baseSectionLevel}{5}
%-----------------------------------------------------------------

%\frontmatter

% \begingroup
% \let\onecolumn\twocolumn
% \sffamily
% \tableofcontents
% \rmfamily
% \endgroup
% 
%-----------------------------------------------------------------
\epigraph{
\textsl{das Wesen der Mathematik liegt gerade in ihrer Freiheit.}
\par
(The essence of mathematics lies precisely in its freedom.)}%
{Cantor~\cite{Cantor1883},
as quoted in~\cite{ferreiros2007labyrinth}.}

\epigraph{
Mathematics is not the rigid and rigidity-producing schema 
that the layman thinks it is; 
rather, in it we find ourselves at that meeting point 
of constraint and freedom 
that is the very essence of human nature.}%
{Hermann Weyl ?}

\epigraph{Mathematics as we practice it is much more formally 
complete and precise than
other sciences, but it is much less formally complete and precise 
for its content
than computer programs.}%
{Thurston, 
\textit{On proof and progress in mathematics
}~\cite{thurston1994proof}}

\epigraph{On the most fundamental level, the foundations of mathematics are much shakier
than the mathematics that we do. Most mathematicians adhere to foundational
principles that are known to be polite fictions. For example, it is a theorem that
there does not exist any way to ever actually construct or even define a well-ordering
of the real numbers. There is considerable evidence (but no proof) that we can get
away with these polite fictions without being caught out, but that doesn’t make
them right. Set theorists construct many alternate and mutually contradictory
“mathematical universes” such that if one is consistent, the others are too. This
leaves very little confidence that one or the other is the right choice or the natural
choice. G¨odel’s incompleteness theorem implies that there can be no formal system
that is consistent, yet powerful enough to serve as a basis for all of the mathematics
that we do.}%
{Thurston, \textit{On proof and progress in mathematics}~\cite{thurston1994proof}}

\epigraph{
\textsl{Und was wird der geduldige Leser am
Schlusse sagen? Dass der Verfasser mit einem Aufwande von uns/iglicher Arbeit es gliicklich
erreicht hat, die klarsten Vorstellungen in ein unheimliches Dunkel zu hiillen!}
\par
(What will the forbearing reader say at the end? That the author, in a squandering of indescribable
work, has happily managed to surround the clearest ideas in a disturbing obscurity!
)}%
{Dedekind to Klein, April 1888, 
from~\cite{dugac1976DedekindFondements},
as quoted in~\cite{ferreiros2007labyrinth}.}

%-----------------------------------------------------------------
\setcounter{currentlevel}{\value{baseSectionLevel}}
\levelstay{Preface 2020-07}
\label{sec:Preface-2020-07}

Capture state of ignorance as of 2020-07,
to record what's confusing to the novice.

For the past $20+$ years, I have, every year or so,
 encountered claims that
category is the ``right way to do \textit{blah}''.
Pick up some recent 
introduction~\cite{adamek-herrlich-strecker-1990,
asperti-longo-1991,awodey-2010,
barr-wells-2020,fong-spivakd-2018-seven-sketches,
geroch-1985,lawvere-schanuel-2009,
leinster-2016-basic-category-theory,
maclane-1998-working-mathematician,nLab-2020,
riehl-2017,
spivakd-2013,spivakd-2014} 
get about halfway and give up.

Recent context: optimization software leads to
exact arithmetic leads to computable reals,
leads to contructive analysis and intuitionistic logic,
leads to history and foundations of math 
leads to topos theory.


Current hypothesis: something is wrong.
Maybe it's me, or maybe \ldots
Directed multigraph perhaps isn't the 'right' language to
expose and use commonalities across mathematics.

\textbf{TODO:} is there a reasonable way to measure 
the complextity of a topic/definition/theorem/textbook,
something perhaps like the depth and branching of the
dependency tree?

Guess: right foundation should be self defining
(like any decent programming language).

motivation from empirical problems; 
see eg~\cite{maclane1981mathModels}.
goal ``breadth, clarity, and depth''.

my point of view: 
mathematics (notation) and programming languages
as \textit{written} languages,
used to formulate real world problems,
and construct their solutions.
express an idea in at most a couple pages of text.
say enough and not too much, especially not irrelevant things.
abstract just enough; 
concrete details of model/representation/implementation.

key issue: is the solution correct?
mathematics evaluated by consensus;
software by testing.

%-----------------------------------------------------------------
\setcounter{currentlevel}{\value{baseSectionLevel}}
\levelstay{Why?}
\label{sec:Why?}

\setlength{\epigraphwidth}{0.95\linewidth}

\epigraph{Category theory provides a language to describe precisely many similar phenomena that
occur in different mathematical fields. For example,
\begin{enumerate}
  \item  Each finite dimensional vector space is isomorphic to its dual and hence also to its
second dual. The second correspondence is considered “natural”, but the first is
not. Category theory allows one to precisely make the distinction via the notion
of natural isomorphism.
\item Topological spaces can be defined in many different ways, e.g., via open sets, via
closed sets, via neighborhoods, via convergent filters, and via closure operations.
Why do these definitions describe “essentially the same” objects? Category theory
provides an answer via the notion of concrete isomorphism.
\item Initial structures, final structures, and factorization structures occur in many different
situations. Category theory allows one to formulate and investigate such
concepts with an appropriate degree of generality.
\end{enumerate}
\leavevmode }
{Ad\'{a}mek, Herrlich, Strecker, 
\textit{Abstract and Concrete Categories:
The Joy of Cats}~\cite{adamek-herrlich-strecker-1990}}

\pagebreak

\epigraph{In addition to its direct relevance to theoretical knowledge and current applications, category theory
is often used as an (implicit) mathematical jargon rather than for its explicit notions and results.
Indeed, category theory may prove useful in construction of a sound, unifying mathematical
environment, one of the purposes of theoretical investigation. As we have all probably experienced, it
is good to know in which “category” one is working, i.e., which are the acceptable morphisms and
constructions, and the language of categories may provide a powerful standardization of methods and
language. In other words, many different formalisms and structures may be proposed for what is
essentially the same concept; the categorical language and approach may simplify through abstraction,
display the generality of concepts, and help to formulate uniform definitions. This has been the case,
for example, in the early applications of category theory to algebraic geometry.}
{Asperi and Longo, \textit{Categories, Types, and Structures: 
an introduction to Category Theory for the working computer scientist}~\cite{asperti-longo-1991}}

%\pagebreak

\epigraph{Categories originally arose in mathematics out of the need of a formalism to
describe the passage from one type of mathematical structure to another. A
category in this way represents a kind of mathematics, and may be described
as category as mathematical workspace.
A category is also a mathematical structure. As such, it is a common
generalization of both ordered sets and monoids (the latter are a simple
type of algebraic structure that include transition systems as examples),
and questions motivated by those topics often have interesting answers for
categories. This is category as mathematical structure.
A third point of view is emphasized in this book. A category can be seen
as a structure that formalizes a mathematician's description of a type of
structure. This is the role of category as theory. Formal descriptions in
mathematical logic are traditionally given as formal languages with rules for
forming terms, axioms and equations. Algebraists long ago invented a formalism
based on tuples, the method of signatures and equations, to describe
algebraic structures. In this book, we advocate categories in their role as formal
theories as being in many ways superior to the others just mentioned.
Continuing along the same path, we advocate sketches as definite specifcations
for the theories.}
{Barr and Wells, \textit{Category theory for computing science}~\cite{barr-wells-2020}}

\pagebreak

\epigraph{Category theory is becoming a central hub for all of pure mathematics. It is unmatched
in its ability to organize and layer abstractions, to find commonalities between structures
of all sorts, and to facilitate communication between different mathematical
communities.}
{Fong and Spivak, \textit{Seven Sketches in Compositionality: 
An Invitation to Applied Category Theory}~\cite{fong-spivakd-2018-seven-sketches}}

%\pagebreak

\epigraph{In each area of mathematics 
(e.g. groups, topological spaces) 
there are available many definitions and constructions.
It turns out, however, that there are a number of notions
(e.g. that of a product) that occur naturally 
in various areas of mathematics, with only slight changes from
one area to another.
It is convenient to take advantage of this observation.
Category theory can be described as that branch of mathematics
in which one studies certain definitions in a broader context---without
reference to the particular area 
in which the definition might be applied.
It is the ``mathematics of mathematics''.}
{Geroch, \textit{Mathematical Physics}~\cite{geroch-1985}}

%\pagebreak
\epigraph{Since its first introduction over 60 years ago, the concept of category has been
increasingly employed in all branches of mathematics, especially in studies where the
relationship between different branches is of importance. The categorical ideas arose
originally from the study of a relationship between geometry and algebra; the
fundamental simplicity of these ideas soon made possible their broader application.
\par
The categorical concepts are latent in elementary mathematics; making them more
explicit helps us to go beyond elementary algebra into more advanced mathematical
sciences.}
{Lawvere and Schanuel, 
\textit{Conceptual Mathematics: 
A First Introduction to Categories}~\cite{lawvere-schanuel-2009}}

\pagebreak
\epigraph{Category theory takes a bird’s eye view of mathematics. From high in the sky,
details become invisible, but we can spot patterns that were impossible to detect
from ground level. How is the lowest common multiple of two numbers
like the direct sum of two vector spaces? What do discrete topological spaces,
free groups, and fields of fractions have in common?We will discover answers
to these and many similar questions, seeing patterns in mathematics that you
may never have seen before.
\par
The most important concept in this book is that of universal property. The
further you go in mathematics, especially pure mathematics, the more universal
properties you will meet. We will spend most of our time studying different
manifestations of this concept.
\par
Like all branches of mathematics, category theory has its own special vocabulary,
which we will meet as we go along. But since the idea of universal
property is so important, I will use this introduction to explain it with no jargon
at all, by means of examples.}
{Leinster, \textit{Basic Category Theory}~\cite{leinster-2016-basic-category-theory}}

%\pagebreak
\epigraph{\ldots One the first level, categories provide
a convenient conceptual language, based on notions of category, 
functor, natural transformation, contravariance, and
functor category.
\ldots Next comes the fundamental idea of an adjoint pair of functors.
This appears in many equivalent forms: that of universal construction,
that of direct and inverse limit, 
and that of a pair of functors with a natural isomorphism between
corresponding sets of arrows. \ldots
The slogan is ``Adjoint functions arise everywhere.''
\par
Alternatively, the fundamental notion of categopry theory is that
of a monoid---a set with a binary operation of multiplcation
that is associative and that has a unit;
a category itself can be regarded as a sort of generalized monoid.
\ldots
\par
Since a category consists of arrows, our subject could be 
described as learning how to live without elements, using arrows
instead.
}
{MacLane, \textit{Categories for the working
mathematician}~\cite{maclane-1998-working-mathematician}}

\pagebreak
\epigraph{Atiyah described mathematics as the
“science of analogy.” In this vein, the purview
of category theory is mathematical analogy. 
Category theory provides a cross-disciplinary
language for mathematics designed to delineate general phenomena,
 which enables the
transfer of ideas from one area of study to another. 
The category-theoretic perspective can
function as a simplifying1 abstraction, isolating propositions 
that hold for formal reasons
from those whose proofs require techniques particular 
to a given mathematical discipline.}
{Riehl, 
\textit{Category Theory in Context, Preface
}~\cite{riehl-2017}}

\pagebreak
\epigraph{This course is an attempt to extol the virtues of a new branch of mathematics,
called category theory, which was invented for powerful communication of ideas between
different fields and subfields within mathematics. By powerful communication of ideas I
actually mean something precise. Different branches of mathematics can be formalized
into categories. These categories can then be connected together by functors. And the
sense in which these functors provide powerful communication of ideas is that facts and
theorems proven in one category can be transferred through a connecting functor to
yield proofs of analogous theorems in another category. A functor is like a conductor of
mathematical truth.
\par
I believe that the language and toolset of category theory can be useful throughout
science. We build scientific understanding by developing models, and category theory is
the study of basic conceptual building blocks and how they cleanly fit together to make
such models. Certain structures and conceptual frameworks show up again and again in
our understanding of reality. No one would dispute that vector spaces are ubiquitous.
But so are hierarchies, symmetries, actions of agents on objects, data models, global
behavior emerging as the aggregate of local behavior, self-similarity, and the effect of
methodological context.
\par
Some ideas are so common that our use of them goes virtually undetected, such as settheoretic
intersections. For example, when we speak of a material that is both lightweight
and ductile, we are intersecting two sets. But what is the use of even mentioning this
set-theoretic fact? The answer is that when we formalize our ideas, our understanding
is almost always clarified. Our ability to communicate with others is enhanced, and the
possibility for developing new insights expands. And if we are ever to get to the point
that we can input our ideas into computers, we will need to be able to formalize these
ideas first.
\par
It is my hope that this course will offer scientists a new vocabulary in which to think
and communicate, and a new pipeline to the vast array of theorems that exist and are
considered immensely powerful within mathematics. These theorems have not made their
way out into the world of science, but they are directly applicable there. Hierarchies are
partial orders, symmetries are group elements, data models are categories, agent actions
are monoid actions, local-to-global principles are sheaves, self-similarity is modeled by
operads, context can be modeled by monads.}
{D. Spivak, \textit{Category Theory for Scientists 
(old version)}~\cite{spivakd-2013}}

An alternative to first order logic, sets, 
and functions as a foundation for all
mathematics~\cite{feferman-1977-FEFCFA},  
particularly topos theory.


%-----------------------------------------------------------------
\setcounter{currentlevel}{\value{baseSectionLevel}}
\levelstay{Naming}
\label{sec:Naming}

Particularly bad in category theory, eg, 
what's ``commuting'' in a commuting diagram.

Prefer self-explanatory names, even if a bit longer:
eg, ``left (right) cancelable'' rather than ``mono(epi)morphism'',
``structure-preserving function'' over ``homomorphism''.

``Morphism'' etymology.
 
%-----------------------------------------------------------------
\setcounter{currentlevel}{\value{baseSectionLevel}}
\levelstay{Graphical language}
\label{sec:Graphical-language}

%-----------------------------------------------------------------
\setcounter{currentlevel}{\value{baseSectionLevel}}
\levelstay{Bottom up}
\label{sec:Bottom-up}

Start from directed (multi)graph.

Not really a foundation, due to 
dependence on \textit{collections} of objects and arrows
and \textit{functions} between these collection. 

Categorical constructs are pretty much impossible to motivate
 in this context.

%-----------------------------------------------------------------
\setcounter{currentlevel}{\value{baseSectionLevel}}
\levelstay{Topos down}
\label{sec:Topos-down}
 
Topos definition, then define what it depends on, \ldots.
 
%-----------------------------------------------------------------
\setcounter{currentlevel}{\value{baseSectionLevel}}
\levelstay{Abstraction}
\label{sec:Abstraction}

\epigraph{Throughout these courses the infusion of a geometrical
point of view is of paramount importance. A vector
is geometrical; it is an element of a vector space, defined
by suitable axioms—whether the scalars be real numbers or
elements of a general field. A vector is not an n-tuple of
numbers until a coordinate system has been chosen. Any
teacher and any text book which starts with the idea that vectors
are n-tuples is committing a crime for which the proper
punishment is ridicule. The n-tuple idea is not ‘easier,’ it is
harder; it is not clearer, it is more misleading. By the same
token, linear transformations are basic and matrices are their
representations\ldots}
{MacLane, \textit{Of course and courses}~\cite{MacLane:1954}}

Start from (all of) conventional mathematics,
remove unnecessary details freom definitions, theorems, proofs.
Express what remains in graphical category language.
Look for ``graphs'' that match, unifying disparate parts of math.

Problem: you need to learn a lot of conventional math before
you can really start on category theory.

 
%-----------------------------------------------------------------
\pagebreak
\appendix
%\part{Appendices}
\import{\texFolder}{typesetting}
%-----------------------------------------------------------------
%\backmatter


%\part{Backmatter}
% un-comment \input and comment \printbibiliography
% to get index, glossary, list of theorems, etc.
%%------------------------------------------------------------------------------
% \begingroup  % Temporarily disable \clearpage to show both lists on one page
%   %\let\clearpage\relax    % http://tex.stackexchange.com/a/14511/104449
%   \renewcommand{\listtheoremname}{List of definitions}
%   \textsf{\listoftheorems[ignoreall, show={definition}]}
% \endgroup
%-------------------------------------------------------------------------------
\pagebreak
\renewcommand{\listfigurename}{Figures}
\addcontentsline{toc}{chapter}{\listfigurename}
\begingroup
\let\onecolumn\twocolumn
\sffamily
\listoffigures
\rmfamily
\endgroup
%-------------------------------------------------------------------------------
\pagebreak
\renewcommand{\lstlistlistingname}{Code samples}
\addcontentsline{toc}{chapter}{\lstlistlistingname}
\begingroup
\let\onecolumn\twocolumn
\sffamily
\lstlistoflistings
\rmfamily
\endgroup
%-------------------------------------------------------------------------------
\pagebreak
\renewcommand{\listtheoremname}{Examples}
\addcontentsline{toc}{chapter}{\lstlistlistingname}
\begingroup
\let\onecolumn\twocolumn
\sffamily
\listoftheorems
\rmfamily
\endgroup
%-------------------------------------------------------------------------------
% \newglossarystyle{mystyle}{%
%  \glossarystyle{altlist}%
%  \renewcommand*{\glossaryentryfield}[5]{%
%    \item[\glsentryitem{##1}\glstarget{##1}{##2}]%
%       :\hspace{1em}##3\glspostdescription\space ##5}%
% }
\pagebreak
\printglossary[title=Glossary,toctitle=Glossary]
%-------------------------------------------------------------------------------
\pagebreak
\printbibliography[heading=bibintoc, title={References}]
%-------------------------------------------------------------------------------
\printindex
%-------------------------------------------------------------------------------

\printbibliography[heading=bibintoc, title={References}]
%-----------------------------------------------------------------
\end{document}
