% !TEX TS-program = arara
% arara: xelatex: { synctex: on, options: [--halt-on-error] } 
% arara: biber
% arara: makeglossaries
% arara: makeindex
% arara: xelatex: { synctex: on, options: [-halt-on-error] } 
% arara: xelatex: { synctex: on, options: [-halt-on-error] } 
% arara: clean: { files: [topos-notes.aux,topos-notes.bbl] }
% arara: clean: { files: [topos-notes.bcf,topos-notes.blg] }
% arara: clean: { files: [topos-notes.glg,topos-notes.glo] }
% arara: clean: { files: [topos-notes.gls,topos-notes.gls] }
% arara: clean: { files: [topos-notes.idx,topos-notes.ilg] }
% arara: clean: { files: [topos-notes.ind,topos-notes.loe] }
% arara: clean: { files: [topos-notes.lof,topos-notes.log] }
% arara: clean: { files: [topos-notes.log,topos-notes.lol] }
% arara: clean: { files: [topos-notes.out,topos-notes.thm] }
% arara: clean: { files: [topos-notes.run.xml] }
% arara: clean: { files: [topos-notes.toc,topos-notes.xdy] }
% arara: clean: { files: [topos-notes.synctex.gz] }
%-------------------------------------------------------------------------------
\documentclass[11pt,openany]{article}
\usepackage{import}
\def \texFolder {../../tex/}
\def \bibFolder {../../bib/}
\def \sharedFolder {../../shared/}
\import{\texFolder}{head}
\import{\texFolder}{landscape-2col-8x17}
\import{\texFolder}{clj-listings}
\import{\sharedFolder}{glossary}
\addbibresource{{\bibFolder}algebra.bib}
\addbibresource{{\bibFolder}arithmetic.bib}
\addbibresource{{\bibFolder}cactus.bib}
\addbibresource{{\bibFolder}proglang.bib}
\addbibresource{{\bibFolder}halmos.bib}
\addbibresource{{\bibFolder}interpolation.bib}
\addbibresource{{\bibFolder}logic.bib}
\addbibresource{{\bibFolder}mcdonald.bib}
\addbibresource{{\bibFolder}mop.bib}
\addbibresource{{\bibFolder}numbers.bib}
\addbibresource{{\bibFolder}sets.bib}
\addbibresource{{\bibFolder}tex.bib}
%\addbibresource{{\bibFolder}ieeestd.bib}
\addbibresource{{\bibFolder}fparith.bib}
\addbibresource{{\bibFolder}topos.bib}
%-----------------------------------------------------------------
\title{Notes on category and topos theory}
\author{\textsc{John Alan McDonald}}
%\email{mcdonald.john.alan at gmail.com}
\date{draft of \today}
%-----------------------------------------------------------------
\begin{document}

\maketitle

%-----------------------------------------------------------------
% 7 for part level
% 6 for chapter level
% 5 for section level
% 4 for subsection level
% 3 for subsubsection level
% 2 for paragraph level
% 1 for subparagraph level
\setcounter{baseSectionLevel}{5}
%-----------------------------------------------------------------

%\frontmatter

% \begingroup
% \let\onecolumn\twocolumn
% \sffamily
% \tableofcontents
% \rmfamily
% \endgroup
% 
%-----------------------------------------------------------------
\epigraph{
\textsl{das Wesen der Mathematik liegt gerade in ihrer Freiheit.}
\par
(The essence of mathematics lies precisely in its freedom.)}%
{Cantor~\cite{Cantor1883},
as quoted in~\cite{ferreiros2007labyrinth}.}

\epigraph{
Mathematics is not the rigid and rigidity-producing schema 
that the layman thinks it is; 
rather, in it we find ourselves at that meeting point 
of constraint and freedom 
that is the very essence of human nature.}%
{Hermann Weyl ?}


motivation from empirical problems; 
see eg~\cite{maclane1981mathModels}.
goal ``breadth, clarity, and depth''.

adjoint functors?

axioms on composition of functions as alternative to set theory
as foundation of mathematics

Cantorian set theory: platonic sets with $2$nd order axioms.

my point of view: 
mathematics (notation) and programming languages
as \textit{written} languages,
used to formulate real world problems,
and construct their solutions.

express an idea in at most a couple pages of text.

say enough and not too much, especially not irrelevant things.
abstract just enough; 
concrete details of model/representation/implementation.

key issue: is the solution correct?
mathematics evaluated by consensus;
software by testing.

\epigraph{Mathematics as we practice it is much more formally 
complete and precise than
other sciences, but it is much less formally complete and precise 
for its content
than computer programs.}%
{Thurston, 
\textit{On proof and progress in mathematics
}~\cite{thurston1994proof}}

%-----------------------------------------------------------------
\setcounter{currentlevel}{\value{baseSectionLevel}}
\levelstay{Preface 2020-07}
\label{sec:Preface-2020-07}

Capture state of ignorance as of 2020-07,
to record what's confusing to the novice.

For the past $20+$ years, I have, every year or so,
 encountered claims that
category is the ``right way to do \textit{blah}''.
Pick up some recent 
introduction~\cite{asperti-longo-1991,awodey-2010,
barr-wells-2020,fong-spivakd-2018-seven-sketches,
geroch-1985,lawvere-schanuel-2009,
leinster-2016-basic-category-theory,
maclane-1998-working-mathematician,nLab-2020,
spivakd-2013,spivakd-2014} 
get about halfway and give up.

Recent context: optimization software leads to
exact arithmetic leads to computable reals,
leads to contructive analysis and intuitionistic logic,
leads to history and foundations of math 
leads to topos theory.

Current hypothesis: some thing is wrong.
Directed multigraph perhaps isn't the 'right' way
expose and use commonalities across mathematics.


%-----------------------------------------------------------------
\setcounter{currentlevel}{\value{baseSectionLevel}}
\levelstay{Why?}
\label{sec:Why?}


An alternative to first order logic, sets, 
and functions as a foundation for all
mathematics~\cite{feferman-1977-FEFCFA},  
particularly topos theory.


%-----------------------------------------------------------------
\setcounter{currentlevel}{\value{baseSectionLevel}}
\levelstay{Naming}
\label{sec:Naming}

Particularly bad in category theory, eg, 
what's ``commuting'' is a commuting diagram.

Prefer self-explanatory names, even if a bit longer:
eg, ``left (right) cancelable'' rather than ``mono(epi)morphism'',
``structure-preserving function'' over ``homomorphism''.

``Morphism'' etymology.
 
%-----------------------------------------------------------------
\setcounter{currentlevel}{\value{baseSectionLevel}}
\levelstay{Graphical language}
\label{sec:Graphical-language}

%-----------------------------------------------------------------
\setcounter{currentlevel}{\value{baseSectionLevel}}
\levelstay{Bottom up}
\label{sec:Bottom-up}

Start from directed (multi)graph.

Not really a foundation, due to 
dependence on \textit{collections} of objects and arrows
and \textit{functions} between these collection. 

Categorical constructs are pretty much impossible to motivate
 in this context.

%-----------------------------------------------------------------
\setcounter{currentlevel}{\value{baseSectionLevel}}
\levelstay{Top down}
\label{sec:Top-down}
 
Topos definition, then define what it depends on, \ldots.
 
%-----------------------------------------------------------------
\setcounter{currentlevel}{\value{baseSectionLevel}}
\levelstay{Abstraction}
\label{sec:Abstraction}

Start from all of conventional mathematics,
remove unnecessary details freom definitions, theorems, proofs.
Express what remains in graphical category language.
Look for ``graphs'' that match, unifying disparate parts of math.

Problem: you need to learn a lot of conventional math before
you can really start on category theory.

 
%-----------------------------------------------------------------
\pagebreak
\appendix
%\part{Appendices}
\import{\texFolder}{typesetting}
%-----------------------------------------------------------------
%\backmatter


%\part{Backmatter}
% un-comment \input and comment \printbibiliography
% to get index, glossary, list of theorems, etc.
%%------------------------------------------------------------------------------
% \begingroup  % Temporarily disable \clearpage to show both lists on one page
%   %\let\clearpage\relax    % http://tex.stackexchange.com/a/14511/104449
%   \renewcommand{\listtheoremname}{List of definitions}
%   \textsf{\listoftheorems[ignoreall, show={definition}]}
% \endgroup
%-------------------------------------------------------------------------------
\pagebreak
\renewcommand{\listfigurename}{Figures}
\addcontentsline{toc}{chapter}{\listfigurename}
\begingroup
\let\onecolumn\twocolumn
\sffamily
\listoffigures
\rmfamily
\endgroup
%-------------------------------------------------------------------------------
\pagebreak
\renewcommand{\lstlistlistingname}{Code samples}
\addcontentsline{toc}{chapter}{\lstlistlistingname}
\begingroup
\let\onecolumn\twocolumn
\sffamily
\lstlistoflistings
\rmfamily
\endgroup
%-------------------------------------------------------------------------------
\pagebreak
\renewcommand{\listtheoremname}{Examples}
\addcontentsline{toc}{chapter}{\lstlistlistingname}
\begingroup
\let\onecolumn\twocolumn
\sffamily
\listoftheorems
\rmfamily
\endgroup
%-------------------------------------------------------------------------------
% \newglossarystyle{mystyle}{%
%  \glossarystyle{altlist}%
%  \renewcommand*{\glossaryentryfield}[5]{%
%    \item[\glsentryitem{##1}\glstarget{##1}{##2}]%
%       :\hspace{1em}##3\glspostdescription\space ##5}%
% }
\pagebreak
\printglossary[title=Glossary,toctitle=Glossary]
%-------------------------------------------------------------------------------
\pagebreak
\printbibliography[heading=bibintoc, title={References}]
%-------------------------------------------------------------------------------
\printindex
%-------------------------------------------------------------------------------

\printbibliography[heading=bibintoc, title={References}]
%-----------------------------------------------------------------
\end{document}
