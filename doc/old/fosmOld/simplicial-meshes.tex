% jam 2004-09-10

\section{Simplicial Meshes}
\label{sec:simplicial-meshes}

An {\it abstract $d$-simplex} (or just {\it simplex})
$\Simplex{S}$, is a set of $d+1$ elements of some ground space.
The elements of the simplex are its {\it vertices}, $\{ \Vertex{V}_0 \ldots \Vertex{V}_d \}$.
A typical ground space might be the non-negative integers,
which is what I will assume in the following.
A typical implementation would be an integer array
whose elements are sorted to facilitate equality/containment tests, etc.

The {\it dimension} of a $d$-simplex is $d$.
The {\it faces} of a simplex $\Simplex{S}$ are the simplices which are subsets of $\Simplex{S}$.
The {\it facets} of $\Simplex{S}$ are $(d-1)$-simplices contained in $\Simplex{S}$.
The {\it edges} of $\Simplex{S}$ are the $1$-simplices contained in $\Simplex{S}$.
If $\Simplex{T}$ is a face of $\Simplex{S}$,
then the face {\it opposite} $\Simplex{T}$ in $\Simplex{S}$
is the simplex contained those vertices of $\Simplex{S}$
not in $\Simplex{T}$.
If $\Simplex{T}$ is the facet of $\Simplex{S} = \{ \Vertex{V}_0 \ldots \Vertex{V}_d \}$
containing, wlog, vertices $\{ \Vertex{V}_0 \ldots \Vertex{V}_{d-1} \}$
then we speak of $\Vertex{V}_d$ as the {\it vertex opposite} $\Simplex{T}$
and $\Simplex{T}$ as the {\it facet opposite} $\Vertex{V}_d$.

An {\it oriented simplex} is a simplex together with a choice
of one of the two possible circular orderings of its vertices.
A typical implementation would add an orientation flag
to the sorted array simplex implementation.

An {\it abstract simplicial complex} (or just {\it simplicial complex}),
$\SimplicialComplex{K}$, is a set of simplices
that obey an containment constraint:
if $\Simplex{S}$ is in $\SimplicialComplex{K}$, then all faces of $\Simplex{S}$ are in $\SimplicialComplex{K}$.
The complex {\it generated} by a set of simplices $\{ \Vertex{V} \}$,
$\SimplicialComplex{K} \left( \{ \Vertex{V} \} \right)$ is the union of all $\{ \Vertex{V} \}$
and their faces.

The {\it dimension} of a simplicial complex is the maximum dimension
of any simplex in the complex.
The {\it facets} of a complex are its $(d-1)$-simplices.
The {\it edges} of a complex are its $1$-simplices.
The {\it $k$-skeleton} of a complex $\SimplicialComplex{K}$
is the simplicial complex generated by the $k$-simplices in $\SimplicialComplex{K}$.

A {\it pure simplicial complex} is an $d$-dimensional complex
in which every simplex is contained in some $d$-simplex in the complex,
that is, a complex which is generated by its $d$-simplices.
The {\it boundary} of a pure $d$-dimensional simplicial complex $\SimplicialComplex{K}$,
$\Boundary \SimplicialComplex{K}$,
is the complex generated by the $(d-1)$-simplices in $\SimplicialComplex{K}$
which are contained in only 1 $d$-simplex in $\SimplicialComplex{K}$.

An {\it orientable simplicial complex} is a pure $d$-dimensional simplicial complex
in which all the $d$-simplices can be assigned consistent orientations,
and an {\it oriented simplicial complex} is one in which the simplices
have been oriented consistently.
Two oriented $d$-simplices that share a $(d-1)$-dimensional facet
are {\it oriented consistently} if they have opposite orientations
relative to the shared facet. The orientation of a simplex relative to
one of its facets can be computed by permuting the vertices so that
the first $d$ vertices belong the the facet and the unshared vertex is last.
The sign of the permutation times the $\pm$ assigned orientation
is the relative orientation.




A {\it geometric realization of an abstract simplex,}
or {\it geometric simplex}, $\p(\Simplex{S})$,  associates points
in a {\it realization space,}
$\p ( \Vertex{V}_i ) \in \vspace$, with the vertices of an abstract simplex.
The points are referred to as the {\it vertex positions.}
For the purposes of this paper,
I assume $\vspace$ is
a finite dimensional real inner product space.

Note that this is not the usual definition,
in which a geometric simplex is the convex hull of its vertex positions
$\convexspan \{ \p(\Vertex{V}_0) \ldots \p(\Vertex{V}_d) \} = \convexspan ( \p(S) )$.
I prefer to keep the simplex distinct from its convex hull,
but I will often omit explicit mention of the convex hull
(eg., the 'volume of a simplex', the 'interior of a simplex', etc.).

With my definition,
a geometric simplex could be viewed as simply an abstract simplex
where the ground space is the realization space, $\vspace$.
However, in the context of mesh optimization,
it is better to preserve the distinction between an abstract simplex
and its geometric realizations.

A {\it simplicial mesh} is a pure simplicial complex, $\SimplicialComplex{K}$,
plus a mapping, $\p()$, from abstract vertices to points.

Note that this differs from the usual definition for
{\it geometric simplicial complex}
(or {\it geometric realization of a simplicial complex}),
which requires that the convex hulls of the geometric
simplices obey a geometric containment constraint:
If $\Simplex{S}_0$ and $\Simplex{S}_1$ are in $\SimplicialComplex{K}$,
then $\convexspan ( \p( \Simplex{S}_0 ) ) \intersection
\convexspan ( \p( \Simplex{S}_1 ) )$
is either empty or
$\convexspan ( \p( \Simplex{S}_2 ) )$ for some $\Simplex{S}_2 \in \SimplicialComplex{K}$.
I distinguish 'simplicial mesh' from 'geometric simplicial complex'
because it isn't feasible or desirable to maintain the geometric
containment constraint during optimization.

My definition differs from many other definitions of 'simplicial mesh',
which are restrictions of 'geometric simplicial complex', for example,
requiring a 'simplicial mesh' to be a geometric simplicial complex
which is an orientable manifold with boundary.




