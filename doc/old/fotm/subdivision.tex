I am only considering subdivision surfaces based on triangles
\cite{HoppeEtal:1994:SIGGRAPH,Hoppe:1994:Phd}.

\subsection{Approximating meshes}
\label{sec:Approximating-meshes}

A common approach to the use of subdivision surfaces is
to approximate the limit surface by the {\it subdivided mesh,} $\M^s$,
a $k$ times subdivided version of the {\it control mesh,} $\M^c$
(a typical value for $k$ is 2).
The positions of the $n^c$ vertices of the control mesh,
$\p^c = (\p^c_0 \ldots \p^c_{n-1}) \\in \Re^{3n^c},$
and the $n^s$ vertices of the subdivided mesh,
$\p^s = (\p^s_0 \ldots \p^s_{n-1}) \in \Re^{3n^s},$
are related by the {\it subdivision transform}
$\S : \Re^{n^c} \mapsto \Re^{n^s}$.
If
$\x^c = (x^c_0 \ldots x^c_{n-1}) \in \Re^{n^c}$,
$\y^c = (y^c_0 \ldots y^c_{n-1}) \in \Re^{n^c}$,
and
$\z^c = (z^c_0 \ldots z^c_{n-1}) \in \Re^{n^c}$,
are the $x, y,$ and $z,$ coordinates of $\p^c$,
and $\x^s, \y^s, \z^s$ are the same coordinates
of $\p^s$, then
\begin{eqnarray}
\x^s & = & \S \x^c
\\
\y^s & = & \S \y^c
\nonumber
\\
\z^s & = & \S \z^c
\nonumber
\end{eqnarray}
We can use the above to define $\S_3 : \Re^{3n^c} \mapsto \Re^{3n^s}$,
so that
\begin{equation}
\p^s = \S_3 \p^c.
\end{equation}


If $f(\p^s) = f(\S_3 \p^c)$ is a penalty function applied to the subdivided mesh,
then the gradient with respect to the positions of
the vertices of the control mesh is simply:
\begin{equation}
\Gc{\p^c}{f(\S_3 \p^c)}{\q^c} = \S_3^{\dagger} \Gc{\p^s}{f(\p^s)}{\q^s = \S_3 \q^c}
\end{equation}
