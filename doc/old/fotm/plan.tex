\label{sec:general}


%--------------------------------------------------------------------

\subsection{Identities for real vector operations}
\label{sec:RX}

See \cite[p.~85, ex.~4-9]{Spivak:1965:CalculusOnManifolds}.

Let $\p, \q \in \Re^{n}$.
Let $\theta(\p,\q)$ be the angle between $\p$ and $\q$.

\begin{itemize}
\item The inner (dot) product:
\begin{equation}
\p \bullet \q \; \equiv \; \sum_{i=0}^{n-1} p_i q_i
\end{equation}

\item The euclidean ($l_2$) norm:
\begin{equation}
\| \p \|^2 \; \equiv \; \p \bullet \p
\end{equation}
\begin{equation}
\p \bullet \q \; = \; \| \p \| \| \q \| \cos(\theta(\p,\q))
\end{equation}

\item Orthogonal complement:
\begin{equation}
\p \perp \q \; \equiv \; \p \; - \; \left( \p \bullet \frac{\q}{\|\q\|}\right) \frac{\q}{\|\q\|}
\end{equation}

\item The tensor product

Let $\p \in \Re^m, \q, \r \in \Re^n.$
$\p \otimes \q$ is a rank 1 linear transformation
from $\Re^n$ to $\Re^m$, defined by:
\begin{equation}
(\p \otimes \q)(\r) \; \equiv \; \p (\q \bullet \r)
\end{equation}

\end{itemize}

%--------------------------------------------------------------------

\subsection{Identities for 3-dimensional vector operations}
\label{sec:R3X}

See \cite[p.~85, ex.~4-9]{Spivak:1965:CalculusOnManifolds}, .

Let $\p, \q, \r \in \Re^3$.
Let $(p_0,p_1,p_2), (q_0,q_1,q_2), (r_0,r_1,r_2), $ be their coordinates
in some orthonormal basis.

The cross product:
\begin{equation}
\p \times \q  \; \equiv \; (p_1 q_2 - p_2 q_1, \; p_2 q_0 - p_0 q_2, \; p_0 q_1 - p_1 q_0)
\end{equation}
\begin{equation}
\p \times \q  \; = \; - \; \q \times \p
\end{equation}
\begin{equation}
\| \p \times \q \| \; = \; \| \p \| \; \| \q \| \; \sin(\theta(\p,\q))
\end{equation}
\begin{equation}
\| \p \times \q \|  \; = \;  \sqrt{\| \p \|^2 \| \q \|^2 \; - \; (\p \bullet \q)^2}
\end{equation}
\begin{equation}
\p \bullet ( \p \times \q ) \; = \; ( \p \times \q ) \bullet \q \; = \; 0
\end{equation}
\begin{equation}
\label{eq:dot_cross}
\p \bullet ( \q \times \r ) \; = \; ( \p \times \q ) \bullet \r \; = \; \q \bullet ( \r \times \p )
\end{equation}
\begin{equation}
\p \times ( \q \times \r ) \; = \; ( \p \bullet \r ) \q \; - \; (\p \bullet \q) \r
\end{equation}
\begin{equation}
( \p \times \q ) \times \r \; = \; ( \p \bullet \r ) \q \; - \; (\q \bullet \r) \p
\end{equation}
\begin{equation}
( \p \times \q ) \times \r \; = \; \left((\q \otimes \p) - (\p \otimes \q)\right) \r
\end{equation}


%-------------------------------------------------------------------------------

\subsection{Functions on real vector spaces}
\label{sec:functions}

This paper describes functions of triangular meshes.

Interesting functions usually depend, directly or indirectly,
on the positions of some subset of the vertices.
I consider the vertex positions to be elements of $\Re^3$,
with an (implied) universal origin,
and thus do not distinguish points and vectors.

In general, the functions discussed here map between real vector spaces:
$\f:{\Re}^{n} \mapsto \Re^{m}$, where $\Re^n$ is the
{\it domain} and $\Re^m$ is the {\it codomain}.
Strictly speaking, the {\it range} of $\f$ is the set $\f(\Re^n)$,
which may be a proper subset of its codomain $\Re^m$.

I typically use $\p$, $\q$, $\r$, etc., for elements of $\Re^n$
and
$\f$, $\g$, $\h$ for vector-valued functions.
I generally do not distinguish $\Re$, the real numbers,
and $\Re^1$, the 1-dimensional real vector space.
I sometimes use $f$, $g$, $h$ for extra clarity in the special
case of real-valued functions.

The domains of many interesting functions,
such as those that depend on vertex positions,
are direct sums of $\Re^3$.
The {\it direct sum} $\Re^n \oplus \Re^m$ is the cartesian product
of $\Re^n$ and $\Re^m$ --- the set of ordered pairs $(\p,\q)$
where $\p \in \Re^n$ and $\q \in \Re^m$ ---
with the restriction that the inner product is the obvious extension of the
inner products on $\Re^n$ and $\Re^m$:
$(\p_0,\q_0) \bullet (\p_1,\q_1) = (\p_0 \bullet \p_1) + (\q_0 \bullet \q_1).$
For simplicity, I identify
$\Re^{3n} = \Re^3 \oplus \Re^3 \oplus \cdots \oplus \Re^3 = \oplus^n \Re^3
= \Re^n \oplus \Re^n \oplus \Re^n $.
I will usually write an element of $\oplus^n \Re^3$ as
$(\p_0,\ldots,\p_{n-1})$
and use
$\f(\p_0,\p_1,\ldots,\p_{n-1})$
for a function that depends on $n$ vertices.
Sometimes it will be useful to separate the $x,$ $y,$ and $z$ coordinates:
$\p = (\x,\y,\z),$
where $\x =(x_0, \ldots x_{n-1}) \in \Re^n$, are the $x$-coordinates
of the positions of the vertices, and similarly for $y$ and $z$.

%-------------------------------------------------------------------------------

\subsection{Derivatives}
\label{sec:derivatives}

One way to view the derivative of a function
$\f:{\Re}^{n} \mapsto \Re^{m}$,
at a point $\p$,
is as the linear transformation $\L:{\Re}^{n} \mapsto \Re^{m}$,
that best approximates the local 'slope' of $\f$ at $\p$.
To be a little more precise, we want
\begin{displaymath}
\lim_{ \|{\bf \delta}  \| \mapsto 0}
{ \|{\f(\p + {\bf \delta}) - (\f(\p) + \L({\bf\delta})) \|}
  \over  \|{\bf \delta}  \|}
 = 0
\end{displaymath}
For a concise and correct discussion, see \cite{Spivak:1965:CalculusOnManifolds}.

\begin{itemize}

\item $\Da{\f}$

In its most general form,
I denote the derivative of $\f$ by $\Da{\f}$.
Note that this is
linear-transformation-valued function of the domain of $\f$.

\item $\Db{\f}{\p}$

I denote the derivative of $\f$ at $\p$ by $\Db{\f}{\p}$.
$\Db{\f}{\p}$ is a specific linear transformation from
the domain of $\f$ to the codomain of $\f$.

\item $\Dc{\f}{\p}{\q}$

The derivative is most often represented by the {\it Jacobian},
the $m \times n$ matrix of partial derivatives
with respect to some bases for $\Re^n$ and $\Re^m$.
However, it's often easier to express the derivative clearly if we
explicitly include the argument of the linear transformation.
In this case, I write $\Dc{\f}{\p}{\q}$
for the derivative of $f$ at the point $\p$
applied to the vector $\q$.

\item $\Dd{\p_i}{\f}{(\q_0,\ldots,\q_{n-1})}{\r_i}$

For functions on direct sum spaces,
$\f(\p_0,\p_1,\ldots,\p_{n-1})$, $\p_i \in \Re^{n_i}$,
it's often easier to consider the derivative
with respect to one argument at a time.
I write $\Dd{\p_i}{\f}{(\q_0,\ldots,\q_{n-1})}{\r_0,\ldots,\r_{n-1}}$
for the derivative of $\f$ with respect to $\p_i$,
at the point $(\q_0,\ldots,\q_{n-1}) \in \oplus_{i=0}^{n-1} \Re^{m_i}$,
applied to the vector $\r_i \in \Re^{n_i}$.
Note that, if you consider $\f$ to be a function
of direct sums of $\Re^1$, we have the usual
partial derivatives.

\end{itemize}


%-------------------------------------------------------------------------------

\subsubsection{Gradients of real-valued functions}
\label{sec:gradients}

\begin{itemize}

\item $\Ga{f}$

In minimizing real-valued functions, $f(\p)$, $\p \in \Re^n$,
we frequently need
the {\it gradient,} $\Ga{f} \in \Re^n$,
the vector pointing in the direction of most rapid increase of $f$,
whose magnitude is the rate of increase, or slope,
of $f$ in that direction.

The gradient, $\Ga{f}$,
has a close relationship to the derivative, $\Da{f}$,
and the two are often confused.
Recall that the derivative is a linear transformation
from the domain of $f$ to its codomain.
In the case of real-valued functions,
this means the derivative is a linear function on $\Re^n$,
an element of the dual space of $\Re^n$, a 'row' vector.
It's easy to see that the gradient is simply the dual (the 'transpose')
of the derivative, $\Ga{f} = (\Da{f})^{\dagger}$
(see \cite[p.~96, ex.~4-18]{Spivak:1965:CalculusOnManifolds}).

Notation for the various versions of the gradient
follows that for derivatives:

\item $\Gb{f}{\q}$

The gradient of $f$ at $\q$.

\item $\Gc{\p_i}{f}{\q}$

The gradient of $f$
with respect to $\p_i$ at $\q$.

\item $(\Gb{f}{\q}) \bullet \; \r$

The analog to exressing the derivative as a linear transformation
with an explicit argument is to write expressions for
the inner product of the gradient and an arbitrary other vector $\r$

\item $(\Gc{\p_i}{f}{\q}) \bullet \;\r_i$

See above.

\end{itemize}


%-------------------------------------------------------------------------------

\subsubsection{Chain rule}
\label{sec:chain}

The most general identity used in computing derivatives is the {\it chain rule.}
Suppose
$\f:\Re^{n_0} \mapsto \Re^{n_1}$,
$\g:\Re^{n_1} \mapsto \Re^{n_2}$,
and
$\h = \g \circ \f : \Re^{n_0} \mapsto \Re^{n_2}.$
Then
\begin{equation}
\label{eq:chain-rule}
\Db{\h}{\p} \;\; = \;\; \Db{(\g \circ \f)}{\p}
            \;\; = \;\; \Db{\g}{\f(\p)} \; \circ \; \Db{\f} {\p}.
\end{equation}

See \cite[Theorem~2-2]{Spivak:1965:CalculusOnManifolds}.


%-------------------------------------------------------------------------------

\subsubsection{Derivatives of multilinear functions}
\label{sec:multilinear}

A function $\f(\p_0,\ldots,\p_k):\Re^{n_0} \oplus \Re^{n_k} \mapsto \Re^m$
is {\it multilinear} if
\begin{equation}
\f(a_{00} \p_{00} + a_{01} \p_{01}, \ldots, a_{k0} \p_{k0} + a_{k1} \p_{k1})
\; = \; \sum_{i_0,\ldots,i_k = 0,1} \;(a_{0i_0} \cdots a_{ki_k}) \f(\p_{0i_0}, \ldots, \p_{ki_k}).
\end{equation}

The derivative of $\f$
at the point $(\p_0,\ldots,\p_k)$, applied to the vector $(\q_0,\ldots,\q_k)$ is

\begin{equation}
\Dc{\f}{(\p_0,\ldots,\p_k)}{\q_0,\ldots,\q_k}
\; = \; \sum_{i=0,k} \f(\p_0,\ldots,\p_{i-1},\q_i,\p_{i+1},\ldots,\p_k).
\end{equation}

See \cite[ex.~2-14]{Spivak:1965:CalculusOnManifolds}.


%-------------------------------------------------------------------------------

\subsubsection{Derivatives of bilinear functions}
\label{sec:bilinear}

Bilinear functions are a useful special case of multilinear functions.

A function $\f(\p,\q):\Re^{n_0} \oplus \Re^{n_1} \mapsto \Re^m$
is {\it bilinear} if
\begin{eqnarray}
\f(a_0 \p_0 + a_1 \p_1, b_0 \q_0 + b_1 \q_1) & = & a_0 b_0 f(\p_0,\q_0)  \\
                                             & + & a_0 b_1 f(\p_0,\q_1) \nonumber \\
                                             & + & a_1 b_0 f(\p_1,\q_0) \nonumber \\
                                             & + & a_1 b_1 f(\p_1,\q_1).\nonumber
\end{eqnarray}

The derivative of $\f$
at the point $(\p_0,\q_0)$, applied to the vector $(\p,\q)$ is

\begin{equation}
\Dc{\f}{(\p_0,\q_0)}{\p,\q} = \f(\p_0,\q) + \f(\p,\q_0).
\end{equation}

See \cite[ex. 2-12]{Spivak:1965:CalculusOnManifolds}.

%-------------------------------------------------------------------------------

\subsubsection{Derivatives of linear functions}
\label{sec:Derivatives-of-linear-functions}

Linear functions are another useful special case of multilinear functions.
A function $\f(\p):\Re^{n} \mapsto \Re^m$
is {\it linear} if
\begin{equation}
\f(a_0 \p_0 + a_1 \p_1)
 =
a_0 \f(\p_0) + a_1 \f(\p_1)
\end{equation}

The derivative of $\f$ is simply $\f$ itself.

%-------------------------------------------------------------------------------

\subsubsection{Derivatives of inner products}
\label{sec:inner}

We can view the inner product on $\Re^m$, $\p \bullet \q$,
as a bilinear function $d(\p,\q) : \Re^m \oplus \Re^m \mapsto \Re$.
Thus
\begin{equation}
\Dc{d}{(\p_0,\q_0)}{\p,\q} = \p_0 \bullet \q + \p \bullet \q_0.
\end{equation}

Suppose
$\f:\Re^{n} \mapsto \Re^{m}$, and
$\g:\Re^{n} \mapsto \Re^{m}$.
The derivative of $\f \bullet \g$ is:
\begin{eqnarray}
\label{eq:dot_derivative}
\Dc{(\f \bullet \g)}{\p_0}{\p}
& =
& \Db{d}{(\f(\p_0),\g(\p_0))} \;\circ \;(\Dc{\f}{\p_0}{\p}, \Dc{\g}{\p_0}{\p})
\\
& =
& \f(\p_0) \bullet \Dc{\g}{\p_0}{\p} \; + \; \g(\p_0) \bullet \Dc{\f}{\p_0}{\p} \nonumber
\end{eqnarray}

See \cite[ex.~2-13]{Spivak:1965:CalculusOnManifolds}.


%-------------------------------------------------------------------------------

\subsubsection{Derivatives of cross products}
\label{sec:cross}

We can view the 3-dimensional cross product
$ \times $
as a bilinear function
$\c(\p,\q) = \p \times \q : \Re^3 \oplus \Re^3 \mapsto \Re^3$.
As with the inner product,
the derivative is
\begin{equation}
\Dc{c}{(\p_0,\q_0)}{\p,\q} = \p_0 \times \q + \p \times \q_0.
\end{equation}

Suppose
$\f:\Re^{n} \mapsto \Re^3$, and
$\g:\Re^{n} \mapsto \Re^3$.
The derivative of $\f \times \g$ is:
\begin{eqnarray}
\Dc{(\f \times \g)}{\p_0}{\p}
& =
& \Db{\c}{(\f(\p_0),\g(\p_0))} \;\circ \;(\Dc{\f}{\p_0}{\p}, \Dc{\g}{\p_0}{\p})
\\
& =
& \f(\p_0) \;\times \;\Dc{\g}{\p_0}{\p} \;+ \;\Dc{\f}{\p_0}{\p} \;\times \;\g(\p_0) \nonumber
\end{eqnarray}

%-------------------------------------------------------------------------------

\subsubsection{Derivatives of scalar products}
\label{sec:scalar}

Suppose
$f:\Re^{n} \mapsto \Re$, and
$\g:\Re^{n} \mapsto \Re^m$.
It follows from the chain rule that the derivative of $\h = f\g$ is:
\begin{eqnarray}
\label{eq:scalar_product_derivative}
\Db{(f\g)}{\p}
& = & f(\p) \;\Db{\g}{\p} \;+ \g(\p) \; \Db{f}{\p}  \\
& = & f(\p) \;\Db{\g}{\p} \;+ \g(\p) \otimes \Gb{f}{\p} \; \nonumber
\end{eqnarray}


%-------------------------------------------------------------------------------

\subsubsection{Derivatives of euclidean norms}
\label{sec:norms}

Let $l_2^2(\p) = \; \| \p  \|^2 : \Re^n \mapsto \Re$
be the square of the usual euclidean norm on $\Re^n$.
($ \| \p  \|^2  = \sum_{i=0,n-1} \p_i^2$.)
\begin{eqnarray}
\label{eq:l2-gradient}
\Gb{l_2^2}{\p} & = & 2\p \\
\Db{l_2^2}{\p} & = & 2\p^\dagger
\end{eqnarray}

Let $l_2(\p) = \; \| \p  \|$ be the usual euclidean norm.

$\Gb{l_2}{\p} = {{ \p } \over { \| \p  \|}}$,

$\Db{l_2}{\p} = {{ \p^\dagger } \over { \| \p  \|}}$.

Let $\f(\p) : \Re^n \mapsto \Re^m$.
By the chain rule:
$\Db{\| \f \|^2}{\p}  =  2 {\f(\p)}^{\dagger} \Db{\f}{\p} $.


\begin{equation}
\Gb{\| \f \|^2}{\p}  =  2 \;\Db{\f}{\p}^\dagger \;\circ \;\f(\p)
\end{equation}

\begin{eqnarray}
\label{eq:norm_derivative}
\Db{\| \f \|}{\p}
& = &
{{\f(\p)^\dagger} \over {\| \f(\p) \|}} \Db{\f}{\p}  \\
\Gb{\| \f \|}{\p}
& = &
\left(\Db{\f}{\p}\right)^\dagger \;\circ \;{{\f(\p)} \over { \| \f(\p)  \|}}
\label{eq:norm_gradient}
\end{eqnarray}

%-------------------------------------------------------------------------------

\subsubsection{Derivatives of normalized functions}
\label{sec:Derivatives-of-normalized-functions}

Let $\tilde{\f}$ be the normalized version of $\f$:
\begin{equation}
\tilde{\f} \;= \;{{\f} \over {\| \f \|}}
\end{equation}

Then, from equations \ref{eq:scalar_product_derivative}
and \ref{eq:norm_derivative}:
\begin{eqnarray}
\Dc{\tilde{\f}}{\p}{\q}
& = &
\Dc{\left({\f \over {\| \f \|}}\right)}{\p}{\q}
\\
& = &
{\Dc{\f}{\p}{\q} \over {\| \f(\p) \|}}
\; + \;
\f(\p) \; \Dc{\left({1 \over {\| \f \|}}\right)}{\p}{\q} \nonumber \\
& = &
{\Dc{\f}{\p}{\q} \over {\| \f(\p) \|}}
\; - \;
\f(\p) {{\Dc{\| \f \|}{\p}{\q}} \over {\|\f(\p)\|^2}} \nonumber \\
& = &
{\Dc{\f}{\p}{\q} \over {\| \f(\p) \|}}
\; - \;
\f(\p) \left( {{\f(\p)^\dagger} \over {\| \f(\p) \|^3}} \;\Dc{\f}{\p}{\q} \right) \nonumber \\
& = &
{\| \f(\p) \|^2 \Dc{\f}{\p}{\q}
\; - \;
\f(\p)\left( \f(\p) \bullet \Dc{\f}{\p}{\q} \right) }
\over {\| \f(\p) \|^3}  \nonumber \\
& = &
{{\| \f(\p) \|^2 \I_{\Re^3} \;- \;\left( \f(\p) \otimes \f(\p) \right)  }
\over {\| \f(\p) \|^3} }
\;\Dc{\f}{\p}{\q} \nonumber \\
& = &
{{\I_{\Re^3} \;- \;\left( \tilde{\f}(\p) \otimes \tilde{\f}(\p) \right)  }
\over {\| \f(\p) \|} }
\;\Dc{\f}{\p}{\q} \nonumber
\end{eqnarray}

$\otimes$ is the elementary tensor product operation.
If you are stuck thinking in terms of row and column vectors,
$\p \otimes \q \;= \;\p\q^\dagger$.
More generally, if $\p \in \Re^m$ and $\q \in \Re^n$,
then $\p \otimes \q$ is the rank 1 linear transformation from $\Re^n \mapsto \Re^m$:
$\left(\p \otimes \q\right) (\r) \;= \;\p \left(\q \bullet \r\right)$.

We can write the derivative above without reference to the argument $\q$:
\begin{eqnarray}
\label{eq:normalized_function_derivative}
\Db{\tilde{\f}}{\p}
& = &
\Db{\left({\f \over {\| \f \|}}\right)}{\p}  \\
& = &
{{\I_{\Re^3} \;- \;\left( \tilde{\f}(\p) \otimes \tilde{\f}(\p) \right) }
\over {\| \f(\p) \|} }
\;\Db{\f}{\p} \nonumber
\end{eqnarray}

A common, trivial, normalized function is the normalized version of
a vector:
\begin{equation}
\tilde{\p} \;= \;{{\p} \over {\| \p \|}}
\end{equation}

From equation \ref{eq:normalized_function_derivative}
it follows that:
\begin{eqnarray}
\label{eq:normalized_vector_derivative}
\Db{\tilde{\p}}{\q}
& = &
\Db{\left({\p \over {\| \p \|}}\right)}{\q}
\\
& = &
{{\I_{\Re^3} \;- \;\left( \tilde{\q} \otimes \tilde{\q} \right) }
\over {\| \q \|} }
\nonumber
\\
& = &
{{\| \q \|^2 \I_{\Re^3} \;- \;\left( \q \otimes \q \right) }
\over {\| \q \|^3} }
\nonumber
\end{eqnarray}

%-------------------------------------------------------------------------------

\subsubsection{Derivatives of angles}
\label{sec:derivatives-of-angles}

The angle between 2 vectors $\p_0, \p_1 \in \Re^m$, is the inverse cosine
of their normalized inner product:
\begin{equation}
\theta(\p_0,\p_1)
=
\cos^{-1}
\left(
{ \p_0 \bullet \p_1 } \over {\|\p_0\| \|\p_1\|}
\right)
\end{equation}
Recall that the derivative of the $\cos^{-1}$ is:
\begin{equation}
\frac{d}{\mathit dx} \cos^{-1}(x) = { -1 \over \sqrt{1 - x^2} }
\end{equation}
It follows that:
\begin{eqnarray}
\label{eq:angle_gradient}
\Gc{\p_0}{\theta(\p_0,\p_1)}{\q}
& = &
{{-1} \over
{ \sqrt{1 - \left( {\q_0 \bullet \q_1} \over {\| \q_0 \| \| \q_1 \|} \right)^2 }}}
\Gc{\p_0}{\left( {\q_0 \bullet \q_1} \over {\| \q_0 \| \| \q_1 \|} \right)} {\q}
\\
& = &
{
{-\|\q_0\|\|\q_1\|}
\over
{ \sqrt{\|\q_0\|^2\|\q_1\|^2 - \left( \q_0 \bullet \q_1 \right)^2 }}
}
\left[
{{\q_1} \over {\|\q_0\|\|\q_1\|}}
+
{{\left( \q_0 \bullet \q_1 \right)} \over {\| \q1 \|}}
\Gc{\p_0}{\left( {1} \over {\| \p_0 \|} \right)} {\q}
\right]
\nonumber
\\
& = &
{
{-\|\q_0\|\|\q_1\|}
\over
{ \sqrt{\|\q_0\|^2\|\q_1\|^2 - \left( \q_0 \bullet \q_1 \right)^2 }}
}
\left[
{{\q_1} \over {\|\q_0\|\|\q_1\|}}
-
{{\left( \q_0 \bullet \q_1 \right) \q0} \over {\| \q1 \| \|\q_0\|^3}}
\right]
\nonumber
\\
& = &
{
{-1}
\over
{ \sqrt{\|\q_0\|^2\|\q_1\|^2 - \left( \q_0 \bullet \q_1 \right)^2 }}
}
\left[
\q_1
-
{{\left( \q_0 \bullet \q_1 \right) \q0} \over {\|\q_0\|^2}}
\right]
\nonumber
\\
& = &
{
{- \q_1 \perp \q_0}
\over
{ \sqrt{\|\q_0\|^2\|\q_1\|^2 - \left( \q_0 \bullet \q_1 \right)^2 }}
}
\nonumber
\\
&  &
\nonumber
\\
\Gc{\p_1}{\theta(\p_0,\p_1)}{\q}
& = &
{
{- \q_0 \perp \q_1}
\over
{ \sqrt{\|\q_0\|^2\|\q_1\|^2 - \left( \q_0 \bullet \q_1 \right)^2 }}
}
\nonumber
\end{eqnarray}
