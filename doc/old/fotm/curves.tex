\subsection{Vertex Bends}
\label{sec:Vertex-Bends}

\begin{figure}[!htp]
\centering
\begin{verbatim}
                   e1
             p1 o--------o p0
                        a \
                           \ e2
                            \
                             o p2
\end{verbatim}
\caption{Polyline vertex neighborhood.
\label{fig:Polyline-vertex-neighborhood}}
\end{figure}

In this section, I disuss measures of curvature
based on the bending at each vertex in a polyline.

%------------------------------------------------------------------

\subsubsection{Cosine}
\label{sec:polyline-vertex-cosine}

Consider the non-boundary vertex
at point $\p_0 \in \Reals^3$ in figure \ref{fig:Polyline-vertex-neighborhood}.
$\v$ has degree $2$;
the incident edges are labeled $\e_1$ and $e_2$;
and the neighboring vertices are at points $\p_1 \in \Reals^3$
and $\p_2 \in \Reals^3$.
The shape of the neighborhood is determined by
$\p = (\p_0, \p_1, \p_2) \in \Reals^9$
The unsigned angle between edges $\e_1$ and $\e_2$ is $\alpha(\p)$.

One measure of the amount of bending is the cosine of $\alpha$:
\begin{equation}
\cos(\alpha(\p)) =
{{(\p_1 - \p_0)} \over {\| \p_1 - \p_0 \|} }
\bullet
{{(\p_2 - \p_0)} \over {\| \p_2 - \p_0 \|} }
\end{equation}

A little calculus shows that the partial gradients are:
\begin{eqnarray}
\label{eq:polyline-vertex-cosine-gradient}
\Gf{\p_0}{\cos(\alpha(\p))}
& = &
-
\left[
{{(\p_1 - \p_0) \perp  (\p_2 - \p_0)}
\over
{\| \p_1 - \p_0 \| \| \p_2 - \p_0 \|} }
+
{{(\p_2 - \p_0) \perp  (\p_1 - \p_0)}
\over
{\| \p_1 - \p_0 \| \| \p_2 - \p_0 \|} }
\right]
\\
\Gf{\p_1}{\cos(\alpha(\p))}
& = &
{{(\p_2 - \p_0) \perp  (\p_1 - \p_0)}
\over
{\| \p_1 - \p_0 \| \| \p_2 - \p_0 \|} }
\nonumber
\\
\Gf{\p_2}{\cos(\alpha(\p))}
& = &
{{(\p_1 - \p_0) \perp  (\p_2 - \p_0)}
\over
{\| \p_1 - \p_0 \| \| \p_2 - \p_0 \|} }
\nonumber
\end{eqnarray}

%------------------------------------------------------------------

\subsubsection{Squared Cosine}
\label{sec:polyline-vertex-squared-cosine}

For any positive measure of bending,
minimizing the sum of squared bends,
rather than simply the sum of bends,
will tend to produce a more even distribution of curvature.
Since $\cos(\alpha)$ ranges over $[-1,1]$,
we use $\left( \frac{1 + \cos(\alpha)}{2} \right)^2$.

The gradient is:
\begin{equation}
\Gf{\p}{\left( {{1 + \cos(\alpha(\p))} \over 2} \right)^2}
=
\left( 1 + \cos(\alpha(\p)) \right)
\Gf{\p}{\cos\left(\alpha(\p)\right)}
\end{equation}

$\Gf{\p}{\cos\left(\alpha(\p)\right)}$ is given
in equation \ref{eq:polyline-vertex-cosine-gradient}.

%------------------------------------------------------------------

\subsubsection{Angle}
\label{sec:polyline-vertex-angle}


The angle $\alpha$ is $\pi$ for a straight polyline,
and $0$ for a maximally bent vertex.
Therefore we may choose to minimize the sum of negative angles:
\begin{equation}
-\alpha(\p) =
-\cos^{-1} \left(
{{(\p_1 - \p_0)} \over {\| \p_1 - \p_0 \|} }
\bullet
{{(\p_2 - \p_0)} \over {\| \p_2 - \p_0 \|} }
\right)
\end{equation}

The gradient is:
\begin{equation}
\Gf{\p}{-\alpha(\p)}
=
{{1} \over {\sqrt{ 1 - \cos(\alpha(\p))^2}}}
\Gf{\p}{\cos\left(\alpha(\p)\right)}
\end{equation}

$\Gf{\p}{\cos\left(\alpha(\p)\right)}$ is given
in equation \ref{eq:polyline-vertex-cosine-gradient}.

%------------------------------------------------------------------

\subsubsection{Squared Angle}
\label{sec:polyline-vertex-squared-angle}

Since $-\alpha$ ranges over $[-\pi,0]$
we square $\pi - \alpha$.

The gradient is:
\begin{equation}
\Gf{\p}{\left( \pi - \alpha(\p) \right)^2}
=
{{2 \left( \pi - \alpha(\p) \right)}
\over
{\sqrt{ 1 - \cos(\alpha(\p))^2}}}
\Gf{\p}{\cos\left(\alpha(\p)\right)}
\end{equation}

$\Gf{\p}{\cos \left( \alpha(\p) \right)}$ is given
in equation \ref{eq:polyline-vertex-cosine-gradient}.

