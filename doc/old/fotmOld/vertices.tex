\subsection{Functions of the vertices}
\label{sec:vertices}

%-----------------------------------------------------------------

\subsubsection{Vertex surround angle}
\label{sec:vertex_surround_angle}

\paragraph{Non-boundary case}
\label{sec:non_boundary_vertex_surround_angle}

\begin{figure}[!htp]
\centering
\begin{verbatim}
                v2/e2 o           o v1/e1
                       \   f1    /
                        \       /
                     f2  \ g1  /   f0
                          \ ^ /
                       g2( \ / )g0
             v3/e3 o--------o--------o v0/e0
                       g3( / \ )gn-1
                          / v \
                     f3  /     \  fn-1
                        /       \
                 v4/e4 o   ...   o vn-1/en-1
\end{verbatim}
\caption{Vertex surround angle.
\label{fig:vertex_surround_angle}}
\end{figure}

Consider the non-boundary vertex
labeled $\v$ in figure \ref{fig:vertex_surround_angle}.
$\v$ has degree $n$;
the incident edges are labeled $\e_0, \e_1, \ldots, \e_{n-1}$;
the incident faces are labeled $\f_0, \f_1, \ldots, \f_{n-1}$;
and the neighboring vertices are labeled $\v_0, \v_1, \ldots, \v_{n-1}$.
The angle between edges $\e_j$ and $\e_{j + 1 \bmod n}$ is $\gamma_j$.

$\gamma_j = \gamma(\f_j,\v)$ is the {\em corner angle} in face $\f_j$
of vertex $\v$.

The {\em surround angle} of $\v$ is the sum of all its corner angles:
\begin{equation}
\alpha(\v) = \sum_{j=0}^{n-1} \gamma(\f_j,\v),
\end{equation}

If the neighborhood of $\v$ is flat, then $\alpha(\v)=2\pi$.
A neighborhood with $\alpha(\v) < 2\pi$, corresponds very roughly
to positive Gaussian curvature,
and, $\alpha(\v) > 2\pi$,
is analogous to negative Gaussian curvature.
However, it's important to remember that, unlike the case
with smooth surfaces, where the most complicated local
shape is a saddle,
the crinkling of a vertex neighborhood can be arbitrarily complicated,
for any non-zero surround angle.

Although a mesh with all surround angles of $2\pi$ is not guaranteed
to be flat,
a mesh with surround angles different from $2\pi$ is guaranteed
to be not-flat,
and, in some sense,
the more different the surround angles are from $2\pi$
the more non-flat is the mesh.
This leads to the following as a mesh roughness measure:
\begin{eqnarray}
f(\M)
& = & \sum_{\v \in \V(\M)} \left[ \alpha(\v) - 2\pi \right]^2
\\
& = & \sum_{\v \in \V(\M)}
\left[ \left(\sum_{\f \in \F(\v)} \gamma(\f,\v)\right)
 - 2\pi \right]^2
\nonumber
\end{eqnarray}
The natural implementation of this function is to first compute
all the corner angles and cache them with the faces,
then compute the surround angles and cache with each vertex,
and then do a lookup while computing the sum.

In computing the gradient, it's simplest to re-arrange the
order of summation:
\begin{eqnarray}
\Ga{f}
& = & \Ga{ \left[ \sum_{\v \in \V(\M)} \left[ \alpha(\v) - 2\pi \right]^2 \right]}
\\
& = & 2 \sum_{\v \in \V(\M)} \left[ \alpha(\v) - 2\pi \right] \Ga{\alpha(\v)}
\nonumber
\\
& = & 2 \sum_{\v \in \V(\M)} \left[ \alpha(\v) - 2\pi \right]
\sum_{\f \in \F(\v)} \Ga{\gamma(\f,\v)}
\nonumber
\\
& = & 2 \sum_{\v \in \V(\M)}
\sum_{\f \in \F(\v)}
\left[ \alpha(\v) - 2\pi \right]
\Ga{\gamma(\f,\v)}
\nonumber
\\
& = & 2
\sum_{\f \in \F(\M)}
\sum_{\v \in \V(\f)}
\left[ \alpha(\v) - 2\pi \right]
\Ga{\gamma(\f,\v)}
\nonumber
\\
& = & 2
\sum_{\f \in \F(\M)}
\sum_{i=0}^2
\left[ \alpha(\v_i(\f)) - 2\pi \right]
\Ga{\gamma(\p_i(\f),\p_{i+1 \% 2}(\f),\p_{i+2 \% 3}(\f))}
\nonumber
\end{eqnarray}

Consider the partial gradient with respect to
location of one of the vertices:
\begin{eqnarray}
\label{eq:surround_angle_partial_gradient}
\Gc{\p_j}{f}{q}
& = 2 \sum_{\f \in \F(\M)} &
\sum_{i=0}^2
\left[ \alpha(\v_i(\f)) - 2\pi \right]
\Gc{\p_j}{\gamma(\p_i(\f),\p_{i+1 \% 2}(\f),\p_{i+2 \% 3}(\f))}{\q}
\\
& = 2 \sum_{\f \in \F(\v_j)} &
\sum_{i=0}^2
\left[ \alpha(\v_i(\f)) - 2\pi \right]
\Gc{\p_j}{\gamma(\p_i(\f),\p_{i+1 \% 2}(\f),\p_{i+2 \% 3}(\f))}{\q}
\nonumber
\\
& = 2 \sum_{\f \in \F(\v_j)} & \left(
\left[ \alpha({\mathrm p}(\v_j,\f) - 2\pi \right]
\Gc{\p_j}{\gamma({\mathrm p}(\p_j,\f),\p_j,{\mathrm s}(\p_j,\f))}{\q}
\right.
\nonumber
\\
& & +
\left[ \alpha(\v_j) - 2\pi \right]
\Gc{\p_j}{\gamma(\p_j,{\mathrm s}(\p_j,\f),{\mathrm p}(\p_j,\f)}{\q}
\nonumber
\\
& & +
\left.
\left[ \alpha({\mathrm s}(\v_j,\f) - 2\pi \right]
\Gc{\p_j}{\gamma({\mathrm s}(\p_j,\f),{\mathrm p}(\p_j,\f),\p_j)}{\q}
\right)
\nonumber
\end{eqnarray}
where ${\mathrm p}(\v,\f)$ is $\v$'s {\em predecessor},
the vertex that comes before $\v$ in the oriented face $\f$,
${\mathrm s}(\v_j,\f)$ the {\em successor}
the vertex that comes after $\v$,
and similarly for the vertex positions $\p$.

Consider the term corresponding to a particular face $\f$ in
the sum in equation \ref{eq:surround_angle_partial_gradient}.
Call that face's vertices and positions $\v_0, \v_1, \v_2$
and $\p_0, \p_1, \p_2$.
\begin{eqnarray}
\Gc{\p_0}{f_{\f}}{q}
& = &
\left[ \alpha(\v_0) - 2\pi \right] \Gc{\p_0}{\gamma(\p_0,\p_1,\p_2)}{\q}
\\
& + &
\left[ \alpha(\v_1) - 2\pi \right] \Gc{\p_0}{\gamma(\p_1,\p_2,\p_0)}{\q}
\nonumber
\\
& + &
\left[ \alpha(\v_2) - 2\pi \right] \Gc{\p_0}{\gamma(\p_2,\p_0,\p_1)}{\q}
\nonumber
\\
&  &
\nonumber
\\
& = &
\left[ \alpha(\v_0) - 2\pi \right]
{{(\q_1 - \q_0) \perp (\q_2 - \q_0) + (\q_2 - \q_0) \perp (\q_1 - \q_0)}
\over
{
\sqrt{
\| \q_2 - \q_0 \|^2 \| \q_1 - \q_0 \|^2
-
\left( ( \q_1 -\q_0 ) \bullet ( \q_2 -\q_0 ) \right)^2
}
}}
\nonumber
\\
& - &
\left[ \alpha(\v_1) - 2\pi \right]
{{(\q_0 - \q_1) \perp (\q_2 - \q_1)}
\over
{
\sqrt{
\| \q_0 - \q_1 \|^2 \| \q_2 - \q_1 \|^2
-
\left( ( \q_0 -\q_1 ) \bullet ( \q_2 - \q_1 ) \right)^2
}
}}
\nonumber
\\
& - &
\left[ \alpha(\v_2) - 2\pi \right]
{{(\q_0 - \q_2) \perp (\q_1 - \q_2)}
\over
{
\sqrt{
\| \q_0 - \q_2 \|^2 \| \q_1 - \q_2 \|^2
-
\left( ( \q_0 -\q_2 ) \bullet ( \q_1 -\q_2 ) \right)^2
}
} }
\nonumber
\end{eqnarray}
