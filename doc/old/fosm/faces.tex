\begin{plSection}{Functions of faces}
\label{sec:faces}

\begin{plDiagram}
{Face labeling.}
{FaceLabeling}
\centering
\begin{verbatim}

          V2/p2
            o
           /U\
          / g2\
     E20 /     \ E12
        / F012  \
       /)g0   g1(\
V0/p0 o-----------o V1/p1
           E01
\end{verbatim}
\end{plDiagram}

%-----------------------------------------------------------------
\begin{plSection}{Corner angles}
\label{sec:corner_angles}

Suppose face $\Face{f}_{012}$ has vertices $\Vertex{v}_0, \Vertex{v}_1, \Vertex{v}_2$,
at points $\Point{p}_0, \Point{p}_1, \Point{p}_2$,
and edges $\Edge{e}_{01}, \Edge{e}_{12}, \Edge{e}_{20}$,
as labeled in \cref{diagram:FaceLabeling}.

The {\em corner angle} $\gamma_i$ in face $\Face{f}_{012}$ of vertex $\Vertex{v}_i$ is
the angle between the two edges in $\Face{f}_{012}$ that meet at $\Vertex{v}_i$:
\begin{eqnarray}
\gamma_i
& = & \gamma(\Face{f}_{012},\Vertex{v}_i)
\\
& = & \gamma(\Point{p}_i,\Point{p}_{(i+1) \% 3},\Point{p}_{(i+2) \% 3})
\nonumber
\\
& = & \theta(\Point{p}_{(i+1) \% 3} - \Point{p}_i,\Point{p}_{(i+2) \% 3} - \Point{p}_i)
\nonumber
\\
& = &
\cos^{-1}
\left[
\frac{ \left( \Point{p}_{(i+1) \% 3} - \Point{p}_i \right)
  \bullet
  \left( \Point{p}_{(i+2) \% 3} - \Point{p}_i \right) }
{ \| \Point{p}_{(i+1) \% 3} - \Point{p}_i \|
  \| \Point{p}_{(i+2) \% 3} - \Point{p}_i \| }
\right]
\nonumber
\end{eqnarray}

Corner angles vary between $0$ and $\pi$, with both extremes
corresponding to singular, zero-area faces.
The sum of corner angles for a given face is always $\pi$.

Using \cref{eq:angle_gradient},
it's easy to see that the partial gradients are:
\begin{eqnarray}
\Gc{\Point{p}_0}{\gamma(\Point{p}_0,\Point{p}_1,\Point{p}_2)}{\Point{q}}
& = &
\frac{
\left[ (\Point{q}_1 -\Point{q}_0) \perp (\Point{q}_2 - \Point{q}_0) \right]
+
\left[ (\Point{q}_2 -\Point{q}_0) \perp (\Point{q}_1 - \Point{q}_0) \right]
}
{ \sqrt{\|\Point{q}_1 - \Point{q}_0\|^2\|\Point{q}_2 - \Point{q}_0\|^2 -
\left( (\Point{q}_1 - \Point{q}_0) \bullet (\Point{q}_2 - \Point{q}_0) \right)^2 }}
\\
\Gc{\Point{p}_1}{\gamma(\Point{p}_0,\Point{p}_1,\Point{p}_2)}{\Point{q}}
& = &
\frac{-(\Point{q}_1 -\Point{q}_0) \perp (\Point{q}_2 - \Point{q}_0)}
{ \sqrt{\|\Point{q}_1 - \Point{q}_0\|^2\|\Point{q}_2 - \Point{q}_0\|^2 -
\left( (\Point{q}_1 - \Point{q}_0) \bullet (\Point{q}_2 - \Point{q}_0) \right)^2 }}
\nonumber
\\
\Gc{\Point{p}_2}{\gamma(\Point{p}_0,\Point{p}_1,\Point{p}_2)}{\Point{q}}
& = &
\frac{-(\Point{q}_2 -\Point{q}_0) \perp (\Point{q}_1 - \Point{q}_0)}
{ \sqrt{\|\Point{q}_1 - \Point{q}_0\|^2\|\Point{q}_2 - \Point{q}_0\|^2 -
\left( (\Point{q}_1 - \Point{q}_0) \bullet (\Point{q}_2 - \Point{q}_0) \right)^2 }}
\nonumber
\end{eqnarray}

\end{plSection}%{Corner angles}
%-----------------------------------------------------------------
\begin{plSection}{Functions of face normals}
\label{sec:normals}

A number of important functions of triangular meshes,
such as surface area,
are based on face normal vectors.

%-----------------------------------------------------------------
\begin{plSection}{Area-weighted face normal}
\label{sec:areanormal}

Suppose we have a face whose 3 vertices are at $\Point{p} = (\Point{p}_0, \Point{p}_1, \Point{p}_2)$,
where $\Point{p}_i \in \Reals^3; i=0,1,2.$
(Note that the order of the $\Point{p}_i$ determines the orientation of the face.
With a face as labeled in \cref{diagram:FaceLabeling},
the normal points out of the page.)

The {\it area-weighted normal} vector is
\Normal{n}opagebreak
\begin{eqnarray}
\Normal{a} (\Point{p}) & = & (\Point{p}_0 \times \Point{p}_1) \ + \ (\Point{p}_1 \times \Point{p}_2) \ + \ (\Point{p}_2 \times \Point{p}_0) \\
        & = & (\Point{p}_1 - \Point{p}_0) \ \times \ (\Point{p}_2 - \Point{p}_0) \nonumber \\
        & = & (\Point{p}_2 - \Point{p}_1) \ \times \ (\Point{p}_0 - \Point{p}_1) \nonumber \\
        & = & (\Point{p}_0 - \Point{p}_2) \ \times \ (\Point{p}_1 - \Point{p}_2) \nonumber
\end{eqnarray}

The 'partial' derivatives of the area-weighted normal are:
\begin{eqnarray}
\Dd{\Point{p}_0}{\Normal{a}}{\Point{q}}{\Point{r}_0} \
& = \ (\Point{r}_0 \times \Point{q}_1) \ + \ (\Point{q}_2 \times \Point{r}_0) & = (\Point{q}_2 - \Point{q}_1) \times \Point{r}_0 \\
\Dd{\Point{p}_1}{\Normal{a}}{\Point{q}}{\Point{r}_1} \
& = \ (\Point{r}_1 \times \Point{q}_2) \ + \ (\Point{q}_0 \times \Point{r}_1) & = (\Point{q}_0 - \Point{q}_2) \times \Point{r}_1 \nonumber \\
\Dd{\Point{p}_2}{\Normal{a}}{\Point{q}}{\Point{r}_2} \
& = \ (\Point{r}_2 \times \Point{q}_0) \ + \ (\Point{q}_1 \times \Point{r}_2) & = (\Point{q}_1 - \Point{q}_0) \times \Point{r}_2 \nonumber
\end{eqnarray}

Note that $\Df{\Point{p}}{\Normal{a}}$ is {\it skew-symmetric}, that is,
$\Df{\Point{p}}{\Normal{a}}^{\dagger} = -\Df{\Point{p}}{\Normal{a}}.$

\end{plSection}%{Area-weighted face normal}
%-----------------------------------------------------------------
\begin{plSection}{Face area}
\label{sec:facearea}

The area of a face is half the length of the area-weighted normal:
\begin{eqnarray}
A(\Point{p})
& = & \frac{1}{2} \| \ \Normal{a}(\Point{p}) \ \|  \\
& = & \frac{1}{2} \| \ (\Point{p}_0 \times \Point{p}_1) \ + \ (\Point{p}_1 \times \Point{p}_2) \ + \ (\Point{p}_2 \times \Point{p}_0) \ \|.
\nonumber
\end{eqnarray}

It follows from \cref{eq:norm_derivative}
that the first 'partial' derivative of the face area is:
\begin{eqnarray}
\label{eq:area_partial_derivative}
\Dd{\Point{p}_0}{A}{\Point{q}}{\Point{r}_0}
& = &
\frac{\Normal{a}(\Point{q})^\dagger}{2\|\Normal{a}(\Point{q})\|}
{\Dd{\Point{p}_0}{\Normal{a}}{\Point{q}}{\Point{r}_0}}  \\
& = &
\frac{\Normal{a}(\Point{q})}{2\|\Normal{a}(\Point{q})\|}
\bullet
\left[(\Point{r}_0 \times \Point{q}_1) + (\Point{q}_2 \times \Point{r}_0)\right] \nonumber \\
& = &
\frac{\Normal{a}(\Point{q})}{2\|\Normal{a}(\Point{q})\|}
\bullet
\left[(\Point{q}_2 - \Point{q}_1) \ \times \  \Point{r}_0)\right] \nonumber \\
& = &
\frac{\Normal{a}(\Point{q}) \times (\Point{q}_2 - \Point{q}_1)}{2\|\Normal{a}(\Point{q})\|}
\bullet
\Point{r}_0 \nonumber
\end{eqnarray}

The last identity follows from \cref{eq:dot_cross}.

The 'partial' gradients of the face area are then:
\begin{eqnarray}
\Gc{\Point{p}_0}{A}{\Point{q}} & = & \frac{\Normal{a}(\Point{q}) \times (\Point{q}_2 - \Point{q}_1)}{2\|\Normal{a}(\Point{q})\|} \\
\Gc{\Point{p}_1}{A}{\Point{q}} & = & \frac{\Normal{a}(\Point{q}) \times (\Point{q}_0 - \Point{q}_2)}{2\|\Normal{a}(\Point{q})\|} \nonumber \\
\Gc{\Point{p}_2}{A}{\Point{q}} & = & \frac{\Normal{a}(\Point{q}) \times (\Point{q}_1 - \Point{q}_0)}{2\|\Normal{a}(\Point{q})\|} \nonumber
\end{eqnarray}

More simply, using the face unit normal \( \Normal{n}(\Point{p})  =  \frac{\Normal{a}(\Point{p})}{\| \Normal{a}(\Point{p}) \|} \):
\begin{eqnarray}
\label{eq:area_gradient}
\Gc{\Point{p}_0}{A}{\Point{q}} & = & \frac{\Normal{n}(\Point{q})}{2} \times (\Point{q}_2 - \Point{q}_1) \\
\Gc{\Point{p}_1}{A}{\Point{q}} & = & \frac{\Normal{n}(\Point{q})}{2} \times (\Point{q}_0 - \Point{q}_2) \nonumber \\
\Gc{\Point{p}_2}{A}{\Point{q}} & = & \frac{\Normal{n}(\Point{q})}{2} \times (\Point{q}_1 - \Point{q}_0) \nonumber
\end{eqnarray}

\end{plSection}%{Face area}
%-----------------------------------------------------------------
\begin{plSection}{Face unit normal vector}
\label{sec:facenormal}

The unit vector normal to a face whose vertices are at
$\Point{p} = (\Point{p}_0,\Point{p}_1,\Point{p}_2)$ is just the area weighted normal (see \cref{sec:areanormal})
adjusted to length 1:
\begin{equation}
\Normal{n}(\Point{p})  =  \frac{\Normal{a}(\Point{p})}{\| \Normal{a}(\Point{p}) \|}
\end{equation}

Following \cref{eq:normalized_function_derivative}, the derivative is:

\begin{eqnarray}
\label{eq:unit_normal_derivative}
\Dc{\Normal{n}}{\Point{p}}{\Point{q}}
&  =
& \frac{ \left( {\| \Normal{a}(\Point{p}) \|^2 \Identity_{\Reals^3}  -  \Normal{a}(\Point{p}) \otimes \Normal{a}(\Point{p}) }
{\| \Normal{a}(\Point{p}) \|^3} \right) }
\; \Dc{\Normal{a}}{\Point{p}}{\Point{q}}
 \\
& & \nonumber\\
&  =
& \frac{ \left( {\| \Normal{a}(\Point{p}) \|^2 \Identity_{\Reals^3}  -  \Normal{a}(\Point{p}) \otimes \Normal{a}(\Point{p}) }
{\| \Normal{a}(\Point{p}) \|^3} \right)} \ast
\nonumber \\
&    &
\left[ \left( \Point{q}_0 \times \Point{p}_1 \right) + \left( \Point{p}_2 \times \Point{q}_0 \right)
+
\left( \Point{q}_1 \times \Point{p}_2 \right) + \left( \Point{p}_0 \times \Point{q}_1 \right)
+
\left( \Point{q}_2 \times \Point{p}_0 \right) + \left( \Point{p}_1 \times \Point{q}_2 \right) \right]
\nonumber \\
& & \nonumber\\
&  =
& \frac{ \left( {\| \Normal{a}(\Point{p}) \|^2 \Identity_{\Reals^3}  -  \Normal{a}(\Point{p}) \otimes \Normal{a}(\Point{p}) }{\| \Normal{a}(\Point{p}) \|^3} \right)} \ast
\nonumber \\
&    &
\left[ \left( (\Point{p}_2 - \Point{p}_1) \times \Point{q}_0 \right)
+
\left( (\Point{p}_0 - \Point{p}_2) \times \Point{q}_1 \right)
+
\left( (\Point{p}_1 - \Point{p}_0) \times \Point{q}_2 \right) \right]
\nonumber \\
& & \nonumber\\
&  =
& { \left( {\Identity_{\Reals^3}  -  \Normal{n}(\Point{p}) \otimes \Normal{n}(\Point{p}) } \over {\| \Normal{a}(\Point{p}) \|} \right)} \ast \; \Dc{\Normal{a}}{\Point{p}}{\Point{q}}
\nonumber
\end{eqnarray}

\end{plSection}%{Face unit normal vector}
%-----------------------------------------------------------------
\end{plSection}%{Functions of face normals}
%-----------------------------------------------------------------
\begin{plSection}{Aspect Ratio}
\label{sec:aspect_ratio}
Minimizing a measure of face aspect ratio can help maintain
a well-conditioned mesh.
Maximizing it may help in discovering collapse-able edges.
%-----------------------------------------------------------------
\begin{plSection}{Squared edge lengths over area}
\label{sec:Squared-edge-lengths-over-area}

One measure of the deviation of a face from equilaterality is:
\begin{equation}
{\mathrm L2A}(\Point{p}_0,\Point{p}_1,\Point{p}_2)
=
\frac{ \| \Point{p}_0 - \Point{p}_1 \|^2 + \| \Point{p}_1 - \Point{p}_2 \|^2 + \| \Point{p}_2 - \Point{p}_0 \|^2 }
{A(\Point{p})}
\end{equation}

Using \cref{eq:area_gradient}, it follows that the
partial gradients of L2A are:
\begin{eqnarray}
\label{eq:L2A_gradient}
\Gc{\Point{p}_0}{L2A}{\Point{q}}
& =
&
\left(
\frac{2\left[ \left( \Point{p}_0 - \Point{p}_1 \right) + \left( \Point{p}_0 - \Point{p}_2 \right) \right]}
{A(\Point{q})}
\right)
\\
& - &
\left(
\frac{ \| \Point{p}_0 - \Point{p}_1 \|^2 + \| \Point{p}_1 - \Point{p}_2 \|^2 + \| \Point{p}_2 - \Point{p}_0 \|^2 }
{2 A^2(\Point{q})}
\left[ \Normal{n}(\Point{q}) \times (\Point{q}_2 - \Point{q}_1)
\right]
\right)
\nonumber \\
\Gc{\Point{p}_1}{L2A}{\Point{q}}
& =
&
\left(
\frac{2\left[ \left( \Point{p}_1 - \Point{p}_2 \right) + \left( \Point{p}_1 - \Point{p}_0 \right) \right]}
{A(\Point{q})}
\right)
\nonumber
\\
& - &
\left(
\frac{ \| \Point{p}_0 - \Point{p}_1 \|^2 + \| \Point{p}_1 - \Point{p}_2 \|^2 + \| \Point{p}_2 - \Point{p}_0 \|^2 }
{2 A^2(\Point{q})}
\left[ \Normal{n}(\Point{q}) \times (\Point{q}_0 - \Point{q}_2) \right]
\right)
\nonumber
\\
\Gc{\Point{p}_2}{L2A}{\Point{q}}
& =
&
\left(
\frac{2\left[ \left( \Point{p}_2 - \Point{p}_0 \right) + \left( \Point{p}_2 - \Point{p}_1 \right) \right]}
{A(\Point{q})}
\right)
\nonumber
\\
& - &
\left(
\frac{ \| \Point{p}_0 - \Point{p}_1 \|^2 + \| \Point{p}_1 - \Point{p}_2 \|^2 + \| \Point{p}_2 - \Point{p}_0 \|^2 }
{2 A^2(\Point{q})}
\left[ \Normal{n}(\Point{q}) \times (\Point{q}_1 - \Point{q}_0) \right]
\right)
\nonumber
\end{eqnarray}
\end{plSection}%{Squared edge lengths over area}
%-----------------------------------------------------------------
\end{plSection}%{Aspect Ratio}
%-----------------------------------------------------------------
\end{plSection}%{Functions of faces}
