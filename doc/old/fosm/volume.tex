% jam 2004-08-28

\section{Simplex Volume}
\label{sec:simplex_volume}

\nocite{Heckbert:1974:GraphicsGems4}
\nocite{GoodmanORourke:2004:HDCG2}

This section describes how to compute
the volume of a $m$-simplex in $\Reals^n$ (with $m \le n$)
and its derivatives with respect to the vertex positions.
In what follows, let $S$ be the geometric $m$-simplex
spanned by the $m+1$ points $\{\p_0 , \ldots , \p_m\}$,
where $\p_i \in \Reals^n$.
As in \autoref{sec:barycentric-coordinates},
let $\v_i = (\p_i - \p_m); i = 0 \ldots (m-1)$
and ${\mathbf V} = \sum_{i=0}^{m-1} \v_i \otimes \e_i$
(which can be represented by a matrix whose columns are the $\v_i$).

To simplify the discussion, I will describe computing the
value of the volume form: $\rho(S) = m! \volume(S)$,
which is the volume of the parallelpiped with sides $\v_i$.

Most discussions of simplex volume consider
only the case of a $n$-simplex in $\Reals^n$.
In that case, the $\rho(S) = \det({\mathbf V)}$
\cite{HenkRichterGebertZiegler:2004:ConvexPolytopes}.

When $m < n$, ${\mathbf V}$ is no longer square,
so $\det({\mathbf V})$ doesn't work.
Hanson \cite{Hanson:1994:NdGraphics} suggests:
\begin{equation}
\rho(S)^2 = \det({\mathbf V}^{\dagger} {\mathbf V})
\end{equation}

