% jam 2004-09-10

\section{Classifying projected simplexes}
\label{sec:classifying}

%--------------------------------------------------------------------

The convex hull of a $(m-1)$ dimensional projection of a $m$-simplex
is either a $(m-1)$-simplex or a $(m-1)$ dimensional cross polytope.

\begin{Lemma}
\label{rambau-lemma}
Any set $Z$ of $(m+2)$ points whose convex hull is of dimension $m$
has exactly two triangulations denoted $T_{Z^+}$ and $T_{Z^-}$.
\end{Lemma}

See Rambau~\cite[Lemma~1.1.2]{rambau-jorg-1996}.

Let $S=(\p_0, \p_1, \ldots , \p_m)$ be an $m$-simplex in $\Reals^{n}$.
Let $\pi$ be a projection from $\Reals^{n}$ to $Q$, an $(m-1)$-dimensional
affine subspace of $\Reals^{n}$.
Assume $\pi$ is chosen so that the points
$\{\pi \p_0, \pi \p_1, \ldots , \pi \p_m\}$ are in {\it general position},
that is, any $l+1$ of the projected points spans an $l$-dimensional
affine subspace of $Q$.

By Lemma \ref{rambau-lemma},
there are two exactly triangulations of the convex hull of the projected points.
The $(m-1)$-simplexes of the triangulations are images of the $(m-1)$-simplexes of $S$,
and two triangulations correspond to a partition of $(m-1)$-simplexes of $S$
into two subsets, the "top" and "bottom" of $S$ with respect to $\pi$.

To see why this is true, and to further classify the triangulations,
consider the fact that the boundary of the convex hull of $\pi S$
must contain either $m$ or $m+1$ of the $\pi \p_i$.
(Any fewer and the points cannot be in general position.)

\begin{Theorem}
\label{one-simplex-case}
If the boundary of the convex hull of $\pi S$
contains $m$ of the $\pi \p_i$,
then it is a $(m-1)$-simplex
and the image of one of the $(m-1)$-simplexes, $F$, in $S$.
The first triangulation is just $\pi F$.
The second triangulation consists of the images of
all the remaining $(m-1)$-simplexes of $S$.
The second triangulation is itself the mutual refinement of both.
\end{Theorem}

The first 2 statements are obvious.
If we label the vertices so that $\{\pi \p_1, \ldots , \pi \p_m\}$
are on the boundary, and $\pi \p_0$ is in the interior,
then the second triangulation results from refining the first
triangulation {\it pulling}
\cite{lee-hdcg-17-2004} the vertex $\pi \p_0$.
The faces of the triangulation formed by pulling
are the images of the $m$ $(m-1)$-simplexes
of $S$ that contain $\p_0$, that is, all the $(m-1)$-simplexes in $S$,
other than $(\p_1, \ldots , \pi \p_m)$.

\begin{Theorem}
\label{two-simplex-case}
If the boundary of the convex hull of $\pi S$
contains all $m+1$ of the $\pi \p_i$,
then it is the image of 2 of the $(m-1)$-simplexes in $S$,
which share a common $(m-2)$-simplex.
These 2 simplexes are the first triangulation.
The second triangulation consists of the images
of the remaining $m-1$ $(m-1)$-simplexes of $S$,
which share a common $1$-simplex.
The mutual refinement is formed by splitting either
the shared $(m-2)$-simplex in the first triangulation
or the shared $1$-simplex in the second.
\end{Theorem}

\subsection{Implementation}

Let $\Vspace$ be a $d$-dimensional real inner product space.
Let $\p() : \Integers^{+} \mapsto \Vspace,$
so that $\p(\Ssimplex)$ is a geometric realization of the $d$-simplex $\Ssimplex$.
Let $\Projection_{\Aspace}$ be the orthogonal projection onto
$\Aspace$, a $(d-1)$-dimensional affine subspace of $\Vspace$.
Assume that the $\affine_span \left( \p(\Ssimplex) \right)$
is $(d-1)$-dimensional.

Let $\Kcomplex_{d-1}$ be an oriented version of the $(d-1)$-skeleton of $\Ssimplex$.

By the results above, the oriented $(d-1)$-simplices,
$\{ \Ffacet_0 \ldots \Ffacet_d \}$ in $\Kcomplex_{d-1}$
can be partitioned into two subsets, $\Kcomplex_{d-1}^{+}$
and $\Kcomplex_{d-1}^{-}$, such that
$\Projection_{\Aspace} \left( \p ( \Kcomplex_{d-1}^{+} ) \right)$
and
$\Projection_{\Aspace} \left( \p ( \Kcomplex_{d-1}^{-} ) \right)$
are each triangulations of
$\convex_span \left( \Projection_{\Aspace} \left( \p ( \Ssimplex ) \right) \right)$.
The two subsets can be determined by computng the signed
volumes of the realizations of the oriented facets,
with the positive volumes forming one triangulation and the negative the other.

What I want to do is to create a mutual refinement of the two subsets,
$\Kcomplex_{d-1}^{1}$,
which triangulates
$\convex_span \left( \Projection_{\Aspace} \left( \p ( \Ssimplex ) \right) \right)$.
I then want to split and collapse simplices in any complex containing $\Ssimplex$
so that the $d$-simplex $\Ssimplex$ is replaced by the $(d-1)$-dimensional
complex $\Kcomplex_{d-1}^{1}$.

There are two cases:
\begin{enumerate}

\item $\convex_span \left( \Projection_{\Aspace} \left( \p ( \Ssimplex ) \right) \right)$
has $d$-corners, which are the projected realizations of $d$
of the vertices of $\Ssimplex$.
The remaining vertex, $\Vvertex_0$,
is such that
$\Projection_{\Aspace} \left( \p ( \Vvertex_0 ) \right)$,
can be expressed
as a convex combination of the projected realizations of the corner vertices.

In this case, one of $\Kcomplex_{d-1}^{\pm}$ is a single facet $\Ffacet_d$
and the other contains the remaining $d$ facets $\{\Ffacet_0 \ldots \Ffacet_{d-1} \}$,
which share $\Vvertex_0$.
To get the mutual refinement, we split $\Ffacet_d$ around a new vertex $\Vvertex_1$.
To reduce the split children of $\Ssimplex$ to $\Kcomplex_{d-1}^{1}$,
we collapse the edge $\{ \Vvertex_0 , \Vvertex_1 \}$.

Note that the number of vertices in any complex containing $\Ssimplex$
is unchanged.

\item All the $\Projection_{\Aspace} \left( \p ( \Vvertex_i ) \right) $
are corners of
$\convex_span \left( \Projection_{\Aspace} \left( \p ( \Ssimplex ) \right) \right)$.

In this case, one of $\Kcomplex_{d-1}^{\pm}$
contains two facets, $\Ffacet_{d-1}$ and $\Ffacet_{d}$,
and the other contains the remaining $(d-1)$ facets, $\{\Ffacet_0 \ldots \Ffacet_{d-2} \}$.
$\Ffacet_{d-1}$ and $\Ffacet_{d}$ share a common $(d-2)$-simplex, $\Tsimplex$.
$\{\Ffacet_0 \ldots \Ffacet_{d-2} \}$ share a common edge.
The common edge connects the vertex $\Vvertex_{d-1}$,
which is opposite $\Tsimplex$ in $\Ffacet_{d-1}$,
and the vertex $\Vvertex_{d}$
which is opposite $\Tsimplex$ in $\Ffacet_{d}$.
To get the mutual refinement, we split $\Eedge$ around a new vertex $\Vvertex_0$
and split $\Tsimplex$ around a new vertex $\Vvertex_1$.
To reduce the split children of $\Ssimplex$ to $\Kcomplex_{d-1}^{1}$,
we collapse the edge $\{ \Vvertex_0 , \Vvertex_1 \}$.

Note that the number of vertices in any complex containing $\Ssimplex$
is increased by 1.

\end{enumerate}








