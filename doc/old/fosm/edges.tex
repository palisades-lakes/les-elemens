\begin{plSection}{Functions of the edges}
\label{sec:edges}

\begin{plDiagram}
{Edge face pair labeling.}
{EdgeFaces}
\centering
\begin{verbatim}
          p1
          o
         /|\
        / | \
       /  |  \e31
   e12/   |   \
     /    |e01 \
    /     |     \
p2 o f012 | f031 o p3
    \     |     /
     \    |    /
   e20\   |   /e03
       \  |  /
        \ | /
         \|/
          o
          p0
\end{verbatim}
\end{plDiagram}

Notation in this section is based on \cref{diagram:EdgeFaces}.
We are discussing functions defined on a neighborhood of edge $e_{01}$.

We assume that, for each edge, an arbitrary order is assigned to
its two vertices, which are then at the positions $\Point{p}_0,\Point{p}_1$ in the diagram.

An interior edge has 2 adjacent faces, $f_{012}$ and $f_{031}$.
We assume that these 2 faces are oriented consistently, with the labels
taken counterclockwise, so that the normal vectors point out of the page.
Each face is represented by an ordered triple of vertices,
but the order is only determined up to a circular permutation;
for example, $f_{012}$ may be represented by the ordered triples
$(\Point{p}_0,\Point{p}_1,\Point{p}_2)$, 
$(\Point{p}_2,\Point{p}_0,\Point{p}_1)$, or 
$(\Point{p}_1,\Point{p}_2,\Point{p}_0)$,
but not by
$(\Point{p}_0,\Point{p}_2,\Point{p}_1)$, 
$(\Point{p}_1,\Point{p}_0,\Point{p}_2)$, or
 $(\Point{p}_2,\Point{p}_1,\Point{p}_0)$.

For the given ordering $(\Point{p}_0,\Point{p}_1)$ of the edge,
$f_{120}$ is the edge's {\it left face}
and $f_{031}$ is the {\it right face}.

Note that we cannot assume any consistent ordering of the 4 neighboring edges;
for example, $e_{12}$ may be represented by either ordered pair
$(\Point{p}_1,\Point{p}_2)$ or $(\Point{p}_2,\Point{p}_1)$.

%-----------------------------------------------------------------
\begin{plSection}{Edge length}
\label{sec:edge_length}

The edge tangent vector is $\Point{p}_1 - \Point{p}_0$.

The gradient of its squared length is:
\begin{equation}
\Gc{\Point{p}_i}{\| \Point{p}_1 - \Point{p}_0 \|^2}{\Point{q}} 
= 2 \left( \Point{p}_i - \Point{p}_{(i+1) \bmod 1} \right)
\end{equation}

The gradient of the edge length, $\|\Point{p}_1 - \Point{p}_0\|$ is:
\begin{equation}
\Gc{\Point{p}_i}{\| \Point{p}_1 - \Point{p}_0 \|}{\Point{q}} =
\frac{\left( \Point{p}_i - \Point{p}_{(i+1) \bmod 1} \right)}
{\|\Point{p}_1 - \Point{p}_0\|}
\end{equation}

\end{plSection}%{Edge length}
%-----------------------------------------------------------------
\begin{plSection}{Difference in face normals}
\label{sec:normal_difference}

One measure of the change in surface normal across an edge
is simply the vector difference of the two normals:

\begin{equation}
\label{eq:deltan}
\DNormal{n} (\Point{p}_0, \Point{p}_1, \Point{p}_2, \Point{p}_3)
=
\Normal{n} (\Point{p}_{012}) - \Normal{n} (\Point{p}_{031})
\end{equation}

The (total) derivative of the squared distance between adjacent face normals is:
\begin{eqnarray}
\Db{\|\DNormal{n}(\Point{p})\|^2}{\Point{q}}
& =
2 \ \DNormal{n} ( \Point{q} )^\dagger &
\left( \Db{ ( \DNormal{n} ) }{\Point{q}} \right)
\\
& =
2 \ \DNormal{n}(\Point{q})^\dagger &
\left( \Db{\Normal{n}(\Point{p}_{012})}{\Point{q}} - \Db{\Normal{n}(\Point{p}_{031})}{\Point{q}} \right)
\nonumber \\
& =
2 \DNormal{n}(\Point{q})^\dagger &
\{ \; \left[ \Identity_{\Reals^3} - \left( \Normal{n}( \Point{q}_{012} ) \otimes \Normal{n}( \Point{q}_{012} ) \right)
\right]
\ast \Db{\Vector{a} \left( \Point{p}_{012} \right) }{\Point{q}}
\nonumber \\
\label{eq:deltan_derivative}
&
& - \left[ \Identity_{\Reals^3} - \left( \Normal{n}( \Point{q}_{031} ) \otimes \Normal{n} ( \Point{q}_{031} ) \right)
\right]
\ast \Db{\Vector{a} ( \Point{p}_{031} ) }{\Point{q}}
\; \}
\nonumber
\end{eqnarray}

The partial derivatives, with respect to one of the vertices,
like $\Dd{\Point{p}_0}{\Vector{a} ( \Point{p}_{012} ) }{\Point{q}}{\Point{r}_0}$,
all have a similar form:
\begin{equation}
\Dd{\Point{p}_0}{\Vector{a} ( \Point{p}_{012} ) }{\Point{q}}{\Point{r}_0}  = (\Point{q}_1 - \Point{q}_3) \times \Point{r}_0
\end{equation}
Using this, equation \ref{eq:deltan_derivative}, equation \ref{eq:dot_cross},
and the facts that
$\DNormal{n}(\Point{q})  \perp  \Normal{n}(\Point{q}_{012}) = - \left( \Normal{n}(\Point{q}_{031})  \perp  \Normal{n}(\Point{q}_{012}) \right)$
and
$\DNormal{n}(\Point{q})  \perp  \Normal{n}(\Point{q}_{031}) = \Normal{n}(\Point{q}_{012})  \perp  \Normal{n}(\Point{q}_{031})$,
we can write the partial gradients without reference to the
derivative's argument $\Point{r}$:
\begin{eqnarray}
\label{eq:normal-difference-gradient}
\Gc{\Point{p}_0}{\|\DNormal{n}\|^2}{\Point{q}}
& = &
\left[
\frac{{ \Normal{n}(\Point{q}_{031})  \perp  \Normal{n}(\Point{q}_{012}) }
{A(\Point{q}_{012})}}
\times (\Point{q}_1 - \Point{q}_2)
\right]
\; + \;
\left[
\frac{{ \Normal{n}(\Point{q}_{012})  \perp  \Normal{n}(\Point{q}_{031}) }
{A(\Point{q}_{031})}}
\times (\Point{q}_3 - \Point{q}_1)
\right]
\\
\Gc{\Point{p}_1}{\|\DNormal{n}\|^2}{\Point{q}}
& = &
\left[
\frac{{ \Normal{n}(\Point{q}_{031})  \perp  \Normal{n}(\Point{q}_{012}) }
{A(\Point{q}_{012})}}
\times (\Point{q}_2 - \Point{q}_0)
\right]
\; + \;
\left[
\frac{{ \Normal{n}(\Point{q}_{012})  \perp  \Normal{n}(\Point{q}_{031}) }
{A(\Point{q}_{031})}}
\times (\Point{q}_0 - \Point{q}_3)
\right]
\nonumber
\\
\Gc{\Point{p}_2}{\|\DNormal{n}\|^2}{\Point{q}}
& = &
\left[
\frac{{ \Normal{n}(\Point{q}_{031})  \perp  \Normal{n}(\Point{q}_{012}) }
{A(\Point{q}_{012})}}
\times (\Point{q}_0 - \Point{q}_1)
\right]
\nonumber
\\
\Gc{\Point{p}_3}{\|\DNormal{n}\|^2}{\Point{q}}
& = &
\left[
\frac{{ \Normal{n}(\Point{q}_{012})  \perp  \Normal{n}(\Point{q}_{031}) }
{A(\Point{q}_{031})}}
\times (\Point{q}_1 - \Point{q}_0)
\right]
\nonumber
\end{eqnarray}

\end{plSection}%{Difference in face normals}
%-----------------------------------------------------------------
\begin{plSection}{Inner product between face normals}
\label{sec:normal_dot}

The inner product $\left( \Normal{n}_{012} \bullet \Normal{n}_{031} \right)$
is another important measure of edge curvature.
It is closely related to the squared distance between adjacent normals:
\begin{equation}
\label{eq:normal-distance-dot}
\| \Normal{n}_{012} - \Normal{n}_{031} \|^2
= \| \Normal{n}_{012} \|^2
+ \| \Normal{n}_{031} \|^2
- 2 \left( \Normal{n}_{012} \bullet \Normal{n}_{031} \right)
= 2 \left[ 1 - \left( \Normal{n}_{012} \bullet \Normal{n}_{031} \right) \right]
\end{equation}

The function $f(\Point{p}) = 1 - \left( \Normal{n}_{012} \bullet \Normal{n}_{031} \right)$
achieves its minimum, $0$, on flat face pairs,
and its maximum, $2$, on face pairs that are folded back on themselves.
It's a reasonable choice the total bending or curvature of a surface.
And $\Da{f} = - \Da{\left( \Normal{n}_{012} \bullet \Normal{n}_{031} \right)}$.

The derivative of
$\left( \Normal{n}_{012} \bullet \Normal{n}_{031} \right)$
can be calculated using equations \ref{eq:dot_derivative} and
\ref{eq:unit_normal_derivative}:
\begin{eqnarray}
\label{normal_dot_derivative}
\Db{\left( \Normal{n}_{012} \bullet \Normal{n}_{031} \right)}{\Point{q}}
& = & \Normal{n}(\Point{q}_{031}) \bullet \Db{\Normal{n}_{012}}{\Point{q}} + \Normal{n}(\Point{q}_{012}) \bullet \Db{\Normal{n}_{031}}{\Point{q}}
\\
\nonumber \\
& = &
\Normal{n}(\Point{q}_{031}) \bullet
\frac{\Identity - \left(\Normal{n}(\Point{q}_{012}) \otimes \Normal{n}(\Point{q}_{012}) \right)}{\| \Vector{a}(\Point{q}_{012}) \|}
\; \Db{\Vector{a}_{012}}{\Point{q}}
\nonumber \\
& + &
\Normal{n}(\Point{q}_{012}) \bullet
\frac{\Identity - \left(\Normal{n}(\Point{q}_{031}) \otimes \Normal{n}(\Point{q}_{031}) \right)}{\| \Vector{a}(\Point{q}_{031}) \|}
\; \Db{\Vector{a}_{031}}{\Point{q}}
\nonumber
\end{eqnarray}

As in \cref{sec:normal_difference}, we can write the partial gradients
without reference to an argument:
\begin{eqnarray}
\label{eq:normal_dot_gradient}
\Gc{\Point{p}_0}{(\Normal{n}_{012} \bullet \Normal{n}_{031})}{\Point{q}}
& = \; &
\frac{ \Normal{n}(\Point{q}_{031}) - 
\left[ \Normal{n}(\Point{q}_{012}) \bullet \Normal{n}(\Point{q}_{031}) \right] 
\Normal{n}(\Point{q}_{012}) }
{\| \Vector{a} (\Point{q}_{012}) \| }
\times (\Point{q}_2 - \Point{q}_1)
\\
& \; + &
\frac{ \Normal{n}(\Point{q}_{012}) - \left[ \Normal{n}(\Point{q}_{012}) \bullet \Normal{n}(\Point{q}_{031}) \right] \Normal{n}(\Point{q}_{031})  }
{\| \Vector{a} (\Point{q}_{031}) \| }
\times (\Point{q}_1 - \Point{q}_3)
\nonumber \\
& & \nonumber \\
\Gc{\Point{p}_1}{(\Normal{n}_{012} \bullet \Normal{n}_{031})}{\Point{q}}
& = \; &
\frac{ \Normal{n}(\Point{q}_{031}) - \left[ \Normal{n}(\Point{q}_{012}) \bullet \Normal{n}(\Point{q}_{031}) \right] \Normal{n}(\Point{q}_{012})  }
{\| \Vector{a} (\Point{q}_{012}) \| }
\times (\Point{q}_0 - \Point{q}_2)
\nonumber \\
& \; + &
\frac{ \Normal{n}(\Point{q}_{012}) - \left[ \Normal{n}(\Point{q}_{012}) \bullet \Normal{n}(\Point{q}_{031}) \right] \Normal{n}(\Point{q}_{031})   }
{\| \Vector{a} (\Point{q}_{031}) \| }
\times (\Point{q}_3 - \Point{q}_0)
\nonumber \\
& & \nonumber \\
\Gc{\Point{p}_2}{(\Normal{n}_{012} \bullet \Normal{n}_{031})}{\Point{q}}
& = \; &
\frac{ \Normal{n}(\Point{q}_{031}) - \left[ \Normal{n}(\Point{q}_{012}) \bullet \Normal{n}(\Point{q}_{031}) \right] \Normal{n}(\Point{q}_{012})  }
{\| \Vector{a} (\Point{q}_{012}) \| }
\times (\Point{q}_1 - \Point{q}_0)
\nonumber \\
& & \nonumber \\
\Gc{\Point{p}_3}{(\Normal{n}_{012} \bullet \Normal{n}_{031})}{\Point{q}}
& = \; &
\frac{ \Normal{n}(\Point{q}_{012}) - \left[ \Normal{n}(\Point{q}_{012}) \bullet \Normal{n}(\Point{q}_{031}) \right] \Normal{n}(\Point{q}_{031}) }
{\| \Vector{a} (\Point{q}_{031}) \| }
\times (\Point{q}_0 - \Point{q}_1)
\nonumber
\end{eqnarray}

This can be simplified using the fact that
\(\Normal{n}_i \perp \Normal{n}_j = \Normal{n}_i - \left[ \Normal{n}_i \bullet \Normal{n}_j \right] \Normal{n}_j\), for unit vectors,
and the face area \(A(\Point{q}) = \frac{1}{2} \| \Vector{a}(\Point{q}) \|\):
\begin{eqnarray}
\label{eq:simplified_normal_dot_gradient}
\Gc{\Point{p}_0}{(\Normal{n}_{012} \bullet \Normal{n}_{031})}{\Point{q}}
& = \;\;\; &
\frac{\left[ \Normal{n}(\Point{q}_{031}) \perp \Normal{n}(\Point{q}_{012}) \right]}{2A(\Point{q}_{012})}
\times (\Point{q}_2 - \Point{q}_1)
\\
& \;\;\; + &
\frac{\left[ \Normal{n}(\Point{q}_{012}) \perp \Normal{n}(\Point{q}_{031}) \right]}{2A(\Point{q}_{031})}
\times (\Point{q}_1 - \Point{q}_3)
\nonumber \\
& & \nonumber \\
\Gc{\Point{p}_1}{(\Normal{n}_{012} \bullet \Normal{n}_{031})}{\Point{q}}
& = \;\;\; &
\frac{\left[ \Normal{n}(\Point{q}_{031}) \perp \Normal{n}(\Point{q}_{012}) \right]}{2A(\Point{q}_{012})}
\times (\Point{q}_0 - \Point{q}_2)
\nonumber \\
& \;\;\; + &
\frac{\left[ \Normal{n}(\Point{q}_{012}) \perp \Normal{n}(\Point{q}_{031}) \right]}{2A(\Point{q}_{031})}
\times (\Point{q}_3 - \Point{q}_0)
\nonumber \\
& & \nonumber \\
\Gc{\Point{p}_2}{(\Normal{n}_{012} \bullet \Normal{n}_{031})}{\Point{q}}
& = \;\;\; &
\frac{\left[ \Normal{n}(\Point{q}_{031}) \perp \Normal{n}(\Point{q}_{012}) \right]}{2A(\Point{q}_{012})}
\times (\Point{q}_1 - \Point{q}_0)
\nonumber \\
& & \nonumber \\
\Gc{\Point{p}_3}{(\Normal{n}_{012} \bullet \Normal{n}_{031})}{\Point{q}}
& = \;\;\; &
\frac{\left[ \Normal{n}(\Point{q}_{012}) \perp \Normal{n}(\Point{q}_{031}) \right]}{2A(\Point{q}_{031})}
\times (\Point{q}_0 - \Point{q}_1)
\nonumber
\end{eqnarray}

\end{plSection}%{Inner product between face normals}
%-----------------------------------------------------------------
\begin{plSection}{Squared inner product between face normals}
\label{sec:squared_normal_dot}

We can get a more even distribution of bending by giving
a higher weight to sharper edge bends.
A simple way to do that is to square some existing function,
for example: $\left(1 - \Normal{n}_{012} \bullet \Normal{n}_{031}\right)^2$.
The derivative is simply:
\begin{equation}
\Da{\left(1 - \Normal{n}_{012} \bullet \Normal{n}_{031}\right)^2}
= -2 \left( 1 - \Normal{n}_{012} \bullet \Normal{n}_{031} \right)
\Da{(\Normal{n}_{012} \bullet \Normal{n}_{031})}
\end{equation}

It follows from equation \ref{eq:simplified_normal_dot_gradient}
that the partial gradients are:
\begin{eqnarray}
\label{eq:squared_normal_dot_gradient}
\Gc{\Point{p}_0}{\left(1 - \Normal{n}_{012} \bullet \Normal{n}_{031}\right)^2}{\Point{q}}
& = \;\;\; &
\frac{\left( \Normal{n}(\Point{q}_{012}) \bullet \Normal{n}(\Point{q}_{031}) - 1\right)
}
{A(\Point{q}_{012}) }
\left[ \Normal{n}(\Point{q}_{031}) \perp \Normal{n}(\Point{q}_{012}) \right]
\times (\Point{q}_2 - \Point{q}_1)
\\
& \;\;\; + &
\frac{\left( \Normal{n}(\Point{q}_{012}) \bullet \Normal{n}(\Point{q}_{031}) - 1\right)
}{A(\Point{q}_{031})}
\left[ \Normal{n}(\Point{q}_{012}) \perp \Normal{n}(\Point{q}_{031}) \right]
\times (\Point{q}_1 - \Point{q}_3)
\nonumber \\
& & \nonumber \\
\Gc{\Point{p}_1}{\left(1 - \Normal{n}_{012} \bullet \Normal{n}_{031}\right)^2}{\Point{q}}
& = \;\;\; &
\frac{\left( \Normal{n}(\Point{q}_{012}) \bullet \Normal{n}(\Point{q}_{031}) - 1\right)
}{A(\Point{q}_{012})}
\left[ \Normal{n}(\Point{q}_{031}) \perp \Normal{n}(\Point{q}_{012}) \right]
\times (\Point{q}_0 - \Point{q}_2)
\nonumber \\
& \;\;\; + &
\frac{\left( \Normal{n}(\Point{q}_{012}) \bullet \Normal{n}(\Point{q}_{031}) - 1\right)
}{A(\Point{q}_{031})}
\left[ \Normal{n}(\Point{q}_{012}) \perp \Normal{n}(\Point{q}_{031}) \right]
\times (\Point{q}_3 - \Point{q}_0)
\nonumber \\
& & \nonumber \\
\Gc{\Point{p}_2}{\left(1 - \Normal{n}_{012} \bullet \Normal{n}_{031}\right)^2}{\Point{q}}
& = \;\;\; &
\frac{\left( \Normal{n}(\Point{q}_{012}) \bullet \Normal{n}(\Point{q}_{031}) - 1\right)
}{A(\Point{q}_{012})}
\left[ \Normal{n}(\Point{q}_{031}) \perp \Normal{n}(\Point{q}_{012}) \right]
\times (\Point{q}_1 - \Point{q}_0)
\nonumber \\
& & \nonumber \\
\Gc{\Point{p}_3}{\left(1 - \Normal{n}_{012} \bullet \Normal{n}_{031}\right)^2}{\Point{q}}
& = \;\;\; &
\frac{\left( \Normal{n}(\Point{q}_{012}) \bullet \Normal{n}(\Point{q}_{031}) - 1\right)
}{A(\Point{q}_{031})}
\left[ \Normal{n}(\Point{q}_{012}) \perp \Normal{n}(\Point{q}_{031}) \right]
\times (\Point{q}_0 - \Point{q}_1)
\nonumber
\end{eqnarray}
\end{plSection}%{Squared inner product between face normals}
%-----------------------------------------------------------------
\end{plSection}%{Functions of the edges}
