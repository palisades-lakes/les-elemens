% jam 2004-09-10

\section{Mesh Operations}
\label{sec:mesh-operations}

\subsection{Split}

Splitting a simplex produces a refined complex by adding a new vertex
and subdivided some of the simplices.


Suppose $\Ssimplex$ is a $d$-simplex
in the pure $n$-dimensional simplicial complex $\Kcomplex_0$.
Let $\Vvertex$ be a vertex not in $\Kcomplex_0$.
{\it Splitting $\Ssimplex$ around $\Vvertex$}
produces a refined pure $n$-dimensional simplicial complex, $\Kcomplex_1$:

$\Kcomplex_1$ contains all the  $n$-simplexes of $\Kcomplex_0$
that do not contain $\Ssimplex$.

The simplexes that do contain $\Ssimplex$ are split:
Let $\Tsimplex$ be an $n$-simplex that contains $\Ssimplex$.
Let $\Rsimplex$ be the $(n-d-1)$-simplex opposite $\Ssimplex$
in $\Tsimplex$.
(If $\Tsimplex = \Ssimplex$, then $\Rsimplex$ is the empty set.)
Let $\{ \Ffacet_0 \ldots \Ffacet_d \}$ be the
d $(d-1)$-simplex facets of $\Ssimplex$.
$\Kcomplex_1$ has
$d$ $n$-simplices formed from
$\Vvertex$, $\Rsimplex$, and
$\Ffacet_i$ where $i=0 \ldots d$.

This is also known as "placing a vertex" 
(See Lee~\cite[sec.~17.2]{Lee:2004:Subdivisions}).

\subsection{Collapse}

Collapsing a simplex produces a reduced complex by mergng
the vertices of the simplex into a single new one.

Suppose $\Ssimplex$ is a $d$-simplex
in the pure $n$-dimensional simplicial complex $\Kcomplex_0$.
Let $\Vvertex$ be a vertex not in $\Kcomplex_0$.
{\it Collapsing $\Ssimplex$ to $\Vvertex$}
produces a reduced pure $n$-dimensional simplicial complex, $\Kcomplex_1$:

$\Kcomplex_1$ contains all the
$n$-simplices of $\Kcomplex_0$ that share no vertices with $\Ssimplex$.
Any $n$-simplex in $\Kcomplex_0$ sharing more than one vertex
with $\Ssimplex$ is ignored.
For every $n$-simplex in $\Kcomplex_0$ sharing one vertex with $\Ssimplex$,
$\Kcomplex_1$ gets an $n$-simplex with $\Vvertex$ replacing the
shared vertex.

