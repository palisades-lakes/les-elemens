%-------------------------------------------------------------------------------
\errorcontextlines 10000
%-------------------------------------------------------------------------------
%input path
%-------------------------------------------------------------------------------
%\makeatletter
%\def\input@path{{../bib/}{../shared/}{../figs/}{../tex/}{../rawtex/}{../}}
%\makeatother
%-------------------------------------------------------------------------------
% TODO: try these
% \usepackage{amsmath,amsfonts,amssymb,mathrsfs,theorem}
% \usepackage{multicol,multirow,calc,achicago,graphicx,color,colortab,rotating,enumerate}
% \usepackage{pstricks,psfrag,tabularx,comment,hyperref}
% \usepackage{boxedminipage}
% \usepackage{bbm}
% Graphics
% \usepackage{graphicx}
% %Tables
% \usepackage{booktabs}
% \usepackage{lscape}
% \usepackage{bbold}
% \usepackage{natbib}
% \def\newblock{\hskip .11em plus .33em minus .07em}
% \usepackage{url}
% \usepackage{citeref}
% \citestyle{authoryear}
% \usepackage{hyperref}
%-------------------------------------------------------------------------------
%\usepackage{bookmark}
\usepackage{coseoul}
% used to revert to sub-document's top level
\newcounter{baseSectionLevel}
%-------------------------------------------------------------------------------
% layout file determines 1/2 col, landscape/portrait...
\usepackage{geometry}
%-------------------------------------------------------------------------------
\usepackage{color}
\usepackage[dvipsnames,svgnames,x11names]{xcolor}
%-------------------------------------------------------------------------------
\usepackage{graphics}
\usepackage{epsfig}
\usepackage{graphicx}
\PassOptionsToPackage{normalem}{ulem}
\usepackage{ulem}
%-------------------------------------------------------------------------------
% category thgeory
\usepackage{tikz-cd}
%\usepackage{tikz-network}
%-------------------------------------------------------------------------------
\usepackage{url}
%-------------------------------------------------------------------------------
\usepackage{csquotes}
%-------------------------------------------------------------------------------
\usepackage[english]{babel}
%-------------------------------------------------------------------------------
\usepackage{epigraph}
\setlength{\epigraphwidth}{0.9\linewidth}
\renewcommand{\epigraphflush}{center}
\renewcommand{\sourceflush}{flushleft}

%-------------------------------------------------------------------------------
\usepackage{fontspec}
%-------------------------------------------------------------------------------
%\setmainfont{Baskerville Old Face}
%\setmainfont{Libre Caslon Text}[Scale=0.85]
%\setmainfont{Centaur}
%\setmainfont{Garamond}
%\setmainfont{Georgia}
%\setmainfont{Perpetua}
%\setmainfont{Poor Richard}

% http://www.impallari.com/projects/overview/libre-caslon-display-and-text
%\setmainfont{Libre Caslon Text}[Scale=0.85]
%\newfontfamily\scshape[Letters=SmallCaps,Scale=1.15]{Crimson}

% http://iginomarini.com/fell/the-revival-fonts/
% \fontspec[
%  SmallCapsFont=IM FELL English SC,
%  SmallCapsFeatures={Letters=SmallCaps},
% ]{IM FELL English}
% \setmainfont{IM FELL English}
 
% https://github.com/CatharsisFonts/Cormorant/releases/tag/v3.3 

% http://www.georgduffner.at/ebgaramond/download.html
% \fontspec[
%  SmallCapsFeatures={Letters=SmallCaps},
% ]{EB Garamond}
\setmainfont[
Path,
UprightFont = *12-Regular,
ItalicFont  = *12-Italic,
BoldFont    = *08-Regular,
BoldItalicFont = *08-Italic ]
{EBGaramond}

% https://www.microsoft.com/typography/fonts/family.aspx?FID=134
% \fontspec[
%  SmallCapsFeatures={Letters=SmallCaps},
% ]{Garamond}
% \setmainfont{Garamond}
%\setmainfont{Palatino Linotype}
%\setmainfont{Perpetua}[Scale=1.1]
%\setmainfont{Times New Roman}
%-------------------------------------------------------------------------------
% https://www.microsoft.com/typography/fonts/family.aspx?FID=155
\setsansfont{Gill Sans MT} 

% http://arkandis.tuxfamily.org/adffonts.html
% \setsansfont{Gillius ADF}
%-------------------------------------------------------------------------------
% \setmonofont{}
%-------------------------------------------------------------------------------
% \usepackage{xeCJK}
% \setCJKmainfont{SimHei}
% \setCJKsansfont{SimHei}
% \setCJKmonofont{Lucida Sans Typewriter}
%-------------------------------------------------------------------------------
\usepackage{amsmath}
\usepackage{amssymb}
\DeclareMathOperator*{\argmin}{argmin}
\DeclareMathOperator*{\argmax}{argmax}
\DeclareMathOperator*{\sign}{sign}
\DeclareMathOperator*{\defeq}
{\overset{\underset{\mathrm{def}}{}}{=}}
%\DeclareMathOperator*{\cdf}{cdf}
%\DeclareMathOperator*{\quantile}{quantile}
\newcommand\bigforall{\mbox{\Large $\mathsurround0pt\forall$}} 
%https://tex.stackexchange.com/questions/83509/hfill-in-math-mode
\makeatletter
\newcommand{\pushright}[1]{\ifmeasuring@#1\else\omit\hfill$\displaystyle#1$\fi\ignorespaces}
\newcommand{\pushleft}[1]{\ifmeasuring@#1\else\omit$\displaystyle#1$\hfill\fi\ignorespaces}
\newcommand{\specialcell}[1]{\ifmeasuring@#1\else\omit$\displaystyle#1$\ignorespaces\fi}\makeatother
\makeatother
% https://tex.stackexchange.com/questions/14071/how-can-i-increase-the-line-spacing-in-a-matrix
\makeatletter
\renewcommand*\env@matrix[1][\arraystretch]{%
  \edef\arraystretch{#1}%
  \hskip -\arraycolsep
  \let\@ifnextchar\new@ifnextchar
  \array{*\c@MaxMatrixCols c}}
\makeatother
% https://tex.stackexchange.com/questions/42726/align-but-show-one-equation-number-at-the-end/42728#42728
\newcommand\numberthis{\addtocounter{equation}{1}\tag{\theequation}}
%-----------------------------------------------------------------
\usepackage{mathtools}
%-----------------------------------------------------------------
%\usepackage{amsthm}
\usepackage[amsthm,amsmath]{ntheorem}
\usepackage{thmtools}
\theoremstyle{definition}
\newtheorem{theorem}{\textsc{Theorem}}[section]
\theoremstyle{definition}
\newtheorem{definition}{\textsc{Definition}}[section]
\theoremstyle{definition}
\newtheorem{example}{\textsc{Example}}[section]
%-----------------------------------------------------------------
% 2019-12-22 thmtools broken by change to kvsetkeys
% \usepackage{thmtools}

% https://tex.stackexchange.com/questions/249963/remove-repeated-theorem-in-the-list-of-theorems
% \usepackage{etoolbox}
% \makeatletter
% \patchcmd\thmt@mklistcmd
%   {\thmt@thmname}
%   {\check@optarg{\thmt@thmname}}
%   {}{}
% \patchcmd\thmt@mklistcmd
%   {\thmt@thmname\ifx}
%   {\check@optarg{\thmt@thmname}\ifx}
%   {}{}
% \protected\def\check@optarg#1{%
%   \@ifnextchar\thmtformatoptarg\@secondoftwo{#1}%
% }
% \makeatother
%-----------------------------------------------------------------
% \newtheoremstyle{break}%
% {}{}%
% {}{}%
% {}{}% % Note that final punctuation is omitted.
% {\newline}{}
% \theoremstyle{break}
% \newtheorem{example}{Example}[section]
% \newtheorem{definition}{\textsf{Definition}}[section]
% \newtheorem{fact}{\textsf{Fact}}[section]
%\DeclareRobustCommand{\rvspace}[1]{\vspace{#1}}
% \theoremstyle{break}
%-----------------------------------------------------------------
% \declaretheoremstyle[
% % TODO: should be relative to parskip, or something like that
% spaceabove=16pt,
% spacebelow=12pt,
% headfont=\normalfont\mdseries,
% headpunct={\vspace{\topsep}\newline\vspace{\topsep}\vspace{\topsep}},
% %notefont=\sffamily\bfseries,
% notefont=\sffamily,
% notebraces={\hspace{1em}}{},
% shaded={bgcolor=GhostWhite},
% bodyfont=\normalfont,
% postheadspace=1em,
% ]{mythmstyle}
% \declaretheorem[style=mythmstyle,title=Theorem,name={Theorem}]{theorem}
% \declaretheorem[style=mythmstyle,title=Definition,name={Definition}]{definition}
% \declaretheorem[style=mythmstyle,title=Example,name={Example}]{example}
%\renewcommand{\thmtformatoptarg}[1]{#1}%
%-----------------------------------------------------------------
% \numberwithin{theorem}{chapter}
% \numberwithin{definition}{chapter}
% \numberwithin{example}{chapter}
% \numberwithin{equation}{chapter}
% \numberwithin{figure}{chapter}
\numberwithin{theorem}{section}
\numberwithin{definition}{section}
\numberwithin{example}{section}
\numberwithin{equation}{section}
\numberwithin{figure}{section}
%-----------------------------------------------------------------
\usepackage{listings}
\lstset{backgroundcolor={\color{GhostWhite}},
basicstyle={\ttfamily\small},
breaklines=false,
captionpos=b,
%frame=tblr,
mathescape=true,
escapechar=\%,
keywordstyle={\ttfamily}}
%\renewcommand{\lstlistingname}{Listing}
%\renewcommand{\lstlistingname}{}
% \providecommand{\algorithmname}{Algorithm}
% \providecommand{\exercisename}{Exercise}
% \providecommand{\theoremname}{Theorem}
% \providecommand{\examplename}{Example}
%-----------------------------------------------------------------
% \makeatletter
% \let\orig@item\item
% 
% \def\item{%
%     \@ifnextchar{[}%
%         {\lstinline@item}%
%         {\orig@item}%
% }
% 
% \begingroup
% \catcode`\]=\active
% \gdef\lstinline@item[{%
%     \setbox0\hbox\bgroup
%         \catcode`\]=\active
%         \let]\lstinline@item@end
% }
% \endgroup
% 
% \def\lstinline@item@end{%
%     \egroup
%     \orig@item[\usebox0]%
% }
% \makeatother
%-----------------------------------------------------------------
\usepackage{algpseudocode,algorithm,algorithmicx}
%-----------------------------------------------------------------
\usepackage{datetime}
\renewcommand{\dateseparator}{-}
\renewcommand{\today}{
\the\year \dateseparator \twodigit\month \dateseparator \twodigit\day}
%-----------------------------------------------------------------
\setlength{\parskip}{5pt}
\setlength{\parindent}{0pt}
\usepackage[parfill]{parskip}
%-----------------------------------------------------------------
% \usepackage{fancyhdr}
% \pagestyle{fancy}
% \setlength{\headwidth}{\textheight}
% \addtolength{\headwidth}{\columnsep}
% %\addtolength{\headwidth}{\marginparsep}
% %\addtolength{\headwidth}{\marginparwidth}
% \fancypagestyle{plain}{
% \fancyhead{} % clear all head fields 
% \fancyfoot{} % clear all foot fields
% \fancyfoot[RO,LE]{\textsf{\thepage}} 
% \fancyfoot[RE,LO]{\textsf{Draft of \today}}
% \renewcommand{\headrulewidth}{0.0pt}
% \renewcommand{\footrulewidth}{0.1pt}}
% \pagestyle{plain}
%-----------------------------------------------------------------
\usepackage{titling}
%\newfontfamily\titlefont[Scale=MatchUppercase]{Gill Sans MT}
%\renewcommand{\maketitlehooka}{\titlefont}
%\pretitle{\begin{flushright}\Huge\sffamily\bfseries}
\pretitle{\begin{flushright}\Huge\scshape\bfseries}
\posttitle{\par\end{flushright}\vskip 0.25em}
%\preauthor{\begin{flushright}\sffamily\scshape\mdseries}
\preauthor{\begin{flushright}\scshape\mdseries}
\postauthor{\par\end{flushright}}
%\predate{\begin{flushright}\sffamily\scshape\mdseries}
\predate{\begin{flushright}\scshape\mdseries}
\postdate{\par\end{flushright}}
\setlength{\droptitle}{-80pt}
%------------------------------------------------------------------
\usepackage{enumitem}
%\setlist[description]{font=\small\sffamily\mdseries,style=unboxed,leftmargin=0cm}
%\setlist[description]{font=\sffamily\mdseries}
\setlist[description]{font=\scshape\bfseries}
\setlist[itemize]{style=unboxed,itemindent=0cm}
\setlist[enumerate]{style=unboxed,itemindent=0cm}
%-----------------------------------------------------------------
%\usepackage[sf,small,compact]{titlesec}
\usepackage[small,compact]{titlesec}
%\newfontfamily\headingfont[]{New Yorker}
%\newfontfamily\headingfont[Scale=MatchUppercase]{Libre Caslon Display}
%\newfontfamily\headingfont[]{Perpetua Titling MT}
%\newfontfamily\headingfont[Scale=MatchUppercase]{Gill Sans MT}
%\newfontfamily\headingfont[Scale=MatchUppercase]{Gillius ADF}

% \titleformat{\part}{\huge\sffamily\bfseries}{\thepart}{0.5em}{}
% \titleformat{\chapter}{\LARGE\sffamily\bfseries}{\thechapter}{0.5em}{}
% \titleformat{\section}{\Large\sffamily\bfseries}{\thesection}{0.5em}{}
% \titleformat{\subsection}{\large\sffamily\bfseries}{\thesubsection}{0.5em}{}
% \titleformat{\subsubsection}{\large\sffamily\mdseries}{\thesubsubsection}{0.5em}{}
% \titleformat{\paragraph}[runin]{\normalsize\sffamily\mdseries}{\theparagraph}{0.5em}{}[\hspace{1em}]
% \titleformat{\subparagraph}[runin]{\normalsize\sffamily\mdseries}{\thesubparagraph}{0.5em}{}[\hspace{1em}]

\titleformat{\part}{\huge\scshape\bfseries}{\thepart}{0.5em}{}
\titleformat{\chapter}{\LARGE\scshape\bfseries}{\thechapter}{0.5em}{}
\titleformat{\section}{\Large\scshape\bfseries}{\thesection}{0.5em}{}
\titleformat{\subsection}{\large\scshape\bfseries}{\thesubsection}{0.5em}{}
\titleformat{\subsubsection}{\large\scshape\mdseries}{\thesubsubsection}{0.5em}{}
\titleformat{\paragraph}[runin]{\normalsize\scshape\mdseries}{\theparagraph}{0.5em}{}[\hspace{1em}]
\titleformat{\subparagraph}[runin]{\normalsize\scshape\mdseries}{\thesubparagraph}{0.5em}{}[\hspace{1em}]

\titlespacing\section{0pt}{12pt plus 12pt minus 2pt}{5pt plus 5pt minus 2pt}
\titlespacing\subsection{0pt}{11pt plus 11pt minus 2pt}{5pt plus 5pt minus 2pt}
\titlespacing\subsubsection{0pt}{10pt plus 10pt minus 2pt}{5pt plus 5pt minus 2pt}

\setcounter{secnumdepth}{7}
%-----------------------------------------------------------------
% \makeatletter
% \let\oldl@chapter\l@chapter
% \def\l@chapter#1#2{\oldl@chapter{#1}{\textsf{#2}}}
% \let\old@dottedcontentsline\@dottedtocline
% \def\@dottedtocline#1#2#3#4#5{%
% \old@dottedcontentsline{#1}{#2}{#3}{#4}{{\textsf{#5}}}}
% \makeatother

\makeatletter
\let\oldl@chapter\l@chapter
\def\l@chapter#1#2{\oldl@chapter{#1}{\textrm{#2}}}
\let\old@dottedcontentsline\@dottedtocline
\def\@dottedtocline#1#2#3#4#5{%
\old@dottedcontentsline{#1}{#2}{#3}{#4}{{\textrm{#5}}}}
\makeatother
%-----------------------------------------------------------------
% \usepackage{tocloft}
% \renewcommand{\cftpartfont}{\sffamily}
% \renewcommand{\cftchapfont}{\sffamily}
% \renewcommand{\cftsecfont}{\sffamily}
% \renewcommand{\cftsubsecfont}{\sffamily}
% \renewcommand{\cftsubsubsecfont}{\sffamily}
% \renewcommand{\cftparafont}{\sffamily}
% \renewcommand{\cftsubparafont}{\sffamily}
%-----------------------------------------------------------------
%https://en.wikibooks.org/wiki/LaTeX/Indexing
\usepackage{makeidx}
\makeindex
\usepackage[totoc]{idxlayout}
%-----------------------------------------------------------------
\usepackage[
backend=biber, 
citestyle=numeric-comp, 
bibstyle=numeric,
%bibstyle=verbose,
%entrykey=false,
labelnumber=true,
sortcites=true,
maxnames=1000,
maxitems=1000,
block=nbpar,
abbreviate=false,
seconds=true,
date=iso,
alldates=iso,
datezeros=true,
timezeros=true,
]{biblatex} 
\renewcommand\mkbibnamefamily[1]{\textsc{#1}}
%-----------------------------------------------------------------
% \usepackage[chapter]{tocbibind}
% \renewcommand{\listfigurename}{Figures}
% \setlofname{Figures}
% \renewcommand{\listoffigures}{\begingroup
% \tocchapter
% \tocfile{\listfigurename}{lof}
% \endgroup}
%-----------------------------------------------------------------
%\usepackage[titletoc]{appendix}
%-----------------------------------------------------------------
\usepackage[unicode=true,
pdfusetitle,
bookmarks=true,
bookmarksnumbered=false,
bookmarksopen=true,
bookmarksopenlevel=1,
breaklinks=false,
pdfborder={0 0 0},
pdftoolbar=false,
pdffitwindow=true,
backref=false,
colorlinks=true]{hyperref}
\hypersetup{unicode=true,
colorlinks=true,
pdfpagemode=UseOutlines,
pdfpagelayout=OneColumn,
pdfstartview=Fit,
linkcolor=MidnightBlue,
urlcolor=Mahogany,
citecolor=OliveGreen}
\usepackage{bookmark}
%-----------------------------------------------------------------
% \usepackage[xindy,toc,style=alttreehypergroup,nolong,nosuper]{glossaries}
%-----------------------------------------------------------------
% doesn't do much for XeTeX
%\usepackage{microtype}
%-----------------------------------------------------------------
%jam 2004-09-10

\def\Boundary{\partial}
\def\Vvertex{\nu}
\def\Ssimplex{\sigma}
\def\Tsimplex{\tau}
\def\Rsimplex{\rho}
\def\Ffacet{\phi}
\def\Eedge{\epsilon}
\def\Kcomplex{\mathcal{K}}  % An abstract simplicial complex
\def\Mmesh{\mathcal{M}}  % A simplicial mesh

\def\Set#1{{\mathcal{#1}}} 
\def\Space#1{{\mathbb{#1}}} 
\def\Vector#1{{\mathsf{#1}}} 
\def\Point#1{{\mathbf{#1}}} 

\def\Sset{\mathcal{S}}    % generic set

\def\Integers{\mathbb{Z}}
\def\Reals{\mathbb{R}}    % Real numbers
\def\Quaternions{\mathbb{Q}}    % set of Quaternions

\def\Uspace{\mathbb{U}}  % vector space
\def\Vspace{\mathbb{V}}  % vector space
\def\Wspace{\mathbb{W}}  % vector space

\def\Aspace{\mathbb{A}}  % affine space
\def\Tspace{\mathbb{T}}  % translation space (of an affine space)

\def\Lspace{\mathbb{L}} % space of of linear maps

\def\t{\mathsf{t}} % translation vector
\def\u{\mathsf{u}} % vector
\def\v{\mathsf{v}} % vector
\def\w{\mathsf{w}} % vector
\def\0{{\mathsf 0}} % zero vector
\def\e{\mathsf{e}} % canonical basis vector
\def\b{\mathsf{b}} % barycentric coord. $\b_i$, $\b_{i,j}$

\def\p{\mathsf{p}}     % generic point
\def\q{\mathsf{q}}     % generic point
\def\r{\mathsf{r}}     % generic point

\def\f{\mathsf{f}}     % generic vector-valued function
\def\g{\mathsf{g}}     % generic vector-valued function
\def\h{\mathsf{h}}     % generic vector-valued function

\def\Umap{\mathsf{U}}  % matrix whose cols are ui
\def\Vmap{\mathsf{V}}  % matrix whose cols are vi
\def\Wmap{\mathsf{W}}  % matrix whose cols are wi

\def\Lmap{\mathsf{L}} % linear transform
\def\Mmap{\mathsf{M}} % another linear transform

\def\Emap{\mathsf{E}} % canonical basis vector for space of linear transforms
\def\Tmap{\mathsf{T}}  % translation
\def\Amap{\mathsf{A}} % affine transform

\def\c{\mathsf{c}} % row of \Lmap linear transform
\def\r{\mathsf{r}} % row of \Lmap linear transform

\def\Identity{\mathsf{I}}   % the identity transformation

\def\union{\cup}
\def\intersection{\cap}
\def\Union{\bigcup}
\def\Intersection{\bigcap}

\def\Transpose{\mathrm{transpose}}   % the transpose transformation
\def\LTL{\mathrm{LTL}}   % a-transpose-a
\def\Inverse{\mathrm{inverse}}
\def\Pseudoinverse{\mathrm{pseudoinverse}}

\def\dimension{\mathrm{dim}}   % dimension of geometric object
\def\Projection{\pi}   % orthogonal projection

\def\kernel{\mathrm{ker}}    % kernel
\def\range{\mathrm{ran}}    % range
\def\linearspan{\mathrm{span}}    % linear span
\def\affinespan{\mathrm{aff}}    % affine span
\def\convexspan{\mathrm{hull}}    % convex span, convex hull

\def\volume{\mathrm{volume}}    % sign function

\def\Da#1{{\mathcal{D}{#1}}}    % derivative operator
\def\Db#1#2{{\mathcal{D}{#1}_{\mid_{#2}}}}    % derivative operator
\def\Dc#1#2#3{{\mathcal{D}{#1}_{\mid_{#2}}({#3})}}  % derivative operator
\def\Dd#1#2#3#4{{\mathcal{D}_{#1}{{#2}}_{\mid_{#3}}({{#4}})}} % derivative operator
\def\De#1#2#3{{\mathcal D}_{#1}{#2}_{\mid_{#3}}}  % derivative operator
\def\Df#1#2{{\mathcal D}_{#1}{#2}}  % derivative operator

\def\da#1#2{{\partial}_{#1}{#2}}  % partial derivative operator
\def\db#1#2#3{{\partial}_{#1}{#2}_{\mid_{#3}}}  % partial derivative operator

\def\Ga#1{{\mathbf \nabla}{#1}}   % derivative operator
\def\Gb#1#2{{\mathbf \nabla}{#1}_{\mid_{#2}}} % derivative operator
\def\Gc#1#2#3{{\mathbf \nabla}_{#1}{{#2}}_{\mid_{#3}}}  % derivative operator
\def\Gf#1#2{{\mathbf \nabla}_{#1}{#2}}  % derivative operator

\def\Edges{\mathcal{E}}                   % edges
\def\Faces{\mathcal{F}}                   % faces

\def\Q{\mathsf{Q}} % Transform corresponding to quaternion
\def\R{\mathsf{R}} % Rotation
\def\G{\mathsf{G}} % riGid transform
\def\Sc{\mathsf{S}}    % scaled rotation or subdivision transform

\def\sign{\mathrm{sign}}    % sign function

\def\dn{{\mathsf \delta \n}} % change in normal
\def\nd{{\mathsf \n^\bullet}} % dot product of adjacent normals

\def\a{\mathsf{a}}     % area-weighted face normal
\def\l{\mathsf{l}}     % Linear map as vector
\def\n{\mathsf{n}}     % unit length face normal
\def\x{\mathsf{x}}     % instance of point $\x$, $\x_i \in X$
\def\c{\mathsf{c}}     % 3d cross product as function
\def\d{\mathsf{d}}     % data record
\def\y{\mathsf{y}}     % another barycentric coordinate
\def\z{\mathsf{z}}     % projection of a point
\def\S{\mathsf{S}}     % local subdivision matrix

%-----------------------------------------------------------------


