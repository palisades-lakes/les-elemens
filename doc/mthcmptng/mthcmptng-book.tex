% !TEX TS-program = arara
% arara: xelatex: { synctex: on, options: '-halt-on-error' } 
% arara: biber
% arara: makeglossaries
% arara: makeindex
% arara: xelatex: { synctex: on, options: '-halt-on-error'  } 
% arara: xelatex: { synctex: on, options: '-halt-on-error'   } 
% arara: clean: { files: [ mthcmptng-book.aux, mthcmptng-book.bbl] }
% arara: clean: { files: [ mthcmptng-book.bcf, mthcmptng-book.blg] }
% arara: clean: { files: [ mthcmptng-book.glg, mthcmptng-book.glo] }
% arara: clean: { files: [ mthcmptng-book.gls, mthcmptng-book.gls] }
% arara: clean: { files: [ mthcmptng-book.idx, mthcmptng-book.ilg] }
% arara: clean: { files: [ mthcmptng-book.ind, mthcmptng-book.loe] }
% arara: clean: { files: [ mthcmptng-book.lof, mthcmptng-book.log ] }
% arara: clean: { files: [ mthcmptng-book.log, mthcmptng-book.lol ] }
% arara: clean: { files: [ mthcmptng-book.out ] }
% arara: clean: { files: [ mthcmptng-book.run.xml] }
% arara: clean: { files: [ mthcmptng-book.toc, mthcmptng-book.xdy] }
% arara: clean: { files: [ mthcmptng-book.synctex.gz] }
%-------------------------------------------------------------------------------
\documentclass[10pt,openany]{book}
%-------------------------------------------------------------------------------
\errorcontextlines 10000
%-------------------------------------------------------------------------------
\usepackage{coseoul}
% used to revert to sub-document's top level
\newcounter{baseSectionLevel}
%-------------------------------------------------------------------------------
% layout file determines 1/2 col, landscape/portrait...
\usepackage{geometry}
%-------------------------------------------------------------------------------
\usepackage{color}
\usepackage[dvipsnames,svgnames,x11names]{xcolor}
%-------------------------------------------------------------------------------
\usepackage{graphics}
\usepackage{epsfig}
\usepackage{graphicx}
\PassOptionsToPackage{normalem}{ulem}
\usepackage{ulem}
%-------------------------------------------------------------------------------
% category thgeory
\usepackage{tikz-cd}
%\usepackage{tikz-network}
%-------------------------------------------------------------------------------
\usepackage{url}
%-------------------------------------------------------------------------------
\usepackage[utf8x]{inputenc}
\usepackage{csquotes}
%-------------------------------------------------------------------------------
\usepackage[english]{babel}
%-------------------------------------------------------------------------------
\usepackage{epigraph}
\setlength{\epigraphwidth}{0.9\linewidth}
\renewcommand{\epigraphflush}{center}
\renewcommand{\sourceflush}{flushleft}

%-------------------------------------------------------------------------------
\usepackage{fontspec}
%-------------------------------------------------------------------------------
%\setmainfont{Baskerville Old Face}
%\setmainfont{Libre Caslon Text}[Scale=0.85]
%\setmainfont{Centaur}
%\setmainfont{Garamond}
%\setmainfont{Georgia}
%\setmainfont{Perpetua}
%\setmainfont{Poor Richard}

% http://www.impallari.com/projects/overview/libre-caslon-display-and-text
%\setmainfont{Libre Caslon Text}[Scale=0.85]
%\newfontfamily\scshape[Letters=SmallCaps,Scale=1.15]{Crimson}

% http://iginomarini.com/fell/the-revival-fonts/
% \fontspec[
%  SmallCapsFont=IM FELL English SC,
%  SmallCapsFeatures={Letters=SmallCaps},
% ]{IM FELL English}
% \setmainfont{IM FELL English}
 
% https://github.com/CatharsisFonts/Cormorant/releases/tag/v3.3 

% http://www.georgduffner.at/ebgaramond/download.html
% \fontspec[
%  SmallCapsFeatures={Letters=SmallCaps},
% ]{EB Garamond}
\setmainfont[
Path,
UprightFont = *12-Regular,
ItalicFont  = *12-Italic,
BoldFont    = *08-Regular,
BoldItalicFont = *08-Italic ]
{EBGaramond}

% https://www.microsoft.com/typography/fonts/family.aspx?FID=134
% \fontspec[
%  SmallCapsFeatures={Letters=SmallCaps},
% ]{Garamond}
% \setmainfont{Garamond}
%\setmainfont{Palatino Linotype}
%\setmainfont{Perpetua}[Scale=1.1]
%\setmainfont{Times New Roman}
%-------------------------------------------------------------------------------
% https://www.microsoft.com/typography/fonts/family.aspx?FID=155
\setsansfont{Gill Sans MT} 

% http://arkandis.tuxfamily.org/adffonts.html
% \setsansfont{Gillius ADF}
%-------------------------------------------------------------------------------
% \usepackage{xeCJK}
% \setCJKmainfont{SimHei}
% \setCJKsansfont{SimHei}
% \setCJKmonofont{Lucida Sans Typewriter}
%-------------------------------------------------------------------------------
\usepackage{amsmath}
\usepackage{amssymb}
\DeclareMathOperator*{\argmin}{argmin}
\DeclareMathOperator*{\argmax}{argmax}
\DeclareMathOperator*{\sign}{sign}
\DeclareMathOperator*{\defeq}
{\overset{\underset{\mathrm{def}}{}}{=}}
%\DeclareMathOperator*{\cdf}{cdf}
%\DeclareMathOperator*{\quantile}{quantile}
\newcommand\bigforall{\mbox{\Large $\mathsurround0pt\forall$}} 
% https://tex.stackexchange.com/questions/14071/how-can-i-increase-the-line-spacing-in-a-matrix
\makeatletter
\renewcommand*\env@matrix[1][\arraystretch]{%
  \edef\arraystretch{#1}%
  \hskip -\arraycolsep
  \let\@ifnextchar\new@ifnextchar
  \array{*\c@MaxMatrixCols c}}
\makeatother
% https://tex.stackexchange.com/questions/42726/align-but-show-one-equation-number-at-the-end/42728#42728
\newcommand\numberthis{\addtocounter{equation}{1}\tag{\theequation}}
%-----------------------------------------------------------------
\usepackage{mathtools}
%-----------------------------------------------------------------
%\usepackage[amsthm,amsmath]{ntheorem}
\usepackage{amsthm}
\usepackage{thmtools}
%-----------------------------------------------------------------
\newtheoremstyle{break}%
{3pt}{3pt}%
{}{}%
{\bfseries}%
{}% % Note that final punctuation is omitted.
{\newline}%
{\thmname{#1}\thmnumber{ #2}\thmnote{ #3}}

\theoremstyle{break}
\newtheorem{definition}{\textsc{Definition}}[section]

\theoremstyle{definition}
\newtheorem{theorem}{\textsc{Theorem}}[section]
\theoremstyle{definition}
\newtheorem{example}{\textsc{Example}}[section]

%-----------------------------------------------------------------
% \numberwithin{theorem}{chapter}
% \numberwithin{definition}{chapter}
% \numberwithin{example}{chapter}
% \numberwithin{equation}{chapter}
% \numberwithin{figure}{chapter}
\numberwithin{theorem}{section}
\numberwithin{definition}{section}
\numberwithin{example}{section}
\numberwithin{equation}{section}
\numberwithin{figure}{section}
%-----------------------------------------------------------------
\usepackage{listings}
\lstset{backgroundcolor={\color{GhostWhite}},
basicstyle={\ttfamily\small},
breaklines=false,
captionpos=b,
%frame=tblr,
mathescape=true,
escapechar=\%,
keywordstyle={\ttfamily}}
%\renewcommand{\lstlistingname}{Listing}
%\renewcommand{\lstlistingname}{}
% \providecommand{\algorithmname}{Algorithm}
% \providecommand{\exercisename}{Exercise}
% \providecommand{\theoremname}{Theorem}
% \providecommand{\examplename}{Example}
%-----------------------------------------------------------------
\usepackage{algpseudocode,algorithm,algorithmicx}
%-----------------------------------------------------------------
\usepackage{datetime}
\renewcommand{\dateseparator}{-}
\renewcommand{\today}{
\the\year \dateseparator \twodigit\month \dateseparator \twodigit\day}
%-----------------------------------------------------------------
\setlength{\parskip}{5pt}
\setlength{\parindent}{0pt}
\usepackage[parfill]{parskip}
%-----------------------------------------------------------------
% \usepackage{fancyhdr}
% \pagestyle{fancy}
% \setlength{\headwidth}{\textheight}
% \addtolength{\headwidth}{\columnsep}
% %\addtolength{\headwidth}{\marginparsep}
% %\addtolength{\headwidth}{\marginparwidth}
% \fancypagestyle{plain}{
% \fancyhead{} % clear all head fields 
% \fancyfoot{} % clear all foot fields
% \fancyfoot[RO,LE]{\textsf{\thepage}} 
% \fancyfoot[RE,LO]{\textsf{Draft of \today}}
% \renewcommand{\headrulewidth}{0.0pt}
% \renewcommand{\footrulewidth}{0.1pt}}
% \pagestyle{plain}
%-----------------------------------------------------------------
\usepackage{titling}
%\newfontfamily\titlefont[Scale=MatchUppercase]{Gill Sans MT}
%\renewcommand{\maketitlehooka}{\titlefont}
%\pretitle{\begin{flushright}\Huge\sffamily\bfseries}
\pretitle{\begin{flushright}\Huge\scshape\bfseries}
\posttitle{\par\end{flushright}\vskip 0.25em}
%\preauthor{\begin{flushright}\sffamily\scshape\mdseries}
\preauthor{\begin{flushright}\scshape\mdseries}
\postauthor{\par\end{flushright}}
%\predate{\begin{flushright}\sffamily\scshape\mdseries}
\predate{\begin{flushright}\scshape\mdseries}
\postdate{\par\end{flushright}}
\setlength{\droptitle}{-80pt}
%-----------------------------------------------------------------
\usepackage{enumitem}
%\setlist[description]{font=\small\sffamily\mdseries,style=unboxed,leftmargin=0cm}
%\setlist[description]{font=\sffamily\mdseries}
\setlist[description]{font=\scshape\bfseries}
\setlist[itemize]{style=unboxed,itemindent=0cm}
\setlist[enumerate]{style=unboxed,itemindent=0cm}
%-----------------------------------------------------------------
%\usepackage[sf,small,compact]{titlesec}
\usepackage[pagestyles,small,compact,sc,rigidchapters]{titlesec}
%\newfontfamily\headingfont[]{New Yorker}
%\newfontfamily\headingfont[Scale=MatchUppercase]{Libre Caslon Display}
%\newfontfamily\headingfont[]{Perpetua Titling MT}
%\newfontfamily\headingfont[Scale=MatchUppercase]{Gill Sans MT}
%\newfontfamily\headingfont[Scale=MatchUppercase]{Gillius ADF}

% \titleformat{\part}{\huge\sffamily\bfseries}{\thepart}{0.5em}{}
% \titleformat{\chapter}{\LARGE\sffamily\bfseries}{\thechapter}{0.5em}{}
% \titleformat{\section}{\Large\sffamily\bfseries}{\thesection}{0.5em}{}
% \titleformat{\subsection}{\large\sffamily\bfseries}{\thesubsection}{0.5em}{}
% \titleformat{\subsubsection}{\large\sffamily\mdseries}{\thesubsubsection}{0.5em}{}
% \titleformat{\paragraph}[runin]{\normalsize\sffamily\mdseries}{\theparagraph}{0.5em}{}[\hspace{1em}]
% \titleformat{\subparagraph}[runin]{\normalsize\sffamily\mdseries}{\thesubparagraph}{0.5em}{}[\hspace{1em}]

\titleformat{\part}{\huge\scshape\bfseries}{\thepart}{0.5em}{}
\titleformat{\chapter}{\LARGE\scshape\bfseries}{\thechapter}{0.5em}{}
%\titleformat{name=\chapter*,numberless}{\LARGE\scshape\bfseries}{\thechapter}{0.5em}{}
\titleformat{\section}{\Large\scshape\bfseries}{\thesection}{0.5em}{}
\titleformat{\subsection}{\large\scshape\bfseries}{\thesubsection}{0.5em}{}
\titleformat{\subsubsection}{\large\scshape\mdseries}{\thesubsubsection}{0.5em}{}
\titleformat{\paragraph}[runin]{\normalsize\scshape\mdseries}{\theparagraph}{0.5em}{}[\hspace{1em}]
\titleformat{\subparagraph}[runin]{\normalsize\scshape\mdseries}{\thesubparagraph}{0.5em}{}[\hspace{1em}]
%\titlespacing\chapter{0pt}{12pt plus 12pt minus 2pt}{5pt plus 5pt minus 2pt}
\titlespacing\section{0pt}{12pt plus 12pt minus 2pt}{5pt plus 5pt minus 2pt}
\titlespacing\subsection{0pt}{11pt plus 11pt minus 2pt}{5pt plus 5pt minus 2pt}
\titlespacing\subsubsection{0pt}{10pt plus 10pt minus 2pt}{5pt plus 5pt minus 2pt}
\newcommand{\chapterbreak}{\vfill\clearpage}
\setcounter{secnumdepth}{7}
%-----------------------------------------------------------------
\renewpagestyle{plain}{
\sethead[][\scshape\chaptertitle][] % even
{} {\scshape\chaptertitle} {} % odd
\setfoot[][\thepage][] % even
{}{\thepage}{} % odd
}
\newpagestyle{front}{
\sethead[][\scshape\chaptertitle][] % even
{} {\scshape\chaptertitle} {} % odd
\setfoot[][\thepage][] % even
{}{\thepage}{} % odd
}
\newpagestyle{back}{
\sethead[][\scshape\chaptertitle][] % even
{} {\scshape\chaptertitle}{} % odd
\setfoot[][\thepage][] % even
{}{\thepage}{} % odd
}
\renewpagestyle{empty}{
\sethead[][][] % even
{} {} {}% odd
}
\pagestyle{empty}
%-----------------------------------------------------------------
\makeatletter
\let\oldl@chapter\l@chapter
\def\l@chapter#1#2{\oldl@chapter{#1}{\textrm{#2}}}
\let\old@dottedcontentsline\@dottedtocline
\def\@dottedtocline#1#2#3#4#5{%
\old@dottedcontentsline{#1}{#2}{#3}{#4}{{\textrm{#5}}}}
\makeatother
%-----------------------------------------------------------------
%https://en.wikibooks.org/wiki/LaTeX/Indexing
\usepackage{makeidx}
\makeindex
\usepackage[totoc]{idxlayout}
%-----------------------------------------------------------------
\usepackage[
backend=biber, 
citestyle=numeric-comp, 
bibstyle=numeric,
%bibstyle=verbose,
%entrykey=false,
labelnumber=true,
sortcites=true,
maxnames=1000,
maxitems=1000,
block=nbpar,
abbreviate=false,
seconds=true,
date=iso,
alldates=iso,
datezeros=true,
timezeros=true,
]{biblatex} 
\renewcommand\mkbibnamefamily[1]{\textsc{#1}}
%-----------------------------------------------------------------
% \usepackage[chapter]{tocbibind}
% \renewcommand{\listfigurename}{Figures}
% \setlofname{Figures}
% \renewcommand{\listoffigures}{\begingroup
% \tocchapter
% \tocfile{\listfigurename}{lof}
% \endgroup}
%-----------------------------------------------------------------
%\usepackage[titletoc]{appendix}
%-----------------------------------------------------------------
\usepackage[unicode=true,
pdfusetitle,
bookmarks=true,
bookmarksnumbered=false,
bookmarksopen=true,
bookmarksopenlevel=1,
breaklinks=false,
pdfborder={0 0 0},
pdftoolbar=false,
pdffitwindow=true,
backref=false,
colorlinks=true]{hyperref}
\hypersetup{unicode=true,
colorlinks=true,
pdfpagemode=UseOutlines,
pdfpagelayout=OneColumn,
pdfstartview=Fit,
linkcolor=MidnightBlue,
urlcolor=Mahogany,
citecolor=OliveGreen}
\usepackage{bookmark}
%-----------------------------------------------------------------
% \usepackage[xindy,toc,style=alttreehypergroup,nolong,nosuper]{glossaries}
%-----------------------------------------------------------------
% doesn't do much for XeTeX
%\usepackage{microtype}
%-----------------------------------------------------------------
\def\M{{\mathcal M}}                   % A mesh
\def\V{{\mathcal V}}                   % vertices
\def\E{{\mathcal E}}                   % edges
\def\F{{\mathcal F}}                   % faces
\def\Ta{{\mathcal T}} % registration target
\def\I{{\mathbf I}}   % the identity transformation
\def\Tr{{\mathbf T}} % registration transform
\def\Eu{{\mathbf E}} % Euclidean transform
\def\Q{{\mathbf Q}} % Transform corresponding to quaternion
\def\R{{\mathbf R}} % Rotation
\def\G{{\mathbf G}} % riGid transform
\def\A{{\mathbf A}} % affine transform
\def\L{{\mathbf L}} % linear transform
\def\Sc{{\mathbf S}}    % scaled rotation or subdivision transform
\def\t{{\mathbf t}} % translation vector
\def\v{{\mathbf v}}                   % vertex
\def\e{{\mathbf e}}                   % edge
\def\f{{\mathbf f}}                   % face
\def\s{{\mathbf s}}                   % simplex
\def\w{{\mathbf w}}                   % vector of unconstrained weights
\def\P{{\mathbf P}}                   % vector of unconstrained weights
\def\u{{\mathbf u}}                   % vector of convex weights

\def\etal{{\it et al.}}

\def\Re{\mathcal{R}}    % set of Reals, $\Reals^3$
\def\Qs{\mathcal{Q}}    % set of Quaternions

\def\sign{\mathrm{sign}}    % sign function

\def\Pr{{\mathcal P}}   % projection

\def\Da#1{{\mathcal{D}{#1}}}    % derivative operator
\def\Db#1#2{{\mathcal{D}{#1}_{\mid_{#2}}}}    % derivative operator
\def\Dc#1#2#3{{\mathcal{D}{#1}_{\mid_{#2}}({#3})}}  % derivative operator
\def\Dd#1#2#3#4{{\mathcal{D}_{#1}{{#2}}_{\mid_{#3}}({{#4}})}} % derivative operator
\def\De#1#2#3{{\mathcal D}_{#1}{#2}_{\mid_{#3}}}  % derivative operator
\def\Df#1#2{{\mathcal D}_{#1}{#2}}  % derivative operator

\def\Ga#1{{\mathbf \nabla}{#1}}   % derivative operator
\def\Gb#1#2{{\mathbf \nabla}{#1}_{\mid_{#2}}} % derivative operator
\def\Gc#1#2#3{{\mathbf \nabla}_{#1}{{#2}}_{\mid_{#3}}}  % derivative operator
\def\Gf#1#2{{\mathbf \nabla}_{#1}{#2}}  % derivative operator

\def\norm#1{{\parallel{#1}\parallel}}   % l2 norm
\def\norm2#1{{\parallel{#1}\parallel^2}}  % squared l2 norm

\def\dn{{\mathbf \delta \n}} % change in normal
\def\nd{{\mathbf \n^\bullet}} % dot product of adjacent normals

\def\a{{\mathbf a}}     % area-weighted face normal
\def\l{{\mathbf l}}                   % Linear map as vector
\def\n{{\mathbf n}}     % unit length face normal
\def\x{{\mathbf x}}     % instance of point $\x$, $\x_i \in X$
\def\p{{\mathbf p}}     % generic point
\def\q{{\mathbf q}}     % generic point
\def\r{{\mathbf r}}     % generic point
\def\f{{\mathbf f}}     % generic vector-valued function
\def\g{{\mathbf g}}     % generic vector-valued function
\def\h{{\mathbf h}}     % generic vector-valued function
\def\c{{\mathbf c}}     % 3d cross product as function
\def\b{{\mathbf b}}     % barycentric coord. $\b_i$, $\b_{i,j}$
\def\a{{\mathbf a}}     % barycentric coord/affine combinations
\def\d{{\mathbf d}}     % data record
\def\e{{\mathbf e}}     % standard basis vectors
\def\y{{\mathbf y}}     % another barycentric coordinate
\def\z{{\mathbf z}}     % projection of a point
\def\S{{\mathbf S}}     % local subdivision matrix

%-----------------------------------------------------------------
% multipage quotes in boxes
\usepackage[framemethod=tikz]{mdframed}
\usepackage[most]{tcolorbox}
\usepackage{chngcntr}
\usetikzlibrary{calc}

\newlength\CapHt
\newlength\CapDp

\newtcolorbox{boxquote}[0]{%
spartan,
breakable,
enhanced,
freelance,
%frame code = {},
colframe=GhostWhite,
% frame code={
%   \draw[line width=0pt]
%     ([yshift=1pt]interior.north west-|frame.north west) --
%     ([yshift=1pt]interior.north east-|frame.north east) --
%     (frame.south east) --
%     (frame.south west) --
%     cycle;
%   },
colback=GhostWhite,
top=4mm,
arc=0pt,
outer arc=0pt,
pad at break=2mm,
fontupper=\normalsize,
notitle,
fontlower=\normalfont\small,
overlay middle={%
  \node[
    inner xsep=0pt,
    anchor=north east,
    font=\footnotesize\color{DarkGray}]
  at (frame.south east) {cont.};
  },
overlay first={%
  \node[
    inner xsep=0pt,
    anchor=north east,
    font=\footnotesize\color{DarkGray}]
  at (frame.south east) {cont.};
  },
% enlarge top by=\topskip,
% enlarge top at break by=0pt,
% enlarge bottom at break by=\baselineskip,
% attach boxed title to bottom left={yshift=2pt},
%   boxed title style={
%     enhanced,
%     arc=0pt,
%     outer arc=0pt,
%   },
%   boxed title style={size=small,colback=Cornsilk}
}
% \makeatletter
% % add tcblistings to \jobname.lol (list of listings)
% \tcbset{
%   addtolol/.style={list entry={\kvtcb@title},add to list={lol}{section}},
%   }
% \makeatother

\newtcbtheorem[
number within=section,
list inside={tcbdef}]
{tcbdefinition}
{Definition}%
{spartan,
breakable,
enhanced,
freelance,
colback=GhostWhite,
colframe=GhostWhite,
coltitle=black,
fonttitle=\scshape\bfseries}
{def}
%

\input{landscape-2col-header}
\input{clj-listings}
%-----------------------------------------------------------------
% The alttree type of glossary styles need to know the
 % widest entry name for each level
 \iftoggle{printglossary}
 {
\glssetwidest{Rational numbers} % level 0 widest name
\glssetwidest[1]{Homogeneous space}      % level 1 widest name
\makeglossaries

\newglossarystyle{cites}
{% based on list style
  \setglossarystyle{list}%
    \renewcommand*{\glossentry}[2]{%
    \item[\glsentryitem{##1}%
          \glstarget{##1}{\glossentryname{##1}}]
       \glossentrydesc{##1}\glspostdescription
    \ifglshasfield{useri}{##1}{\space
     % in the event of multiple cites (as in the vestibulum2
     % sample entry), \glsentryuseri{##1} needs to be expanded
     % before being passed to \cite.
     \glsletentryfield{\thiscite}{##1}{useri}%
     (See \expandafter\cite\expandafter{\thiscite})}{}%
    \space ##2}%
}

\newglossarystyle{citeshyper}
{% based on list style
  \setglossarystyle{alttreehypergroup}%
    \renewcommand*{\glossentry}[2]{%
    \item[\glsentryitem{##1}%
          \glstarget{##1}{\glossentryname{##1}}]
       \glossentrydesc{##1}\glspostdescription
    \ifglshasfield{useri}{##1}{\space
     % in the event of multiple cites (as in the vestibulum2
     % sample entry), \glsentryuseri{##1} needs to be expanded
     % before being passed to \cite.
     \glsletentryfield{\thiscite}{##1}{useri}%
     (See \expandafter\cite\expandafter{\thiscite})}{}%
    \space ##2}}
}
{}
%-----------------------------------------------------------------
\newglossaryentry{Sets}{
  name={Sets},
  text={Sets},
  description={\nopostdesc},
  %user1=Halmos:1960:NaiveSetTheory
  }

\newglossaryentry{Spaces}{
  name={Spaces},
  text={Spaces},
  description={\nopostdesc}}

  
\newglossaryentry{Set}{
  name={Set},
  text={set},
  first={a generic set},
  symbol={\Set{S}},
  description={a generic \emph{set}},
  sort=set,
  parent=Sets}
  
\newglossaryentry{HomogeneousSpace}{
  name={Homogeneous Space},
  text={Homogeneous space},
  %first={a generic set},
  %symbol={\ensuremath{\mathcal{S}}},
  description={a \emph{space} where every point looks the same},
  %sort=set,
  parent=Spaces}
  
\newglossaryentry{elementOf}{
  name={elementOf},
  text={elementOf},
  first={elementOf},
  description={$x \in \mathcal{S}$ means $x$ is an element of
  the set $\mathcal{S}$}, 
  sort=element,
  symbol={\ensuremath{\in}},
  parent=Sets}

\newglossaryentry{Integers}{
  name={integers},
  text={integers},
  symbol={\Space{Z}},
  description={the integers}}

\newglossaryentry{PositiveIntegers}{
  name={positive integers},
  text={positive integers},
  symbol={\ensuremath{\Space{Z}_{+}}},
  description={\SetSpec{i\glssymbol{elementOf}\glssymbol{Integers}}{i>0}},
  parent=Integers}

\newglossaryentry{NaturalNumbers}{
  name={natural numbers},
  text={natural numbers},
  symbol={\Space{N}},
  description={\SetSpec{i\glssymbol{elementOf}\glssymbol{Integers}}{i \geq 0}}}

\newglossaryentry{RationalNumbers}{
  name={rational numbers},
  text={rational numbers},
  symbol={\Space{Q}},
  description={
  \SetSpec{i/j}{
  i \glssymbol{elementOf} \glssymbol{Integers}, 
  j \glssymbol{elementOf} \glssymbol{PositiveIntegers}}}}

\newglossaryentry{RealNumbers}{
  name={real numbers},
  text={real numbers},
  symbol={\Space{R}},
  description={the real numbers}}

\newglossaryentry{DoublePrecisionFloat}{
  name={doubles},
  text={doubles},
  symbol={\Space{D}},
  description={the {IEEE} 754 64 bit floating point numbers}}

\newglossaryentry{SinglePrecisionFloat}{
  name={floats},
  text={floats},
  symbol={\Space{F}},
  description={the {IEEE} 754 32 bit floating point numbers}}

\newglossaryentry{GenericSpace}{
  name={a generic space},
  text={a generic space},
  symbol={\Space{S}},
  description={a generic space}}

\addbibresource{cactus.bib}
\addbibresource{clojure.bib}
\addbibresource{halmos.bib}
\addbibresource{ieeestd.bib}
\addbibresource{tex.bib}
\addbibresource{fparith.bib}
%-------------------------------------------------------------------------------
\title{Math Computing (working title)}
\author{\textsc{John Alan McDonald}}
%\email{mcdonald.john.alan at gmail.com}
\date{draft of \today}
%-------------------------------------------------------------------------------
\begin{document}

\maketitle

\frontmatter

\begingroup
\let\onecolumn\twocolumn
\sffamily
\tableofcontents
\rmfamily
\endgroup

\mainmatter

\part{Aspirations and Inspirations}
\chapter{Good writing}

\index{Good writing|see{Set}}
Primary goal of writing, whether in a natural language (eg
English), mathematics, or a programming language, is communicating
an idea to the reader.
This is (perhaps) obvious for natural and mathematical language,
but not univerally accepted for programming languages.

\cite{Halmos1970HowToWrite}

('Literate' programming confuses stream of consciousness writing
with communication with other people.)

Express the idea completely enough, without excess.
and specify no more than
necessary --- the he/she problem.
\index{Type problems@\textsl{Sam}!he/she@\textbf{he/she}}
Make it clear what you aren't making clear.

\part{Foundations}

Most readers will be familiar with most of what's here, 
but it's worth at least skimming, 
because the point of view is different.

\chapter{Mathematics}
\section{Ambiguity}
\section{Abstraction, instantiation, and representation}

pragmatic, semi-axiomatic definition of things.

\chapter{A model for computation}		

Transformation of data structures.

\section{Data Structures}

A \textit{data structure} is a collection of \textit{places}
that hold values. 
Using Clojure/Java for specificity.
A value might be primitive
(\texttt{int}, \texttt{double}, \ldots)
or a references to an 'independent' data structure.

\begin{example}[Java array]
Places correspond to \texttt{int}s from 0 to length minus 1.
Values restricted to instances of array's type.
\end{example}  

\begin{example}[Instance of Java class]
Places correspond to the class's fields.
Values restricted to each field's type.
\end{example}  

\begin{example}[EDN data]
Nested sequences and hashmaps. Similar to JSON.

Places correspond to the class's fields.
Values restricted to each field's type.
\end{example}  

\subsection{Mutability}
\subsection{Random access}

Means different things. 
Constant time to any place.
Absolute origin, rather than relative references.
Places known/easily enumerable, rather than requring traversal
to find out what places there are.
No general guarantee that you can visit everything, once.

\subsection{An unsolved problem}

Where's the boundary of a thing? 
Is a reference part of a path, or
an encapsulated value?
No language I know does this well.

Deep copy vs shallow copy.


\section{Code and data}

Data is just bits. 
Meaning determined by the code that manipulates it. 

Code and meaning always changing.

Correctness difficulty increases
with 'distance' between code and data. 

Multiple implementations of meaning really bad.
Data standards (eg IEEE 754 \cite{Higham2002ASNA, IEEE:1985:AIS,
P754:2008:ISF, Muller-et-al-2010}) major undertaking even for
small simple unchanging data semantics.

Services bad; shared libraries good.
Crossing programming language boundaries bad; 
monolingual environments \cite{Heering:1985:TMP:3318.3321} good.

\section{Types and prototypes}

The ``he or she'' problem: forced to specify irrelevant details.




\section{Clojure and Java}

\chapter{Comparisons}

\part{Spaces and functions between them}

I'm assuming that you are generally familiar with things like:
\begin{itemize}
  \item what a \gls{Set} is, at least in general terms.
  \item the usual number systems: 
the natural numbers, the integers, the rationals, and the reals, 
and the arithmetic operations ($+$, $-$, $*$, $/$) on them.
\item the difference between the
countable infinities of the natural, integral, and rational numbers,
and the difference from the uncountable infinity of the real numbers.
\end{itemize} 

If not, it would be good to review some basic source, like Wikipedia.

In the rest of this book, I'm going to use the word 'space' to refer to
mathematical structures, like the usual number systems, vector and affine
spaces, groups, fields, \ldots, that consist of one, or two, or a few, sets,
plus operations on the elements of those sets.

The first concrete examples of this will be the usual number systems 
in~\autoref{ch:Numbers}.

\input{sets}
%-----------------------------------------------------------------
\levelstay{Functions}
\label{sec:Functions}

In mathematics literature, \textit{function} is usually defined as
a special kind of \textit{relation} --- essentially a set of
ordered pairs with certain properties.

This approach is useful for some purposes, but here we will be
more interested in ``computable'' functions, at least in the sense
that a function is a ``machine'' that takes an element of one set
as input and returns an element of another, with the constraint
that a given input always returns the same output.

Because we need the notion of ``relation'' anyway, I'm going to
provide both definitions.

%-----------------------------------------------------------------
\leveldown{Relations}
\label{sec:Relations}

A \textit{relation}, 
$\Set{R}$, on $\Set{S}_0, \Set{S}_1, \ldots \Set{S}_{n-1}$,  
is any subset 
$\Set{R} \subseteq \Set{S}_0 \times \Set{S}_1 \times \ldots 
\times \Set{S}_{n-1}$,
that is, a set of tuples.

Ambiguity note: a given set of tuples, $\Set{R}$, can be
considered as a relation over many cartesian product sets.
The minimal such set is 
$\Set{R}_0 \times \Set{R}_1 \times \ldots \times
\Set{R}_{n-1}$, 
where 
$\Set{R}_i = \{ x | \exists r 
\in \Set{R} \text{ s.t. } r_i = x\}$.

A general \textit{binary relation} is a set of ordered
pairs $\Set{R} \subseteq \Set{S}_0 \times \Set{S}_1$.
It's common to write a binary relation as a predicate, 
$r(s_0,s_1) = ([s_0 \, s_1] \in \Set{R})$,
or $(r \, s_0 \, s_1)$ in pseudo-code,
or as a binary operation $s_0 \, R \, s_1$.

An important special case
are binary relations on $\Set{S}^2 = \Set{S} \times \Set{S}$
(often just written as ``binary relation on $\Set{S}$''). 
In this case we can define certain properties:

\begin{description}
\item[Transitive]
$r(s_0,s_1) \text{ and } r(s_1,s_2) \Rightarrow r(s_0,s_2)$.
\item[Reflexive] $r(s,s)$ is always true.
\item[Symmetric] For $s_0 \neq s_1$, $r(s_0,s_1)$ implies
$r(s_1,s_0)$.
\item[Antisymmetric] For $s_0 \neq s_1$, $r(s_0,s_1)$ implies not
$r(s_1,s_0)$.
\end{description}
These properties determine 2 important classes of relations:
\begin{description}
\item[Equivalence] Transitive, reflexive, and symmetric.
\item[Partial order] \label{def:partial-order}
Transitive, reflexive, and antisymmetric.
\end{description}

%-----------------------------------------------------------------
\levelstay{Functions}

A \textit{functional} binary relation, $\Set{F}$ on $\Set{X}
\times \Set{Y}$ has exactly one $y \in \Set{Y}$ for each
$x \in \Set{X}$ such that $[x \, y] \in \Set{F}$.
More conventional notation writes the \textit{function}, 
$f : \Set{X} \rightarrow \Set{Y}$ as $y = f(x)$.

Any function, $f : \Set{X} \rightarrow \Set{Y}$ defines an
equivalence relation on $\Set{X}$ via 
$\Set{E}_f = {[x_0 \, x_1] | f(x_0) = f(x_1)}$

%-----------------------------------------------------------------
\levelstay{Equivalence classes and quotient sets}

If $\Set{E}$ is an equivalence relation on $\Set{S}^2$, then we
can define the \textit{equivalence class} of $s_0$, $E(s_0) = \{
s_1 \in \Set{S} | [s_0 \, s_1] \in \Set{E} \}$.
The set of distinct equivalence classes partitions $\Set{S}$,
and is called the \textit{quotient set}: $\Set{S} / \Set{E}$.

In the case where the equivalence relation is derived from a
function, we write $\Set{S} / f$.

Equivalence classes and quotient sets will turn out to be
important. A common representational/implementation trick is to
use a larger but simpler (in some sense) space for calculations
which are meant to apply to a quotient space that actually has the
properties of interest. Homogeneous coordinates for affine and
projective spaces  are an important example.
The first one we will encounter here will be the rational numbers,
which are represented/implemented as pairs of integers
$[\text{numerator} \, \text{denominator}]$, but the actual
rational numbers are the equivalence classes defined by 
$\{ [[p_0 \, q_0] \, [p_1 \, q_1]] | p_0*q_1 = p_1*q_0 \}$,
that is, $[1 \, 2]$ and $[2 \, 4]$ represent the same rational.

%-----------------------------------------------------------------
\levelstay{Inverses and pseudo-inverses}
\label{sec:Inverses-and-pseudo-inverses}

Given a function $f : \Set{D} \mapsto \Set{C}$,
the \textit{inverse} of $f$ is 
$f^{-1}(c) = \SetSpec{d}{f(d) = c}$.
Note that $f^{-1} : \Set{C}  \mapsto \PowerSet{\Set(D)}$.

$\Set{L}\left( f , c \right) = f^{-1}(c)$ is a \textit{level set} of
$f$ at $c$.

For more symmetry, we can extend $f$ to 
$\PowerSet{\Set{D}}  \mapsto \PowerSet{\Set(C)}$
by defining
$f(\Set{S}) = 
\SetSpec{c}{\exists \, d \, \in \Set{S} \; \text{s.t.} \; c = f \left( d \right)}$

The usual definition of inverse treats $f^{-1}$
as a function from $\Set{C} \mapsto \Set{D}$,
which is undefined where the value of the true
inverse is not a set containing a single point.

\textbf{TODO:} When is a function 
$\PowerSet{\Set{D}}  \mapsto \PowerSet{\Set(C)}$
derivable from a function $\Set{D} \mapsto \Set{C}$?

%-----------------------------------------------------------------
\levelstay{Implicit functions}
\label{sec:Implicit-functions}

Suppose 
$f : \left( \Set{D}_0 \times \Set{D}_1 \right) \mapsto \Set{C}$.
Then each level set $f^{-1}\left( c \right)$ defines a relation on 
$\Set{R}_{f^{-1}(c)} \left( \Set{D}_0 , \Set{D}_1 \right)$.
If that relation is a function
(one $d_1$ paired to each $d_0$),
we call it an \textit{implicit function},
which we might write $g_{f^{-1}(c)} : \Set{D}_0 \mapsto \Set{D}_1$

Implicit functions are generally difficult to use,
because they don't tell us how to compute 
$g_{f^{-1}(c)} \left( d_0 \right)$.
Essentially, one has to use an iterativc zero-finding 
algorithm, which is difficult and non-robust except
for very low dimensional problems.
An alternative is to minimize something like
$\| c - f\left( d_0, d_1 \right) \|^2$ over $d_0$,
and check that the minimum value is close enough to $0$.




%-----------------------------------------------------------------
%\levelstay{Computable functions}

%-----------------------------------------------------------------
%\levelstay{implementation}

%-----------------------------------------------------------------
%\levelstay{examples}


\input{numbers}

\chapter{Simplexes}
\section{mathematics}
\subsection{Orientation}
\section{computation}
\section{examples}

\chapter{Linear (vector) spaces and functions}
\section{mathematics}
\cite{Halmos1960Finite}
\section{computation}
\section{examples}

\chapter{Affine (flat) spaces and functions}
\section{mathematics}
\section{computation}
\section{examples}

\chapter{Projective spaces and functions}
\section{mathematics}
\section{computation}
\section{examples}

\chapter{Barycentric (convex) spaces and functions}
\section{mathematics}
\section{computation}
\section{examples}

\chapter{Spherical spaces and functions}
\section{mathematics}
\section{computation}
\section{examples}


%\chapter{Notation and general results}
%\begin{plSection}{Notation and general results}
\label{sec:general}
%-----------------------------------------------------------------
\begin{plSection}{Identities for real vector operations}
\label{sec:RX}

See \citeAuthorYearTitle[p. 85, ex. 4-9]{Spivak:1965:CalculusOnManifolds}.

Let $\Vector{p}, \Vector{q} \in \Reals^{n}$.
Let $\theta(\Vector{p},\Vector{q})$ be the angle between $\Vector{p}$ and $\Vector{q}$.

\begin{itemize}
\item The inner (dot) product:
\begin{equation}
\Vector{p} \bullet \Vector{q} \; \equiv \; \sum_{i=0}^{n-1} p_i q_i
\end{equation}

\item The euclidean ($l_2$) norm:
\begin{equation}
\| \Vector{p} \|^2 \; \equiv \; \Vector{p} \bullet \Vector{p}
\end{equation}
\begin{equation}
\Vector{p} \bullet \Vector{q} \; = \; \| \Vector{p} \| \| \Vector{q} \| \cos(\theta(\Vector{p},\Vector{q}))
\end{equation}

\item Orthogonal complement:
\begin{equation}
\Vector{p} \perp \Vector{q} 
\; \equiv \; \Vector{p} 
\; - \; 
\left( 
\Vector{p} \bullet 
\frac{\Vector{q}}{\|\Vector{q}\|}
\right) 
\end{equation}

\item The tensor product

Let $\Vector{p} \in \Reals^m, \Vector{q}, \Vector{r} \in \Reals^n.$
$\Vector{p} \otimes \Vector{q}$ is a rank 1 linear transformation
from $\Reals^n$ to $\Reals^m$, defined by:
\begin{equation}
(\Vector{p} \otimes \Vector{q})(\Vector{r}) \; \equiv \; \Vector{p} (\Vector{q} \bullet \Vector{r})
\end{equation}

\end{itemize}

\end{plSection}%{Identities for real vector operations}
%-----------------------------------------------------------------
\begin{plSection}{Identities for 3-dimensional vector operations}
\label{sec:R3X}

See \citeAuthorYearTitle[p. 85, ex. 4-9]{Spivak:1965:CalculusOnManifolds}.

Let $\Vector{p}, \Vector{q}, \Vector{r} \in \Reals^3$.
Let $(p_0,p_1,p_2), (q_0,q_1,q_2), (r_0,r_1,r_2), $ be their coordinates
in some orthonormal basis.

The cross product:
\begin{equation}
\Vector{p} \times \Vector{q}  
\; \equiv \; 
(p_1 q_2 - p_2 q_1, \; p_2 q_0 - p_0 q_2, \; p_0 q_1 - p_1 q_0)
\end{equation}
\begin{equation}
\Vector{p} \times \Vector{q}  
\; = \; - \; 
\Vector{q} \times \Vector{p}
\end{equation}
\begin{equation}
\| \Vector{p} \times \Vector{q} \| \; = \; \| \Vector{p} \| \;
 \| \Vector{q} \| \; \sin(\theta(\Vector{p},\Vector{q}))
\end{equation}
\begin{equation}
\| \Vector{p} \times \Vector{q} \|  
\; = \;  
\sqrt{\| \Vector{p} \|^2 \| \Vector{q} \|^2
 \; - \; (\Vector{p} \bullet \Vector{q})^2}
\end{equation}
\begin{equation}
\Vector{p} \bullet ( \Vector{p} \times \Vector{q} ) 
\; = \; ( \Vector{p} \times \Vector{q} ) \bullet \Vector{q} \; = \; 0
\end{equation}
\begin{equation}
\label{eq:dot_cross}
\Vector{p} \bullet ( \Vector{q} \times \Vector{r} ) 
\; = \; ( \Vector{p} \times \Vector{q} ) \bullet \Vector{r} 
\; = \; \Vector{q} \bullet ( \Vector{r} \times \Vector{p} )
\end{equation}
\begin{equation}
\Vector{p} \times ( \Vector{q} \times \Vector{r} )
 \; = \; 
 ( \Vector{p} \bullet \Vector{r} ) \Vector{q}
  \; - \; (\Vector{p} \bullet \Vector{q}) \Vector{r}
\end{equation}
\begin{equation}
( \Vector{p} \times \Vector{q} ) \times \Vector{r} 
\; = \; 
( \Vector{p} \bullet \Vector{r} ) \Vector{q} 
\; - \; (\Vector{q} \bullet \Vector{r}) \Vector{p}
\end{equation}
\begin{equation}
( \Vector{p} \times \Vector{q} ) \times \Vector{r} 
\; = \;
 \left((\Vector{q} \otimes \Vector{p})
  - (\Vector{p} \otimes \Vector{q})\right) \Vector{r}
\end{equation}

\end{plSection}%{Identities for 3-dimensional vector operations}
%-----------------------------------------------------------------
\begin{plSection}{Functions on real vector spaces}
\label{sec:functions}

This paper describes functions of triangular meshes.

Interesting functions usually depend, directly or indirectly,
on the positions of some subset of the vertices.
I consider the vertex positions to be elements of $\Reals^3$,
with an (implied) universal origin,
and thus do not distinguish points and vectors.

In general, the functions discussed here map between real vector spaces:
$\Vector{f}:{\Re}^{n} \mapsto \Reals^{m}$, where $\Reals^n$ is the
{\it domain} and $\Reals^m$ is the {\it codomain}.
Strictly speaking, the {\it range} of $\Vector{f}$ is the set $\Vector{f}(\Reals^n)$,
which may be a proper subset of its codomain $\Reals^m$.

I typically use $\Vector{p}$, $\Vector{q}$, $\Vector{r}$, etc., for elements of $\Reals^n$
and
$\Vector{f}$, $\Vector{g}$, $\Vector{h}$ for vector-valued functions.
I generally do not distinguish $\Re$, the real numbers,
and $\Reals^1$, the 1-dimensional real vector space.
I sometimes use $f$, $g$, $h$ for extra clarity in the special
case of real-valued functions.

The domains of many interesting functions,
such as those that depend on vertex positions,
are direct sums of $\Reals^3$.
The {\it direct sum} $\Reals^n \oplus \Reals^m$ is the cartesian product
of $\Reals^n$ and $\Reals^m$ --- the set of ordered pairs $(\Vector{p},\Vector{q})$
where $\Vector{p} \in \Reals^n$ and $\Vector{q} \in \Reals^m$ ---
with the restriction that the inner product is the obvious extension of the
inner products on $\Reals^n$ and $\Reals^m$:
$(\Vector{p}_0,\Vector{q}_0) \bullet (\Vector{p}_1,\Vector{q}_1) = (\Vector{p}_0 \bullet \Vector{p}_1) + (\Vector{q}_0 \bullet \Vector{q}_1).$
For simplicity, I identify
$\Reals^{3n} = \Reals^3 \oplus \Reals^3 \oplus \cdots \oplus \Reals^3 = \oplus^n \Reals^3
= \Reals^n \oplus \Reals^n \oplus \Reals^n $.
I will usually write an element of $\oplus^n \Reals^3$ as
$(\Vector{p}_0,\ldots,\Vector{p}_{n-1})$
and use
$\Vector{f}(\Vector{p}_0,\Vector{p}_1,\ldots,\Vector{p}_{n-1})$
for a function that depends on $n$ vertices.
Sometimes it will be useful to separate the $x,$ $y,$ and $z$ coordinates:
$\Vector{p} = (\Vector{x},\Vector{y},\Vector{z}),$
where $\Vector{x} =(x_0, \ldots x_{n-1}) \in \Reals^n$, are the $x$-coordinates
of the positions of the vertices, and similarly for $y$ and $z$.

\end{plSection}%{Functions on real vector spaces}
%-----------------------------------------------------------------
\begin{plSection}{Derivatives}
\label{sec:derivatives}

One way to view the derivative of a function
$\Vector{f}:{\Re}^{n} \mapsto \Reals^{m}$,
at a point $\Vector{p}$,
is as the linear transformation 
$\Vector{L}:{\Re}^{n} \mapsto \Reals^{m}$,
that best approximates the local 'slope' of 
$\Vector{f}$ at $\Vector{p}$.
To be a little more precise, we want
\begin{displaymath}
\lim_{ \|\deltaBold\| \mapsto 0}
\frac{\|
\Vector{f}(\Vector{p} + \deltaBold) 
- \left( \Vector{f}(\Vector{p}) + \Vector{L}(\deltaBold) \right) 
\|}
{\|\deltaBold\|}
 = 0
\end{displaymath}
For a concise and correct discussion, see \citeAuthorYearTitle{Spivak:1965:CalculusOnManifolds}.

\begin{itemize}

\item $\Derivative{\Vector{f}}$

In its most general form,
I denote the derivative of $\Vector{f}$ 
by $\Derivative{\Vector{f}}$.
Note that this is
linear-transformation-valued function 
of the domain of $\Vector{f}$.

\item $\Derivative{\Vector{f}}[\Vector{p}]$

I denote the derivative of $\Vector{f}$ 
at $\Vector{p}$ by $\Derivative{\Vector{f}}[\Vector{p}]$.
$\Derivative{\Vector{f}}[\Vector{p}]$ 
is a specific linear transformation from
the domain of $\Vector{f}$ to the codomain of $\Vector{f}$.

\item $\Derivative{\Vector{f}}[\Vector{p}][\Vector{q}]$

The derivative is most often represented by the {\it Jacobian},
the $m \times n$ matrix of partial derivatives
with respect to some bases for $\Reals^n$ and $\Reals^m$.
However, it's often easier to express the derivative clearly if we
explicitly include the argument of the linear transformation.
In this case, I write $\Derivative{\Vector{f}}[\Vector{p}][\Vector{q}]$
for the derivative of $f$ at the point $\Vector{p}$
applied to the vector $\Vector{q}$.

\item $\Derivative[\Vector{p}_i]
{\Vector{f}}
[(\Vector{q}_0,\ldots,\Vector{q}_{n-1})]
[\Vector{r}_i]$

For functions on direct sum spaces,
$\Vector{f}(\Vector{p}_0,\Vector{p}_1,\ldots,
\Vector{p}_{n-1})$, 
$\Vector{p}_i \in \Reals^{n_i}$,
it's often easier to consider the derivative
with respect to one argument at a time.
I write 
$\Derivative[\Vector{p}_i]
{\Vector{f}}
[(\Vector{q}_0,\ldots,\Vector{q}_{n-1})]
[\Vector{r}_0,\ldots,\Vector{r}_{n-1}]$
for the derivative of $\Vector{f}$ 
with respect to $\Vector{p}_i$,
at the point 
$(\Vector{q}_0,\ldots,\Vector{q}_{n-1})
 \in \oplus_{i=0}^{n-1} \Reals^{m_i}$,
applied to the vector $\Vector{r}_i \in \Reals^{n_i}$.
Note that, if you consider $\Vector{f}$ to be a function
of direct sums of $\Reals^1$, we have the usual
partial derivatives.

\end{itemize}

%-----------------------------------------------------------------
\begin{plSection}{Gradients of real-valued functions}
\label{sec:gradients}

\begin{itemize}

\item $\Gradient{f}$

In minimizing real-valued functions, $f(\Vector{p})$, $\Vector{p} \in \Reals^n$,
we frequently need
the {\it gradient,} $\Gradient{f} \in \Reals^n$,
the vector pointing in the direction of most rapid increase of $f$,
whose magnitude is the rate of increase, or slope,
of $f$ in that direction.

The gradient, $\Gradient{f}$,
has a close relationship to the derivative, $\Derivative{f}$,
and the two are often confused.
Recall that the derivative is a linear transformation
from the domain of $f$ to its codomain.
In the case of real-valued functions,
this means the derivative is a linear function on $\Reals^n$,
an element of the dual space of $\Reals^n$, a 'row' vector.
It's easy to see that the gradient is simply the dual (the 'transpose')
of the derivative, $\Gradient{f} = (\Derivative{f})^{\dagger}$
(see \citeAuthorYearTitle[p. 96, ex. 4-18]{Spivak:1965:CalculusOnManifolds}).

Notation for the various versions of the gradient
follows that for derivatives:

\item $\Gradient{f}[\Vector{q}]$

The gradient of $f$ at $\Vector{q}$.

\item $\Gradient[\Vector{p}_i]{f}[\Vector{q}]$

The gradient of $f$
with respect to $\Vector{p}_i$ at $\Vector{q}$.

\item $(\Gradient{f}[\Vector{q}]) \bullet \; \Vector{r}$

The analog to expressing the derivative as a linear transformation
with an explicit argument is to write expressions for
the inner product of the gradient and an arbitrary other vector 
$\Vector{r}$

\item $(\Gradient[\Vector{p}_i]{f}[\Vector{q}]) \bullet \;\Vector{r}_i$

See above.

\end{itemize}

\end{plSection}%{Gradients of real-valued functions}
%-----------------------------------------------------------------
\begin{plSection}{Chain rule}
\label{sec:chain}

The most general identity used in computing derivatives i
s the {\it chain rule.}
Suppose
$\Vector{f}:\Reals^{n_0} \mapsto \Reals^{n_1}$,
$\Vector{g}:\Reals^{n_1} \mapsto \Reals^{n_2}$,
and
$\Vector{h} =
 \Vector{g} \circ \Vector{f} 
 : \Reals^{n_0} \mapsto \Reals^{n_2}.$
Then
\begin{equation}
\label{eq:chain-rule}
\Derivative{\Vector{h}}[\Vector{u}]
=  \Derivative{(\Vector{g} \circ \Vector{f})}[\Vector{v}]
=  \Derivative{\Vector{g}}[\Vector{f}(\Vector{v})]
  \circ  \Derivative{\Vector{f}}[\Vector{v}].
\end{equation}

See \citeAuthorYearTitle[Theorem~2-2]{Spivak:1965:CalculusOnManifolds}, .

\end{plSection}%{Chain rule}
%-----------------------------------------------------------------
\begin{plSection}{Derivatives of multilinear functions}
\label{sec:multilinear}

A function 
$\Vector{f}(\Vector{p}_0,\ldots,\Vector{p}_k) 
: \Reals^{n_0} \oplus \Reals^{n_k} \mapsto \Reals^m$
is {\it multilinear} if
\begin{equation}
\Vector{f}(
a_{00} \Vector{p}_{00} + a_{01} \Vector{p}_{01}, 
\ldots, 
a_{k0} \Vector{p}_{k0} + a_{k1} \Vector{p}_{k1})
\; = \; 
\sum_{i_0,\ldots,i_k = 0,1} \;
(a_{0i_0} \cdots a_{ki_k}) 
\Vector{f}(\Vector{p}_{0i_0}, \ldots, \Vector{p}_{ki_k}).
\end{equation}

The derivative of $\Vector{f}$
at the point $(\Vector{p}_0,\ldots,\Vector{p}_k)$, 
applied to the vector $(\Vector{q}_0,\ldots,\Vector{q}_k)$ is

\begin{equation}
\Derivative{\Vector{f}}
[(\Vector{p}_0,\ldots,\Vector{p}_k)]
[\Vector{q}_0,\ldots,\Vector{q}_k]
\; = \; 
\sum_{i=0,k} 
\Vector{f}(
\Vector{p}_0,
\ldots,
\Vector{p}_{i-1},
\Vector{q}_i,
\Vector{p}_{i+1},
\ldots,
\Vector{p}_k).
\end{equation}

See \citeAuthorYearTitle[ex.~2-14]{Spivak:1965:CalculusOnManifolds}, .

\end{plSection}%{Derivatives of multilinear functions}
%-----------------------------------------------------------------
\begin{plSection}{Derivatives of bilinear functions}
\label{sec:bilinear}

Bilinear functions are a useful special case 
of multilinear functions.

A function 
$\Vector{f}(\Vector{p},\Vector{q}) :
\Reals^{n_0} \oplus \Reals^{n_1} \mapsto \Reals^m$
is {\it bilinear} if
\begin{eqnarray}
\Vector{f}(
a_0 \Vector{p}_0 + a_1 \Vector{p}_1, 
b_0 \Vector{q}_0 + b_1 \Vector{q}_1) 
& = & a_0 b_0 f(\Vector{p}_0,\Vector{q}_0)  \\
& + & a_0 b_1 f(\Vector{p}_0,\Vector{q}_1) \nonumber \\
& + & a_1 b_0 f(\Vector{p}_1,\Vector{q}_0) \nonumber \\
& + & a_1 b_1 f(\Vector{p}_1,\Vector{q}_1).\nonumber
\end{eqnarray}

The derivative of $\Vector{f}$
at the point $(\Vector{p}_0,\Vector{q}_0)$, 
applied to the vector $(\Vector{p},\Vector{q})$ is

\begin{equation}
\Derivative{\Vector{f}}
[(\Vector{p}_0,\Vector{q}_0)]
[\Vector{p},\Vector{q}]
 = \Vector{f}(\Vector{p}_0,\Vector{q})
 + \Vector{f}(\Vector{p},\Vector{q}_0).
\end{equation}

See \citeAuthorYearTitle[ex.~2-12]{Spivak:1965:CalculusOnManifolds}.

\end{plSection}%{Derivatives of bilinear functions}
%-----------------------------------------------------------------
\begin{plSection}{Derivatives of linear functions}
\label{sec:Derivatives-of-linear-functions}

Linear functions are another useful special case of multilinear functions.
A function $\Vector{f}(\Vector{p}):\Reals^{n} \mapsto \Reals^m$
is {\it linear} if
\begin{equation}
\Vector{f}(a_0 \Vector{p}_0 + a_1 \Vector{p}_1)
 =
a_0 \Vector{f}(\Vector{p}_0) + a_1 \Vector{f}(\Vector{p}_1)
\end{equation}

The derivative of $\Vector{f}$ is simply $\Vector{f}$ itself.

\end{plSection}%{Derivatives of linear functions}
%-----------------------------------------------------------------
\begin{plSection}{Derivatives of inner products}
\label{sec:inner}

We can view the inner product on $\Reals^m$, $\Vector{p} \bullet \Vector{q}$,
as a bilinear function $d(\Vector{p},\Vector{q}) : \Reals^m \oplus \Reals^m \mapsto \Re$.
Thus
\begin{equation}
\Derivative{d}[(\Vector{p}_0,\Vector{q}_0)][\Vector{p},\Vector{q}]
 = \Vector{p}_0 \bullet \Vector{q} + \Vector{p} \bullet \Vector{q}_0.
\end{equation}

Suppose
$\Vector{f}:\Reals^{n} \mapsto \Reals^{m}$, and
$\Vector{g}:\Reals^{n} \mapsto \Reals^{m}$.
The derivative of $\Vector{f} \bullet \Vector{g}$ is:
\begin{eqnarray}
\label{eq:dot_derivative}
\Derivative{(\Vector{f} \bullet \Vector{g})}[\Vector{p}_0][\Vector{p}]
& =
& \Derivative{d}[(\Vector{f}(\Vector{p}_0),\Vector{g}(\Vector{p}_0))]
 \;\circ \;
 (\Derivative{\Vector{f}}[\Vector{p}_0][\Vector{p}], 
 \Derivative{\Vector{g}}[\Vector{p}_0][\Vector{p}])
\\
& =
& \Vector{f}(\Vector{p}_0) \bullet 
\Derivative{\Vector{g}}[\Vector{p}_0][\Vector{p}] 
\; + \; \Vector{g}(\Vector{p}_0) 
\bullet \Derivative{\Vector{f}}[\Vector{p}_0][\Vector{p}] \nonumber
\end{eqnarray}

See \citeAuthorYearTitle[ex.~2-13]{Spivak:1965:CalculusOnManifolds}.

\end{plSection}%{Derivatives of inner products}
%-----------------------------------------------------------------
\begin{plSection}{Derivatives of cross products}
\label{sec:cross}

We can view the 3-dimensional cross product
$ \times $
as a bilinear function
$\Vector{c}(\Vector{p},\Vector{q}) = 
\Vector{p} \times \Vector{q}
 : \Reals^3 \oplus \Reals^3 \mapsto \Reals^3$.
As with the inner product,
the derivative is
\begin{equation}
\Derivative{c}
[(\Vector{p}_0,\Vector{q}_0)]
[\Vector{p},\Vector{q}] 
= \Vector{p}_0 \times \Vector{q} 
+ \Vector{p} \times \Vector{q}_0.
\end{equation}

Suppose
$\Vector{f}:\Reals^{n} \mapsto \Reals^3$, and
$\Vector{g}:\Reals^{n} \mapsto \Reals^3$.
The derivative of $\Vector{f} \times \Vector{g}$ is:
\begin{eqnarray}
\Derivative{(\Vector{f} \times \Vector{g})}
[\Vector{p}_0][\Vector{p}]
& =
& \Derivative{\Vector{c}}
[(\Vector{f}(\Vector{p}_0),\Vector{g}(\Vector{p}_0))]
\;\circ \;
(\Derivative{\Vector{f}}[\Vector{p}_0][\Vector{p}],
 \Derivative{\Vector{g}}[\Vector{p}_0][\Vector{p}])
\\
& =
& \Vector{f}(\Vector{p}_0) 
\;\times \;
\Derivative{\Vector{g}}[\Vector{p}_0][\Vector{p}] 
\;+ \;
\Derivative{\Vector{f}}[\Vector{p}_0][\Vector{p}] 
\;\times \;
\Vector{g}(\Vector{p}_0) \nonumber
\end{eqnarray}

\end{plSection}%{Derivatives of cross products}
%-----------------------------------------------------------------
\begin{plSection}{Derivatives of scalar products}
\label{sec:scalar}

Suppose
$f:\Reals^{n} \mapsto \Re$, and
$\Vector{g}:\Reals^{n} \mapsto \Reals^m$.
It follows from the chain rule that the derivative of 
$\Vector{h} = f\Vector{g}$ is:
\begin{eqnarray}
\label{eq:scalar_product_derivative}
\Derivative{(f\Vector{g})}[\Vector{p}]
& = & f(\Vector{p})
\;\Derivative{\Vector{g}}[\Vector{p}] 
\;+ \Vector{g}(\Vector{p}) \; 
\Derivative{f}[\Vector{p}]  \\
& = & f(\Vector{p}) \;
\Derivative{\Vector{g}}[\Vector{p}] 
\;+ \Vector{g}(\Vector{p}) 
\otimes 
\Gradient{f}[\Vector{p}] \; \nonumber
\end{eqnarray}

\end{plSection}%{Derivatives of scalar products}
%-----------------------------------------------------------------
\begin{plSection}{Derivatives of euclidean norms}
\label{sec:norms}

Let $l_2(\Vector{p}) = \; \| \Vector{p}  \|
: \Reals^n \mapsto \Reals$ 
be the usual euclidean norm on $\Reals^n$.
Let $l_2^2(\Vector{p}) = \; \| \Vector{p}  \|^2 $
be its square
($ \| \Vector{p}  \|^2  = \sum_{i=0,n-1} \Vector{p}_i^2$),
and $ \| \Vector{p}  \|^3$ the cube.
\begin{eqnarray}
\label{eq:l2-gradient}
\Gradient{l_2}[\Vector{p}] 
& = & \frac{\Vector{p}}{\|\Vector{p}\|} \\
\Derivative{l_2}[\Vector{p}]
 & = & \frac{ \Vector{p}^\dagger }{ \|\Vector{p}\|} \nonumber \\
\Gradient{l_2^2}[\Vector{p}] & = & 2\Vector{p} \nonumber \\ 
\Derivative{l_2^2}[\Vector{p}] & = & 2\Vector{p}^\dagger \nonumber \\
\Gradient{l_2^3}[\Vector{p}] 
& = & 3 \| \Vector{p}  \| \Vector{p} \nonumber \\
\Derivative{l_2^3}[\Vector{p}] 
& = & 3 \| \Vector{p}  \| \Vector{p}^\dagger \nonumber
\end{eqnarray}

Let $\Vector{f}(\Vector{p}) : \Reals^n \mapsto \Reals^m$.
By the chain rule:
$\Derivative{\| \Vector{f} \|^2}[\Vector{p}]
=  
2 {\Vector{f}(\Vector{p})}^{\dagger}
 \Derivative{\Vector{f}}[\Vector{p}] $.

\begin{equation}
\Gradient{\| \Vector{f} \|^2}[\Vector{p}]  = 
 2 \;\Derivative{\Vector{f}}[\Vector{p}]^\dagger 
 \;\circ \;\Vector{f}(\Vector{p})
\end{equation}

\begin{eqnarray}
\label{eq:norm_derivative}
\Derivative{\| \Vector{f} \|}[\Vector{p}]
& = &
\frac{\Vector{f}(\Vector{p})^\dagger} 
{\| \Vector{f}(\Vector{p}) \|} 
\Derivative{\Vector{f}}[\Vector{p}]  \\
\Gradient{\| \Vector{f} \|}[\Vector{p}]
& = &
\left(\Derivative{\Vector{f}}[\Vector{p}]\right)^\dagger
\;\circ \;
\frac{\Vector{f}(\Vector{p})} 
{\|\Vector{f}(\Vector{p})\|}
\label{eq:norm_gradient}
\end{eqnarray}

\end{plSection}%{Derivatives of euclidean norms}
%-----------------------------------------------------------------
\begin{plSection}{Derivatives of normalized functions}
\label{sec:Derivatives-of-normalized-functions}

Let $\tilde{\Vector{f}}$ be the normalized version of 
$\Vector{f}$:
\begin{equation}
\tilde{\Vector{f}} \;= \;
\frac{\Vector{f}}{\|\Vector{f}\|}
\end{equation}

Then, from \cref{eq:scalar_product_derivative}
and \cref{eq:norm_derivative}:
\begin{eqnarray}
\Derivative{\tilde{\Vector{f}}}[\Vector{p}][\Vector{q}]
& = &
\Derivative
{\left(\frac{\Vector{f}}{\|\Vector{f}\|}\right)}
[\Vector{p}]
[\Vector{q}]
\\
& = &
\frac{\Derivative{\Vector{f}}[\Vector{p}][\Vector{q}]}
{\|\Vector{f}(\Vector{p})\|}
\; + \;
\Vector{f}(\Vector{p}) \; 
\Derivative
{\left[\frac{1}{\|\Vector{f}\|}\right]}
[\Vector{p}]
[\Vector{q}] \nonumber \\
& = &
\frac{\Derivative{\Vector{f}}[\Vector{p}][\Vector{q}]} 
{\| \Vector{f}(\Vector{p}) \|}
\; - \;
\Vector{f}(\Vector{p}) 
\frac{\Derivative{\| \Vector{f} \|}[\Vector{p}][\Vector{q}]}
{\|\Vector{f}(\Vector{p})\|^2} \nonumber \\
& = &
\frac{\Derivative{\Vector{f}}[\Vector{p}][\Vector{q}]}
{\|\Vector{f}(\Vector{p})\|}
\; - \;
\Vector{f}(\Vector{p})
\left( 
\frac{\Vector{f}(\Vector{p})^\dagger} 
{\|\Vector{f}(\Vector{p})\|^3} 
\;
\Derivative{\Vector{f}}[\Vector{p}][\Vector{q}] 
\right) \nonumber \\
& = &
\frac{
\| \Vector{f}(\Vector{p}) \|^2 
\Derivative{\Vector{f}}[\Vector{p}][\Vector{q}]
\; - \;
\Vector{f}(\Vector{p})
\left( 
\Vector{f}(\Vector{p}) 
\bullet 
\Derivative{\Vector{f}}[\Vector{p}][\Vector{q}] \right) 
}
{\| \Vector{f}(\Vector{p}) \|^3}  
\nonumber \\
& = &
\frac{
\| \Vector{f}(\Vector{p}) \|^2 
\Identity_{\Reals^3} 
\;- \;
\left( 
\Vector{f}(\Vector{p})
\otimes 
\Vector{f}(\Vector{p}) 
\right) 
}
{\| \Vector{f}(\Vector{p}) \|^3} 
\;
\Derivative{\Vector{f}}[\Vector{p}][\Vector{q}] 
\nonumber \\
& = &
\frac{
\Identity_{\Reals^3} 
\;- \;
\left( 
\tilde{\Vector{f}}(\Vector{p})
\otimes 
\tilde{\Vector{f}}(\Vector{p}) 
\right)  
}
{\| \Vector{f}(\Vector{p}) \|} 
\;\Derivative{\Vector{f}}[\Vector{p}][\Vector{q}] \nonumber
\end{eqnarray}

where $\otimes$ is the elementary tensor product operation.
If you are stuck thinking in terms of row and column vectors,
$\Vector{p} \otimes \Vector{q} \;= \;\Vector{p}\Vector{q}^\dagger$.
More generally, if $\Vector{p} \in \Reals^m$ and $\Vector{q} \in \Reals^n$,
then $\Vector{p} \otimes \Vector{q}$ 
is the rank 1 linear transformation from
 $\Reals^n \mapsto \Reals^m$:
$\left(\Vector{p} \otimes \Vector{q}\right) 
(\Vector{r}) \;= \;\Vector{p} \left(\Vector{q} \bullet \Vector{r}\right)$.

We can write the derivative above without reference 
to the argument $\Vector{q}$:
\begin{eqnarray}
\label{eq:normalized_function_derivative}
\Derivative{\tilde{\Vector{f}}}[\Vector{p}]
& = &
\Derivative
{\left[
\frac{\Vector{f}}{\|\Vector{f}\|}
\right]}
[\Vector{p}] \\
& = &
\frac{
\Identity_{\Reals^3} 
\;- \;
\left[
\tilde{\Vector{f}}(\Vector{p})
\otimes 
\tilde{\Vector{f}}(\Vector{p}) 
\right]
}
{\|\Vector{f}(\Vector{p})\|} 
\;
\Derivative{\Vector{f}}[\Vector{p}] \nonumber
\end{eqnarray}

A common, trivial, normalized function is the normalized version of
a vector:
\begin{equation}
\tilde{\Vector{p}} \;= \; \frac{\Vector{p}}{\|\Vector{p}\|}
\end{equation}

From equation \cref{eq:normalized_function_derivative}
it follows that:
\begin{eqnarray}
\label{eq:normalized_vector_derivative}
\Derivative{\tilde{\Vector{p}}}[\Vector{q}]
& = &
\Derivative
{\left(
\frac{\Vector{p}}{\|\Vector{p}\|}
\right)}
[\Vector{q}]
\\
& = &
\frac{
\Identity_{\Reals^3} 
\;- \;
\left( \tilde{\Vector{q}} \otimes \tilde{\Vector{q}} \right)
}
{\| \Vector{q} \|}
\nonumber
\\
& = &
\frac{
\|\Vector{q}\|^2
 \Identity_{\Reals^3} 
 \;- \;
 \left( \Vector{q} \otimes \Vector{q} \right) 
 }
{\| \Vector{q} \|^3} 
\nonumber
\end{eqnarray}

\end{plSection}%{Derivatives of normalized functions}
%-----------------------------------------------------------------
\begin{plSection}{Derivatives of angles}
\label{sec:derivatives-of-angles}

The angle between 2 vectors 
$\Vector{p}_0, \Vector{p}_1 \in \Reals^m$, 
is the inverse cosine
of their normalized inner product:
\begin{equation}
\theta(\Vector{p}_0,\Vector{p}_1)
=
\cos^{-1}
\left(
\frac{ \Vector{p}_0 \bullet \Vector{p}_1 } 
{\|\Vector{p}_0\| \|\Vector{p}_1\|}
\right)
\end{equation}
Recall that the derivative of the $\cos^{-1}$ is:
\begin{equation}
\frac{\mathit d}{\mathit dx} 
\cos^{-1}(x) 
= 
\frac{-1}{\sqrt{1-x^2}} 
\end{equation}
It follows that:
\begin{eqnarray}
\label{eq:angle_gradient}
\Gradient[\Vector{p}_0]{\theta(\Vector{p}_0,\Vector{p}_1)}[\Vector{q}]
& = &
\frac{-1}
{
\sqrt{1-
\left(
\frac{\Vector{q}_0 \bullet \Vector{q}_1} 
{\| \Vector{q}_0 \| \| \Vector{q}_1 \|}
\right)^2}
}
\Gradient[\Vector{p}_0]
{\left( 
\frac{\Vector{q}_0 \bullet \Vector{q}_1} 
{\| \Vector{q}_0 \| \| \Vector{q}_1 \|} 
\right)}
[\Vector{q}]
\\
& = &
\frac{-\|\Vector{q}_0\|\|\Vector{q}_1\|}
{ 
\sqrt{\|\Vector{q}_0\|^2\|\Vector{q}_1\|^2 
- \left( \Vector{q}_0 \bullet \Vector{q}_1 \right)^2 }
}
\left[
\frac{\Vector{q}_1}{\|\Vector{q}_0\|\|\Vector{q}_1\|}
+
\frac{
\left( 
\Vector{q}_0 \bullet \Vector{q}_1 
\right)
} 
{\| \Vector{q}1 \|}
\Gradient[\Vector{p}_0]
{\left(\frac{1}{\| \Vector{p}_0 \|} \right)} 
[\Vector{q}]
\right]
\nonumber
\\
& = &
\frac{-\|\Vector{q}_0\|\|\Vector{q}_1\|}
{ 
\sqrt{
\|\Vector{q}_0\|^2\|\Vector{q}_1\|^2 
\,-\,
\left( \Vector{q}_0 \bullet \Vector{q}_1 \right)^2 
}
}
\left[
\frac{\Vector{q}_1}{\|\Vector{q}_0\|\|\Vector{q}_1\|}
-
\frac{\left( \Vector{q}_0 \bullet \Vector{q}_1 \right) \Vector{q}_0} 
{\| \Vector{q}_1 \| \|\Vector{q}_0\|^3}
\right]
\nonumber
\\
& = &
\frac{-1}
{ 
\sqrt{
\|\Vector{q}_0\|^2\|\Vector{q}_1\|^2 
\,-\,
\left( \Vector{q}_0 \bullet \Vector{q}_1 \right)^2 
}
}
\left[
\Vector{q}_1
-
\frac{\left( \Vector{q}_0 \bullet \Vector{q}_1 \right) \Vector{q}0} 
{\|\Vector{q}_0\|^2}
\right]
\nonumber
\\
& = &
\frac{-\Vector{q}_1 \perp \Vector{q}_0}
{ \sqrt{
\|\Vector{q}_0\|^2\|\Vector{q}_1\|^2
 - \left( \Vector{q}_0 \bullet \Vector{q}_1 \right)^2 
 }
 }
\nonumber
\\
&  &
\nonumber
\\
\Gradient[\Vector{p}_1]{\theta(\Vector{p}_0,\Vector{p}_1)}[\Vector{q}]
& = &
\frac{- \Vector{q}_0 \perp \Vector{q}_1}
{ \sqrt{\|\Vector{q}_0\|^2\|\Vector{q}_1\|^2 
- \left( \Vector{q}_0 \bullet \Vector{q}_1 \right)^2 }}
\nonumber
\end{eqnarray}
\end{plSection}%{Derivatives of angles}
%-----------------------------------------------------------------
\end{plSection}%{Derivatives}
%-----------------------------------------------------------------
\end{plSection}%{Notation and general results}


%\part{Smoothness}
%\cleardoublepage

%\chapter{Polylines}
%\subsection{Vertex Bends}
\label{sec:Vertex-Bends}

\begin{figure}[!htp]
\centering
\begin{verbatim}
                   e1
             p1 o--------o p0
                        a \
                           \ e2
                            \
                             o p2
\end{verbatim}
\caption{Polyline vertex neighborhood.
\label{fig:Polyline-vertex-neighborhood}}
\end{figure}

In this section, I disuss measures of curvature
based on the bending at each vertex in a polyline.

%-----------------------------------------------------------------

\subsubsection{Cosine}
\label{sec:polyline-vertex-cosine}

Consider the non-boundary vertex
at point $\p_0 \in \Re^3$ in figure \ref{fig:Polyline-vertex-neighborhood}.
$\v$ has degree $2$;
the incident edges are labeled $\e_1$ and $e_2$;
and the neighboring vertices are at points $\p_1 \in \Re^3$
and $\p_2 \in \Re^3$.
The shape of the neighborhood is determined by
$\p = (\p_0, \p_1, \p_2) \in \Re^9$
The unsigned angle between edges $\e_1$ and $\e_2$ is $\alpha(\p)$.

One measure of the amount of bending is the cosine of $\alpha$:
\begin{equation}
\cos(\alpha(\p)) =
{{(\p_1 - \p_0)} \over {\| \p_1 - \p_0 \|} }
\bullet
{{(\p_2 - \p_0)} \over {\| \p_2 - \p_0 \|} }
\end{equation}

A little calculus shows that the partial gradients are:
\begin{eqnarray}
\label{eq:polyline-vertex-cosine-gradient}
\Gf{\p_0}{\cos(\alpha(\p))}
& = &
-
\left[
{{(\p_1 - \p_0) \perp  (\p_2 - \p_0)}
\over
{\| \p_1 - \p_0 \| \| \p_2 - \p_0 \|} }
+
{{(\p_2 - \p_0) \perp  (\p_1 - \p_0)}
\over
{\| \p_1 - \p_0 \| \| \p_2 - \p_0 \|} }
\right]
\\
\Gf{\p_1}{\cos(\alpha(\p))}
& = &
{{(\p_2 - \p_0) \perp  (\p_1 - \p_0)}
\over
{\| \p_1 - \p_0 \| \| \p_2 - \p_0 \|} }
\nonumber
\\
\Gf{\p_2}{\cos(\alpha(\p))}
& = &
{{(\p_1 - \p_0) \perp  (\p_2 - \p_0)}
\over
{\| \p_1 - \p_0 \| \| \p_2 - \p_0 \|} }
\nonumber
\end{eqnarray}

%-----------------------------------------------------------------

\subsubsection{Squared Cosine}
\label{sec:polyline-vertex-squared-cosine}

For any positive measure of bending,
minimizing the sum of squared bends,
rather than simply the sum of bends,
will tend to produce a more even distribution of curvature.
Since $\cos(\alpha)$ ranges over $[-1,1]$,
we use $\left( \frac{1 + \cos(\alpha)}{2} \right)^2$.

The gradient is:
\begin{equation}
\Gf{\p}{\left( {{1 + \cos(\alpha(\p))} \over 2} \right)^2}
=
\left( 1 + \cos(\alpha(\p)) \right)
\Gf{\p}{\cos\left(\alpha(\p)\right)}
\end{equation}

$\Gf{\p}{\cos\left(\alpha(\p)\right)}$ is given
in equation \ref{eq:polyline-vertex-cosine-gradient}.

%-----------------------------------------------------------------

\subsubsection{Angle}
\label{sec:polyline-vertex-angle}


The angle $\alpha$ is $\pi$ for a straight polyline,
and $0$ for a maximally bent vertex.
Therefore we may choose to minimize the sum of negative angles:
\begin{equation}
-\alpha(\p) =
-\cos^{-1} \left(
{{(\p_1 - \p_0)} \over {\| \p_1 - \p_0 \|} }
\bullet
{{(\p_2 - \p_0)} \over {\| \p_2 - \p_0 \|} }
\right)
\end{equation}

The gradient is:
\begin{equation}
\Gf{\p}{-\alpha(\p)}
=
{{1} \over {\sqrt{ 1 - \cos(\alpha(\p))^2}}}
\Gf{\p}{\cos\left(\alpha(\p)\right)}
\end{equation}

$\Gf{\p}{\cos\left(\alpha(\p)\right)}$ is given
in equation \ref{eq:polyline-vertex-cosine-gradient}.

%-----------------------------------------------------------------

\subsubsection{Squared Angle}
\label{sec:polyline-vertex-squared-angle}

Since $-\alpha$ ranges over $[-\pi,0]$
we square $\pi - \alpha$.

The gradient is:
\begin{equation}
\Gf{\p}{\left( \pi - \alpha(\p) \right)^2}
=
{{2 \left( \pi - \alpha(\p) \right)}
\over
{\sqrt{ 1 - \cos(\alpha(\p))^2}}}
\Gf{\p}{\cos\left(\alpha(\p)\right)}
\end{equation}

$\Gf{\p}{\cos \left( \alpha(\p) \right)}$ is given
in equation \ref{eq:polyline-vertex-cosine-gradient}.


%\cleardoublepage

%\chapter{Triangular Meshes}
%\subsection{Functions of faces}
\label{sec:faces}


%-------------------------------------------------------------------------------

\subsubsection{Corner angles}
\label{sec:corner_angles}

\begin{figure}[!htp]
\centering
\begin{verbatim}

          V2/p2
            o
           /U\
          / g2\
     E20 /     \ E12
        / F012  \
       /)g0   g1(\
V0/p0 o-----------o V1/p1
           E01

\end{verbatim}
\caption{Face labeling.
\label{fig:face_labeling}}
\end{figure}

Suppose face $\F_{012}$ has vertices $\V_0, \V_1, \V_2$,
at points $\p_0, \p_1, \p_2$,
and edges $\E_{01}, \E_{12}, \E_{20}$,
as labeled in figure \ref{fig:face_labeling}.

The {\em corner angle} $\gamma_i$ in face $\F_{012}$ of vertex $\V_i$ is
the angle between the two edges in $\F_{012}$ that meet at $\V_i$:
\begin{eqnarray}
\gamma_i
& = & \gamma(\F_{012},\V_i)
\\
& = & \gamma(\p_i,\p_{(i+1) \% 3},\p_{(i+2) \% 3})
\nonumber
\\
& = & \theta(\p_{(i+1) \% 3} - \p_i,\p_{(i+2) \% 3} - \p_i)
\nonumber
\\
& = &
\cos^{-1}
\left[
{ \left( \p_{(i+1) \% 3} - \p_i \right)
  \bullet
  \left( \p_{(i+2) \% 3} - \p_i \right) }
\over
{ \| \p_{(i+1) \% 3} - \p_i \|
  \| \p_{(i+2) \% 3} - \p_i \| }
\right]
\nonumber
\end{eqnarray}

Corner angles vary between $0$ and $\pi$, with both extremes
corresponding to singular, zero-area faces.
The sum of corner angles for a given face is always $\pi$.

Using equation \ref{eq:angle_gradient},
it's easy to see that the partial gradients are:
\begin{eqnarray}
\Gc{\p_0}{\gamma(\p_0,\p_1,\p_2)}{\q}
& = &
{{
\left[ (\q_1 -\q_0) \perp (\q_2 - \q_0) \right]
+
\left[ (\q_2 -\q_0) \perp (\q_1 - \q_0) \right]
}
\over
{ \sqrt{\|\q_1 - \q_0\|^2\|\q_2 - \q_0\|^2 -
\left( (\q_1 - \q_0) \bullet (\q_2 - \q_0) \right)^2 }}
}
\\
\Gc{\p_1}{\gamma(\p_0,\p_1,\p_2)}{\q}
& = &
{{-(\q_1 -\q_0) \perp (\q_2 - \q_0)}
\over
{ \sqrt{\|\q_1 - \q_0\|^2\|\q_2 - \q_0\|^2 -
\left( (\q_1 - \q_0) \bullet (\q_2 - \q_0) \right)^2 }}
}
\nonumber
\\
\Gc{\p_2}{\gamma(\p_0,\p_1,\p_2)}{\q}
& = &
{{-(\q_2 -\q_0) \perp (\q_1 - \q_0)}
\over
{ \sqrt{\|\q_1 - \q_0\|^2\|\q_2 - \q_0\|^2 -
\left( (\q_1 - \q_0) \bullet (\q_2 - \q_0) \right)^2 }}
}
\nonumber
\end{eqnarray}

%-------------------------------------------------------------------------------

\subsubsection{Functions of face normals}
\label{sec:normals}

A number of important functions of triangular meshes,
such as surface area,
are based on face normal vectors.

%-------------------------------------------------------------------------------

\paragraph{Area-weighted face normal}
\label{sec:areanormal}

Suppose we have a face whose 3 vertices are at $\p = (\p_0, \p_1, \p_2)$,
where $\p_i \in \Re^3; i=0,1,2.$
(Note that the order of the $\p_i$ determines the orientation of the face.
With a face as labeled in figure \ref{fig:face_labeling},
the normal points out of the page.)

The {\it area-weighted normal} vector is
\nopagebreak
\begin{eqnarray}
\a (\p) & = & (\p_0 \times \p_1) \ + \ (\p_1 \times \p_2) \ + \ (\p_2 \times \p_0) \\
        & = & (\p_1 - \p_0) \ \times \ (\p_2 - \p_0) \nonumber \\
        & = & (\p_2 - \p_1) \ \times \ (\p_0 - \p_1) \nonumber \\
        & = & (\p_0 - \p_2) \ \times \ (\p_1 - \p_2) \nonumber
\end{eqnarray}

The 'partial' derivatives of the area-weighted normal are:
\begin{eqnarray}
\Dd{\p_0}{\a}{\q}{r_0} \
& = \ (\r0 \times \q_1) \ + \ (\q_2 \times \r_0) & = (\q_2 - \q_1) \times \r_0 \\
\Dd{\p_1}{\a}{\q}{r_1} \
& = \ (\r1 \times \q_2) \ + \ (\q_0 \times \r_1) & = (\q_0 - \q_2) \times \r_1 \nonumber \\
\Dd{\p_2}{\a}{\q}{r_2} \
& = \ (\r2 \times \q_0) \ + \ (\q_1 \times \r_2) & = (\q_1 - \q_0) \times \r_2 \nonumber
\end{eqnarray}

Note that $\Df{\p}{\a}$ is {\it skew-symmetric}, that is,
$\Df{\p}{\a}^{\dagger} = -\Df{\p}{\a}.$

%-------------------------------------------------------------------------------

\paragraph{Face area}
\label{sec:facearea}

The area of a face is half the length of the area-weighted normal:
\begin{eqnarray}
A(\p)
& = & {1 \over 2} \| \ \a(\p) \ \|  \\
& = & {1 \over 2} \| \ (\p_0 \times \p_1) \ + \ (\p_1 \times \p_2) \ + \ (\p_2 \times \p_0) \ \|.
\nonumber
\end{eqnarray}

It follows from equation \ref{eq:norm_derivative}
that the first 'partial' derivative of the face area is:
\begin{eqnarray}
\label{eq:area_partial_derivative}
\Dd{\p_0}{A}{\q}{\r_0}
& = &
{{\a(\q)^\dagger} \over {2\|\a(\q)\|}}
{\Dd{\p_0}{\a}{\q}{\r_0}}  \\
& = &
{{\a(\q)} \over {2\|\a(\q)\|}}
\bullet
\left[(\r_0 \times \q_1) + (\q_2 \times \r_0)\right] \nonumber \\
& = &
{{\a(\q)} \over {2\|\a(\q)\|}}
\bullet
\left[(\q_2 - \q_1) \ \times \  \r_0)\right] \nonumber \\
& = &
{{\a(\q) \times (\q_2 - \q_1)} \over {2\|\a(\q)\|}}
\bullet
\r_0 \nonumber
\end{eqnarray}

The last identity follows from equation \ref{eq:dot_cross}.

The 'partial' gradients of the face area are then:
\begin{eqnarray}
\Gc{\p_0}{A}{\q} & = & {{\a(\q) \times (\q_2 - \q_1)} \over {2\|\a(\q)\|}} \\
\Gc{\p_1}{A}{\q} & = & {{\a(\q) \times (\q_0 - \q_2)} \over {2\|\a(\q)\|}} \nonumber \\
\Gc{\p_2}{A}{\q} & = & {{\a(\q) \times (\q_1 - \q_0)} \over {2\|\a(\q)\|}} \nonumber
\end{eqnarray}

More simply, using the face unit normal \( \n(\p)  =  {{\a(\p)} \over {\| \a(\p) \|}} \):
\begin{eqnarray}
\label{eq:area_gradient}
\Gc{\p_0}{A}{\q} & = & \frac{\n(\q)}{2} \times (\q_2 - \q_1) \\
\Gc{\p_1}{A}{\q} & = & \frac{\n(\q)}{2} \times (\q_0 - \q_2) \nonumber \\
\Gc{\p_2}{A}{\q} & = & \frac{\n(\q)}{2} \times (\q_1 - \q_0) \nonumber
\end{eqnarray}

%-------------------------------------------------------------------------------

\paragraph{Face unit normal vector}
\label{sec:facenormal}

The unit vector normal to a face whose vertices are at
$\p = (\p_0,\p_1,\p_2)$ is just the area weighted normal (see section \ref{sec:areanormal})
adjusted to length 1:
\begin{equation}
\n(\p)  =  {{\a(\p)} \over {\| \a(\p) \|}}
\end{equation}

Following equation \ref{eq:normalized_function_derivative}, the derivative is:

\begin{eqnarray}
\label{eq:unit_normal_derivative}
\Dc{\n}{\p}{\q}
&  =
& { \left( {\| \a(\p) \|^2 \I_{\Re^3}  -  \a(\p) \otimes \a(\p) } \over {\| \a(\p) \|^3} \right) }
\; \Dc{\a}{\p}{\q}
 \\
& & \nonumber\\
&  =
& { \left( {\| \a(\p) \|^2 \I_{\Re^3}  -  \a(\p) \otimes \a(\p) } \over {\| \a(\p) \|^3} \right)} \ast
\nonumber \\
&    &
\left[ \left( \q_0 \times \p_1 \right) + \left( \p_2 \times \q_0 \right)
+
\left( \q_1 \times \p_2 \right) + \left( \p_0 \times \q_1 \right)
+
\left( \q_2 \times \p_0 \right) + \left( \p_1 \times \q_2 \right) \right]
\nonumber \\
& & \nonumber\\
&  =
& { \left( {\| \a(\p) \|^2 \I_{\Re^3}  -  \a(\p) \otimes \a(\p) } \over {\| \a(\p) \|^3} \right)} \ast
\nonumber \\
&    &
\left[ \left( (\p_2 - \p_1) \times \q_0 \right)
+
\left( (\p_0 - \p_2) \times \q_1 \right)
+
\left( (\p_1 - \p_0) \times \q_2 \right) \right]
\nonumber \\
& & \nonumber\\
&  =
& { \left( {\I_{\Re^3}  -  \n(\p) \otimes \n(\p) } \over {\| \a(\p) \|} \right)} \ast \; \Dc{\a}{\p}{\q}
\nonumber
\end{eqnarray}

%-------------------------------------------------------------------------------

\subsubsection{Aspect Ratio}
\label{sec:aspect_ratio}

Minimizing a measure of face aspect ratio can help maintain
a well-conditioned mesh.
Maximizing it may help in discovering collapse-able edges.

%-------------------------------------------------------------------------------

\paragraph{Squared edge lengths over area}
\label{sec:Squared-edge-lengths-over-area}

One measure of the deviation of a face from equilaterality is:
\begin{equation}
{\mathrm L2A}(\p_0,\p_1,\p_2)
=  {{ \| \p_0 - \p_1 \|^2 + \| \p_1 - \p_2 \|^2 + \| \p_2 - \p_0 \|^2 }
\over
{A(\p)} }
\end{equation}

Using \ref{eq:area_gradient}, it follows that the
partial gradients of L2A are:
\begin{eqnarray}
\label{eq:L2A_gradient}
\Gc{\p_0}{L2A}{\q}
& =
&
\left(
\frac{2\left[ \left( \p_0 - \p_1 \right) + \left( \p_0 - \p_2 \right) \right]}
{A(\q)}
\right)
\\
& - &
\left(
\frac{ \| \p_0 - \p_1 \|^2 + \| \p_1 - \p_2 \|^2 + \| \p_2 - \p_0 \|^2 }
{2 A^2(\q)}
\left[ \n(\q) \times (\q_2 - \q_1)
\right]
\right)
\nonumber \\
\Gc{\p_1}{L2A}{\q}
& =
&
\left(
\frac{2\left[ \left( \p_1 - \p_2 \right) + \left( \p_1 - \p_0 \right) \right]}
{A(\q)}
\right)
\nonumber
\\
& - &
\left(
\frac{ \| \p_0 - \p_1 \|^2 + \| \p_1 - \p_2 \|^2 + \| \p_2 - \p_0 \|^2 }
{2 A^2(\q)}
\left[ \n(\q) \times (\q_0 - \q_2) \right]
\right)
\nonumber
\\
\Gc{\p_2}{L2A}{\q}
& =
&
\left(
\frac{2\left[ \left( \p_2 - \p_0 \right) + \left( \p_2 - \p_1 \right) \right]}
{A(\q)}
\right)
\nonumber
\\
& - &
\left(
\frac{ \| \p_0 - \p_1 \|^2 + \| \p_1 - \p_2 \|^2 + \| \p_2 - \p_0 \|^2 }
{2 A^2(\q)}
\left[ \n(\q) \times (\q_1 - \q_0) \right]
\right)
\nonumber
\end{eqnarray}
%\begin{plSection}{Functions of the edges}
\label{sec:edges}

\begin{plDiagram}
{Edge face pair labeling.}
{EdgeFaces}
\centering
\begin{verbatim}
          p1
          o
         /|\
        / | \
       /  |  \e31
   e12/   |   \
     /    |e01 \
    /     |     \
p2 o f012 | f031 o p3
    \     |     /
     \    |    /
   e20\   |   /e03
       \  |  /
        \ | /
         \|/
          o
          p0
\end{verbatim}
\end{plDiagram}

Notation in this section is based on \cref{diagram:EdgeFaces}.
We are discussing functions defined on a neighborhood of edge $e_{01}$.

We assume that, for each edge, an arbitrary order is assigned to
its two vertices, which are then at the positions $\Point{p}_0,\Point{p}_1$ in the diagram.

An interior edge has 2 adjacent faces, $f_{012}$ and $f_{031}$.
We assume that these 2 faces are oriented consistently, with the labels
taken counterclockwise, so that the normal vectors point out of the page.
Each face is represented by an ordered triple of vertices,
but the order is only determined up to a circular permutation;
for example, $f_{012}$ may be represented by the ordered triples
$(\Point{p}_0,\Point{p}_1,\Point{p}_2)$, 
$(\Point{p}_2,\Point{p}_0,\Point{p}_1)$, or 
$(\Point{p}_1,\Point{p}_2,\Point{p}_0)$,
but not by
$(\Point{p}_0,\Point{p}_2,\Point{p}_1)$, 
$(\Point{p}_1,\Point{p}_0,\Point{p}_2)$, or
 $(\Point{p}_2,\Point{p}_1,\Point{p}_0)$.

For the given ordering $(\Point{p}_0,\Point{p}_1)$ of the edge,
$f_{120}$ is the edge's {\it left face}
and $f_{031}$ is the {\it right face}.

Note that we cannot assume any consistent ordering of the 4 neighboring edges;
for example, $e_{12}$ may be represented by either ordered pair
$(\Point{p}_1,\Point{p}_2)$ or $(\Point{p}_2,\Point{p}_1)$.

%-----------------------------------------------------------------
\begin{plSection}{Edge length}
\label{sec:edge_length}

The edge tangent vector is $\Point{p}_1 - \Point{p}_0$.

The gradient of its squared length is:
\begin{equation}
\Gc{\Point{p}_i}{\| \Point{p}_1 - \Point{p}_0 \|^2}{\Point{q}} 
= 2 \left( \Point{p}_i - \Point{p}_{(i+1) \bmod 1} \right)
\end{equation}

The gradient of the edge length, $\|\Point{p}_1 - \Point{p}_0\|$ is:
\begin{equation}
\Gc{\Point{p}_i}{\| \Point{p}_1 - \Point{p}_0 \|}{\Point{q}} =
\frac{\left( \Point{p}_i - \Point{p}_{(i+1) \bmod 1} \right)}
{\|\Point{p}_1 - \Point{p}_0\|}
\end{equation}

\end{plSection}%{Edge length}
%-----------------------------------------------------------------
\begin{plSection}{Difference in face normals}
\label{sec:normal_difference}

One measure of the change in surface normal across an edge
is simply the vector difference of the two normals:

\begin{equation}
\label{eq:deltan}
\DNormal{n} (\Point{p}_0, \Point{p}_1, \Point{p}_2, \Point{p}_3)
=
\Normal{n} (\Point{p}_{012}) - \Normal{n} (\Point{p}_{031})
\end{equation}

The (total) derivative of the squared distance between adjacent face normals is:
\begin{eqnarray}
\Db{\|\DNormal{n}(\Point{p})\|^2}{\Point{q}}
& =
2 \ \DNormal{n} ( \Point{q} )^\dagger &
\left( \Db{ ( \DNormal{n} ) }{\Point{q}} \right)
\\
& =
2 \ \DNormal{n}(\Point{q})^\dagger &
\left( \Db{\Normal{n}(\Point{p}_{012})}{\Point{q}} - \Db{\Normal{n}(\Point{p}_{031})}{\Point{q}} \right)
\nonumber \\
& =
2 \DNormal{n}(\Point{q})^\dagger &
\{ \; \left[ \Identity_{\Reals^3} - \left( \Normal{n}( \Point{q}_{012} ) \otimes \Normal{n}( \Point{q}_{012} ) \right)
\right]
\ast \Db{\Vector{a} \left( \Point{p}_{012} \right) }{\Point{q}}
\nonumber \\
\label{eq:deltan_derivative}
&
& - \left[ \Identity_{\Reals^3} - \left( \Normal{n}( \Point{q}_{031} ) \otimes \Normal{n} ( \Point{q}_{031} ) \right)
\right]
\ast \Db{\Vector{a} ( \Point{p}_{031} ) }{\Point{q}}
\; \}
\nonumber
\end{eqnarray}

The partial derivatives, with respect to one of the vertices,
like $\Dd{\Point{p}_0}{\Vector{a} ( \Point{p}_{012} ) }{\Point{q}}{\Point{r}_0}$,
all have a similar form:
\begin{equation}
\Dd{\Point{p}_0}{\Vector{a} ( \Point{p}_{012} ) }{\Point{q}}{\Point{r}_0}  = (\Point{q}_1 - \Point{q}_3) \times \Point{r}_0
\end{equation}
Using this, equation \ref{eq:deltan_derivative}, equation \ref{eq:dot_cross},
and the facts that
$\DNormal{n}(\Point{q})  \perp  \Normal{n}(\Point{q}_{012}) = - \left( \Normal{n}(\Point{q}_{031})  \perp  \Normal{n}(\Point{q}_{012}) \right)$
and
$\DNormal{n}(\Point{q})  \perp  \Normal{n}(\Point{q}_{031}) = \Normal{n}(\Point{q}_{012})  \perp  \Normal{n}(\Point{q}_{031})$,
we can write the partial gradients without reference to the
derivative's argument $\Point{r}$:
\begin{eqnarray}
\label{eq:normal-difference-gradient}
\Gc{\Point{p}_0}{\|\DNormal{n}\|^2}{\Point{q}}
& = &
\left[
\frac{{ \Normal{n}(\Point{q}_{031})  \perp  \Normal{n}(\Point{q}_{012}) }
{A(\Point{q}_{012})}}
\times (\Point{q}_1 - \Point{q}_2)
\right]
\; + \;
\left[
\frac{{ \Normal{n}(\Point{q}_{012})  \perp  \Normal{n}(\Point{q}_{031}) }
{A(\Point{q}_{031})}}
\times (\Point{q}_3 - \Point{q}_1)
\right]
\\
\Gc{\Point{p}_1}{\|\DNormal{n}\|^2}{\Point{q}}
& = &
\left[
\frac{{ \Normal{n}(\Point{q}_{031})  \perp  \Normal{n}(\Point{q}_{012}) }
{A(\Point{q}_{012})}}
\times (\Point{q}_2 - \Point{q}_0)
\right]
\; + \;
\left[
\frac{{ \Normal{n}(\Point{q}_{012})  \perp  \Normal{n}(\Point{q}_{031}) }
{A(\Point{q}_{031})}}
\times (\Point{q}_0 - \Point{q}_3)
\right]
\nonumber
\\
\Gc{\Point{p}_2}{\|\DNormal{n}\|^2}{\Point{q}}
& = &
\left[
\frac{{ \Normal{n}(\Point{q}_{031})  \perp  \Normal{n}(\Point{q}_{012}) }
{A(\Point{q}_{012})}}
\times (\Point{q}_0 - \Point{q}_1)
\right]
\nonumber
\\
\Gc{\Point{p}_3}{\|\DNormal{n}\|^2}{\Point{q}}
& = &
\left[
\frac{{ \Normal{n}(\Point{q}_{012})  \perp  \Normal{n}(\Point{q}_{031}) }
{A(\Point{q}_{031})}}
\times (\Point{q}_1 - \Point{q}_0)
\right]
\nonumber
\end{eqnarray}

\end{plSection}%{Difference in face normals}
%-----------------------------------------------------------------
\begin{plSection}{Inner product between face normals}
\label{sec:normal_dot}

The inner product $\left( \Normal{n}_{012} \bullet \Normal{n}_{031} \right)$
is another important measure of edge curvature.
It is closely related to the squared distance between adjacent normals:
\begin{equation}
\label{eq:normal-distance-dot}
\| \Normal{n}_{012} - \Normal{n}_{031} \|^2
= \| \Normal{n}_{012} \|^2
+ \| \Normal{n}_{031} \|^2
- 2 \left( \Normal{n}_{012} \bullet \Normal{n}_{031} \right)
= 2 \left[ 1 - \left( \Normal{n}_{012} \bullet \Normal{n}_{031} \right) \right]
\end{equation}

The function $f(\Point{p}) = 1 - \left( \Normal{n}_{012} \bullet \Normal{n}_{031} \right)$
achieves its minimum, $0$, on flat face pairs,
and its maximum, $2$, on face pairs that are folded back on themselves.
It's a reasonable choice the total bending or curvature of a surface.
And $\Da{f} = - \Da{\left( \Normal{n}_{012} \bullet \Normal{n}_{031} \right)}$.

The derivative of
$\left( \Normal{n}_{012} \bullet \Normal{n}_{031} \right)$
can be calculated using equations \ref{eq:dot_derivative} and
\ref{eq:unit_normal_derivative}:
\begin{eqnarray}
\label{normal_dot_derivative}
\Db{\left( \Normal{n}_{012} \bullet \Normal{n}_{031} \right)}{\Point{q}}
& = & \Normal{n}(\Point{q}_{031}) \bullet \Db{\Normal{n}_{012}}{\Point{q}} + \Normal{n}(\Point{q}_{012}) \bullet \Db{\Normal{n}_{031}}{\Point{q}}
\\
\nonumber \\
& = &
\Normal{n}(\Point{q}_{031}) \bullet
\frac{\Identity - \left(\Normal{n}(\Point{q}_{012}) \otimes \Normal{n}(\Point{q}_{012}) \right)}{\| \Vector{a}(\Point{q}_{012}) \|}
\; \Db{\Vector{a}_{012}}{\Point{q}}
\nonumber \\
& + &
\Normal{n}(\Point{q}_{012}) \bullet
\frac{\Identity - \left(\Normal{n}(\Point{q}_{031}) \otimes \Normal{n}(\Point{q}_{031}) \right)}{\| \Vector{a}(\Point{q}_{031}) \|}
\; \Db{\Vector{a}_{031}}{\Point{q}}
\nonumber
\end{eqnarray}

As in \cref{sec:normal_difference}, we can write the partial gradients
without reference to an argument:
\begin{eqnarray}
\label{eq:normal_dot_gradient}
\Gc{\Point{p}_0}{(\Normal{n}_{012} \bullet \Normal{n}_{031})}{\Point{q}}
& = \; &
\frac{ \Normal{n}(\Point{q}_{031}) - 
\left[ \Normal{n}(\Point{q}_{012}) \bullet \Normal{n}(\Point{q}_{031}) \right] 
\Normal{n}(\Point{q}_{012}) }
{\| \Vector{a} (\Point{q}_{012}) \| }
\times (\Point{q}_2 - \Point{q}_1)
\\
& \; + &
\frac{ \Normal{n}(\Point{q}_{012}) - \left[ \Normal{n}(\Point{q}_{012}) \bullet \Normal{n}(\Point{q}_{031}) \right] \Normal{n}(\Point{q}_{031})  }
{\| \Vector{a} (\Point{q}_{031}) \| }
\times (\Point{q}_1 - \Point{q}_3)
\nonumber \\
& & \nonumber \\
\Gc{\Point{p}_1}{(\Normal{n}_{012} \bullet \Normal{n}_{031})}{\Point{q}}
& = \; &
\frac{ \Normal{n}(\Point{q}_{031}) - \left[ \Normal{n}(\Point{q}_{012}) \bullet \Normal{n}(\Point{q}_{031}) \right] \Normal{n}(\Point{q}_{012})  }
{\| \Vector{a} (\Point{q}_{012}) \| }
\times (\Point{q}_0 - \Point{q}_2)
\nonumber \\
& \; + &
\frac{ \Normal{n}(\Point{q}_{012}) - \left[ \Normal{n}(\Point{q}_{012}) \bullet \Normal{n}(\Point{q}_{031}) \right] \Normal{n}(\Point{q}_{031})   }
{\| \Vector{a} (\Point{q}_{031}) \| }
\times (\Point{q}_3 - \Point{q}_0)
\nonumber \\
& & \nonumber \\
\Gc{\Point{p}_2}{(\Normal{n}_{012} \bullet \Normal{n}_{031})}{\Point{q}}
& = \; &
\frac{ \Normal{n}(\Point{q}_{031}) - \left[ \Normal{n}(\Point{q}_{012}) \bullet \Normal{n}(\Point{q}_{031}) \right] \Normal{n}(\Point{q}_{012})  }
{\| \Vector{a} (\Point{q}_{012}) \| }
\times (\Point{q}_1 - \Point{q}_0)
\nonumber \\
& & \nonumber \\
\Gc{\Point{p}_3}{(\Normal{n}_{012} \bullet \Normal{n}_{031})}{\Point{q}}
& = \; &
\frac{ \Normal{n}(\Point{q}_{012}) - \left[ \Normal{n}(\Point{q}_{012}) \bullet \Normal{n}(\Point{q}_{031}) \right] \Normal{n}(\Point{q}_{031}) }
{\| \Vector{a} (\Point{q}_{031}) \| }
\times (\Point{q}_0 - \Point{q}_1)
\nonumber
\end{eqnarray}

This can be simplified using the fact that
\(\Normal{n}_i \perp \Normal{n}_j = \Normal{n}_i - \left[ \Normal{n}_i \bullet \Normal{n}_j \right] \Normal{n}_j\), for unit vectors,
and the face area \(A(\Point{q}) = \frac{1}{2} \| \Vector{a}(\Point{q}) \|\):
\begin{eqnarray}
\label{eq:simplified_normal_dot_gradient}
\Gc{\Point{p}_0}{(\Normal{n}_{012} \bullet \Normal{n}_{031})}{\Point{q}}
& = \;\;\; &
\frac{\left[ \Normal{n}(\Point{q}_{031}) \perp \Normal{n}(\Point{q}_{012}) \right]}{2A(\Point{q}_{012})}
\times (\Point{q}_2 - \Point{q}_1)
\\
& \;\;\; + &
\frac{\left[ \Normal{n}(\Point{q}_{012}) \perp \Normal{n}(\Point{q}_{031}) \right]}{2A(\Point{q}_{031})}
\times (\Point{q}_1 - \Point{q}_3)
\nonumber \\
& & \nonumber \\
\Gc{\Point{p}_1}{(\Normal{n}_{012} \bullet \Normal{n}_{031})}{\Point{q}}
& = \;\;\; &
\frac{\left[ \Normal{n}(\Point{q}_{031}) \perp \Normal{n}(\Point{q}_{012}) \right]}{2A(\Point{q}_{012})}
\times (\Point{q}_0 - \Point{q}_2)
\nonumber \\
& \;\;\; + &
\frac{\left[ \Normal{n}(\Point{q}_{012}) \perp \Normal{n}(\Point{q}_{031}) \right]}{2A(\Point{q}_{031})}
\times (\Point{q}_3 - \Point{q}_0)
\nonumber \\
& & \nonumber \\
\Gc{\Point{p}_2}{(\Normal{n}_{012} \bullet \Normal{n}_{031})}{\Point{q}}
& = \;\;\; &
\frac{\left[ \Normal{n}(\Point{q}_{031}) \perp \Normal{n}(\Point{q}_{012}) \right]}{2A(\Point{q}_{012})}
\times (\Point{q}_1 - \Point{q}_0)
\nonumber \\
& & \nonumber \\
\Gc{\Point{p}_3}{(\Normal{n}_{012} \bullet \Normal{n}_{031})}{\Point{q}}
& = \;\;\; &
\frac{\left[ \Normal{n}(\Point{q}_{012}) \perp \Normal{n}(\Point{q}_{031}) \right]}{2A(\Point{q}_{031})}
\times (\Point{q}_0 - \Point{q}_1)
\nonumber
\end{eqnarray}

\end{plSection}%{Inner product between face normals}
%-----------------------------------------------------------------
\begin{plSection}{Squared inner product between face normals}
\label{sec:squared_normal_dot}

We can get a more even distribution of bending by giving
a higher weight to sharper edge bends.
A simple way to do that is to square some existing function,
for example: $\left(1 - \Normal{n}_{012} \bullet \Normal{n}_{031}\right)^2$.
The derivative is simply:
\begin{equation}
\Da{\left(1 - \Normal{n}_{012} \bullet \Normal{n}_{031}\right)^2}
= -2 \left( 1 - \Normal{n}_{012} \bullet \Normal{n}_{031} \right)
\Da{(\Normal{n}_{012} \bullet \Normal{n}_{031})}
\end{equation}

It follows from equation \ref{eq:simplified_normal_dot_gradient}
that the partial gradients are:
\begin{eqnarray}
\label{eq:squared_normal_dot_gradient}
\Gc{\Point{p}_0}{\left(1 - \Normal{n}_{012} \bullet \Normal{n}_{031}\right)^2}{\Point{q}}
& = \;\;\; &
\frac{\left( \Normal{n}(\Point{q}_{012}) \bullet \Normal{n}(\Point{q}_{031}) - 1\right)
}
{A(\Point{q}_{012}) }
\left[ \Normal{n}(\Point{q}_{031}) \perp \Normal{n}(\Point{q}_{012}) \right]
\times (\Point{q}_2 - \Point{q}_1)
\\
& \;\;\; + &
\frac{\left( \Normal{n}(\Point{q}_{012}) \bullet \Normal{n}(\Point{q}_{031}) - 1\right)
}{A(\Point{q}_{031})}
\left[ \Normal{n}(\Point{q}_{012}) \perp \Normal{n}(\Point{q}_{031}) \right]
\times (\Point{q}_1 - \Point{q}_3)
\nonumber \\
& & \nonumber \\
\Gc{\Point{p}_1}{\left(1 - \Normal{n}_{012} \bullet \Normal{n}_{031}\right)^2}{\Point{q}}
& = \;\;\; &
\frac{\left( \Normal{n}(\Point{q}_{012}) \bullet \Normal{n}(\Point{q}_{031}) - 1\right)
}{A(\Point{q}_{012})}
\left[ \Normal{n}(\Point{q}_{031}) \perp \Normal{n}(\Point{q}_{012}) \right]
\times (\Point{q}_0 - \Point{q}_2)
\nonumber \\
& \;\;\; + &
\frac{\left( \Normal{n}(\Point{q}_{012}) \bullet \Normal{n}(\Point{q}_{031}) - 1\right)
}{A(\Point{q}_{031})}
\left[ \Normal{n}(\Point{q}_{012}) \perp \Normal{n}(\Point{q}_{031}) \right]
\times (\Point{q}_3 - \Point{q}_0)
\nonumber \\
& & \nonumber \\
\Gc{\Point{p}_2}{\left(1 - \Normal{n}_{012} \bullet \Normal{n}_{031}\right)^2}{\Point{q}}
& = \;\;\; &
\frac{\left( \Normal{n}(\Point{q}_{012}) \bullet \Normal{n}(\Point{q}_{031}) - 1\right)
}{A(\Point{q}_{012})}
\left[ \Normal{n}(\Point{q}_{031}) \perp \Normal{n}(\Point{q}_{012}) \right]
\times (\Point{q}_1 - \Point{q}_0)
\nonumber \\
& & \nonumber \\
\Gc{\Point{p}_3}{\left(1 - \Normal{n}_{012} \bullet \Normal{n}_{031}\right)^2}{\Point{q}}
& = \;\;\; &
\frac{\left( \Normal{n}(\Point{q}_{012}) \bullet \Normal{n}(\Point{q}_{031}) - 1\right)
}{A(\Point{q}_{031})}
\left[ \Normal{n}(\Point{q}_{012}) \perp \Normal{n}(\Point{q}_{031}) \right]
\times (\Point{q}_0 - \Point{q}_1)
\nonumber
\end{eqnarray}
\end{plSection}%{Squared inner product between face normals}
%-----------------------------------------------------------------
\end{plSection}%{Functions of the edges}

%\subsection{Functions of the vertices}
\label{sec:vertices}

%-----------------------------------------------------------------

\subsubsection{Vertex surround angle}
\label{sec:vertex_surround_angle}

\paragraph{Non-boundary case}
\label{sec:non_boundary_vertex_surround_angle}

\begin{figure}[!htp]
\centering
\begin{verbatim}
                v2/e2 o           o v1/e1
                       \   f1    /
                        \       /
                     f2  \ g1  /   f0
                          \ ^ /
                       g2( \ / )g0
             v3/e3 o--------o--------o v0/e0
                       g3( / \ )gn-1
                          / v \
                     f3  /     \  fn-1
                        /       \
                 v4/e4 o   ...   o vn-1/en-1
\end{verbatim}
\caption{Vertex surround angle.
\label{fig:vertex_surround_angle}}
\end{figure}

Consider the non-boundary vertex
labeled $\v$ in figure \ref{fig:vertex_surround_angle}.
$\v$ has degree $n$;
the incident edges are labeled $\e_0, \e_1, \ldots, \e_{n-1}$;
the incident faces are labeled $\f_0, \f_1, \ldots, \f_{n-1}$;
and the neighboring vertices are labeled $\v_0, \v_1, \ldots, \v_{n-1}$.
The angle between edges $\e_j$ and $\e_{j + 1 \bmod n}$ is $\gamma_j$.

$\gamma_j = \gamma(\f_j,\v)$ is the {\em corner angle} in face $\f_j$
of vertex $\v$.

The {\em surround angle} of $\v$ is the sum of all its corner angles:
\begin{equation}
\alpha(\v) = \sum_{j=0}^{n-1} \gamma(\f_j,\v),
\end{equation}

If the neighborhood of $\v$ is flat, then $\alpha(\v)=2\pi$.
A neighborhood with $\alpha(\v) < 2\pi$, corresponds very roughly
to positive Gaussian curvature,
and, $\alpha(\v) > 2\pi$,
is analogous to negative Gaussian curvature.
However, it's important to remember that, unlike the case
with smooth surfaces, where the most complicated local
shape is a saddle,
the crinkling of a vertex neighborhood can be arbitrarily complicated,
for any non-zero surround angle.

Although a mesh with all surround angles of $2\pi$ is not guaranteed
to be flat,
a mesh with surround angles different from $2\pi$ is guaranteed
to be not-flat,
and, in some sense,
the more different the surround angles are from $2\pi$
the more non-flat is the mesh.
This leads to the following as a mesh roughness measure:
\begin{eqnarray}
f(\M)
& = & \sum_{\v \in \V(\M)} \left[ \alpha(\v) - 2\pi \right]^2
\\
& = & \sum_{\v \in \V(\M)}
\left[ \left(\sum_{\f \in \F(\v)} \gamma(\f,\v)\right)
 - 2\pi \right]^2
\nonumber
\end{eqnarray}
The natural implementation of this function is to first compute
all the corner angles and cache them with the faces,
then compute the surround angles and cache with each vertex,
and then do a lookup while computing the sum.

In computing the gradient, it's simplest to re-arrange the
order of summation:
\begin{eqnarray}
\Ga{f}
& = & \Ga{ \left[ \sum_{\v \in \V(\M)} \left[ \alpha(\v) - 2\pi \right]^2 \right]}
\\
& = & 2 \sum_{\v \in \V(\M)} \left[ \alpha(\v) - 2\pi \right] \Ga{\alpha(\v)}
\nonumber
\\
& = & 2 \sum_{\v \in \V(\M)} \left[ \alpha(\v) - 2\pi \right]
\sum_{\f \in \F(\v)} \Ga{\gamma(\f,\v)}
\nonumber
\\
& = & 2 \sum_{\v \in \V(\M)}
\sum_{\f \in \F(\v)}
\left[ \alpha(\v) - 2\pi \right]
\Ga{\gamma(\f,\v)}
\nonumber
\\
& = & 2
\sum_{\f \in \F(\M)}
\sum_{\v \in \V(\f)}
\left[ \alpha(\v) - 2\pi \right]
\Ga{\gamma(\f,\v)}
\nonumber
\\
& = & 2
\sum_{\f \in \F(\M)}
\sum_{i=0}^2
\left[ \alpha(\v_i(\f)) - 2\pi \right]
\Ga{\gamma(\p_i(\f),\p_{i+1 \% 2}(\f),\p_{i+2 \% 3}(\f))}
\nonumber
\end{eqnarray}

Consider the partial gradient with respect to
location of one of the vertices:
\begin{eqnarray}
\label{eq:surround_angle_partial_gradient}
\Gc{\p_j}{f}{q}
& = 2 \sum_{\f \in \F(\M)} &
\sum_{i=0}^2
\left[ \alpha(\v_i(\f)) - 2\pi \right]
\Gc{\p_j}{\gamma(\p_i(\f),\p_{i+1 \% 2}(\f),\p_{i+2 \% 3}(\f))}{\q}
\\
& = 2 \sum_{\f \in \F(\v_j)} &
\sum_{i=0}^2
\left[ \alpha(\v_i(\f)) - 2\pi \right]
\Gc{\p_j}{\gamma(\p_i(\f),\p_{i+1 \% 2}(\f),\p_{i+2 \% 3}(\f))}{\q}
\nonumber
\\
& = 2 \sum_{\f \in \F(\v_j)} & \left(
\left[ \alpha({\mathrm p}(\v_j,\f) - 2\pi \right]
\Gc{\p_j}{\gamma({\mathrm p}(\p_j,\f),\p_j,{\mathrm s}(\p_j,\f))}{\q}
\right.
\nonumber
\\
& & +
\left[ \alpha(\v_j) - 2\pi \right]
\Gc{\p_j}{\gamma(\p_j,{\mathrm s}(\p_j,\f),{\mathrm p}(\p_j,\f)}{\q}
\nonumber
\\
& & +
\left.
\left[ \alpha({\mathrm s}(\v_j,\f) - 2\pi \right]
\Gc{\p_j}{\gamma({\mathrm s}(\p_j,\f),{\mathrm p}(\p_j,\f),\p_j)}{\q}
\right)
\nonumber
\end{eqnarray}
where ${\mathrm p}(\v,\f)$ is $\v$'s {\em predecessor},
the vertex that comes before $\v$ in the oriented face $\f$,
${\mathrm s}(\v_j,\f)$ the {\em successor}
the vertex that comes after $\v$,
and similarly for the vertex positions $\p$.

Consider the term corresponding to a particular face $\f$ in
the sum in equation \ref{eq:surround_angle_partial_gradient}.
Call that face's vertices and positions $\v_0, \v_1, \v_2$
and $\p_0, \p_1, \p_2$.
\begin{eqnarray}
\Gc{\p_0}{f_{\f}}{q}
& = &
\left[ \alpha(\v_0) - 2\pi \right] \Gc{\p_0}{\gamma(\p_0,\p_1,\p_2)}{\q}
\\
& + &
\left[ \alpha(\v_1) - 2\pi \right] \Gc{\p_0}{\gamma(\p_1,\p_2,\p_0)}{\q}
\nonumber
\\
& + &
\left[ \alpha(\v_2) - 2\pi \right] \Gc{\p_0}{\gamma(\p_2,\p_0,\p_1)}{\q}
\nonumber
\\
&  &
\nonumber
\\
& = &
\left[ \alpha(\v_0) - 2\pi \right]
{{(\q_1 - \q_0) \perp (\q_2 - \q_0) + (\q_2 - \q_0) \perp (\q_1 - \q_0)}
\over
{
\sqrt{
\| \q_2 - \q_0 \|^2 \| \q_1 - \q_0 \|^2
-
\left( ( \q_1 -\q_0 ) \bullet ( \q_2 -\q_0 ) \right)^2
}
}}
\nonumber
\\
& - &
\left[ \alpha(\v_1) - 2\pi \right]
{{(\q_0 - \q_1) \perp (\q_2 - \q_1)}
\over
{
\sqrt{
\| \q_0 - \q_1 \|^2 \| \q_2 - \q_1 \|^2
-
\left( ( \q_0 -\q_1 ) \bullet ( \q_2 - \q_1 ) \right)^2
}
}}
\nonumber
\\
& - &
\left[ \alpha(\v_2) - 2\pi \right]
{{(\q_0 - \q_2) \perp (\q_1 - \q_2)}
\over
{
\sqrt{
\| \q_0 - \q_2 \|^2 \| \q_1 - \q_2 \|^2
-
\left( ( \q_0 -\q_2 ) \bullet ( \q_1 -\q_2 ) \right)^2
}
} }
\nonumber
\end{eqnarray}

%\cleardoublepage

%\chapter{Subdivision Surfaces}
%\begin{plSection}{Subdivision Surfaces}
I am only considering subdivision surfaces based on triangles
\cite{HoppeEtal:1994:SIGGRAPH,Hoppe:1994:Phd}.
%-----------------------------------------------------------------
\begin{plSection}{Approximating meshes}
\label{sec:Approximating-meshes}

A common approach to the use of subdivision surfaces is
to approximate the limit surface by the {\it subdivided mesh,} $\M^s$,
a $k$ times subdivided version of the {\it control mesh,} $\M^c$
(a typical value for $k$ is 2).
The positions of the $n^c$ vertices of the control mesh,
$\p^c = (\p^c_0 \ldots \p^c_{n-1}) \\in \Reals^{3n^c},$
and the $n^s$ vertices of the subdivided mesh,
$\p^s = (\p^s_0 \ldots \p^s_{n-1}) \in \Reals^{3n^s},$
are related by the {\it subdivision transform}
$\S : \Reals^{n^c} \mapsto \Reals^{n^s}$.
If
$\x^c = (x^c_0 \ldots x^c_{n-1}) \in \Reals^{n^c}$,
$\y^c = (y^c_0 \ldots y^c_{n-1}) \in \Reals^{n^c}$,
and
$\z^c = (z^c_0 \ldots z^c_{n-1}) \in \Reals^{n^c}$,
are the $x, y,$ and $z,$ coordinates of $\p^c$,
and $\x^s, \y^s, \z^s$ are the same coordinates
of $\p^s$, then
\begin{eqnarray}
\x^s & = & \S \x^c
\\
\y^s & = & \S \y^c
\nonumber
\\
\z^s & = & \S \z^c
\nonumber
\end{eqnarray}
We can use the above to define $\S_3 : \Reals^{3n^c} \mapsto \Reals^{3n^s}$,
so that
\begin{equation}
\p^s = \S_3 \p^c.
\end{equation}


If $f(\p^s) = f(\S_3 \p^c)$ is a penalty function applied to the subdivided mesh,
then the gradient with respect to the positions of
the vertices of the control mesh is simply:
\begin{equation}
\Gc{\p^c}{f(\S_3 \p^c)}{\q^c} = \S_3^{\dagger} \Gc{\p^s}{f(\p^s)}{\q^s = \S_3 \q^c}
\end{equation}

\end{plSection}%{Approximating meshes}
%-----------------------------------------------------------------
\end{plSection}%{Subdivision Surfaces}


%\part{Fitting}
%\cleardoublepage

%\chapter{Data Fitting}
%\label{sec:data-fitting}

We can fit a triangular mesh, $\M$, to a set of data points, $\{\d_i; i=0 \ldots n-1\}$
by minimizing the sum of the $l_2$ distances from the points to the mesh:
\begin{equation}
f(\M) = \sum_{i=0}^{n-1} \| \d_i - \Pr_\M (\d_i) \|^2 ,
\end{equation}
where $\Pr_\M (\d_i)$ is the point on $\M$ closest to $\d_i$,
that is, the {\em projection} of $\d_i$ on $\M$.
Note that $f:\Re^{3n} \mapsto \Re$,
where $n$ is the number of vertices in $\M$.

We compute $\Pr_\M (\d)$ by minimizing  $\| \d - \Pr_\s (\d) \|^2$
over all simplices $\s \in \M$.
We need only consider the faces of $\M$,
those edges not in any face,
and the vertices not in any edge,
because the closest point on a face must at least
as the closest point on any of its edges,
and similarly for vertices.
More generally, spatial binning of the simplices and data can greatly
reduce the number of simplices that need to be examined.

The following sections consider the projection of a single
data point $\d$ on a mesh $\M$,
and the gradient of the squared distance,
as a function of the vertex positions $\p(\v)$.

Unfortunately, derivatives of the distance function are not continuous.
Second derivative discontinuities occur
when a data point is on the boundary
of a 'watershed' region, the set of points
projecting on a vertex or the interior of an edge or face.
Gradient discontinuities are encountered
when a data point is equidistant from 2 distinct closest mesh points.

\subsection{Distance to vertex}
\label{sec:Distance-to-vertex}

Let $\p = \p(\v)$ be the position of a particular vertex $\v$,
and $\d$ the 3d data point.
It follows from equation \ref{eq:l2-gradient} that
\begin{equation}
\label{eq:vertex-distance-gradient}
\Gc{\p}{\| \p - \d \|^2}{\q} = 2 ( \q - \d ).
\end{equation}

The distance to the nearest vertex in a set of vertices $\V$ is:
$\min_{\v \in \V} \| \p(\v) - \d \|^2$.
If $\v^{\mathrm min}$ is the minimizing vertex,
and
$\p^{\mathrm min}$ its position,
then the partial gradient with respect
to the position of any other vertex is zero,
and the partial gradient with respect to $\p^{\mathrm min}$
is given in equation \ref{eq:vertex-distance-gradient}.
Note that the gradient is only defined and continuous
while $\d$ is within the interior of the
Voronoi regions surrounding the vertices.

\subsection{Distance to edge}
\label{sec:Distance-to-edge}

Let the edge $\e$ have end points $\p = (\p_0, \p_1) \in \Re^6$.
We can write the projection of a data point $\d$ on $\e$ as:
\begin{equation}
\Pr_\p (\d) = b_0(\p) \p_0 + b_1(\p) \p_1
\end{equation}
where
\begin{eqnarray}
b_0(\p) & = &
\min\left(0,\max\left(1,
{{ (\d - \p_1) \bullet (\p_0 - \p_1) }
\over
{ \| \p_0 - \p_1 \|^2 }
}\right) \right) \\
b_1(\p) & = & 1 - b_0(\p)
\nonumber
\end{eqnarray}

\begin{eqnarray}
\label{eq:edge-distance-gradient-derivation}
\De{\p_0}{ \| \Pr_{\p} (\d) - \d \|^2 }{\q}
& = &
2 \left( \Pr_{\q} (\d) - \d \right)^\dagger
\De{\p_0}{\Pr_{\p} (\d) }{\q}
\\
& = &
2 \left( \Pr_{\q} (\d) - \d \right)^\dagger
\De{\p_0}{\left[ b_0(\p)\p_0 + b_1(\p)\p_1 \right]}{\q}
\nonumber \\
& = &
2 \left( \Pr_{\q} (\d) - \d \right)^\dagger
\De{\p_0}{\left[ b_0(\p)\p_0 + (1 - b_0(\p))\p_1 \right]}{\q}
\nonumber \\
& = &
2 \left( \Pr_{\q} (\d) - \d \right)^\dagger
\De{\p_0}{\left[ b_0(\p)(\p_0 - \p_1) \right]}{\q}
\nonumber \\
& = &
2 \left( \Pr_{\q} (\d) - \d \right)^\dagger
\left[ b_0(\q) \I + (\q_0 - \q_1) \otimes \Gc{\p_0}{b_0(\p)}{\q} \right]
\nonumber
\end{eqnarray}
Because $\left( \Pr_{\q} (\d) - \d \right)$ is orthogonal to
$\left( \q_0 - \q_1 \right)$, we get:
\begin{eqnarray}
\label{eq:edge-distance-gradient}
\Gc{\p_0}{ \| \Pr_{\p} (\d) - \d \|^2 }{\q}
& = & 2 b_0(\q) \left[ \Pr_{\q} (\d) - \d \right]
\\
\Gc{\p_1}{ \| \Pr_{\p} (\d) - \d|^2 }{\q}
& = & 2 b_1(\q) \left[ \Pr_\q (\d) - \d \right]
\nonumber
\end{eqnarray}

As in the vertex case,
the distance to the nearest edge in a set of edges $\E$ is:
\begin{equation}
\| \Pr_{\E} (\d) - \d|^2 = \min_{\e \in \E} \| \Pr_{\p(\e)}(\d) - \d \|^2
\end{equation}
If $\e^{\min}$ is the minimizing edge,
$\v_0^{\min}$ and $\v_1^{\min}$ its vertices,
and $\p_0^{\min}$ and $\p_1^{\min}$
the corresponding endpoints,
then the partial gradient with respect to
the position of any
other vertex is zero,
and the partial gradient with respect to $\p_0^{\min}$ and $\p_1^{\min}$
is given in equation \ref{eq:edge-distance-gradient}.

The total gradient is defined and continuous
when $\d$ is within the union of the watershed regions
of $\e^{\min}$ and its vertices.
It is also continuous where the watershed of one of the vertices
meets the watershed of any of the edges containing that vertex.
It is not if $\d$ lies on the boundary of the
watershed of $\e^{\min}$ and the watershed of an
edge with which it does not share a vertex.

\subsection{Distance to face}
\label{sec:Distance-to-face}

Let the face $\f$ have corner points $\p = (\p_0, \p_1, \p_2) \in \Re^9$.
As in the edge case,
we can write the projection of a data point $\d$ on $\f$
in terms of the barycentric coordinates as:
\begin{equation}
\Pr_\p (\d) = b_0(\p) \p_0 + b_1(\p) \p_1 + b_2(\p) \p_2,
\end{equation}
and, by an argument simlar to that used in
equation \ref{eq:edge-distance-gradient-derivation},
we can show that
\begin{eqnarray}
\label{eq:face-distance-gradient}
\Gc{\p_0}{ \| \Pr_{\p} (\d) - \d \|^2 }{\q}
& = & 2 b_0(\q) \left[ \Pr_{\q} (\d) - \d \right]
\\
\Gc{\p_1}{ \| \Pr_{\p} (\d) - \d|^2 }{\q}
& = & 2 b_1(\q) \left[ \Pr_\q (\d) - \d \right]
\nonumber
\\
\Gc{\p_2}{ \| \Pr_{\p} (\d) - \d|^2 }{\q}
& = & 2 b_2(\q) \left[ \Pr_\q (\d) - \d \right]
\nonumber
\end{eqnarray}

Computing the barycentric coordinates for the projection
on a face (triangle) is slightly more complicated than
for an edge (line segment).

First center the problem by letting
$\v = \p - \p_0$,
$\v_1 = \p_1 - \p_0$, and $\v_2 = \p_2 - \p_0$.
Then compute the raw, unbounded barycentric coordinates
of the projection of $\p$ onto the plane
spanned by the triangle:
\begin{eqnarray}
r_0(\p) & = & 1 - r_1(\p) - r_2(\p)
\\
r_1(\p) & = & v \bullet {{\v_1 \perp \v_2} \over {\| \v_1 \perp \v_2 \|^2} }
\nonumber
\\
r_2(\p) & = & v \bullet {{\v_2 \perp \v_1} \over {\| \v_2 \perp \v_1 \|^2} }
\nonumber
\end{eqnarray}
To correctly bound the raw coordinates to numbers between 0 and 1,
we need to determine whether the projected point is in
the interior of the triangle, on one of the edges,
or on one of the vertices.

\begin{description}

\item[Vertex case:]
If any 2 of the $r_i$ are negative,
then $\p$ projects on the remaining vertex.
Set the 2 $b_i$ corresponding to the negative $r_i$
to 0 and the remaining $b_i$ to 1.

\item[Edge case:]
If any single 1 of the $r_i$ is negative,
then $\p$ projects on the opposite edge.
Set the $b_i$ corresponding to the negative $r_i$
to 0.
Go to section \ref{sec:Distance-to-edge} to see
how to compute the remaining barycentric coordinates
by projecting on the edge

\item[Interior case:]
If none of the $r_i$ is negative,
then $\p$ projects on the interior
and each $b_i = r_i$

\end{description}

%\cleardoublepage

%\chapter{Image Fitting}
%\label{sec:image-fitting}

In this section, I discuss fitting a polygon to a binary classification
image, or a monochrome 'probability' image.
In the first case, the pixel values are either $0$ or $1$,
indicating assignment to 1 of 2 possible classes.
In the monochrome case, I take the the value ($0--255$) to indicate
the probability that the pixel belong to class $1$,
or some other score, such as the fraction of the pixel area that belongs
to class $1$.

The basic idea is to optimize the polygon's vertex positions
to maximize the agreement of the inside-outside pixel classification
induced by the polygon with the classification, or class score
given by the image.

\subsection{Inside-Outside}
\label{sec:inside-outside}

I use a non-standard definition of polygon inside-outside,
based on projection:

\begin{figure}[!htp]
\centering
\begin{verbatim}
                            .p
                           .
                   e01    .
             v0 o--------o
                       v1 \
                           \ e12
                            \
                             o v2
\end{verbatim}
\caption{Convex vertex.
\label{fig:convex-vertex}}
\end{figure}

\begin{figure}[!htp]
\centering
\begin{verbatim}
                             o v2
                            /
                           / e12
                          /
             v0 o--------o v1
                   e01    .
                           .
                            .p
\end{verbatim}
\caption{Concave vertex.
\label{fig:concave-vertex}}
\end{figure}

Suppose we have a polygon, $\M$, consisting of a set of vertices,
$\{\v_i\}$, and consistently directed edges $\{\e_{i,i+1}:(\v_i,\v_{i+1})\}$.
To determine whether a point $\p$ is inside or outside,
find the projection of (the closest point to) $\p$ on the polygon.
If the projection is on a convex vertex (see \ref{fig:convex-vertex}),
then the point is outside.
If the projection is on a concave vertex (see \ref{fig:concave-vertex}),
then the point is inside.
If the projection is to the interior of edge $\e_{i,i+1}$,
then the point is outside if the cross product
$(\v_i-\p) \times (\v_{i+1}-\p) > 0$.

The {\em signed distance,} $\d(\p_i,\M)$, from $\p$ to the polygon $\M$,
is the distance to the closest point, times $-1$ if $\p$ is inside $\M$.
I choose this sign so that the positive part of the signed distance
is the same as the distance to the set corresponding to the polygon interior.
Correspondingly, I define a classified pixel's sign $\sign(\p)$ to be +1
if the pixel is in class 0 and -1 if the pixel is in class 1,
because it's most common to consider the interior of the shape
to be class 1.


\subsection{Counting penalties}
\label{sec:counting-penalties}


\subsection{Signed distance penalties}
\label{sec:signed-distance-penalties}

In this section, I describe a class of penalties the depend on the
signed distance of a classified pixel to the polygon.

These penalties sum over (a sample of) the pixels ($\p_i$) in the class image:
\begin{equation}
f(\M) = \sum_{\p_i} \phi( \sign(\p_i) \ast \d(\p_i,\M) ) ,
\end{equation}

The function $\phi(x)$ is typically zero for all $x \le 0$,
monotone increasing in $x$, and bounded by $1$ as $x\rightarrow\infty$.
$\frac{d\phi}{dx}\mid_{x=0} = 0$ and $\frac{d\phi}{dx}\mid_{x} \rightarrow 0$
as $x\rightarrow\infty$.

Examples are:

\begin{eqnarray}
\phi_s(x) & = 3 \left(\frac{x}{r}\right)^2 - 2 \left(\frac{x}{r}\right)^3 & {x \geq 0} \\
          & = 0 & {x \leq 0}
\end{eqnarray}

\begin{eqnarray}
\phi_g(x) & = 1 - e^{ \frac{-x^2}{2\pi r^2} } & x \geq 0 \\
          & = 0                               & x \leq 0 \nonumber
\end{eqnarray}

%\cleardoublepage

%\chapter{Registration}
%\label{sec:registration}

This section discusses functions that are minimized
to register (align) one mesh to another,
to register a mesh to a data set,
or to register a data set to a mesh.
The general approach is to minimize
a measure of distance between the {\it docking} mesh and its {\it target},
over some set of registration transforms,
for example, the rigid transformations of $\Re^3$.

We most often compute a registration transform, $\Tr$,
to minimize a distance between the transformed mesh,
$\Tr\M$ and its target, $\Ta$:
\begin{equation}
\min_{\Tr} d(\M(\Tr),\Ta)
\end{equation}
The target, $\Ta$, may be another mesh, a set of data points,
or some other geometric object.
The transformed mesh
$\M(\Tr) = \M(\Tr\v) = \M(\Tr(\v_0 \ldots \v_{n-1}))$,
where $\v = (\v_0, \ldots , \v_{n-1})$
are the positions of the vertices of $\M$.

We are also sometimes interested in registering a data set
$\{\x_i\}$ to a mesh.
In this case, we minimize
\begin{equation}
\min_{\Tr} d(\M,\Tr\{\x_i\})
\end{equation}

Both are special cases of choosing a transform on $\Re^{3n}$
by minimizing:
\begin{equation}
\min_{\Tr} f( \Tr \p),
\end{equation}
where
$\p = (\p_0, \ldots , \p_{n-1}) \in \Re^{3n}$,
$\Tr : \Re^{3n} \mapsto \Re^{3n}$,
and
$f : \Re^{3n} \mapsto \Re$.

\subsection{Distance measures}
\label{sec:Distance-measures}

\subsubsection{Vertex distance}
\label{sec:Vertex-distance}

Suppose 2 meshes have corresponding vertices,
that is,
for each vertex $\v_{0i}$ in the docking mesh $\M_0$,
there is a corresponding vertex $\v_{1i}$ in the target mesh $\M_1$,
and vice versa.
Then an obvious measure of distance
is the sum of distances between the corresponding points:
\begin{equation}
d_{\V}(\M_0,\M_1) = \sum_{i=0}^{n-1} \| \p_{0i} - \p_{1i} \|^2
\end{equation}
The gradient of $d_v$ with respect to the positions
of the $i$th vertex of $\M_0$ is:
\begin{equation}
\Gc{\p_{0i}}{d_v}{q} = 2 \left[ \q_{0i} - \p_{1i} \right]
\end{equation}

\subsubsection{Projection distance}
\label{sec:Projection-distance}

In many problems, the vertex correspondence used in
\autoref{sec:Vertex-distance} is not appropriate.
Or we may want to register a mesh to a target data set.
In such cases, we can use the projection distance:
\begin{equation}
d_{\Pr}(\M,\p_1) = \sum_{i=0}^{n-1} \| \Pr_{\M}(\p_{1i}) - \p_{1i} \|^2,
\end{equation}
where the target $\p_1 = (\p_{10}, \ldots \p_{1(n-1)})$
is either a set of data points,
or the set of positions of the vertices of a target mesh.
$\Pr_{\M}(\p)$ is the projection of the point $\p \in \Re^3$
on the mesh $\M$.
This is the same as the general data fitting distance, and
derivatives with respect to the positions of the vertices
of $\M$ are given in \autoref{sec:data-fitting}.

\subsection{Transforms}
\label{sec:Transforms}

In this section, I describe common families
of transforms, $\{\Tr\}$, over which to minimize:
\begin{equation}
\min_{\Tr} f( \Tr \p),
\end{equation}
To do the minimization, we need to compute
the derivative:
\begin{equation}
\De{\Tr}{g(\Tr)}{\Tr_0}
= \De{\Tr}{f(\Tr\p)}{\Tr_0}
= \De{\q}{f(\q)}{\q=\Tr_0\p}
\circ
\De{\Tr}{\Tr\p}{\Tr_0}.
\end{equation}
o, equivalently, the gradient:
\begin{equation}
\Gc{\Tr}{f( \M(\Tr\p) )}{\Tr_0}
 =
\De{\Tr}{\Tr\p}{\Tr_0}^\dagger
\Gc{\q}{f(\M(\q)))}{\Tr_0\p}
\end{equation}

I assume in the following
that $\Df{\p}{f}$ (or $\Gb{\p}{f}$) is known,
so the main task is computing $\Df{\Tr}{\Tr\p}$.

As mentioned in above,
in registration problems
the function $f$ is usually a distance between
two meshes, or between a mesh and a data set,
with either the location of one mesh's vertices
or the data points allowed to vary.
Derivatives and gradients of distance functions,
with respect to vertex positions,
are given in sections \ref{sec:Distance-measures}
and \ref{sec:data-fitting}.

\subsubsection{Direct sum transforms}
\label{sec:Direct-sum-transforms}

In registration problems,we usually want to
apply the same transform to each vertex or data point.
Let $\Tr : \Re^3 \mapsto \Re^3$ be an element
of some family of transformations on $\Re^3$.
A simple direct sum transform applies the same 3-dimensional
transform to each vertex point,
so the full mesh transform is
$\Tr_{3n} = \bigoplus^n \Tr$,
and $\Tr_{3n} (\bigoplus_{i=0}^{n-1} \p_i) = \bigoplus_{i=0}^{n-1} \Tr \p_i$
by transforming the locations of each of the vertices
of the mesh.

In general, suppose $\Tr_i :
{\mathcal D}_i \mapsto {\mathcal C}_i; i = 0 \ldots k-1$
are $k$ maps.
The direct sum of the $\Tr_i$ is:
\begin{equation}
\label{eq:diagonal-blocks}
\Tr =
\left( \bigoplus_{i=0}^{k-1} \Tr_i \right) :
\left( \bigoplus_{i=0}^{k-1}{\mathcal D}_i \right)
\mapsto
\left( \bigoplus_{i=0}^{k-1}{\mathcal C}_i \right)
\end{equation}
$\Tr$ is a transform whose
$k$ 'diagonal blocks' are the $\Tr_i$.

Using this notation,
$\Df{\p}{f(\p)} = \bigoplus_i \Df{\p_i}{f(\p)}$,
$\Df{\Tr}{\Tr\p} = \bigoplus_i \Df{\Tr}{\Tr\p_i}$,
and:
\begin{equation}
\label{eq:total-registration-transform-derivative}
\De{\Tr}{f( \Tr \p )}{\Tr_0}
 =
\sum_i
\De{\q_i}{f(\q)}{\q=\Tr_0\p}
\circ
\De{\Tr}{\Tr\p_i}{\Tr_0}
\end{equation}
For the gradient, the equivalent is:
\begin{equation}
\label{eq:registration-gradient-sum}
\Gc{\Tr}{f( \Tr \p )}{\Tr_0}
 =
\sum_i
\left( \De{\Tr}{\Tr\p_i}{\Tr_0} \right)^{\dagger} \;
\left( \Gc{\q_i}{f(\q)}{\q=\Tr_0\p} \right)
\end{equation}

The derivative with respect to vertex or data points,
$\Df{\p_i}{f(\p)}$,
is assumed to be known.
For example, the derivatives of functions
related to data fitting
are discussed in chapter \ref{sec:data-fitting}.

In this chapter, I focus on computing
$\De{\Tr}{\Tr\p}{\Tr_0}$
for $\Tr : \Re^3 \mapsto \Re^3$
and $\p \in \Re^3$.

\subsubsection{General linear registration}
\label{sec:General-linear-registration}

Suppose $\Tr = \L$ a linear transform on $\Re^3$.
We can represent $\L$ as a vector in $\Re^9$:
\begin{equation}
\label{eq:L-vector}
\l = \left(\L_{00},\L_{01},\L_{02},
       \L_{10},\L_{11},\L_{12},
       \L_{20},\L_{21},\L_{22}\right),
\end{equation}
where $\L_{ij}$ is the $ij$-th element of a
matrix representation of $\L$.
We can identify a vector $\p \in \Re^3$
with a linear transform $\Tr_{\p} : \Re^9 \mapsto \Re^3$
by defining $\Tr_{\p}\l = \L\p$.
In the coordinate system defined by equation \ref{eq:L-vector},
the matrix for $\Tr_{\p}$ is:
\begin{equation}
\label{eq:Tp-matrix}
\Tr_{\p} =
\left(
\begin{array}{lllllllll}
\p_0 & \p_1 & \p_2 &  0   &  0   &  0   &  0   &  0   &  0 \\
 0   &  0   &  0   & \p_0 & \p_1 & \p_2 &  0   &  0   &  0 \\
 0   &  0   &  0   &  0   &  0   &  0   & \p_0 & \p_1 & \p_2 \\
\end{array}
\right)
\end{equation}

Because linear transforms are their own derivatives
(see section\ref{sec:Derivatives-of-linear-functions}),
it follows that the derivative with respect to the
unconstrained set of linear registration transforms is:
\begin{equation}
\label{eq:linear-transform-derivative}
\Df{\L}{\left( \L \, \p \right)}
 \; = \;
\Df{\L}{\left( \Tr_{\p} \, \L \right)}
 \; = \;
\Tr_{\p}
\end{equation}

\subsubsection{Scaled rotations}
\label{sec:Scaled-rotations}

A scaled rotation
is a linear transform $\Sc = s \R$,
where $s \in \Re$ and $\R : \Re^n \mapsto \Re^n$
is a rotation.

The quaternions $\Qs$ are a convenient representation
for the scaled rotations on $\Re^3$.
(See Faugeras~\cite[sec.~5.5.2]{Faugeras:1993:3dVision}.)

A quaternion is a 4-tuple:
\begin{equation}
\q = (w, x, y, z) = (w, \v),
\end{equation}
where $w, x, y, z \in \Re$ and $\v \in \Re^3$.
The set of quaternions has several operations:
\begin{itemize}
\item Quaternion conjugation $\dagger$:
\begin{equation}
\q^\dagger = (w, \v)^\dagger = (w, - \v)
\end{equation}
\item The quaternion product $\diamond$:
\begin{equation}
\q_0 \diamond \q_1 = (w_0w_1 - \v_0 \bullet \v_1, w_0 \v_1 + w_1 \v_0 + \v_0 \times \v_1)
\end{equation}
\item Quaternion norm $\| \|_{\Qs}$:
\begin{equation}
\| \q \|_{\Qs}^2
= \q^\dagger \bullet \q
= \q \bullet \q^\dagger
= w^2 + \|\v\|^2
= w^2 + x^2 + y^2 + z^2
\end{equation}
\end{itemize}

The quaternion product can be extended to $\Re^3$
by identifying $\p \in \Re^3$ with
the quaternion $(0,\p)$.
This allows us to define a linear transform
on $\Re^3$ for any quaternion:
\begin{equation}
\Sc(\q) \p = \q \diamond \p \diamond \q^\dagger
\end{equation}

It turns out that transforms $\Sc(\q)$ so defined are scaled rotations.
The scale is the squared norm of the quaternion $\| \q \|_{\Qs}^2$.
The rotation is about the axis of $\v$
by $\cos^{-1}(w / \| \q \|_{\Qs})$.
(Note that $\q$ and $-\q$ correspond to the same scaled rotation.)

$\Sc(\q)$ can be written as a matrix:
\begin{equation}
\label{eq:quaternion-matrix}
\Sc(\q) =
\left(
\begin{array}{ccc}
w^2 + x^2 - y^2 - z^2 & 2(xy - wz)            & 2(xz + wy)           \\
2(xy + wz)            & w^2 - x^2 + y^2 - z^2 & 2(yz - wx)           \\
2(xz - wy)            & 2(yz + wx)            & w^2 - x^2 - y^2 +z^2
\end{array}
\right)
\end{equation}

Notice that the adjoint (transpose) of $\Sc(\q)$
is the linear transform corresponding to the conjugate quaternion:
$\Sc^{\dagger}(\q) =  \Sc(\q^{\dagger})$.

If we consider $\Sc(\q)$ to be a 9-dimensional vector,
as in equation \ref{eq:L-vector},
then the derivative can be expressed as the matrix:
\begin{equation}
\label{eq:quaternion-derivative-matrix}
\Df{q}{\Sc(\q)}
 = 2 \left(
\begin{array}{rrrr}
  w &  x & -y & -z \\
  z &  y &  x &  w \\
 -y &  z & -w &  x \\
 -z &  y &  x & -w \\
  w & -x &  y & -z \\
  x &  w &  z &  y \\
  y &  z &  w &  x \\
 -z & -w &  z &  y \\
  w & -x & -y &  z \\
\end{array}
\right)
\end{equation}

It's also useful to express the derivative $\Df{q}{\Sc(\q)}$
as a set of partial derivative matrices,
which are computed
by differentiating the elements of matrix in equation
\ref{eq:quaternion-matrix}:
\begin{eqnarray}
\label{eq:quaternion-matrix-partial-derivatives}
\Df{w}{\Sc(\q)}
& = &
2 \left(
\begin{array}{rrr}
 w & -z &  y \\
 z &  w & -x \\
-y &  x &  w
\end{array}
\right)
\\
\nonumber
\\
\Df{x}{\Sc(\q)}
& = &
2 \left(
\begin{array}{rrr}
 x &  y &  z \\
 y & -x & -w \\
 z &  w & -x
\end{array}
\right)
\nonumber
\\
\nonumber
\\
\Df{y}{\Sc(\q)}
& = &
2 \left(
\begin{array}{rrr}
-y &  x &  w \\
 x &  y &  z \\
-w &  z & -y
\end{array}
\right)
\nonumber
\\
\nonumber
\\
\Df{z}{\Sc(\q)}
& = &
2 \left(
\begin{array}{rrr}
-z & -w &  x \\
 w & -z &  y \\
 x &  y &  z
\end{array}
\right)
\nonumber
\end{eqnarray}

Note that
$\Df{w}{\left( \Sc(\q) \; \p \right)}
 = \left( \Df{w}{\Sc(\q)}\right) \; \p$
(and similarly for the partials with respect to $x,y,$ and $z$).

We can write the total derivative in terms of the partials as:
\begin{eqnarray}
\Df{\q}{\left( \Sc(\q) \; \p \right)}
& = &
\left( \Df{\w}{\Sc(\q)} \; \p \right) \otimes \e_w
\\
& + &
\left( \Df{\x}{\Sc(\q)} \; \p \right) \otimes \e_x
\nonumber
\\
& + &
\left( \Df{\y}{\Sc(\q)} \; \p \right) \otimes \e_y
\nonumber
\\
& + &
\left( \Df{\z}{\Sc(\q)} \; \p \right) \otimes \e_z
\nonumber
\end{eqnarray}

In computing the gradients of registration penalties,
we sum expressions like
$\Df{\q}{\left( \Sc(\q)\p_i \right)}^{\dagger} \;
\Gf{\p_i}{f}$
(see equation \ref{eq:registration-gradient-sum}).
In terms of the partial derivative matrices,
this is:
\begin{equation}
\Df{\q}{\left( \Sc(\q)\p_i \right)}^{\dagger} \;
\Gf{\p_i}{f}
=
\left(
\begin{array}{c}
\left( \Df{\w}{\Sc(\q)} \; \p \right) \bullet \Gf{\p_i}{f} \\
\left( \Df{\x}{\Sc(\q)} \; \p \right) \bullet \Gf{\p_i}{f} \\
\left( \Df{\y}{\Sc(\q)} \; \p \right) \bullet \Gf{\p_i}{f} \\
\left( \Df{\z}{\Sc(\q)} \; \p \right) \bullet \Gf{\p_i}{f} \\
\end{array}
\right)
\end{equation}

We sometimes encounter the inverse transform, $\Sc^{-1}(\q)$,
which is the same as the linear transform,
$\Sc(\q^{-1})$,
corresponding to the inverse quaternion
(in the sense of the quaternion product):
$\q^{-1} = (w, x, y, z)^{-1}
         = {1 \over {\| q \|_{\Qs}^2}} (w, -x, -y, -z)$.

To compute derivatives of expressions
involving an inverse quaternion,
we can use the derivative of $\q^{-1}$
with respect to $\q$:
\begin{equation}
\Df{\q}{\q^{-1}}
=
{1 \over {\| \q \|_{\Qs}^4}}
\left(
\begin{array}{cccc}
-w^2+x^2+y^2+z^2 & -2wx              & -2wy              & -2wz \\
 2wx             & -w^2+x^2-y^2-z^2  &  2xy              &  2xz \\
 2wy             &   2xy             & -w^2-x^2+y^2-z^2  &  2yz \\
 2wz             &   2xz             &  2yz              &  -w^2-x^2-y^2+z^2 \\
\end{array}
\right)
\end{equation}

\paragraph{Rotations}
\label{sec:Rotations}

A common representation for the rotations on $\Re^3$
is the set of {\it unit quaternions},
that is, the quaternions satisfying $\| \q \|_{\Qs} = 1$.
However, optimization under a nonlinear constraint
like $\| \q \|_{\Qs} = 1$ is relatively difficult,
so I instead use a redundant representation by general quaternion:
\begin{equation}
\R(\q) = {{\Sc(\q)} \over {\| \q \|_{\Qs}^2}}
\end{equation}
To avoid numerical instability, it's usually enough
to add a small penalty for $\| \q \|_{\Qs}$ far from $1$.

The partial derivative matrices of $\R(\q)$ are used in the
same way as,
and can be expressed in terms of,
the partials of $\Sc(\q)$:
\begin{equation}
\Df{v}{\R(\q)}
\; = \;
\Df{v}{\left(
{{\Sc(\q)} \over {\| \q \|_{\Qs}^2}}
\right)}
\; = \;
{{\Df{v}{\Sc(\q)}} \over {\| \q \|_{\Qs}^2}}
-
{{2v \Sc(\q)} \over {\| \q \|_{\Qs}^4}}
\end{equation}
where $v$ is any of $w$, $x$, $y$, or $z$.

\subsubsection{Shift registration}
\label{sec:Shift-registration}

A simple shift, or translation, adds a constant vector
to its argument: $\Tr \p = \p + \t,$
for some $\t \in \Re^3$.
The derivative with respect to
an unconstrained translation vector
is simply
$\Df{\t}{(\p + \t)} = \I_3,$
where $\I_3$ is the identity on $\Re^3$.

\subsubsection{Affine registration}
\label{sec:affine-registration}

An {\it affine transformation,} $\A : \Re^m \mapsto \Re^n$,
is a linear transformation plus a translation:
$\A \p = \L \p + \t$,
where $\L : \Re^m \mapsto \Re^n$ is a linear transform,
the {\it linear part} of the affine tranform,
and $\t \in \Re^n$ is $\A$'s {\it translation}.

Note that, if $\A = (\L,\t)$, then its inverse is
$\A^{-1} = (\L^{-1}, - \L^{-1}\t)$.

(It is sometimes useful to use a redundant representation:
$\A \p = \L (\p + \t_0) + \t_1$,
where $\t_0 \in \Re^m$ and $\t_1 \in \Re^n$,
but, for simplicity, I'll stick to the minimal one-translation
representation in this discussion.)

In the context of mesh registration,
where $\A : \Re^3 \mapsto \Re^3$,
we can view $\L \in \Re^9$ and
$\A = (\L, \t) \in \left(\Re^9 \oplus \Re^3 \right)= \Re^{12}$,
and we can express derivatives with respect to $\A$
in terms of the independent partial derivatives
with respect to $\L$ and $\t$.

It follows from the results in the preceding sections,
that the derivative with respect to the
unconstrained set of affine registration transforms is:
\begin{equation}
\label{eq:affine-transform-derivative}
\Df{(\L,\t)}{\left( \A \, \p \right)}
 \; = \;
\Df{(\L,\t)}{\left( \L \p + \t \right)}
 \; = \;
\Tr_{\p} \oplus \I_3,
\end{equation}
where $\oplus$ indicates,
as in equation \ref{eq:diagonal-blocks},
that the derivative is formed from the 2
'diagonal blocks'.

\subsubsection{Euclidean registration}
\label{sec:euclidean-registration}

A {\it euclidean transform} is an affine transform
$\Eu : \Re^n \mapsto \Re^n$,
$\Eu \p = \Sc \p + \t $,
whose linear part is a scaled rotation:
$\Sc = s \R$,
where $s \in \Re$ and $\R : \Re^n \mapsto \Re^n$
is a rotation.

Euclidean transforms are easy to invert.

If $\Eu = (\Sc,\t)$, then its inverse is
$\Eu^{-1} = (\Sc^{-1}, - \Sc^{-1}\t)$.
The inverse of a rotation, $\R$, is its adjoint
(tranpose) $\R^{-1} = \R^{\dagger}$.
The inverse of a scaled rotation $\Sc = s \R$
is therefore
$\Sc^{-1} = (s \R)^{-1}
         = {1 \over s} \R^{\dagger}
         = {1 \over {s^2}} \Sc^{\dagger}$.
We can then write the inverse of a euclidean transform as
$\Eu^{-1} = {1 \over {s^2}}(\Sc^{\dagger}, - \Sc^{\dagger}\t)$

Using quaternions, we can represent euclidean transforms with
7 dimensional points:
$\Eu = (w_{\q}, x_{\q}, y_{\q}, z_{\q}, x_{\t}, y_{\t}, z_{\t})$,
where $\q = (w_{\q}, x_{\q}, y_{\q}, z_{\q})$ is a quaternion corresponding
to $\S$, the linear part of $\Eu$,
and $\t = (x_{\t}, y_{\t}, z_{\t})$ is the translation part.

In the 7-dimensional representation, the inverse is:
\begin{equation}
\Eu^{-1} =
{1 \over {\| \q \|_{\Qs}^4}}
\left(
\begin{array}{rcrcr}
& &  w_{\q} & & \\
& & -x_{\q} & & \\
& & -y_{\q} & & \\
& & -z_{\q} & & \\
(-w_{\q}^2 - x_{\q}^2 + y_{\q}^2 + z_{\q}^2)
x_{\t}
&
-
&
2 (w_{\q}z_{\q} + x_{\q}y_{\q})
y_{\t}
&
+
&
2 (w_{\q}y_{\q} - x_{\q}z_{\q})
z_{\t}
\\
 2 (w_{\q}z_{\q} - x_{\q}y_{\q})
x_{\t}
&
+
&
(-w_{\q}^2 + x_{\q}^2 - y_{\q}^2 + z_{\q}^2)
y_{\t}
&
-
&
2 (w_{\q}x_{\q} + y_{\q}z_{\q})
z_{\t}
\\
- 2 (w_{\q}y_{\q} + x_{\q}z_{\q})
x_{\t}
&
+
&
2 (w_{\q}x_{\q} - y_{\q}z_{\q})
y_{\t}
&
+
&
(-w_{\q}^2 + x_{\q}^2 + y_{\q}^2 - z_{\q}^2)
z_{\t}
\end{array}
\right)
\end{equation}

In computing the gradients of registration penalties,
we sum expressions like
$\Df{\q, \t}{\left( \Eu(\q,\t)\p_i \right)}^{\dagger} \;
\Gf{\p_i}{f}$
(see equation \ref{eq:registration-gradient-sum}).
Using the results in sections
\ref{sec:Scaled-rotations}
and
\ref{sec:Shift-registration},
it's not hard to see that,
in terms of the partial derivative matrices,
this is:
\begin{equation}
\label{eq:euclidean-transform-gradient}
\left[
\Df{\q,\t}{\; \Eu(\q,\t)\p_i}
\right]^{\dagger} \;
\Gf{\p_i}{f}
=
\left(
\begin{array}{c}
\left( \Df{\w_{\q}}{\Sc(\q)} \; \p_i \right) \bullet \Gf{\p_i}{f} \\
\left( \Df{\x_{\q}}{\Sc(\q)} \; \p_i \right) \bullet \Gf{\p_i}{f} \\
\left( \Df{\y_{\q}}{\Sc(\q)} \; \p_i \right) \bullet \Gf{\p_i}{f} \\
\left( \Df{\z_{\q}}{\Sc(\q)} \; \p_i \right) \bullet \Gf{\p_i}{f} \\
\end{array}
\right)
\oplus
\Gf{\p_i}{f}
\end{equation}

\subsubsection{Rigid registration}
\label{sec:rigid-registration}

A {\it rigid transform} is an affine transform
$\G \p  = \R \p + \t $,
whose linear part is a rotation, $\R$.
Using the representation, $\R(\q)$,
using {\em non-unit} quaternions,
presented in \autoref{sec:Rotations},
we get the same results for the gradient terms
as in equation \ref{eq:euclidean-transform-gradient},
with the partials of $\Sc(\q)$ replaced by the
partials of $\R(\q)$, that is:
\begin{equation}
\label{eq:rigid-transform-gradient}
\left[
\Df{\q,\t}{\; \G(\q,\t)\p_i}
\right]^{\dagger} \;
\Gf{\p_i}{f}
=
\left(
\begin{array}{c}
\left( \Df{\w_{\q}}{\R(\q)} \; \p_i \right) \bullet \Gf{\p_i}{f} \\
\left( \Df{\x_{\q}}{\R(\q)} \; \p_i \right) \bullet \Gf{\p_i}{f} \\
\left( \Df{\y_{\q}}{\R(\q)} \; \p_i \right) \bullet \Gf{\p_i}{f} \\
\left( \Df{\z_{\q}}{\R(\q)} \; \p_i \right) \bullet \Gf{\p_i}{f} \\
\end{array}
\right)
\oplus
\Gf{\p_i}{f}
\end{equation}

%\cleardoublepage

%\chapter{Averaging}
%\label{sec:Averaging}

A {\it mesh catalog} is a set $\{ \M_0 \ldots \M_{n-1} \}$
of $n$ meshes which are all embeddings
of the same simplicial complex
(that is, they have the same vertices, edges, and faces,
but the vertex positions may be different).

\subsection{Linear weights}
\label{sec:Linear-weights}

We can compute a {\it weighted average mesh} by averaging the vertex positions:
\begin{equation}
\p_i(\w) = \sum_{j=0}^{n-1} w_j \p_{ij}
\end{equation}
where $\p_{i}; (i=0 \ldots m-1)$ is the position of the $i$th vertex of the average mesh,
and $\p_{ij}$ is the position of the $i$th vertex of the $j$th catalog element.

In {\it catalog fitting} we fit a registered weighted average mesh,
to a set of data, $\{ \d_k \in \Reals^3; k=0 \ldots p-1 \}$
by minimizing
\begin{equation}
f(\M(\Tr,\w)) = \sum_{k=0}^{p-1} \| \d_k - \Pr_{\M(\Tr,\w)} (\d_k) \|^2 ,
\end{equation}
over a family of registration transforms $\{\Tr\}$
and vectors of weights $\w \in \Reals^n$.
The vertex positions $\p(\Tr,\w) = (\p_0(\Tr,\w), \ldots , \p_{n-1}(\Tr,\w))$
of the registered average mesh, $\M(\Tr,\w)$, are given by:
\begin{equation}
\p_i(\Tr,\w) = \Tr ( \sum_{j=0}^{n-1} w_j \p_{ij} )
\end{equation}

The partial derivative with respect to $\Tr$
is:
\begin{eqnarray}
\De{\Tr}{f(\M(\Tr,\w))}{\Tr^0,\w^0}
& = &
\De{\p}{f(\M))}{\M(\Tr^0,\w^0)}
\circ
\De{\Tr}{\p(\Tr,\w)}{\Tr^0,\w^0}
\\
& = &
\De{\p}{f(\M))}{\M(\Tr^0,\w^0)}
\circ
\De{\Tr}{\Tr\p(\w^0)}{\Tr^0}
\nonumber
\end{eqnarray}

$\De{\p}{f(\M))}{\M(\Tr^0,\w^0)}$ is the derivative of $f$ with respect to
the vertex positions of of the registered average mesh.
It's value for data fitting distance functions
is given in \autoref{sec:data-fitting}.

$\Df{\Tr}{\Tr\p}$ is given in \autoref{sec:Transforms}
for various families of transforms $\{\Tr\}$: affine, euclidean, and rigid.

Using the chain rule, the partial derivative with respect to $\w$ is:
\begin{eqnarray}
\De{\w}{f(\M(\Tr,\w))}{\Tr^0,\w^0}
& = &
\De{\w}{f(\Tr(\M(\w)))}{\Tr^0,\w^0}
\\
& = &
\De{\p}{f(\M))}{\M(\Tr^0,\w^0)}
\circ
\De{\Tr}{\Tr\p(\w^0)}{\Tr^0}
\circ
\De{\w}{\p(\w)}{\w^0}
\nonumber
\end{eqnarray}

$\Df{\p}{f}$ and $\Df{\Tr}{\Tr\p}$ are already known
so we need only determine $\Df{\w}{\p(\w)}$.
Using reasoning similar
to equation \ref{eq:total-registration-transform-derivative},
we have $\p(\w) = \bigoplus_{j=0}^{m-1} \p_i(\w)$,
and
$\Df{\w}{\p(\w)}
=
\Df{\w}{\bigoplus_{j=0}^{m-1} \p_i(\w)}
=
\bigoplus_{j=0}^{m-1} \Df{\w}{\p_i(\w)}$,
so we can restrict ourselves to
$\Df{\w}{\p(\w)}$, where $\p(\w) = \sum_{i=0}^{n-1} w_i \p_i$,
and $\p, \p_i \in \Reals^3$.

Note that $\p(\w)$ is a linear transform from $\Reals^n \mapsto \Reals^3$,
which can be expressed as $\P = \sum_{i=0}^{n-1} \p_i \otimes \e_i$,
where $\e_i$ are the canonical basis vectors of $\Reals^n$.
($\P$ can be written as a matrix whose $i$th column is the vector $\p_i$.)
Thus it follows that
\begin{equation}
\Df{\w}{\p(\w)} = \Df{\w}{\P\w} = \P
\end{equation}

\subsection{Convex weights}
\label{sec:Convex-weights}

It may be desireable to restrict the weights to be convex,
that is, $0 \leq w_i \leq 1; \sum w_i = 1$.
To use unconstrained optimization methods with convex weights,
we re-parameterize:
\begin{equation}
u_i(\w) = {{w_i^2} \over {\| \w \|^2}}
\end{equation}
where, as usual, $\| \w \|^2 = \sum_{j=0}^{n-1} w_j^2$.

Then we need to compute
\begin{eqnarray}
\De{\w}{\p(\u(\w))}{\w^0}
& = &
\De{\u}{\p(\u)}{\u(\w^0)}
\circ
\De{\w}{\u(\w)}{\w^0}
\\
& = &
\P
\circ
\De{\w}{\u(\w)}{\w^0}
\nonumber
\end{eqnarray}

The partial derivatives are
\begin{eqnarray}
\Df{w_j}{u_i(\w))}
& = &
\Df{w_j}{{w_i^2} \over {\| \w \|^2}}
\\
& = &
{{2 w_j} \over {\| \w \|^4}} \left( \delta_{ij} \| \w \|^2 - w_i^2 \right)
\nonumber
\end{eqnarray}


\glsaddallunused
%-------------------------------------------------------------------------------
\appendix
\part{Appendices}
\input{typesetting}
\chapter{Annoyances}

\section{mathematics}

\section{computation}

\subsection{Java}
\lstset{language=Java}

Generics

Auto boxing/unboxing.

Sorted collections can't handle partial ordering.

BigDecimal.valueOf(x).doubleValue() != x.

\subsection{Clojure}
\lstset{language=Clojure}

Standard math notation: 
enumerated finite set: $\aSet{S} = \{0, 1, 2\}$ 
and cardinality $\#\{0, 1, 2\}
\rightarrow 3$.

\lstinline|#{0 1 2}| $\rightarrow$ \lstinline|java.util.Set|
\lstinline|(count #{0 1 2})| $\rightarrow$
\lstinline|3|.

and
\lstinline|{:x 1 :y 2}| $\rightarrow$ \lstinline|java.util.Map|.

(double (rationalize (double x))) != (double x)
%-------------------------------------------------------------------------------
\backmatter


\part{Backmatter}
%------------------------------------------------------------------------------
% \begingroup  % Temporarily disable \clearpage to show both lists on one page
%   %\let\clearpage\relax    % http://tex.stackexchange.com/a/14511/104449
%   \renewcommand{\listtheoremname}{List of definitions}
%   \textsf{\listoftheorems[ignoreall, show={definition}]}
% \endgroup
%-------------------------------------------------------------------------------
\pagebreak
\renewcommand{\listfigurename}{\scshape Figures}
\addcontentsline{toc}{chapter}{\listfigurename}
\begingroup
\let\onecolumn\twocolumn
\scshape
%\sffamily
\listoffigures
\upshape
%\rmfamily
\endgroup
%-------------------------------------------------------------------------------
\pagebreak
\renewcommand{\lstlistlistingname}{\scshape Code samples}
\addcontentsline{toc}{chapter}{\lstlistlistingname}
\begingroup
\let\onecolumn\twocolumn
\scshape
%\sffamily
\lstlistoflistings
\upshape
%\rmfamily
\endgroup
%-------------------------------------------------------------------------------
\pagebreak
\renewcommand*{\listtheoremname}{\scshape Definitions}
\addcontentsline{toc}{chapter}{\listtheoremname}
\begingroup
\let\onecolumn\twocolumn
\scshape
%\sffamily
\listoftheorems[ignoreall,show={definition}]
\upshape
%\rmfamily
\endgroup
%-------------------------------------------------------------------------------
\pagebreak
\renewcommand{\listtheoremname}{\scshape Theorems}
\addcontentsline{toc}{chapter}{\listtheoremname}
\begingroup
\let\onecolumn\twocolumn
\scshape
%\sffamily
\listoftheorems[ignoreall,show={theorem}]
\upshape
%\rmfamily
\endgroup
%-------------------------------------------------------------------------------
\pagebreak
\renewcommand{\listtheoremname}{\scshape Examples}
\addcontentsline{toc}{chapter}{\listtheoremname}
\begingroup
\let\onecolumn\twocolumn
\scshape
%\sffamily
\listoftheorems[ignoreall,show={example}]
\upshape
%\rmfamily
\endgroup
%-------------------------------------------------------------------------------
% \newglossarystyle{mystyle}{%
%  \glossarystyle{altlist}%
%  \renewcommand*{\glossaryentryfield}[5]{%
%    \item[\glsentryitem{##1}\glstarget{##1}{##2}]%
%       :\hspace{1em}##3\glspostdescription\space ##5}%
% }
\pagebreak
\printglossary[title=Glossary,toctitle=Glossary]
%-------------------------------------------------------------------------------
\pagebreak
\printbibliography[heading=bibintoc, title={References}]
%-------------------------------------------------------------------------------
\printindex
%-------------------------------------------------------------------------------

%-------------------------------------------------------------------------------
\end{document}
