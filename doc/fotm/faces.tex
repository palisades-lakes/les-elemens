\subsection{Functions of faces}
\label{sec:faces}


%-------------------------------------------------------------------------------

\subsubsection{Corner angles}
\label{sec:corner_angles}

\begin{figure}[!htp]
\centering
\begin{verbatim}

          V2/p2
            o
           /U\
          / g2\
     E20 /     \ E12
        / F012  \
       /)g0   g1(\
V0/p0 o-----------o V1/p1
           E01

\end{verbatim}
\caption{Face labeling.
\label{fig:face_labeling}}
\end{figure}

Suppose face $\F_{012}$ has vertices $\V_0, \V_1, \V_2$,
at points $\p_0, \p_1, \p_2$,
and edges $\E_{01}, \E_{12}, \E_{20}$,
as labeled in figure \ref{fig:face_labeling}.

The {\em corner angle} $\gamma_i$ in face $\F_{012}$ of vertex $\V_i$ is
the angle between the two edges in $\F_{012}$ that meet at $\V_i$:
\begin{eqnarray}
\gamma_i
& = & \gamma(\F_{012},\V_i)
\\
& = & \gamma(\p_i,\p_{(i+1) \% 3},\p_{(i+2) \% 3})
\nonumber
\\
& = & \theta(\p_{(i+1) \% 3} - \p_i,\p_{(i+2) \% 3} - \p_i)
\nonumber
\\
& = &
\cos^{-1}
\left[
{ \left( \p_{(i+1) \% 3} - \p_i \right)
  \bullet
  \left( \p_{(i+2) \% 3} - \p_i \right) }
\over
{ \| \p_{(i+1) \% 3} - \p_i \|
  \| \p_{(i+2) \% 3} - \p_i \| }
\right]
\nonumber
\end{eqnarray}

Corner angles vary between $0$ and $\pi$, with both extremes
corresponding to singular, zero-area faces.
The sum of corner angles for a given face is always $\pi$.

Using equation \ref{eq:angle_gradient},
it's easy to see that the partial gradients are:
\begin{eqnarray}
\Gc{\p_0}{\gamma(\p_0,\p_1,\p_2)}{\q}
& = &
{{
\left[ (\q_1 -\q_0) \perp (\q_2 - \q_0) \right]
+
\left[ (\q_2 -\q_0) \perp (\q_1 - \q_0) \right]
}
\over
{ \sqrt{\|\q_1 - \q_0\|^2\|\q_2 - \q_0\|^2 -
\left( (\q_1 - \q_0) \bullet (\q_2 - \q_0) \right)^2 }}
}
\\
\Gc{\p_1}{\gamma(\p_0,\p_1,\p_2)}{\q}
& = &
{{-(\q_1 -\q_0) \perp (\q_2 - \q_0)}
\over
{ \sqrt{\|\q_1 - \q_0\|^2\|\q_2 - \q_0\|^2 -
\left( (\q_1 - \q_0) \bullet (\q_2 - \q_0) \right)^2 }}
}
\nonumber
\\
\Gc{\p_2}{\gamma(\p_0,\p_1,\p_2)}{\q}
& = &
{{-(\q_2 -\q_0) \perp (\q_1 - \q_0)}
\over
{ \sqrt{\|\q_1 - \q_0\|^2\|\q_2 - \q_0\|^2 -
\left( (\q_1 - \q_0) \bullet (\q_2 - \q_0) \right)^2 }}
}
\nonumber
\end{eqnarray}

%-------------------------------------------------------------------------------

\subsubsection{Functions of face normals}
\label{sec:normals}

A number of important functions of triangular meshes,
such as surface area,
are based on face normal vectors.

%-------------------------------------------------------------------------------

\paragraph{Area-weighted face normal}
\label{sec:areanormal}

Suppose we have a face whose 3 vertices are at $\p = (\p_0, \p_1, \p_2)$,
where $\p_i \in \Re^3; i=0,1,2.$
(Note that the order of the $\p_i$ determines the orientation of the face.
With a face as labeled in figure \ref{fig:face_labeling},
the normal points out of the page.)

The {\it area-weighted normal} vector is
\nopagebreak
\begin{eqnarray}
\a (\p) & = & (\p_0 \times \p_1) \ + \ (\p_1 \times \p_2) \ + \ (\p_2 \times \p_0) \\
        & = & (\p_1 - \p_0) \ \times \ (\p_2 - \p_0) \nonumber \\
        & = & (\p_2 - \p_1) \ \times \ (\p_0 - \p_1) \nonumber \\
        & = & (\p_0 - \p_2) \ \times \ (\p_1 - \p_2) \nonumber
\end{eqnarray}

The 'partial' derivatives of the area-weighted normal are:
\begin{eqnarray}
\Dd{\p_0}{\a}{\q}{r_0} \
& = \ (\r0 \times \q_1) \ + \ (\q_2 \times \r_0) & = (\q_2 - \q_1) \times \r_0 \\
\Dd{\p_1}{\a}{\q}{r_1} \
& = \ (\r1 \times \q_2) \ + \ (\q_0 \times \r_1) & = (\q_0 - \q_2) \times \r_1 \nonumber \\
\Dd{\p_2}{\a}{\q}{r_2} \
& = \ (\r2 \times \q_0) \ + \ (\q_1 \times \r_2) & = (\q_1 - \q_0) \times \r_2 \nonumber
\end{eqnarray}

Note that $\Df{\p}{\a}$ is {\it skew-symmetric}, that is,
$\Df{\p}{\a}^{\dagger} = -\Df{\p}{\a}.$

%-------------------------------------------------------------------------------

\paragraph{Face area}
\label{sec:facearea}

The area of a face is half the length of the area-weighted normal:
\begin{eqnarray}
A(\p)
& = & {1 \over 2} \| \ \a(\p) \ \|  \\
& = & {1 \over 2} \| \ (\p_0 \times \p_1) \ + \ (\p_1 \times \p_2) \ + \ (\p_2 \times \p_0) \ \|.
\nonumber
\end{eqnarray}

It follows from equation \ref{eq:norm_derivative}
that the first 'partial' derivative of the face area is:
\begin{eqnarray}
\label{eq:area_partial_derivative}
\Dd{\p_0}{A}{\q}{\r_0}
& = &
{{\a(\q)^\dagger} \over {2\|\a(\q)\|}}
{\Dd{\p_0}{\a}{\q}{\r_0}}  \\
& = &
{{\a(\q)} \over {2\|\a(\q)\|}}
\bullet
\left[(\r_0 \times \q_1) + (\q_2 \times \r_0)\right] \nonumber \\
& = &
{{\a(\q)} \over {2\|\a(\q)\|}}
\bullet
\left[(\q_2 - \q_1) \ \times \  \r_0)\right] \nonumber \\
& = &
{{\a(\q) \times (\q_2 - \q_1)} \over {2\|\a(\q)\|}}
\bullet
\r_0 \nonumber
\end{eqnarray}

The last identity follows from equation \ref{eq:dot_cross}.

The 'partial' gradients of the face area are then:
\begin{eqnarray}
\Gc{\p_0}{A}{\q} & = & {{\a(\q) \times (\q_2 - \q_1)} \over {2\|\a(\q)\|}} \\
\Gc{\p_1}{A}{\q} & = & {{\a(\q) \times (\q_0 - \q_2)} \over {2\|\a(\q)\|}} \nonumber \\
\Gc{\p_2}{A}{\q} & = & {{\a(\q) \times (\q_1 - \q_0)} \over {2\|\a(\q)\|}} \nonumber
\end{eqnarray}

More simply, using the face unit normal \( \n(\p)  =  {{\a(\p)} \over {\| \a(\p) \|}} \):
\begin{eqnarray}
\label{eq:area_gradient}
\Gc{\p_0}{A}{\q} & = & \frac{\n(\q)}{2} \times (\q_2 - \q_1) \\
\Gc{\p_1}{A}{\q} & = & \frac{\n(\q)}{2} \times (\q_0 - \q_2) \nonumber \\
\Gc{\p_2}{A}{\q} & = & \frac{\n(\q)}{2} \times (\q_1 - \q_0) \nonumber
\end{eqnarray}

%-------------------------------------------------------------------------------

\paragraph{Face unit normal vector}
\label{sec:facenormal}

The unit vector normal to a face whose vertices are at
$\p = (\p_0,\p_1,\p_2)$ is just the area weighted normal (see \autoref{sec:areanormal})
adjusted to length 1:
\begin{equation}
\n(\p)  =  {{\a(\p)} \over {\| \a(\p) \|}}
\end{equation}

Following equation \ref{eq:normalized_function_derivative}, the derivative is:

\begin{eqnarray}
\label{eq:unit_normal_derivative}
\Dc{\n}{\p}{\q}
&  =
& { \left( {\| \a(\p) \|^2 \I_{\Re^3}  -  \a(\p) \otimes \a(\p) } \over {\| \a(\p) \|^3} \right) }
\; \Dc{\a}{\p}{\q}
 \\
& & \nonumber\\
&  =
& { \left( {\| \a(\p) \|^2 \I_{\Re^3}  -  \a(\p) \otimes \a(\p) } \over {\| \a(\p) \|^3} \right)} \ast
\nonumber \\
&    &
\left[ \left( \q_0 \times \p_1 \right) + \left( \p_2 \times \q_0 \right)
+
\left( \q_1 \times \p_2 \right) + \left( \p_0 \times \q_1 \right)
+
\left( \q_2 \times \p_0 \right) + \left( \p_1 \times \q_2 \right) \right]
\nonumber \\
& & \nonumber\\
&  =
& { \left( {\| \a(\p) \|^2 \I_{\Re^3}  -  \a(\p) \otimes \a(\p) } \over {\| \a(\p) \|^3} \right)} \ast
\nonumber \\
&    &
\left[ \left( (\p_2 - \p_1) \times \q_0 \right)
+
\left( (\p_0 - \p_2) \times \q_1 \right)
+
\left( (\p_1 - \p_0) \times \q_2 \right) \right]
\nonumber \\
& & \nonumber\\
&  =
& { \left( {\I_{\Re^3}  -  \n(\p) \otimes \n(\p) } \over {\| \a(\p) \|} \right)} \ast \; \Dc{\a}{\p}{\q}
\nonumber
\end{eqnarray}

%-------------------------------------------------------------------------------

\subsubsection{Aspect Ratio}
\label{sec:aspect_ratio}

Minimizing a measure of face aspect ratio can help maintain
a well-conditioned mesh.
Maximizing it may help in discovering collapse-able edges.

%-------------------------------------------------------------------------------

\paragraph{Squared edge lengths over area}
\label{sec:Squared-edge-lengths-over-area}

One measure of the deviation of a face from equilaterality is:
\begin{equation}
{\mathrm L2A}(\p_0,\p_1,\p_2)
=  {{ \| \p_0 - \p_1 \|^2 + \| \p_1 - \p_2 \|^2 + \| \p_2 - \p_0 \|^2 }
\over
{A(\p)} }
\end{equation}

Using \ref{eq:area_gradient}, it follows that the
partial gradients of L2A are:
\begin{eqnarray}
\label{eq:L2A_gradient}
\Gc{\p_0}{L2A}{\q}
& =
&
\left(
\frac{2\left[ \left( \p_0 - \p_1 \right) + \left( \p_0 - \p_2 \right) \right]}
{A(\q)}
\right)
\\
& - &
\left(
\frac{ \| \p_0 - \p_1 \|^2 + \| \p_1 - \p_2 \|^2 + \| \p_2 - \p_0 \|^2 }
{2 A^2(\q)}
\left[ \n(\q) \times (\q_2 - \q_1)
\right]
\right)
\nonumber \\
\Gc{\p_1}{L2A}{\q}
& =
&
\left(
\frac{2\left[ \left( \p_1 - \p_2 \right) + \left( \p_1 - \p_0 \right) \right]}
{A(\q)}
\right)
\nonumber
\\
& - &
\left(
\frac{ \| \p_0 - \p_1 \|^2 + \| \p_1 - \p_2 \|^2 + \| \p_2 - \p_0 \|^2 }
{2 A^2(\q)}
\left[ \n(\q) \times (\q_0 - \q_2) \right]
\right)
\nonumber
\\
\Gc{\p_2}{L2A}{\q}
& =
&
\left(
\frac{2\left[ \left( \p_2 - \p_0 \right) + \left( \p_2 - \p_1 \right) \right]}
{A(\q)}
\right)
\nonumber
\\
& - &
\left(
\frac{ \| \p_0 - \p_1 \|^2 + \| \p_1 - \p_2 \|^2 + \| \p_2 - \p_0 \|^2 }
{2 A^2(\q)}
\left[ \n(\q) \times (\q_1 - \q_0) \right]
\right)
\nonumber
\end{eqnarray}