% jam 2004-09-05

\section{Barycentric coordinates}
\label{sec:barycentric-coordinates}

In this section, I describe how to compute
the barycentric coordinates of a point $\q$
with respect to a $n$-simplex in
a finite dimensional inner product space $\Vspace$
(with $\dimension (\Vspace ) = m \ge n$) ---
and the derivatives with respect to the vertex positions.

Let $S$ be the geometric $n$-simplex
with vertex positions $\{\p_0 , \ldots , \p_n\}$;
in $\Vspace$.

The {\it span} of $S$, $\affine_span ( S )$,
is the affine span of its vertex positions
$\affine_span \{\p_0 , \ldots , \p_n\}$,
that is, the set of points $\q$ such that
$\q = \sum_{j=n}^{m} b_j \p_j $ where $1 = \sum_{j=0}^{n} b_j $.
$b_j$ are {\it barycentric coordinates} of $\q$ with respect to $S$.
Barycentric coordinates of an arbitrary point $\q \in \Vspace$,
are barycentric coordinates of the closest point in $\affine_span(S)$.
If $S$ is non-degenerate (the dimension of its span is $m$),
then the barycentric coordinates of any point are unique.

Assume in what follows that $S$ is non-degenerate.
When faced with for a degenerate simplex, there are several options.
One can use the minimum norm $\b$.
It is often possible to use the barycentric coordinates
with respect to the 'largest' non-degenerate face instead.
We may also want to choose $\b$ to be as close to convex,
$0 \le \b_j \le 1$, as possible. {\it (How?)}

Let $\{ \v_0 \ldots \v_{n-1} \} = \{ (\p_0 - \p_n) \ldots (\p_{n-1} - \p_n)\},$
and let $\Vmap = \sum_{j=0}^{n-1} ( \v_j \otimes \e_j^{\Reals^n} )$,
the linear map from $\Reals^n \mapsto \Vspace$,
whose 'columns' are the $\v_j$.
Let $\b = \left( \b_0 \ldots \b_{n-1} \right) \in \Reals^n$
be the unconstrained vector of the first $n$ barycentric coordinates.
For $\q \in \affine_span ( S )$, $\q = \Vmap \b + \p_m$
and $\b = \Vmap^{-1} \left ( \q - \p_m \right)$.
For arbitrary $\q \in \Vspace$,
$\b = \Vmap^{-} (\q - \p_m)$
(see section \ref{sec:Inverses-and-pseudo-inverses}).













